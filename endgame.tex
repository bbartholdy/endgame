% Options for packages loaded elsewhere
\PassOptionsToPackage{unicode}{hyperref}
\PassOptionsToPackage{hyphens}{url}
%
\documentclass[
  letterpaper,
]{book}
\usepackage{amsmath,amssymb}
\usepackage{lmodern}
\usepackage{iftex}
\ifPDFTeX
  \usepackage[T1]{fontenc}
  \usepackage[utf8]{inputenc}
  \usepackage{textcomp} % provide euro and other symbols
\else % if luatex or xetex
  \usepackage{unicode-math}
  \defaultfontfeatures{Scale=MatchLowercase}
  \defaultfontfeatures[\rmfamily]{Ligatures=TeX,Scale=1}
\fi
% Use upquote if available, for straight quotes in verbatim environments
\IfFileExists{upquote.sty}{\usepackage{upquote}}{}
\IfFileExists{microtype.sty}{% use microtype if available
  \usepackage[]{microtype}
  \UseMicrotypeSet[protrusion]{basicmath} % disable protrusion for tt fonts
}{}
\makeatletter
\@ifundefined{KOMAClassName}{% if non-KOMA class
  \IfFileExists{parskip.sty}{%
    \usepackage{parskip}
  }{% else
    \setlength{\parindent}{0pt}
    \setlength{\parskip}{6pt plus 2pt minus 1pt}}
}{% if KOMA class
  \KOMAoptions{parskip=half}}
\makeatother
\usepackage{xcolor}
\usepackage{longtable,booktabs,array}
\usepackage{calc} % for calculating minipage widths
% Correct order of tables after \paragraph or \subparagraph
\usepackage{etoolbox}
\makeatletter
\patchcmd\longtable{\par}{\if@noskipsec\mbox{}\fi\par}{}{}
\makeatother
% Allow footnotes in longtable head/foot
\IfFileExists{footnotehyper.sty}{\usepackage{footnotehyper}}{\usepackage{footnote}}
\makesavenoteenv{longtable}
\usepackage{graphicx}
\makeatletter
\def\maxwidth{\ifdim\Gin@nat@width>\linewidth\linewidth\else\Gin@nat@width\fi}
\def\maxheight{\ifdim\Gin@nat@height>\textheight\textheight\else\Gin@nat@height\fi}
\makeatother
% Scale images if necessary, so that they will not overflow the page
% margins by default, and it is still possible to overwrite the defaults
% using explicit options in \includegraphics[width, height, ...]{}
\setkeys{Gin}{width=\maxwidth,height=\maxheight,keepaspectratio}
% Set default figure placement to htbp
\makeatletter
\def\fps@figure{htbp}
\makeatother
\setlength{\emergencystretch}{3em} % prevent overfull lines
\providecommand{\tightlist}{%
  \setlength{\itemsep}{0pt}\setlength{\parskip}{0pt}}
\setcounter{secnumdepth}{5}
\newlength{\cslhangindent}
\setlength{\cslhangindent}{1.5em}
\newlength{\csllabelwidth}
\setlength{\csllabelwidth}{3em}
\newlength{\cslentryspacingunit} % times entry-spacing
\setlength{\cslentryspacingunit}{\parskip}
\newenvironment{CSLReferences}[2] % #1 hanging-ident, #2 entry spacing
 {% don't indent paragraphs
  \setlength{\parindent}{0pt}
  % turn on hanging indent if param 1 is 1
  \ifodd #1
  \let\oldpar\par
  \def\par{\hangindent=\cslhangindent\oldpar}
  \fi
  % set entry spacing
  \setlength{\parskip}{#2\cslentryspacingunit}
 }%
 {}
\usepackage{calc}
\newcommand{\CSLBlock}[1]{#1\hfill\break}
\newcommand{\CSLLeftMargin}[1]{\parbox[t]{\csllabelwidth}{#1}}
\newcommand{\CSLRightInline}[1]{\parbox[t]{\linewidth - \csllabelwidth}{#1}\break}
\newcommand{\CSLIndent}[1]{\hspace{\cslhangindent}#1}
\usepackage{booktabs}
\usepackage{amsthm}
\makeatletter
\def\thm@space@setup{%
  \thm@preskip=8pt plus 2pt minus 4pt
  \thm@postskip=\thm@preskip
}
\makeatother

%% Chapter formatting

%Options: Sonny, Lenny, Glenn, Conny, Rejne, Bjarne, Bjornstrup
%\usepackage[Lenny]{fncychap}

\usepackage{titlesec, blindtext, color} % titlesec needs `pandoc --variable subparagraph` to work
\definecolor{gray75}{gray}{0.75}
\newcommand{\hsp}{\hspace{20pt}}
\titleformat{\chapter}[hang]{\Huge\bfseries}{\thechapter\hsp\textcolor{gray75}{|}\hsp}{0pt}{\Huge\bfseries}
\makeatletter
\@ifpackageloaded{tcolorbox}{}{\usepackage[skins,breakable]{tcolorbox}}
\@ifpackageloaded{fontawesome5}{}{\usepackage{fontawesome5}}
\definecolor{quarto-callout-color}{HTML}{909090}
\definecolor{quarto-callout-note-color}{HTML}{0758E5}
\definecolor{quarto-callout-important-color}{HTML}{CC1914}
\definecolor{quarto-callout-warning-color}{HTML}{EB9113}
\definecolor{quarto-callout-tip-color}{HTML}{00A047}
\definecolor{quarto-callout-caution-color}{HTML}{FC5300}
\definecolor{quarto-callout-color-frame}{HTML}{acacac}
\definecolor{quarto-callout-note-color-frame}{HTML}{4582ec}
\definecolor{quarto-callout-important-color-frame}{HTML}{d9534f}
\definecolor{quarto-callout-warning-color-frame}{HTML}{f0ad4e}
\definecolor{quarto-callout-tip-color-frame}{HTML}{02b875}
\definecolor{quarto-callout-caution-color-frame}{HTML}{fd7e14}
\makeatother
\makeatletter
\makeatother
\makeatletter
\@ifpackageloaded{bookmark}{}{\usepackage{bookmark}}
\makeatother
\makeatletter
\@ifpackageloaded{caption}{}{\usepackage{caption}}
\AtBeginDocument{%
\ifdefined\contentsname
  \renewcommand*\contentsname{Table of contents}
\else
  \newcommand\contentsname{Table of contents}
\fi
\ifdefined\listfigurename
  \renewcommand*\listfigurename{List of Figures}
\else
  \newcommand\listfigurename{List of Figures}
\fi
\ifdefined\listtablename
  \renewcommand*\listtablename{List of Tables}
\else
  \newcommand\listtablename{List of Tables}
\fi
\ifdefined\figurename
  \renewcommand*\figurename{Figure}
\else
  \newcommand\figurename{Figure}
\fi
\ifdefined\tablename
  \renewcommand*\tablename{Table}
\else
  \newcommand\tablename{Table}
\fi
}
\@ifpackageloaded{float}{}{\usepackage{float}}
\floatstyle{ruled}
\@ifundefined{c@chapter}{\newfloat{codelisting}{h}{lop}}{\newfloat{codelisting}{h}{lop}[chapter]}
\floatname{codelisting}{Listing}
\newcommand*\listoflistings{\listof{codelisting}{List of Listings}}
\makeatother
\makeatletter
\@ifpackageloaded{caption}{}{\usepackage{caption}}
\@ifpackageloaded{subcaption}{}{\usepackage{subcaption}}
\makeatother
\makeatletter
\@ifpackageloaded{tcolorbox}{}{\usepackage[skins,breakable]{tcolorbox}}
\makeatother
\makeatletter
\@ifundefined{shadecolor}{\definecolor{shadecolor}{rgb}{.97, .97, .97}}
\makeatother
\makeatletter
\makeatother
\makeatletter
\makeatother
\ifLuaTeX
  \usepackage{selnolig}  % disable illegal ligatures
\fi
\IfFileExists{bookmark.sty}{\usepackage{bookmark}}{\usepackage{hyperref}}
\IfFileExists{xurl.sty}{\usepackage{xurl}}{} % add URL line breaks if available
\urlstyle{same} % disable monospaced font for URLs
\hypersetup{
  pdftitle={Putting Dental Calculus Under the Microscope},
  pdfauthor={Bjørn Peare Bartholdy},
  hidelinks,
  pdfcreator={LaTeX via pandoc}}

\title{Putting Dental Calculus Under the Microscope}
\author{Bjørn Peare Bartholdy}
\date{}

\begin{document}

%% Mandatory proefschrift page for Leiden PhD dissertations %%
\clearpage
\thispagestyle{empty}
\begin{center}
\Huge\textbf{Putting Dental Calculus Under the Microscope}\par
\vspace{\baselineskip}
\huge\textit{}\par
\vfill % this space will be whatever is left on the page
    \Large{Proefschrift}\par
    \vspace{\baselineskip}
    \linespread{1.3}
    \large{ter verkrijging van \\
    de graad van Doctor aan de Universiteit Leiden, \\
    op gezag van Rector Magnificus Dr.~H.J. Farnsworth, \\
    volgens besluit van het College voor Promoties \\
    te verdedigen op Wednesday 21 March 3020 \\
    klokke  uur \\[1.5cm]
    door} \\[1.5cm]
    \Large{Bjørn Peare Bartholdy}\par
    \vspace{\baselineskip}
    \large{geboren te Earth \\
    in the 20th century}
\end{center}
%% End: Proefschrift %%

%% Promotor and committee page %%
\clearpage
\thispagestyle{empty}

\noindent\begin{tabular}{p{8em} l}
    \large
    \textbf{Promotor}  & \large Dr.~Amanda G. Henry  \\ 
    \rule{0pt}{4ex}\large\textbf{Co-Promotor}  & \large Dr.~Annelou van
Gijn  \\ 
    \large
    \rule{0pt}{8ex}\textbf{Committee}  & \rule{0pt}{4ex}\large Dr.~Bernadette
Rostenkowski-Wolowitz  \\
    & \indent\textit{ZanGen
Pharmaceuticals} \\  & \rule{0pt}{4ex}\large Dr.~Amy Farrah Fowler  \\
    & \indent\textit{California Institute of
Technology} \\  & \rule{0pt}{4ex}\large Dr.~Sheldon Cooper  \\
    & \indent\textit{California Institute of
Technology} \\  & \rule{0pt}{4ex}\large Dr.~Leonard Hofstadter  \\
    & \indent\textit{California Institute of Technology} \\ 
\end{tabular}

\begingroup
\hspace{0.000001cm}
\vfill
\begin{flushleft}
\line(1,0){225} \\ %%%% Change colour of line to match chapters %%%%
\textbf{Front cover image:} Petra Korlevic \\[0.5cm]
\textbf{Funding:} true 
\end{flushleft}
\endgroup
%% End: promotor and committee page %%

\frontmatter
\maketitle

\ifdefined\Shaded\renewenvironment{Shaded}{\begin{tcolorbox}[borderline west={3pt}{0pt}{shadecolor}, sharp corners, interior hidden, enhanced, frame hidden, breakable, boxrule=0pt]}{\end{tcolorbox}}\fi

\renewcommand*\contentsname{Table of contents}
{
\setcounter{tocdepth}{2}
\tableofcontents
}
\listoffigures
\listoftables
\mainmatter
\hypertarget{hello}{}
\bookmarksetup{startatroot}

\chapter*{Hello}
\addcontentsline{toc}{chapter}{Hello}

\markboth{Hello}{Hello}

\frontmatter

\bookmarksetup{startatroot}

\hypertarget{acknowledgements}{%
\chapter*{Acknowledgements}\label{acknowledgements}}
\addcontentsline{toc}{chapter}{Acknowledgements}

\markboth{Acknowledgements}{Acknowledgements}

Where to begin? So many people helped shape this thesis, and are
therefore also to blame for this work.

\hypertarget{open-science-statement}{}
\bookmarksetup{startatroot}

\chapter*{Open Science Statement}
\addcontentsline{toc}{chapter}{Open Science Statement}

\markboth{Open Science Statement}{Open Science Statement}

All materials and data, including the dissertation itself, are made
available to the best of my ability. All articles in association with
the dissertation are/will be Open Access.

Protocols available: \url{https://protocols.io/workspaces/byoc}\\
Code available: \url{https://github.com/bbartholdy}\\
Data available: TBD

\clearpage
\thispagestyle{empty}
\vspace*{3cm}

\textit{En bar røv at trutte i} \vspace*{\fill}

\par
\vspace*{4cm}

\begin{itemize}
\tightlist
\item
  Jan Bartholdy
\end{itemize}

\mainmatter

\bookmarksetup{startatroot}

\hypertarget{chap-intro}{%
\chapter{Introduction}\label{chap-intro}}

Dental calculus is becoming a popular substance in research on the
behaviour and biology of people in the past. You may also know it as
tartar or mineralised plaque. In other languages the word is often
related to ``tooth stones''. In fact, calculus is itself latin for
`pebble'. This was orginially used as a term for mathematical
calculations using counting stones, and only later used to describe
calicifications in the human body
(\url{https://www.etymonline.com/word/calculus}). This can be the cause
of some confusion, as calculus is also a branch of mathematics. If you
see the the term `calculus' in this disseration, you can safely assume
that I'm referring to stuff that grows on your teeth and for which you
receive lectures from your dentist, and not the topic you dreaded in
high school.

I will briefly describe the formation of dental calculus here, but for a
more thorough review of the entire process I refer you to
\protect\hyperlink{chap-background}{chapter 2}. Dental calculus is
formed from dental plaque, a substance that forms on you teeth and
consists mainly of bacteria and a surrounding structure called the
extracellular matrix. When the local environment within and around the
plaque reaches a favourable alkaline pH, both the extracellular matrix
and bacteria within will calcify
(\protect\hyperlink{ref-jinSupragingivalCalculus2002}{Jin \& Yip, 2002};
\protect\hyperlink{ref-whiteDentalCalculus1997}{D. J. White, 1997}). The
alkaline pH causes minerals (especially calcium and phosphate) from
saliva to enter the plaque, causing the extracellular matrix and
eventually also the bacteria to harden, resulting in a concrete-like
deposit on the surface of the teeth. The process repeats itself when new
bacteria colonise the surface of the newly formed dental calculus,
creating a layered structure, though somewhat disorganised
(\protect\hyperlink{ref-akcaliDentalCalculus2018}{Akcalı \& Lang, 2018};
\protect\hyperlink{ref-jepsenCalculusRemoval2011}{Jepsen et al., 2011}).
Dental plaque can accumulate more easily on teeth (and dental calculus)
because they are a hard, non-shedding surface. Most of the surfaces in
our mouth are covered by a layer of cells called the oral epithelium.
These cells are continuously renewed as new cells are formed and dead
cells fall off (\protect\hyperlink{ref-squierOralMucosa1998}{Squier \&
Finkelstein, 1998}). This constant turnover means that it is difficult
for bacteria to build the communities they require for producing
biofilms. Enamel, the white substance that covers the crown of your
teeth, behaves differently. It stops growing when the tooth has fully
formed. After that, there is no renewal. This allows bacteria to
continue to grow and develop communities if there is no intervention
from you (or your dentist). Dental plaque can trap a variety of
different microparticles, including bacteria, human proteins, and small
debris from the food we eat
(\protect\hyperlink{ref-delafuenteDNAHuman2013}{De La Fuente et al.,
2013}; \protect\hyperlink{ref-hendyProteomicCalculus2018}{Hendy et al.,
2018}; \protect\hyperlink{ref-henryCalculusSyria2008}{Henry \& Piperno,
2008}). When the plaque mineralises, it can preserve these
microparticles over long periods of time, even after the person whose
teeth provided a home for the calculus has died.

Also, the main crystal forms in calculus strongly bind DNA, making
calculus a fantastic source of ancient DNA (aDNA) from the mouth
(\protect\hyperlink{ref-warinnerNewEra2015}{Warinner et al., 2015}).
This is probably why archaeologists have become increasingly interested
in dental calculus.

\hypertarget{intro-arch}{%
\section{Dental calculus in archaeology}\label{intro-arch}}

The main archaeological interest in dental calculus is to explore
research questions involving diet and the evolution of the oral biome
and oral health
(\protect\hyperlink{ref-adlerSequencingAncient2013}{Adler et al., 2013};
\protect\hyperlink{ref-yatesOralMicrobiome2021}{Fellows Yates et al.,
2021}; \protect\hyperlink{ref-henryCalculusSyria2008}{Henry \& Piperno,
2008}; \protect\hyperlink{ref-warinnerPathogensHost2014}{Warinner,
Rodrigues, et al., 2014};
\protect\hyperlink{ref-warinnerEvidenceMilk2014}{Warinner, Hendy, et
al., 2014}). These topics have not changed much since the early uses of
dental calculus in archaeological research, but the methods certainly
have.

In the past, some researchers would consider calculus a nuissance
because it obscured tooth and root morphology
(\protect\hyperlink{ref-scottBriefHistory2015}{G. R. Scott, 2015}); this
had made a lot of people very angry and been widely regarded as a bad
move (\protect\hyperlink{ref-adamsRestaurantEnd2002}{Adams, 2002d, p.
1}). A simple three-stage archaeolgy-specific scoring method was
developed (\protect\hyperlink{ref-brothwellDiggingBones1981}{Brothwell,
1981}), similar to a common clinical scoring sytem
(\protect\hyperlink{ref-greeneSimplifiedOral1964}{J. G. Greene \&
Vermillion, 1964}), and is still widely used today. More detailed
methods are also available
(\protect\hyperlink{ref-dobneyMethodEvaluating1987}{Dobney \& Brothwell,
1987}; \protect\hyperlink{ref-greeneQuantifyingCalculus2005}{T. R.
Greene et al., 2005}).

\begin{figure}

{\centering \includegraphics{01-intro_files/figure-pdf/fig-plot-and-wordclouds-1.pdf}

}

\caption{\label{fig-plot-and-wordclouds}Plot of articles with the term
`dental calculus' in the title.}

\end{figure}

Perhaps the most common uses of dental calculus is to try and recreate
the diet of past people and populations
(Figure~\ref{fig-plot-and-wordclouds}B). One of the ways to do this is
dissolving the calculus in a weak acid, a decalcifant, or mechanically
breaking it up. This process releases any fragments of plants that were
trapped within the calculus and can be identified, for example with a
microscope. The tricky part is not destroying the plant fragments when
releasing them from the calculus. As far as I can tell, the first
attempt at this was the extraction of phytoliths (silicified plant
remains) from the teeth of cows, sheep, and horses
(\protect\hyperlink{ref-armitageExtractionIdentification1975}{Armitage,
1975}). This was a somewhat isolated use-case, and it didn't really
catch on until the 1990s
(\protect\hyperlink{ref-ciochonOpalPhytoliths1990}{Ciochon et al.,
1990}; Middleton 1990, in
\protect\hyperlink{ref-middletonOpalPhytoliths1994}{W. D. Middleton \&
Rovner, 1994}). The first extractions from human teeth followed shortly
(\protect\hyperlink{ref-foxPhytolithCalculus1996}{Fox et al., 1996}),
and there are now studies using plant microremains (especially starch
granules and phytoliths) from dental calculus to infer diet in past
peoples from across the world, including Rapa Nui
(\protect\hyperlink{ref-dudgeonDietGeography2014}{Dudgeon \& Tromp,
2014}), China (\protect\hyperlink{ref-chenStarchGrains2021}{T. Chen et
al., 2021}), Europe
(\protect\hyperlink{ref-fiorinCombiningDental2021}{Fiorin et al.,
2021}), and more
(\protect\hyperlink{ref-buckleyDentalCalculus2014}{Buckley et al.,
2014}; \protect\hyperlink{ref-henryCalculusSyria2008}{Henry \& Piperno,
2008}; \protect\hyperlink{ref-mickleburghNewInsights2012}{Mickleburgh \&
Pagán-Jiménez, 2012}). The durable nature of dental calculus also means
that microremains within it can survive for millenea, allowing us to
look at the diets of early humans and other hominins
(\protect\hyperlink{ref-buckleyDentalCalculus2014}{Buckley et al.,
2014}; \protect\hyperlink{ref-chenStarchGrains2021}{T. Chen et al.,
2021}; \protect\hyperlink{ref-hardyStarchGranules2009}{Hardy et al.,
2009}; \protect\hyperlink{ref-hardyNeanderthalMedics2012}{Hardy et al.,
2012}; \protect\hyperlink{ref-henryDietAustralopithecus2012}{Henry et
al., 2012}, \protect\hyperlink{ref-henryNeanderthalCalculus2014}{2014};
\protect\hyperlink{ref-henryCalculusSyria2008}{Henry \& Piperno, 2008};
\protect\hyperlink{ref-pipernoStarchGrains2008}{Piperno \& Dillehay,
2008}). It's also considered useful because it represents a more recent
and direct source of diet than teeth or other bones, since the turnover
is much quicker in calculus than bone. Calculus can form within weeks at
any point during an individual's life and may indicate direct
consumption, while bone can take years to incorporate a dietary signal.
Enamel stops forming after the last tooth has developed, and the
turnover of dentin is very limited
(\protect\hyperlink{ref-hillsonDentalAnthropology1996}{Hillson, 1996}).

That bacteria can become trapped within calculus has been known to
archaeologists for a while
(\protect\hyperlink{ref-brothwellDiggingBones1981}{Brothwell, 1981}, ;
\protect\hyperlink{ref-vandermeerschMiddlePaleolithic1994}{Vandermeersch
et al., 1994}), but it wasn't utilised in archaeological research until
DNA extraction started to become more accessible
(\protect\hyperlink{ref-delafuenteDNAHuman2013}{De La Fuente et al.,
2013}). Dental calculus then became part of the third scientific
revolution in archaeology. The early studies focused on oral health in
the past (\protect\hyperlink{ref-adlerSequencingAncient2013}{Adler et
al., 2013}; \protect\hyperlink{ref-delafuenteDNAHuman2013}{De La Fuente
et al., 2013};
\protect\hyperlink{ref-warinnerPathogensHost2014}{Warinner, Rodrigues,
et al., 2014}). Bacteria have shorter lifespans than humans which makes
them useful when studying the evolution of bacteria in the human mouth
(\protect\hyperlink{ref-delafuenteDNAHuman2013}{De La Fuente et al.,
2013}; \protect\hyperlink{ref-yatesOralMicrobiome2021}{Fellows Yates et
al., 2021}). Diet has also been a focus of paleogenetic research. This
has mainly been addressed by considering how long-term changes in the
patterns of bacteria within the mouths of our ancestors have changed
that could be related to changes in diet. Just like we adapt to deal
with various diseases, climates, etc., we also adapt to changes in our
diet (\protect\hyperlink{ref-adlerSequencingAncient2013}{Adler et al.,
2013}; \protect\hyperlink{ref-yatesOralMicrobiome2021}{Fellows Yates et
al., 2021}). Directly identifying genetic markers of plants and animals
within dental calulus is difficult, but not impossible (see Warinner,
Hendy, et al. (\protect\hyperlink{ref-warinnerEvidenceMilk2014}{2014})).
Most of the DNA within dental calculus will be oral bacteria, and this
will overwhelm the signal from plant DNA. Plants are also really
complicated because plant DNA only makes up a fraction of the DNA
extracted from calculus, which is overwhelmed by the presence of
endogenous and bacterial DNA (personal communication with Zandra
Fagernäs, James Fellows Yates, and Nikolay Oskolkov in the SPAAM
community)
(\protect\hyperlink{ref-fagernasMicrobialBiogeography2022}{Fagernäs et
al., 2022}). A newer field of biomolecular archaeology, paleoproteomics,
may be able to address this issue by targeting to plant proteins. Hendy
and coauthors were able to identify a number of these in dental
calculus, as well as proteins from cereals, and milk proteins from
different sources
(\protect\hyperlink{ref-hendyProteomicCalculus2018}{Hendy et al.,
2018}).

To a lesser extent, the presence and amount of dental calculus on teeth
has been used as an indicator of dental health
(\protect\hyperlink{ref-drewettExcavationOval1975}{Drewett, 1975};
\protect\hyperlink{ref-lieverseDentalHealth2007}{Lieverse et al., 2007};
\protect\hyperlink{ref-sagneStudiesPeriodontal1977}{Sagne \& Olsson,
1977}; \protect\hyperlink{ref-zhangDentalDisease1982}{Y. Zhang, 1982}).
Pilloud \& Fancher
(\protect\hyperlink{ref-pilloudOutliningDefinition2019}{2019}) explored
the terms associated with a number publications on dental or oral
health, dental calculus came up as one of them; albeit not the most
common, which was (unsurprisingly) dental caries
(Figure~\ref{fig-dental-terms}). More recently Yaussy \& DeWitte
(\protect\hyperlink{ref-yaussyCalculusSurvivorship2019}{2019}) looked at
how it relates to overall health in a population, not just oral health.
They suggest that individuals with more calculus are more at risk than
individuals with less or no calculus.

\begin{figure}

{\centering \includegraphics{figures/wordcloud.png}

}

\caption{\label{fig-dental-terms}Word cloud of most common dental terms
in articles. Figure is from Pilloud \& Fancher
(\protect\hyperlink{ref-pilloudOutliningDefinition2019}{2019}), Figure
1}

\end{figure}

This wide range of applications, and the fact that it's pretty much
ubiquitous in the past (thanks to poor oral hygiene), makes it a really
exciting target for future paleodietary research. That being said, the
study of dental calculus doesn't seem to fit into any predefined areas
of study within (and beyond) archaeology. Most researchers seem to see
it as a means to the information contained within, rather than being
worth studying in its own right (with some exceptions, of course). This
can be problematic. Other than what we can see with our current methods,
what do we really know about dental calculus and how its growth and
structure affect the reliability of these methods and potentially
distort our interpretations of the past?

\hypertarget{intro-what}{%
\section{What is dental calculus?}\label{intro-what}}

First, we must answer a single, surprisingly difficult question: What is
dental calculus? I'm not referring to its formation or composition,
which I briefly described \protect\hyperlink{intro}{above}. How do we
categorise it? Is it a dental disease? An oral health condition? A
byproduct of oral conditions? To answer this, it's necessary to look at
various definitions of oral health. Definitions in an introduction are a
little cliché and tedious, but often necessary. Since oral health is a
complex topic, definitions of oral health are often purposefully (and
confusingly) broad, and they extend beyond physical well-being and into
the realms of emotional and social comfort. The World Dental Federation
(FDI) defines oral health as the ability to perform mouth- and
face-related functions with confidence and without pain (including
smiling, speaking, eating, etc.)
(\protect\hyperlink{ref-fdiOralHealth}{\emph{{FDI}'s Definition of Oral
Health \textbar{} {FDI}}, n.d.})
(\url{https://www.fdiworlddental.org/fdis-definition-oral-health}). Both
the World Health Organisation (WHO) and FDI take a similar approach to
defining oral conditions, giving a list of conditions that cause
discomfort, pain, disfigurement, or death. The list includes the dental
conditions tooth decay (caries), gum disease (periodontal disease), and
dental trauma, but not dental calculus
(\protect\hyperlink{ref-whoOralHealth}{\emph{Oral Health}, n.d.})
(\url{https://www.who.int/news-room/fact-sheets/detail/oral-health}).
While these are not likely to cause death, they are often the source of
physical and emotional discomfort, and may cause further health
complications if they are not dealt with in a timely fashion.

Dental calculus and dental plaque are not considered oral conditions
according to WHO. In fact, dental plaque is part of the normal
functioning of our oral biome
(\protect\hyperlink{ref-marshDentalPlaque2006}{Philip D. Marsh, 2006}).
When plaque reaches a certain level of acidity over a prolonged period
of time, the normal functioning of the bacteria within the plaque may
shift towards a disease-causing function. The biofilm will cause the
surface of the enamel to demineralise, eventually resulting in a cavity
(or caries). Dental caries are unequivocally considered a dental
disease. If, instead, the biofilm calcifies, dental calculus is the
result. Its status in oral health is questionable.

Dental calculus is not known to be painful, nor does it affect the
ability to perform the functions listed above. However, with continued
accumulation, it may affect the confidence of the person performing
these tasks (\protect\hyperlink{ref-collinsHomelessDental2007}{Collins
\& Freeman, 2007}), and in extreme cases it can affect function
(\protect\hyperlink{ref-balajiUnusualPresentation2019}{Balaji et al.,
2019}). Most of the virulence and disease-causing potential is lost when
the bacteria within dental plaque calcify
(\protect\hyperlink{ref-akcaliDentalCalculus2018}{Akcalı \& Lang,
2018}). It has been shown to contain pockets of living bacteria that can
be detrimental to oral and dental health
(\protect\hyperlink{ref-tanCalculusUltrastructure2004}{B. T. K. Tan,
Gillam, et al., 2004};
\protect\hyperlink{ref-tanBacterialViability2004}{B. T. K. Tan, Mordan,
et al., 2004}). The rough, porous surface of dental calculus is also a
great place for bacteria to attach more easily and develop a new layer
of plaque on the surface of the calculus. This is likely why there is
often a correlation (NOT causation) between dental calculus and
periodontitis, especially subgingival calculus
(\protect\hyperlink{ref-jepsenCalculusRemoval2011}{Jepsen et al., 2011};
\protect\hyperlink{ref-whiteDentalCalculus1997}{D. J. White, 1997}).
Since it seems to fulfill some of the criteria of an oral condition, it
should be considered as such, at least under the definitions provided by
WHO and FDI. Whether or not dental calculus can be considered an oral
disease is more questionable. While it does grow on the surface of
teeth, it doesn't seem to affect the underlying enamel. And while there
is a relationship with periodontal disease (which has been defined as a
dental disease), the nature of this relationship is still under debate,
with calculus likely being a secondary contributor
(\protect\hyperlink{ref-jepsenCalculusRemoval2011}{Jepsen et al.,
2011}). As such, we can probably limit the definition to an oral
condition and not necessarily a dental disease
(\protect\hyperlink{ref-pilloudOutliningDefinition2019}{Pilloud \&
Fancher, 2019}). In fact, dental calculus is quite hard, so a layer of
dental calculus on a tooth can actually protect it from wearing down
(although there are better options).

\hypertarget{intro-study}{%
\section{The study of dental calculus}\label{intro-study}}

It seems that the researchers who are studying dental calculus approach
it from a wide range of different fields and backgounds, including
genetics, proteomics, botany, and (bio)archaeology. The paleogeneticists
mine it for the wealth of information it contains on oral health and
disease in the past
(\protect\hyperlink{ref-yatesOralMicrobiome2021}{Fellows Yates et al.,
2021}; \protect\hyperlink{ref-warinnerPathogensHost2014}{Warinner,
Rodrigues, et al., 2014}). Paleodiet researchers extract microremains
and residues from food
(\protect\hyperlink{ref-henryCalculusSyria2008}{Henry \& Piperno, 2008};
\protect\hyperlink{ref-mickleburghNewInsights2012}{Mickleburgh \&
Pagán-Jiménez, 2012}) to infer dietary practices. Paleopathologists use
it to infer overall health in a given population
(\protect\hyperlink{ref-yaussyCalculusSurvivorship2019}{Yaussy \&
DeWitte, 2019}). This leaves research output from studies of calculus
scattered across multiple venues, with no clear gathering point. I think
it's fair to say that dental calculus should be included in discussions
of pathological oral conditions, even if its role is secondary. But who
is currently studying dental calculus as a substance in its own right?
And why do we need to learn more about it if we're just interested in
what's inside? Related discussions have started to take place in recent
years (\protect\hyperlink{ref-bucchiComparisonsMethods2019}{Bucchi et
al., 2019}; \protect\hyperlink{ref-radiniDirtyTeeth2022}{Radini \&
Nikita, 2022};
\protect\hyperlink{ref-wrightAdvancingRefining2021}{Wright et al.,
2021}).

The lack of a specific field of study for dental calculus to belong may
be related to how it's taught to students (and if it's taught at all).
Textbooks from the more established fields in bioarchaeology are
probably a good indicator of the teaching curricula, which also impacts
research focus. The most popular osteoarchaeology textbooks only briefly
mention dental calculus as more of a footnote than anything else. A
couple of lines describing what it is (usually `mineralised plaque') and
that it can contain food debris and bacteria T. D. White et al.
(\protect\hyperlink{ref-whiteHumanOsteology2011}{2011}). They're not
wrong. Diseases that manifest themselves in the skeleton as lesions on
the bones have a very clear home in paleopathology. No one questions
whether or not the degeneration of vertebrae from tuberculosis should be
included in the paleopathology textbooks (at least not as far as I'm
aware).

These textbooks often include chapters on dental disease, where more
detailed descriptions of dental calculus are usually found (e.g.
\protect\hyperlink{ref-robertsDentalDisease2007}{Roberts \& Manchester,
2007}; \protect\hyperlink{ref-waldronPalaeopathology2020}{Waldron,
2020}). Dental caries, calculus' more famous sibling, will often get a
few pages. In some cases, dental calculus may even be hidden within a
section on periodontal disease or plaque
(\protect\hyperlink{ref-aufderheidePaleopathology1998}{Aufderheide et
al., 1998}; e.g.
\protect\hyperlink{ref-ortnerIdentificationPathological2003}{Ortner,
2003}). The focus of these (sub)sections is varied, with some simply
describing what it is, and others giving brief discussion on the
relationship between calculus and periodontal disease. A more detailed
section was dedicated to dental calculus in \emph{Ortner's
Identification of Pathological Conditions in Human Skeletal Remains},
with a detailed description of formation, structure, and application in
(biomolecular) archaeology
(\protect\hyperlink{ref-kinastonOrtnerDentition2019}{Kinaston et al.,
2019}). The description extends well beyond any (paleo)pathological
significance of dental calculus. Can we fault the authors/editors for
not giving it more attention? After all, it's not a dental disease, and
its relationship with other dental diseases is unclear. What is clear,
is that it has implications for oral health, and, for that very reason,
could be addressed more extensively in paleopathology; certainly in the
textbooks that include dental disease.

On the surface, dental anthropology seems like a more suitable home for
the study of dental calculus. However, it's not included in \emph{A
Companion to Dental Anthropology}, an otherwise great resource on
studying archaeological teeth. The editors briefly acknowledge the
valuable information gained from calculus and that it holds a lot of
potential; but that's it
(\protect\hyperlink{ref-scottBriefHistory2015}{G. R. Scott, 2015}).
Other notable absences include textbooks such as \emph{Technique and
Application in Dental Anthropology} and \emph{New Direction in Dental
Anthropology}
(\protect\hyperlink{ref-townsendDentalAnthropology2012}{Townsend et al.,
2012}), both of which dedicate considerable attention to dental caries.
Hillson's \emph{Dental Anthropology}, a book that I consider to be the
`bible' for dental anthropology, has a section on dental calculus in the
Dental Disease chapter. It covers a basic description, the composition,
microscopic structure, methods used for recording archaeological
calculus, and the distribution in the dentition (i.e.~which teeth are
more prone to calculus buildup)
(\protect\hyperlink{ref-hillsonDentalAnthropology1996}{Hillson, 1996}).
Considering these are entire books devoted to the dentition, it seems
odd that there is often only a few paragraphs (if that) on dental
calculus. Granted, the only function teeth serve in the growth of dental
calculus is as a suitable surface on which to attach; though the role of
substratum is an important role, as dental calculus is seemingly unable
to form on other surfaces in the oral cavity.

Since the use of dental calculus in biomolecular archaeology is
relatively new, there are fewer available textbooks, and it rarely has a
dedicated course. The most common place to find descriptions of dental
calculus is, therefore, journal articles. There will be a short
paragraph on dental calculus formation (and sometimes composition) in
the introduction section. These are quite variable and are often limited
by the word count of the journal. Despite this, the descriptions will
often be as long, if not longer, than the sections in textbooks devoted
to dental calculus
(\protect\hyperlink{ref-velskoMicrobialDifferences2019}{Velsko et al.,
2019}). The focus of these paragraphs are generally the same. They
describe the formation and mineral composition of dental calculus, and
provide some examples of how dental calculus has been used in related
studies (not unlike the beginning of this chapter). The contribution of
dental calculus to archaeology has been significant, so it is likely to
receive more and more attention going forward. In fact, an entire
chapter was recently devoted to dental calculus in the second edition of
\emph{Handbook of Archaeological Sciences}
(\protect\hyperlink{ref-fagernasDentalCalculus2023}{Fagernäs \&
Warinner, 2023}). Take that, dental caries!

\hypertarget{the-challenges-of-studying-dental-calculus}{%
\section{The challenges of studying dental
calculus}\label{the-challenges-of-studying-dental-calculus}}

What we know about dental calculus and the influence of diet was
reviewed in an article aimed at (bio)archaeologists. The overall
conclusion reached in the article: it's still pretty unclear
(\protect\hyperlink{ref-lieverseDietAetiology1999}{Lieverse, 1999}).
Now, 20-some years later, there has been limited progress on this point.
High-protein diets are linked to an increase of urea, which is linked to
an increase in oral pH, which is linked to mineral deposition
(\protect\hyperlink{ref-dibdinOralUrea1998}{Dibdin \& Dawes, 1998};
\protect\hyperlink{ref-wongCalciumPhosphate2002}{Wong et al., 2002}).
BUT, protein may also inhibit crystalisation
(\protect\hyperlink{ref-hidakaDietCalculus2007}{S. Hidaka \& Oishi,
2007}). Starch consumption has been linked to increased rates of caries
in early farming populations. This is consistent with \emph{in vitro}
testing, at least for starches high in amylose content. So a high-starch
diet causes caries, not calculus, right? Well, starches with a high
amylopectin content are linked to increased calcification
(\protect\hyperlink{ref-hidakaDietCalculus2007}{S. Hidaka \& Oishi,
2007}). It likely depends on what is consumed along with the starch
(\protect\hyperlink{ref-hidakaStarchRole2008}{Saburo Hidaka et al.,
2008}). There is also some (\emph{in vitro}) evidence to suggest that
silica may promote dental calculus formation by promoting mineral
precipitation, i.e.~the transfer of minerals from saliva to the biofilm
(\protect\hyperlink{ref-damenSilicicAcid1989}{Damen \& Ten Cate, 1989}).

Another aspect of diet and dental calculus where we are still looking
for answers, is the process that causes fragments of food and other
environmental materials to become entrapped in the dental calculus. We
know that it happens. Decades of research has shown dental calculus to
be a seemingly unlimited resource for dietary substances. We don't know
exactly how this happens, and herein lies the potential for bias.
Efforts have been made to understand how much of the consumed food makes
it into the calculus. These include studies on modern humans
(\protect\hyperlink{ref-leonardPlantMicroremains2015}{Leonard et al.,
2015}) and non-human primates
(\protect\hyperlink{ref-powerChimpCalculus2015}{R. C. Power et al.,
2015}; \protect\hyperlink{ref-powerRepresentativenessDental2021}{Robert
C. Power et al., 2021}), where food intake is meticulously documented,
and calculus subsequently analysed. These studies have common findings;
the amount of the diet that becomes trapped in the dental calculus of
any one person has no clear relationship to the amount of food that was
consumed. The most likely reason is that the formation of dental
calculus differs between people
(\protect\hyperlink{ref-powerChimpCalculus2015}{R. C. Power et al.,
2015}). So, it's not a great way to study the diet of a single person,
but generally suitable to study patterns in the diet of a population.
The more people you study, the more likely you are to gain a complete
picture of the diet in a population The fact that we can still see (in
some cases, literally) remains that were consumed thousands of years ago
is pretty cool. We just need a better understanding of why the record of
diet from dental calculus differs from the actual intake of food. This
will allow us to make more robust interpretations about past dietary
practices. Something that may influence the dietary record that we get
from calculus is the method we use to extract the dietary remains from
calculus. Our understanding of dental calculus extraction methods is
improving, with studies looking at the effect of various acids used to
dissolve calculus (commonly EDTA or HCl)
(\protect\hyperlink{ref-bucchiComparisonsMethods2019}{Bucchi et al.,
2019};
\protect\hyperlink{ref-sotoCharacterizationDecontamination2019}{Soto et
al., 2019}; \protect\hyperlink{ref-trompEDTACalculus2017}{Tromp et al.,
2017}); as is our understanding of how the choice of tooth may affect
our results
(\protect\hyperlink{ref-fagernasMicrobialBiogeography2021}{Fagernäs et
al., 2021}).

These studies provide valuable insights into potential biases of our
sampling methods and the representation of diet within dental calculus,
with a minor caveat. Most of these studies have been conducted on living
primates or archaeological remains. An issue with using living (or once
living) organisms is the inability to control factors related to the
variability between subjects. Basically, studying humans is messy and
complicated because we're all unique. It's a lovely sentiment but it can
make for some messy science. Not bad science (not at all!). Just messy.
A method of study that offers more control, is the growth of plaque and
calculus in a lab. This allows us to control many of the things that are
difficult to control in humans, such as the bacteria that colonise our
mouth, where each person has a pretty unique makeup of bacteria. We also
have a very unique genome (with the exception of identical twins) that
plays a role in how quickly we form calculus in our mouth (if at all).
Certain enzymes start digesting our food before as soon as it enters our
mouth, and the activity of these enzymes fluctuates throughout the day,
causing a lot of variability both within and between individuals.
Finally, the number of microremains that enter our mouth over days,
weeks, months, can be very different between people, even with the same
diet. All these things can muddy the results of research on living
subjects, where a lab-grown approach can help tease out confounding
factors. I don't believe research conducted on lab-grown biofilms can in
any way replace studies with modern or archaeological individuals, nor
should they. But it can complement these studies by zooming in on
certain aspects that are too difficult to isolate in (once-)living
people.

Often we can draw from clinical studies as there are common goals,
e.g.~discovering the aetiology and/or presentation of a disease.
However, the motivation driving the studies in archaeology and dental
research are inherently different; although, there is certainly overlap
in some areas (Figure~\ref{fig-plot-and-wordclouds}B and C). There is
more interest in preventing dental calculus from forming in the first
place, so most studies focus on short-duration models to explore
anti-microbial treatments and inhibition of biofilm formation and plaque
buildup (\protect\hyperlink{ref-extercateAAA2010}{Exterkate et al.,
2010}). As shown in a previous study, calculus and plaque have distinct
microbial profiles
(\protect\hyperlink{ref-velskoMicrobialDifferences2019}{Velsko et al.,
2019}), so the applicability of short-term models to explore
archaeological questions on dental calculus are limited, since plaque is
rarely (if ever) preserved. Archaeologists are more interested in
questions related to how diet influences the growth of biofilms, and how
fragments become embedded inside, and what we can say about diet.
Further, the interest in dental calculus as a field of clinical research
has been declining since the 2000s, which, as far as I'm aware, is when
the last studies growing dental calculus in a lab were conducted. We can
see this by the number of clinical articles with the term dental
calculus in the title (Figure~\ref{fig-plot-and-wordclouds}A). And they
certainly aren't interested in how food debris becomes trapped inside
our calculus. Dental calculus has also become less of a problem with the
use of modern dental hygiene practices and regular visits to the dentist
(\protect\hyperlink{ref-velskoMicrobialDifferences2019}{Velsko et al.,
2019}).

To summarise: Bioarchaeologists are interested in how dental calculus
relates to dental and general health; paleodietary researchers are
interested in the food remains that are trapped inside; paleogeneticists
are interested in accessing the oral bacteria that have been fossilised
within; clinial dentistry views it as a nuisance to be removed and,
ideally, prevented from forming in the first place. This lack of
systematic research specifically devoted to dental calculus as a
substance, rather than a means to an end, leaves a lot of questions
regarding the expected behaviour of dental calculus and how information
from the past becomes trapped inside. To summarise the summary: we need
to ask more basic questions about dental calculus.

\hypertarget{intro-aims}{%
\section{Aims}\label{intro-aims}}

This dissertation is a contribution to a dental-calculus-centric body of
knowledge, and addresses a gap in the fundamental research on dental
calculus to further our understanding of how we can use dental calculus
to reconstruct the diets of people in the past. The main aim is the
development, validation, and application of a calcifying oral biofilm
model to improve interpretations on archaeological dental calculus. By
developing a model system we can isolate the effects of confounding
factors in dental calculus and diet, and explore new uses for dental
calculus in paleodietary reconstructions through fundamental
experimentation. I also aim to assess the potential and limitations of
dental calculus to explore dietary activities past populations.

With these aims, I hope to address the following research questions:

\emph{How can we improve the resolution of our interpretations using
dental calculus on individuals and populations?}

\emph{Can we trust the system? (i.e., using dental calculus to
reconstruct diet)} - We don't know the mechanism of incorporation -
There are hidden biases and limitations - We don't know the starting
point, so we have difficulty validating our methods. What was originally
trapped inside?

\emph{How can a model improve our understanding of reconstructions of
diet in studies of dental calculus and diet?} - How can it address
current challenges in paleodietary reconstructions? - Can it help us
produce a better understanding of how dietary intake relates to the
record of diet from archaeological dental calculus?

\hypertarget{thesis-outline-and-structure}{%
\section{Thesis outline and
structure}\label{thesis-outline-and-structure}}

If you have made it to this point, you have probably read most of
\protect\hyperlink{chap-intro}{Chapter 1}, in which I provide some
context to the study of dental calculus in archaeology and identify some
areas that could benefit from further investigation.

\protect\hyperlink{chap-background}{Chapter 2} provides some background
information on oral biofilms and oral biofilm models in more detail than
I can do in the research articles included in Chapters 3 and 4. So if
you're already well-versed in oral microbiology, feel free to skip to
Chapter 3. If not, I recommend picking up a textbook written by actual
experts in the field of oral microbiology. If, for some reason, you
can't access one of these, feel free to read
\protect\hyperlink{chap-background}{Chapter 2}. I suppose there are
worse options than something written by a PhD student in archaeology.

To address the aims of the dissertation outlined earlier in
\protect\hyperlink{intro-aims}{this chapter}, I developed a protocol to
grow dental calculus in a lab on plastic tubes instead of looking at the
real stuff you usually see on teeth inside a mouth. The reason for using
lab-grown biofilms instead of humans is that the \emph{in vitro} lab
model offers more control over all the factors that go into the growth
of dental calculus, at least in theory. The real world is messy, and
sometimes you need to remove things from the real world to break it down
and really get into the nitty gritty of how it works. There are many
different kinds of biofilm models, including single species of bacteria,
select species determined by the researchers (defined consortium), and
multiple species from some natural source (the human mouth, for
example). I will cover the different types of models in more detail in
\protect\hyperlink{background}{Chapter 2}. Since there are many biofilm
models to choose from, developing a new protocol may seem
counter-productive; however, few are developed for long-term growth and
even fewer with the purpose of mineralising the biofilm to form dental
calculus. One of the exceptions involves a highly complex setup that is
unlikely to be supported by budgets and facilities available to most
archaeological laboratories
(\protect\hyperlink{ref-sissonsMultistationPlaque1991}{Sissons et al.,
1991}).

After developing a working protocol, the next step was to determine if
the stuff I grew in the lab is actually dental calculus. Or at least
something close enough that we can use it to explore our research
questions. To do this, we (myself and coauthors) determined the mineral
and bacterial composition of our model using Fourier Transform Infrared
(FTIR) spectroscopy and metagenomic classification
\protect\hyperlink{byoc-valid}{Chapter 3}. We then compared the results
of these analyses to naturally grown dental calculus, both modern and
archaeological.

Being confident that our model looks and behaves like human dental
calculus, we then set out to test some very basic {[}properties{]} of
starch grains within dental calculus.
\protect\hyperlink{byoc-starch}{Chapter 4} is a research article where
we `fed' the biofilm with a known quantity of starch granules during the
growth period to see if the input quantity/ratio matched the extracted
quantity (or output). Those who are familiar with dental calculus
research will not be surprised that it did not. The more interesting
outcome of the study is the more detailed explanation of how the input
and output starch quantities were mismatched.

\protect\hyperlink{mb11CalculusPilot}{Chapter 5} is a separate article,
in the sense that it doesn't involve the biofilm model in any way.
Rather, it addresses the theme of the overall utility of dental calculus
in archaeological research. We look at possible medicinal compounds in
the dental calculus of a Post-medieval Dutch population. We employed
Ultra High Performance Liquid Chromatography coupled with tandem Mass
Spectrometry (UHPLC-MS/MS) to identify various compounds in dental
calculus, including alkaloids and other compounds. It shows the
potential of dental calculus to inform about past practices, but also
highlights some of the limitations we are currently experiencing in the
field. \protect\hyperlink{chap-discussion}{Chapter 6} is a discussion on
the limitations and future potential of dental calculus in the field of
archaeology, and what biofilm models can contribute to our understanding
of past diet.

\hypertarget{references}{%
\section*{References}\label{references}}
\addcontentsline{toc}{section}{References}

\markright{References}

\hypertarget{refs-1}{}
\begin{CSLReferences}{1}{0}
\leavevmode\vadjust pre{\hypertarget{ref-adamsRestaurantEnd2002}{}}%
Adams, D. (2002). \emph{The {Restaurant} at the {End} of the
{Universe}}. {Picador}.

\leavevmode\vadjust pre{\hypertarget{ref-adlerSequencingAncient2013}{}}%
Adler, C. J., Dobney, K., Weyrich, L. S., Kaidonis, J., Walker, A. W.,
Haak, W., Bradshaw, C. J., Townsend, G., Sołtysiak, A., Alt, K. W.,
Parkhill, J., \& Cooper, A. (2013). Sequencing ancient calcified dental
plaque shows changes in oral microbiota with dietary shifts of the
{Neolithic} and {Industrial} revolutions. \emph{Nature Genetics},
\emph{45}(4), 450--455, 455e1. \url{https://doi.org/10.1038/ng.2536}

\leavevmode\vadjust pre{\hypertarget{ref-akcaliDentalCalculus2018}{}}%
Akcalı, A., \& Lang, N. P. (2018). Dental calculus: The calcified
biofilm and its role in disease development. \emph{Periodontology 2000},
\emph{76}(1), 109--115. \url{https://doi.org/10.1111/prd.12151}

\leavevmode\vadjust pre{\hypertarget{ref-armitageExtractionIdentification1975}{}}%
Armitage, P. L. (1975). The {Extraction} and {Identification} of {Opal
Phytoliths} from the {Teeth} of {Ungulates}. \emph{Journal of
Archaeological Science}, \emph{2}, 187--197.

\leavevmode\vadjust pre{\hypertarget{ref-aufderheidePaleopathology1998}{}}%
Aufderheide, A. C., Rodriguez-Martin, C., \& Langsjoen, O. (1998).
\emph{The {Cambridge} encyclopedia of human paleopathology} (Vol. 478).
{Cambridge University Press Cambridge}.

\leavevmode\vadjust pre{\hypertarget{ref-balajiUnusualPresentation2019}{}}%
Balaji, V. R., Niazi, T. M., \& Dhanasekaran, M. (2019). An unusual
presentation of dental calculus. \emph{Journal of Indian Society of
Periodontology}, \emph{23}(5), 484--486.
\url{https://doi.org/10.4103/jisp.jisp_680_18}

\leavevmode\vadjust pre{\hypertarget{ref-brothwellDiggingBones1981}{}}%
Brothwell, D. (1981). \emph{Digging up {Bones}: {The} excavation,
treatment and study of human skeletal remains} (3rd ed.). {British
Museum (Natural History)}.

\leavevmode\vadjust pre{\hypertarget{ref-bucchiComparisonsMethods2019}{}}%
Bucchi, A., Burguet-Coca, A., Expósito, I., Aceituno Bocanegra, F. J.,
\& Lozano, M. (2019). Comparisons between methods for analyzing dental
calculus samples from {El Mirador} cave ({Sierra} de {Atapuerca},
{Spain}). \emph{Archaeological and Anthropological Sciences},
\emph{11}(11), 6305--6314.
\url{https://doi.org/10.1007/s12520-019-00919-z}

\leavevmode\vadjust pre{\hypertarget{ref-buckleyDentalCalculus2014}{}}%
Buckley, S., Usai, D., Jakob, T., Radini, A., \& Hardy, K. (2014).
Dental {Calculus Reveals Unique Insights} into {Food Items}, {Cooking}
and {Plant Processing} in {Prehistoric Central Sudan}. \emph{PLOS ONE},
\emph{9}(7), e100808. \url{https://doi.org/10.1371/journal.pone.0100808}

\leavevmode\vadjust pre{\hypertarget{ref-chenStarchGrains2021}{}}%
Chen, T., Hou, L., Jiang, H., Wu, Y., \& Henry, A. G. (2021). Starch
grains from human teeth reveal the plant consumption of proto-{Shang}
people (c. 2000--1600 {BC}) from {Nancheng} site, {Hebei}, {China}.
\emph{Archaeological and Anthropological Sciences}, \emph{13}(9), 153.
\url{https://doi.org/10.1007/s12520-021-01416-y}

\leavevmode\vadjust pre{\hypertarget{ref-ciochonOpalPhytoliths1990}{}}%
Ciochon, R. L., Piperno, D. R., \& Thompson, R. G. (1990). Opal
phytoliths found on the teeth of the extinct ape {Gigantopithecus}
blacki: Implications for paleodietary studies. \emph{Proceedings of the
National Academy of Sciences}, \emph{87}(20), 8120--8124.
\url{https://doi.org/10.1073/pnas.87.20.8120}

\leavevmode\vadjust pre{\hypertarget{ref-collinsHomelessDental2007}{}}%
Collins, J., \& Freeman, R. (2007). Homeless in {North} and {West
Belfast}: An oral health needs assessment. \emph{British Dental
Journal}, \emph{202}(12, 12), E31--E31.
\url{https://doi.org/10.1038/bdj.2007.473}

\leavevmode\vadjust pre{\hypertarget{ref-damenSilicicAcid1989}{}}%
Damen, J. J. M., \& Ten Cate, J. M. (1989). The {Effect} of {Silicic
Acid} on {Calcium Phosphate Precipitation}. \emph{Journal of Dental
Research}, \emph{68}(9), 1355--1359.
\url{https://doi.org/10.1177/00220345890680091301}

\leavevmode\vadjust pre{\hypertarget{ref-delafuenteDNAHuman2013}{}}%
De La Fuente, C., Flores, S., \& Moraga, M. (2013). {DNA From Human
Ancient Bacteria}: {A} novel source of genetic evidence from
archaeological dental calculus. \emph{Archaeometry}, \emph{55}(4),
767--778. \url{https://doi.org/10.1111/j.1475-4754.2012.00707.x}

\leavevmode\vadjust pre{\hypertarget{ref-dibdinOralUrea1998}{}}%
Dibdin, G. H., \& Dawes, C. (1998). A {Mathematical Model} of the
{Influence} of {Salivary Urea} on the {pH} of {Fasted Dental Plaque} and
on the {Changes Occurring} during a {Cariogenic Challenge}. \emph{Caries
Research}, \emph{32}(1), 70--74. \url{https://doi.org/10.1159/000016432}

\leavevmode\vadjust pre{\hypertarget{ref-dobneyMethodEvaluating1987}{}}%
Dobney, K., \& Brothwell, D. (1987). A method for evaluating the amount
of dental calculus on teeth from archaeological sites. \emph{Journal of
Archaeological Science}, \emph{14}(4), 343--351.
\url{https://doi.org/10.1016/0305-4403(87)90024-0}

\leavevmode\vadjust pre{\hypertarget{ref-drewettExcavationOval1975}{}}%
Drewett, P. (1975). \emph{The {Excavation} of an {Oval Burial Mound} of
the {Third Millennium} be at {Alfriston}, {East Sussex}, 1974}. 38.

\leavevmode\vadjust pre{\hypertarget{ref-dudgeonDietGeography2014}{}}%
Dudgeon, J. V., \& Tromp, M. (2014). Diet, {Geography} and {Drinking
Water} in {Polynesia}: {Microfossil Research} from {Archaeological Human
Dental Calculus}, {Rapa Nui} ({Easter Island}). \emph{International
Journal of Osteoarchaeology}, \emph{24}(5), 634--648.
\url{https://doi.org/10.1002/oa.2249}

\leavevmode\vadjust pre{\hypertarget{ref-extercateAAA2010}{}}%
Exterkate, R. A. M., Crielaard, W., \& Ten Cate, J. M. (2010). Different
{Response} to {Amine Fluoride} by {Streptococcus} mutans and
{Polymicrobial Biofilms} in a {Novel High-Throughput Active Attachment
Model}. \emph{Caries Research}, \emph{44}(4), 372--379.
\url{https://doi.org/10.1159/000316541}

\leavevmode\vadjust pre{\hypertarget{ref-fagernasMicrobialBiogeography2021}{}}%
Fagernäs, Z., Salazar-García, D. C., Avilés, A., Haber, M., Henry, A.,
Maurandi, J. L., Ozga, A., Velsko, I. M., \& Warinner, C. (2021).
Understanding the microbial biogeography of ancient human dentitions to
guide study design and interpretation. \emph{bioRxiv},
2021.08.16.456492. \url{https://doi.org/10.1101/2021.08.16.456492}

\leavevmode\vadjust pre{\hypertarget{ref-fagernasMicrobialBiogeography2022}{}}%
Fagernäs, Z., Salazar-García, D. C., Haber Uriarte, M., Avilés
Fernández, A., Henry, A. G., Lomba Maurandi, J., Ozga, A. T., Velsko, I.
M., \& Warinner, C. (2022). Understanding the microbial biogeography of
ancient human dentitions to guide study design and interpretation.
\emph{FEMS Microbes}, \emph{3}, xtac006.
\url{https://doi.org/10.1093/femsmc/xtac006}

\leavevmode\vadjust pre{\hypertarget{ref-fagernasDentalCalculus2023}{}}%
Fagernäs, Z., \& Warinner, C. (2023). Dental {Calculus}. In A. M.
Pollard, R. A. Armitage, \& C. Makarevicz (Eds.), \emph{Handbook of
{Archaeological Sciences}} (Second edition).
\url{https://onlinelibrary.wiley.com/doi/epub/10.1002/9781119592112}

\leavevmode\vadjust pre{\hypertarget{ref-fdiOralHealth}{}}%
\emph{{FDI}'s definition of oral health \textbar{} {FDI}}. (n.d.). {FDI
World Dental Federation}. Retrieved March 14, 2022, from
\url{https://www.fdiworlddental.org/fdis-definition-oral-health}

\leavevmode\vadjust pre{\hypertarget{ref-yatesOralMicrobiome2021}{}}%
Fellows Yates, J. A., Velsko, I. M., Aron, F., Posth, C., Hofman, C. A.,
Austin, R. M., Parker, C. E., Mann, A. E., Nägele, K., Arthur, K. W.,
Arthur, J. W., Bauer, C. C., Crevecoeur, I., Cupillard, C., Curtis, M.
C., Dalén, L., Bonilla, M. D.-Z., Fernández-Lomana, J. C. D., Drucker,
D. G., \ldots{} Warinner, C. (2021). The evolution and changing ecology
of the {African} hominid oral microbiome. \emph{Proceedings of the
National Academy of Sciences}, \emph{118}(20).
\url{https://doi.org/10.1073/pnas.2021655118}

\leavevmode\vadjust pre{\hypertarget{ref-fiorinCombiningDental2021}{}}%
Fiorin, E., Moore, J., Montgomery, J., Lippi, M. M., Nowell, G., \&
Forlin, P. (2021). Combining dental calculus with isotope analysis in
the {Alps}: {New} evidence from the {Roman} and medieval cemeteries of
{Lamon}, northern {Italy}. \emph{Quaternary International}.
\url{https://doi.org/10.1016/j.quaint.2021.11.022}

\leavevmode\vadjust pre{\hypertarget{ref-foxPhytolithCalculus1996}{}}%
Fox, C. L., Juan, J., \& Albert, R. M. (1996). Phytolith analysis on
dental calculus, enamel surface, and burial soil: {Information} about
diet and paleoenvironment. \emph{American Journal of Physical
Anthropology}, \emph{101}(1), 101--113.
\url{https://doi.org/10.1002/(SICI)1096-8644(199609)101:1\%3C101::AID-AJPA7\%3E3.0.CO;2-Y}

\leavevmode\vadjust pre{\hypertarget{ref-greeneSimplifiedOral1964}{}}%
Greene, J. G., \& Vermillion, J. R. (1964). The {Simplified Oral Hygiene
Index}. \emph{The Journal of the American Dental Association},
\emph{68}(1), 7--13.
\url{https://doi.org/10.14219/jada.archive.1964.0034}

\leavevmode\vadjust pre{\hypertarget{ref-greeneQuantifyingCalculus2005}{}}%
Greene, T. R., Kuba, C. L., \& Irish, J. D. (2005). Quantifying
calculus: {A} suggested new approach for recording an important
indicator of diet and dental health. \emph{HOMO - Journal of Comparative
Human Biology}, \emph{56}(2), 119--132.
\url{https://doi.org/10.1016/j.jchb.2005.02.002}

\leavevmode\vadjust pre{\hypertarget{ref-hardyStarchGranules2009}{}}%
Hardy, K., Blakeney, T., Copeland, L., Kirkham, J., Wrangham, R., \&
Collins, M. (2009). Starch granules, dental calculus and new
perspectives on ancient diet. \emph{Journal of Archaeological Science},
\emph{36}(2), 248--255. \url{https://doi.org/10.1016/j.jas.2008.09.015}

\leavevmode\vadjust pre{\hypertarget{ref-hardyNeanderthalMedics2012}{}}%
Hardy, K., Buckley, S., Collins, M. J., Estalrrich, A., Brothwell, D.,
Copeland, L., García-Tabernero, A., García-Vargas, S., de la Rasilla,
M., Lalueza-Fox, C., Huguet, R., Bastir, M., Santamaría, D., Madella,
M., Wilson, J., Cortés, Á. F., \& Rosas, A. (2012). Neanderthal medics?
{Evidence} for food, cooking, and medicinal plants entrapped in dental
calculus. \emph{Naturwissenschaften}, \emph{99}(8), 617--626.
\url{https://doi.org/10.1007/s00114-012-0942-0}

\leavevmode\vadjust pre{\hypertarget{ref-hendyProteomicCalculus2018}{}}%
Hendy, J., Warinner, C., Bouwman, A., Collins, M. J., Fiddyment, S.,
Fischer, R., Hagan, R., Hofman, C. A., Holst, M., Chaves, E., Klaus, L.,
Larson, G., Mackie, M., McGrath, K., Mundorff, A. Z., Radini, A., Rao,
H., Trachsel, C., Velsko, I. M., \& Speller, C. F. (2018). Proteomic
evidence of dietary sources in ancient dental calculus.
\emph{Proceedings. Biological Sciences}, \emph{285}(1883), 20180977.
\url{https://doi.org/10.1098/rspb.2018.0977}

\leavevmode\vadjust pre{\hypertarget{ref-henryNeanderthalCalculus2014}{}}%
Henry, A. G., Brooks, A. S., \& Piperno, D. R. (2014). Plant foods and
the dietary ecology of {Neanderthals} and early modern humans.
\emph{Journal of Human Evolution}, \emph{69}, 44--54.
\url{https://doi.org/10.1016/j.jhevol.2013.12.014}

\leavevmode\vadjust pre{\hypertarget{ref-henryCalculusSyria2008}{}}%
Henry, A. G., \& Piperno, D. R. (2008). Using plant microfossils from
dental calculus to recover human diet: A case study from {Tell}
al-{Raqā}'i, {Syria}. \emph{Journal of Archaeological Science},
\emph{35}(7), 1943--1950.
\url{https://doi.org/10.1016/j.jas.2007.12.005}

\leavevmode\vadjust pre{\hypertarget{ref-henryDietAustralopithecus2012}{}}%
Henry, A. G., Ungar, P. S., Passey, B. H., Sponheimer, M., Rossouw, L.,
Bamford, M., Sandberg, P., de Ruiter, D. J., \& Berger, L. (2012). The
diet of {Australopithecus} sediba. \emph{Nature}, \emph{487}(7405,
7405), 90--93. \url{https://doi.org/10.1038/nature11185}

\leavevmode\vadjust pre{\hypertarget{ref-hidakaDietCalculus2007}{}}%
Hidaka, S., \& Oishi, A. (2007). An in vitro study of the effect of some
dietary components on calculus formation: Regulation of calcium
phosphate precipitation. \emph{Oral Diseases}, \emph{13}(3), 296--302.
\url{https://doi.org/10.1111/j.1601-0825.2006.01283.x}

\leavevmode\vadjust pre{\hypertarget{ref-hidakaStarchRole2008}{}}%
Hidaka, Saburo, Okamoto, Y., Tsukamoto, S., \& Oishi, A. (2008). The
{Possible Role} of {Starch} in {Oral Calcification}: {The In Vitro
Formation} of {Hydroxyapatite} is {Regulated} by a {Combination} of
{Protein} and {Mineral Content} in {Dietary Starch Flour}. \emph{The
Open Food Science Journal}, \emph{2}(1), 10--22.
\url{https://doi.org/10.2174/1874256400802010010}

\leavevmode\vadjust pre{\hypertarget{ref-hillsonDentalAnthropology1996}{}}%
Hillson, S. (1996). \emph{Dental {Anthropology}}. {Cambridge University
Press}.

\leavevmode\vadjust pre{\hypertarget{ref-jepsenCalculusRemoval2011}{}}%
Jepsen, S., Deschner, J., Braun, A., Schwarz, F., \& Eberhard, J.
(2011). Calculus removal and the prevention of its formation.
\emph{Periodontology 2000}, \emph{55}(1), 167--188.
\url{https://doi.org/10.1111/j.1600-0757.2010.00382.x}

\leavevmode\vadjust pre{\hypertarget{ref-jinSupragingivalCalculus2002}{}}%
Jin, Y., \& Yip, H.-K. (2002). Supragingival {Calculus}: {Formation} and
{Control}. \emph{Critical Reviews in Oral Biology \& Medicine}.
\url{https://doi.org/10.1177/154411130201300506}

\leavevmode\vadjust pre{\hypertarget{ref-kinastonOrtnerDentition2019}{}}%
Kinaston, R., Willis, A., Miszkiewicz, J. J., Tromp, M., \& Oxenham, M.
F. (2019). The {Dentition}: {Development}, {Disturbances}, {Disease},
{Diet}, and {Chemistry}. In J. E. Buikstra (Ed.), \emph{Ortner's
{Identification} of {Pathological Conditions} in {Human Skeletal
Remains} ({Third Edition})} (pp. 749--797). {Academic Press}.
\url{https://doi.org/10.1016/B978-0-12-809738-0.00021-1}

\leavevmode\vadjust pre{\hypertarget{ref-leonardPlantMicroremains2015}{}}%
Leonard, C., Vashro, L., O'Connell, J. F., \& Henry, A. G. (2015). Plant
microremains in dental calculus as a record of plant consumption: {A}
test with {Twe} forager-horticulturalists. \emph{Journal of
Archaeological Science: Reports}, \emph{2}, 449--457.
\url{https://doi.org/10.1016/j.jasrep.2015.03.009}

\leavevmode\vadjust pre{\hypertarget{ref-lieverseDietAetiology1999}{}}%
Lieverse, A. R. (1999). Diet and the aetiology of dental calculus.
\emph{International Journal of Osteoarchaeology}, \emph{9}(4), 219--232.
\url{https://doi.org/10.1002/(SICI)1099-1212(199907/08)9:4\%3C219::AID-OA475\%3E3.0.CO;2-V}

\leavevmode\vadjust pre{\hypertarget{ref-lieverseDentalHealth2007}{}}%
Lieverse, A. R., Link, D. W., Bazaliiskiy, V. I., Goriunova, O. I., \&
Weber, A. W. (2007). Dental health indicators of hunter--gatherer
adaptation and cultural change in {Siberia}'s {Cis-Baikal}.
\emph{American Journal of Physical Anthropology}, \emph{134}(3),
323--339. \url{https://doi.org/10.1002/ajpa.20672}

\leavevmode\vadjust pre{\hypertarget{ref-marshDentalPlaque2006}{}}%
Marsh, P. D. (2006). Dental plaque as a biofilm and a microbial
community -- implications for health and disease. \emph{BMC Oral
Health}, \emph{6}(S1), S14.
\url{https://doi.org/10.1186/1472-6831-6-S1-S14}

\leavevmode\vadjust pre{\hypertarget{ref-mickleburghNewInsights2012}{}}%
Mickleburgh, H. L., \& Pagán-Jiménez, J. R. (2012). New insights into
the consumption of maize and other food plants in the pre-{Columbian
Caribbean} from starch grains trapped in human dental calculus.
\emph{Journal of Archaeological Science}, \emph{39}(7), 2468--2478.
\url{https://doi.org/10.1016/j.jas.2012.02.020}

\leavevmode\vadjust pre{\hypertarget{ref-middletonOpalPhytoliths1994}{}}%
Middleton, W. D., \& Rovner, I. (1994). Extraction of {Opal Phytoliths}
from {Herbivore Dental Calculus}. \emph{Journal of Archaeological
Science}, \emph{21}(4), 469--473.
\url{https://doi.org/10.1006/jasc.1994.1046}

\leavevmode\vadjust pre{\hypertarget{ref-whoOralHealth}{}}%
\emph{Oral health}. (n.d.). {World Health Organization}. Retrieved March
14, 2022, from
\url{https://www.who.int/news-room/fact-sheets/detail/oral-health}

\leavevmode\vadjust pre{\hypertarget{ref-ortnerIdentificationPathological2003}{}}%
Ortner, D. J. (2003). \emph{Identification of {Pathological Conditions}
in {Human Skeletal Remains}}. {Academic Press}.

\leavevmode\vadjust pre{\hypertarget{ref-pilloudOutliningDefinition2019}{}}%
Pilloud, M. A., \& Fancher, J. P. (2019). Outlining a {Definition} of
{Oral Health} within the {Study} of {Human Skeletal Remains}: {Defining
Oral Health}. \emph{Dental Anthropology Journal}, \emph{32}(2, 2),
3--11. \url{https://doi.org/10.26575/daj.v32i2.297}

\leavevmode\vadjust pre{\hypertarget{ref-pipernoStarchGrains2008}{}}%
Piperno, D. R., \& Dillehay, T. D. (2008). Starch grains on human teeth
reveal early broad crop diet in northern {Peru}. \emph{Proceedings of
the National Academy of Sciences}, \emph{105}(50), 19622--19627.
\url{https://doi.org/10.1073/pnas.0808752105}

\leavevmode\vadjust pre{\hypertarget{ref-powerChimpCalculus2015}{}}%
Power, R. C., Salazar-Garcia, D. C., Wittig, R. M., Freiberg, M., \&
Henry, A. G. (2015). Dental calculus evidence of {Tai Forest Chimpanzee}
plant consumption and life history transitions. \emph{Scientific
Reports}, \emph{5}, 15161. \url{https://doi.org/10.1038/srep15161}

\leavevmode\vadjust pre{\hypertarget{ref-powerRepresentativenessDental2021}{}}%
Power, Robert C., Wittig, R. M., Stone, J. R., Kupczik, K., \&
Schulz-Kornas, E. (2021). The representativeness of the dental calculus
dietary record: Insights from {Taï} chimpanzee faecal phytoliths.
\emph{Archaeological and Anthropological Sciences}, \emph{13}(6), 104.
\url{https://doi.org/10.1007/s12520-021-01342-z}

\leavevmode\vadjust pre{\hypertarget{ref-radiniDirtyTeeth2022}{}}%
Radini, A., \& Nikita, E. (2022). Beyond dirty teeth: {Integrating}
dental calculus studies with osteoarchaeological parameters.
\emph{Quaternary International}.
\url{https://doi.org/10.1016/j.quaint.2022.03.003}

\leavevmode\vadjust pre{\hypertarget{ref-robertsDentalDisease2007}{}}%
Roberts, C. A., \& Manchester, K. (2007). Dental {Disease}. In \emph{The
{Archaeology} of {Disease}} (3rd Edition, pp. 63--83). {Cornell
University Press}.

\leavevmode\vadjust pre{\hypertarget{ref-sagneStudiesPeriodontal1977}{}}%
Sagne, S., \& Olsson, G. (1977). Studies of the {Periodontal Status} of
a {Medieval Population}. \emph{Dentomaxillofacial Radiology},
\emph{6}(1), 46--52. \url{https://doi.org/10.1259/dmfr.1977.0006}

\leavevmode\vadjust pre{\hypertarget{ref-scottBriefHistory2015}{}}%
Scott, G. R. (2015). A {Brief History} of {Dental Anthropology}. In J.
D. Irish \& G. R. Scott (Eds.), \emph{A {Companion} to {Dental
Anthropology}} (pp. 7--17). {John Wiley \& Sons, Ltd}.
\url{https://doi.org/10.1002/9781118845486.ch18}

\leavevmode\vadjust pre{\hypertarget{ref-sissonsMultistationPlaque1991}{}}%
Sissons, C. H., Cutress, T. W., Hoffman, M. P., \& Wakefield, J. S. J.
(1991). A {Multi-station Dental Plaque Microcosm} ({Artificial Mouth})
for the {Study} of {Plaque Growth}, {Metabolism}, {pH}, and
{Mineralization}: \emph{Journal of Dental Research}.
\url{https://doi.org/10.1177/00220345910700110301}

\leavevmode\vadjust pre{\hypertarget{ref-sotoCharacterizationDecontamination2019}{}}%
Soto, M., Inwood, J., Clarke, S., Crowther, A., Covelli, D., Favreau,
J., Itambu, M., Larter, S., Lee, P., Lozano, M., Maley, J., Mwambwiga,
A., Patalano, R., Sammynaiken, R., Vergès, J. M., Zhu, J., \& Mercader,
J. (2019). Structural characterization and decontamination of dental
calculus for ancient starch research. \emph{Archaeological and
Anthropological Sciences}, \emph{11}(9), 4847--4872.
\url{https://doi.org/10.1007/s12520-019-00830-7}

\leavevmode\vadjust pre{\hypertarget{ref-squierOralMucosa1998}{}}%
Squier, C. A., \& Finkelstein, M. W. (1998). Oral {Mucosa}. In A. R. Ten
Cate (Ed.), \emph{Oral {Histology}: {Development}, {Structure}, and
{Function}} (5th ed., pp. 345--385). {Mosby}.

\leavevmode\vadjust pre{\hypertarget{ref-tanCalculusUltrastructure2004}{}}%
Tan, B. T. K., Gillam, D. G., Mordan, N. J., \& Galgut, P. N. (2004). A
preliminary investigation into the ultrastructure of dental calculus and
associated bacteria. \emph{Journal of Clinical Periodontology},
\emph{31}(5), 364--369.
\url{https://doi.org/10.1111/j.1600-051X.2004.00484.x}

\leavevmode\vadjust pre{\hypertarget{ref-tanBacterialViability2004}{}}%
Tan, B. T. K., Mordan, N. J., Embleton, J., Pratten, J., \& Galgut, P.
N. (2004). Study of {Bacterial Viability} within {Human Supragingival
Dental Calculus}. \emph{Journal of Periodontology}, \emph{75}(1),
23--29. \url{https://doi.org/10.1902/jop.2004.75.1.23}

\leavevmode\vadjust pre{\hypertarget{ref-townsendDentalAnthropology2012}{}}%
Townsend, G., Kanazawa, E., \& Takayama, H. (Eds.). (2012). \emph{New
{Directions} in {Dental Anthropology}: {Paradigms}, {Methodologies} and
{Outcomes}}. {The University of Adelaide Press}.
\url{https://doi.org/10.1017/9780987171870}

\leavevmode\vadjust pre{\hypertarget{ref-trompEDTACalculus2017}{}}%
Tromp, M., Buckley, H., Geber, J., \& Matisoo-Smith, E. (2017). {EDTA}
decalcification of dental calculus as an alternate means of
microparticle extraction from archaeological samples. \emph{Journal of
Archaeological Science: Reports}, \emph{14}, 461--466.
\url{https://doi.org/10.1016/j.jasrep.2017.06.035}

\leavevmode\vadjust pre{\hypertarget{ref-vandermeerschMiddlePaleolithic1994}{}}%
Vandermeersch, B., Arensburg, B., Tillier, A. M., Rak, Y., Weiner, S.,
Spiers, M., \& Aspillaga, E. (1994). Middle {Paleolithic Dental Bacteria
From Kebara}, {Israel}. \emph{Comptes Rendus De L Academie Des Sciences
Serie Ii}, \emph{319}(6), 727--731.
\url{https://weizmann.esploro.exlibrisgroup.com/esploro/outputs/journalArticle/MIDDLE-PALEOLITHIC-DENTAL-BACTERIA-FROM-KEBARA/993266802803596}

\leavevmode\vadjust pre{\hypertarget{ref-velskoMicrobialDifferences2019}{}}%
Velsko, I. M., Fellows Yates, J. A., Aron, F., Hagan, R. W., Frantz, L.
A. F., Loe, L., Martinez, J. B. R., Chaves, E., Gosden, C., Larson, G.,
\& Warinner, C. (2019). Microbial differences between dental plaque and
historic dental calculus are related to oral biofilm maturation stage.
\emph{Microbiome}, \emph{7}(1), 102.
\url{https://doi.org/10.1186/s40168-019-0717-3}

\leavevmode\vadjust pre{\hypertarget{ref-waldronPalaeopathology2020}{}}%
Waldron, T. (2020). \emph{Palaeopathology}. {Cambridge University
Press}.

\leavevmode\vadjust pre{\hypertarget{ref-warinnerEvidenceMilk2014}{}}%
Warinner, C., Hendy, J., Speller, C., Cappellini, E., Fischer, R.,
Trachsel, C., Arneborg, J., Lynnerup, N., Craig, O. E., Swallow, D. M.,
Fotakis, A., Christensen, R. J., Olsen, J. V., Liebert, A., Montalva,
N., Fiddyment, S., Charlton, S., Mackie, M., Canci, A., \ldots{}
Collins, M. J. (2014). Direct evidence of milk consumption from ancient
human dental calculus. \emph{Scientific Reports}, \emph{4}, 7104.
\url{https://doi.org/10.1038/srep07104}

\leavevmode\vadjust pre{\hypertarget{ref-warinnerPathogensHost2014}{}}%
Warinner, C., Rodrigues, J. F., Vyas, R., Trachsel, C., Shved, N.,
Grossmann, J., Radini, A., Hancock, Y., Tito, R. Y., Fiddyment, S.,
Speller, C., Hendy, J., Charlton, S., Luder, H. U., Salazar-Garcia, D.
C., Eppler, E., Seiler, R., Hansen, L. H., Castruita, J. A., \ldots{}
Cappellini, E. (2014). Pathogens and host immunity in the ancient human
oral cavity. \emph{Nature Genetics}, \emph{46}(4), 336--344.
\url{https://doi.org/10.1038/ng.2906}

\leavevmode\vadjust pre{\hypertarget{ref-warinnerNewEra2015}{}}%
Warinner, C., Speller, C., \& Collins, M. J. (2015). A new era in
palaeomicrobiology: Prospects for ancient dental calculus as a long-term
record of the human oral microbiome. \emph{Philosophical Transactions of
the Royal Society B: Biological Sciences}, \emph{370}(1660), 20130376.
\url{https://doi.org/10.1098/rstb.2013.0376}

\leavevmode\vadjust pre{\hypertarget{ref-whiteDentalCalculus1997}{}}%
White, D. J. (1997). Dental calculus: Recent insights into occurrence,
formation, prevention, removal and oral health effects of supragingival
and subgingival deposits. \emph{European Journal of Oral Sciences},
\emph{105}(5), 508--522.
\url{https://doi.org/10.1111/j.1600-0722.1997.tb00238.x}

\leavevmode\vadjust pre{\hypertarget{ref-whiteHumanOsteology2011}{}}%
White, T. D., Black, M. T., \& Folkens, P. A. (2011). \emph{Human
{Osteology}} (3rd edition). {Academic Press}.

\leavevmode\vadjust pre{\hypertarget{ref-whiteBoneManual2005}{}}%
White, T. D., \& Folkens, P. A. (2005). \emph{The {Human Bone Manual}}
(1st edition). {Academic Press}.

\leavevmode\vadjust pre{\hypertarget{ref-wongCalciumPhosphate2002}{}}%
Wong, L., Sissons, C. H., Pearce, E. I. F., \& Cutress, T. W. (2002).
Calcium phosphate deposition in human dental plaque microcosm biofilms
induced by a ureolytic {pH-rise} procedure. \emph{Archives of Oral
Biology}, \emph{47}(11), 779--790.
\url{https://doi.org/10.1016/S0003-9969(02)00114-0}

\leavevmode\vadjust pre{\hypertarget{ref-wrightAdvancingRefining2021}{}}%
Wright, S. L., Dobney, K., \& Weyrich, L. S. (2021). Advancing and
refining archaeological dental calculus research using multiomic
frameworks. \emph{STAR: Science \& Technology of Archaeological
Research}, \emph{7}(1), 13--30.
\url{https://doi.org/10.1080/20548923.2021.1882122}

\leavevmode\vadjust pre{\hypertarget{ref-yaussyCalculusSurvivorship2019}{}}%
Yaussy, S. L., \& DeWitte, S. N. (2019). Calculus and survivorship in
medieval {London}: {The} association between dental disease and a
demographic measure of general health. \emph{American Journal of
Physical Anthropology}, \emph{168}(3), 552--565.
\url{https://doi.org/10.1002/ajpa.23772}

\leavevmode\vadjust pre{\hypertarget{ref-zhangDentalDisease1982}{}}%
Zhang, Y. (1982). Dental disease of neolithic age skulls excavated in
shaanxi province. \emph{Chinese Medical Journal}, \emph{95}(06),
391--396. \url{https://doi.org/10.5555/cmj.0366-6999.95.06.p391.01}

\end{CSLReferences}

\bookmarksetup{startatroot}

\hypertarget{chap-background}{%
\chapter{Background}\label{chap-background}}

The human mouth, or oral cavity, contains many different types of
surfaces on which bacteria can attach and grow. These surfaces are both
hard (teeth) and soft (mucosa, tongue, gingiva), and are exposed to the
external environment. For this reason, the conditions within the oral
cavity can vary considerably, resulting in a unique range of habitats
for a wide variety of microbes. In fact, the oral biome contains
bacteria from over 700 different species, some of which still haven't
been named. There are so many bacteria in our mouth that it's actually
hard to determine how many there are at any given time, but most
estimates are in the billions. The oral biome is complex. You just won't
believe how vastly, hugely, mind-bogglingly complex it is. I mean, you
may think quantum physics is complicated, but that's just peanuts to the
oral biome (\protect\hyperlink{ref-adamsHitchhikersGuide2002}{Adams,
2002c, p. 66}). As such, this chapter reflects the knowledge at the time
of writing, and no warranty is given for the inevitable new developments
that will change what we now believe to be true.

\hypertarget{biofilms}{%
\section{Biofilms}\label{biofilms}}

The concept of biofilms represents a recent paradigm shift in
microbiology (\protect\hyperlink{ref-costertonBacterialBiofilms1987}{J.
W. Costerton et al., 1987};
\protect\hyperlink{ref-costertonMicrobialBiofilms1995}{J. William
Costerton et al., 1995}). Previously, researchers believed that you
could isolate the organism of interest and learn about its growth,
metabolism, etc. They assumed bacteria would behave the same as a
free-floating organism in a lab test tube as it would in a real-world
environment (such as the human mouth). More recently researchers have
discovered that the behaviour of bacteria differs when they are part of
a larger community, compared to when they are grown in isolation. One
such baterial community is a biofilm. Biofilms consist of large,
intricate, multi-species communities of bacteria enclosed in an
extracellular matrix of their own creation. The ability to produce this
matrix gives the bacteria living within it an adaptive advantage
compared to free-floating (planktonic) organisms. It equips them with
resistance to both antimicrobials (such as antibiotic medication) and
immune responses from the host that would normally be detrimental to
their ability to survive
(\protect\hyperlink{ref-marshDentalPlaque2005}{Philip D. Marsh, 2005};
\protect\hyperlink{ref-marshPhysiologicalApproaches1997}{Philip D. Marsh
\& Bradshaw, 1997}). Resistence to varying conditions is especially
important in the oral cavity, which is a site of frequent fluctuations
in temperature, pH, and oxygen availability. The viscoelastic nature of
the biofilm provides some protection against mechanical destruction and
dislodgement caused by, for example, the tongue and dental hygience
practices
(\protect\hyperlink{ref-petersonViscoelasticityBiofilms2015}{Peterson et
al., 2015}). It also allows them to acquire nutrients from outside the
biofilm, as well as generate and distribute nutrients within the biofilm
to different communities of bacteria
(\protect\hyperlink{ref-flemmingBiofilmsEmergent2016}{Flemming et al.,
2016}). Biofilms are quite persistent structures, and very few surfaces
exist that can completely prevent bacterial colonisation and biofilm
formation
(\protect\hyperlink{ref-rennerPhysicochemicalRegulation2011}{Renner \&
Weibel, 2011}).

\hypertarget{dental-plaque}{%
\subsection{Dental plaque}\label{dental-plaque}}

Dental calculus forms from a specific oral biofilm known as dental
plaque. After we clean our teeth, our saliva coats the surface of our
teeth (enamel) with a layer of proteins known as the dental pellicle (or
acquired enamel pellicle). The pellicle is a film that protects our
teeth from both mechanical wear and chemical decay, but in doing so,
provides a viable surface for microorganisms to attach and initiate
biofilm growth (\protect\hyperlink{ref-yaoIdentificationProtein2003}{Yao
et al., 2003}). Biofilm formation goes through several, often abitrarily
defined, stages of growth. They are arbitrary because they are defined
by the researchers who study them, but are also necessary as a
foundation to explain the development of a biofilm. Rather than thinking
about the stages as occurring sequentially, you should think of them as
occurring concurrently across different areas of the tooth surface.
Biofilm formation is a very dynamic process, and is often
over-simplified in visualisations (not unlike
Figure~\ref{fig-biofilm-form}).

\begin{figure}

{\centering \includegraphics{./figures/biofilm_formation.png}

}

\caption{\label{fig-biofilm-form}A simplified overview of biofilm
formation. Created with BioRender.com. Still under construction.}

\end{figure}

The pellicle contains molecules (known as adhesins) that enable specific
bacteria to attach to complementary receptors on the pellicle. So when
the pellicle adsorbs to the tooth, it becomes a surface for bacterial
attachment (\protect\hyperlink{ref-yaoIdentificationProtein2003}{Yao et
al., 2003}). The first bacteria to attach are known as early coloniser
bacteria (or pioneer colonisers) and include \emph{Streptococcus}
species (spp.), \emph{Actinomyces} spp., and \emph{Haemophilus} spp
(\protect\hyperlink{ref-uzelMicrobialShifts2011}{Uzel et al., 2011};
\protect\hyperlink{ref-zijngeBiofilmArchitecture2010}{Zijnge et al.,
2010}). The initial attachment occurs when the random movement of
bacteria and the flow of saliva brings them close enough to the pellicle
to attach. Some bacteria have a limited, often random, ability to move
if they have long tail-like structures known as flagella, but most are
brought to the surface by saliva.

As bacteria approach the pellicle-coated surface of a tooth, there are
both attractive and repulsive forces at work. Repulsion because both the
bacteria and pellicle proteins have a net negative charge
(\protect\hyperlink{ref-songEffectsMaterial2015}{Song et al., 2015}),
causing eloctrostatic repulsive force; and attraction from van der Waals
forces. Bacteria may be more or less likely to attach depending on the
distance from the bacteria to the surface. If the bacteria come too
close to the surface, the initial attraction (primary maximum) will most
likely be overcome by repulsion (primary maximum). Bacteria are more
likely to attach when they encounter attractive forces at a further
distance (secondary minimum), ultimately leading to a game of
`will-they-won't-they' between the bacteria and pellicle. This initial
attachment is a weak physicochemical long-distance (10--20 nm; it's a
long distance for bacteria) attraction; therefore, attachment is
initially reversible, as bacteria can become detached by salivary flow
or shearing action by the tongue
(\protect\hyperlink{ref-marshDentalPlaque2016}{Philip D. Marsh et al.,
2016}). This model of bacterial attachment, also known as the DLVO
theory, can partially explain the aspects involved in microbial
adhesion. Further explanation includes hydrodynamic forces, where
hydrophobic components of the pellicle and cell surface interact
(\protect\hyperlink{ref-bosPhysicochemistryInitial1999}{R. Bos, 1999};
\protect\hyperlink{ref-vigeantReversibleIrreversible2002}{Vigeant et
al., 2002}). Overcoming the repulsive forces may be in part facilitated
by motility in some organisms. The aforementioned flagellum, for
example, may give the necessary `push' to reach a region of net
attractive forces
(\protect\hyperlink{ref-jinSupragingivalCalculus2002}{Jin \& Yip,
2002}). Additionally, the ionic strength of saliva may play a role in
reducing electrostatic repulsion with increasing ionic strength
(\protect\hyperlink{ref-rennerPhysicochemicalRegulation2011}{Renner \&
Weibel, 2011}).

\includegraphics{figures/bacterial-structure.png}

Attachment becomes stronger and colonisation becomes more solidified at
a shorter distance, as surface molecules on the bacteria interact with
complementary receptors on the pellicle, and the interactions between
bacteria and pellicle become more direct. Some bacteria have components
on their surface that allow them to attach directly to complementary
components on the dental pellicle (adhesin-receptor interactions). These
attachments are very specific because only certain bacteria have the
right molecules on their surface
(\protect\hyperlink{ref-jinSupragingivalCalculus2002}{Jin \& Yip,
2002}). These receptors are often carbohydrates formed by the host,
meaning us. Early colonisers are also able to attach to proteins and
enzymes present in saliva, as well as onto the surface of other bacteria
already attached to the pellicle
(\protect\hyperlink{ref-jinSupragingivalCalculus2002}{Jin \& Yip, 2002};
\protect\hyperlink{ref-nikitkovaStarchBiofilms2013}{Nikitkova et al.,
2013}). When bacteria come within a shorter distance of the pellicle
they may also attach directly to the surface with other hair-like
structures (fimbriae) that are present on the surface of some bacteria.
These hair-like structures attach to matching receptors that are present
in the pellicle
(\protect\hyperlink{ref-nobbsStreptococcusAdherence2009}{Nobbs et al.,
2009}).

While some bacteria specialise in attaching to surfaces, not all of them
possess this ability. However, once the specialists have attached, they
facilitate the adhesion of other bacteria (secondary colonisers) by
allowing them to attach to their surface (coadhesion) rather than
directly to the pellicle. For example, \emph{S. gordonii} can attach to
the pellicle and facilitate coadhesion with \emph{A. naeslundii}
(\protect\hyperlink{ref-palmerCoaggregationInteractions2003}{Robert J.
Palmer Jr. et al., 2003}). Not all attachments involve proteins. They
can also involve carbohydrates, enzymes, and various appendages on the
surface of the bacteria, although these appendages often consist of
proteins in their structure, for example the already mentioned pili and
fimbriae (\protect\hyperlink{ref-nobbsStreptococcusAdherence2009}{Nobbs
et al., 2009}). This can occur on a large scale, causing the number and
types of bacteria on the tooth surface to grow, due to the ability of
different species to attach to one another (coaggregation)
(\protect\hyperlink{ref-jinSupragingivalCalculus2002}{Jin \& Yip, 2002};
\protect\hyperlink{ref-marshDentalPlaque2006}{Philip D. Marsh, 2006}).
Coaggregation and coadhesion is an important part of the growing oral
biofilm. Most taxa don't have the necessary morphology to attach
directly to a substrate, however most oral taxa CAN coaggregate with
other species through cell-cell interactions, usually involving
polysaccharides on the bacterial-cell surfaces
(\protect\hyperlink{ref-kolenbranderOralMultispecies2010}{Kolenbrander
et al., 2010};
\protect\hyperlink{ref-palmerInterbacterialAdhesion2017}{Robert J.
Palmer et al., 2017}).

As the biofilm formed by early colonisers grows through continued
multiplication and coadhesion/coaggregation, the diversity of the
biofilm increases. The proportion of early-colonising streptococci
gradually decreases while there is an increase of \emph{Tannerella
forsythia} \emph{Actinomyces} spp. and \emph{Fusobacterium nucleatum}
(\protect\hyperlink{ref-zijngeBiofilmArchitecture2010}{Zijnge et al.,
2010}). \emph{F. nucleatum} is a bacterium also known as the `bridging
species', as it's believed to play an important part in linking together
early and late coloniser species---including \emph{Prevotella} spp.,
\emph{S. gordonii}, and \emph{Porphyromonas gingivalis}--- which might
not otherwise be able to coaggregate
(\protect\hyperlink{ref-kolenbranderOralMultispecies2010}{Kolenbrander
et al., 2010};
\protect\hyperlink{ref-kolenbranderAdhereToday1993}{Kolenbrander \&
London, 1993}). The increasing diversity of bacteria adhering to a
surface results in communities of bacteria with the ability to
communicate with each other, distribute nutrients, and alter the local
environment for more favourable conditions. This is made possible by the
presence of an extracellular matrix, formed by the production of
polymers by certain bacterial species
(\protect\hyperlink{ref-marshMicrobiologyDental2010}{Philip D. Marsh,
2010}). Microenvironmental changes can allow species to survive in
otherwise unfavourable environments; for example, the survival of many
obligate anaerobes in an environment which is largely aerobic (oxygen
continuously enters the oral cavity as we breathe). Bacteria with the
ability to consume oxygen and produce carbon dioxide allow bacteria with
a lower oxygen tolerance to thrive
(\protect\hyperlink{ref-marshDentalPlaque2005}{Philip D. Marsh, 2005}).
In fact, dental plaque predominantly consists of obligate and
facultative anaerobes and is especially true for
periodontitis-associated biofilms, which tend to be dominated by more
species with a lower oxygen tolerance than their non-periodontitis
counterparts (\protect\hyperlink{ref-curtisRoleMicrobiota2020}{Curtis et
al., 2020}). A pH balance may be maintained by species that are able to
consume acidic metabolic products produced by other species, and convert
them to weaker acids. \emph{Veillonella} spp. especially
(\protect\hyperlink{ref-marshDentalPlaque2005}{Philip D. Marsh, 2005}).
Metabolic products of some bacteria are used by others as nutrients.
By-products of urea metabolism can be used by some organisms, who
further break down the by-products, which can be used by yet other
organisms (\protect\hyperlink{ref-flemmingBiofilmsEmergent2016}{Flemming
et al., 2016}). Working as a community can increase survivability in the
harsh and dynamic environment of the oral cavity, with rapid changes in
pH, oxygen, nutrient availability, etc.

Perhaps ironically, an important part of the maturation of a biofilm is
the removal of bacteria from the biofilm itself. Removal can occur
through both internal and external mechanisms. It's likely that there is
a continuous loss of microbes near/on the surface of the biofilm caused
by shear forces from saliva and mechanical removal by the tongue. There
can be multiple motivating factors involved in the active detachment by
bacteria, including increasingly adverse conditions within the biofilm,
such as nutrient depletion or an unfavourable local environment. If
sufficiently adverse conditions persist, certain bacteria may make the
active decision to `peace out'. Dispersion of bacteria from a biofilm
requires production of matrix-degrading enzymes, and, as such, not all
bacteria can actively disperse from a biofilm
(\protect\hyperlink{ref-petrovaEscapingBiofilm2016}{Petrova \& Sauer,
2016}). The detached bacteria then colonise other parts of the biofilm,
making the biofilm a highly dynamic structure undergoing continuous
remodelling
(\protect\hyperlink{ref-flemmingBiofilmsEmergent2016}{Flemming et al.,
2016}).

So far, the picture of biofilm formation is one of peaceful
coexsistance, collaboration, and even neighbourly interspecies actions.
A basis for this cooperation is increased overall benefits to the
communities
(\protect\hyperlink{ref-renduelesMechanismsCompetition2015}{Rendueles \&
Ghigo, 2015}). However, competition between bacteria still exists within
the biofilm. The metabolic by-products produced by some bacteria may be
toxic for others, allowing the producers to gain a competitive
advantage. The aforementioned acid-production by some bacteria can cause
unfavourable conditions for species that prefer more neutral pH
environments, particularly in the absence of the secondary feeders that
would normally neutralise these compounds. A more direct example of
bacterial competition is the ability of bacteria to produce substances
that are toxic to other bacteria. These are often proteins or peptides
termed bacteriocins, and can either inhibit or even kill other bacteria
(\protect\hyperlink{ref-dawBacteriocinsNature1996}{Daw \& Falkiner,
1996}; \protect\hyperlink{ref-grahamEnterococcusFaecalis2017}{Graham et
al., 2017}). \emph{S. sanguinis} and \emph{S. gordonii} can produce
H\textsubscript{2}O\textsubscript{2} that is toxic to \emph{S. mutans},
a member of their own genus. \emph{S. mutans} can, in turn, produce
mutacin, which inhibits the growth of \emph{S. sorbrinus}. There is no
love lost among these close relatives
(\protect\hyperlink{ref-chenSpecificGenes1999}{P. Chen et al., 1999}).
In addition to H\textsubscript{2}O\textsubscript{2}, oral streptococci
can produce lactate by consuming carbohydrates, giving them a
competitive advantage over acid-sensitive species by altering the local
environment. Some species are resistant to specific metabolic
by-products that others consider toxic, and may even consider them a
delicacy (so to speak). \emph{Veillonella} spp. are an example of
organisms that thrive under these conditions, allowing both streptococci
and \emph{Veillonella} spp. to accumulate in the biofilm and create a
favourable environment to select species
(\protect\hyperlink{ref-edlundUncoveringComplex2018}{Edlund et al.,
2018}). These are simplistic examples, and often competition involves
more interactions between multiple species taking on various roles of
`sensing', `mediating', and `killing'
(\protect\hyperlink{ref-renduelesMechanismsCompetition2015}{Rendueles \&
Ghigo, 2015}). Competition between and within species will ultimately
shape the wider biofilm communities.

\hypertarget{dental-calculus}{%
\subsection{Dental calculus}\label{dental-calculus}}

The exact mechanism of dental calculus formation is not fully
understood, but involves processes of biomineralisation and crystal
formation within dental plaque. The main mineral components of calculus
are crystals containing various combinations of calcium and phosphate
ions. Other salts are also present, but the bulk of the crystals are
made up of calcium phosphates. Initial mineralisation of dental plaque
is a chemical process in which equilibrium of minerals in saliva and
gingival crevicular fluid tips towards saturation with regard to calcium
and phosphate, causing an increase of precipitation relative to
dissolution. This means, that when the concentration of ions increases
and tips the balance between dissolution and precipitation, salts will
accumulate within and on the surface of the biofilm. An increase in
concentration of minerals within the biofilm reaches a critical
threshold (supersaturation) and nucleation is triggered within the
plaque matrix, initiating crystal growth. This may or may not involve
spontaneous (or homogenous) nucleation, as it's unclear whether mineral
concentrations are sufficient to cause spontaneous nucleation, or
whether other biochemical processes act as a catalyst
(\protect\hyperlink{ref-omelonReviewPhosphate2013}{Omelon et al.,
2013}). That it's a chemical process can be shown by the ability to
produce calculus deposits in germ-free rats
(\protect\hyperlink{ref-glasBiophysicalStudies1962}{Glas \& Krasse,
1962}; \protect\hyperlink{ref-theiladeGermfreeCalculus1964}{Theilade et
al., 1964}). Although it's unclear how the germ-free calculus compares
to conventional calculus, and, to my knowledge there have only been
studies on rats. Just because calculus growth can be induced in sterile
conditions, doesn't mean bacteria are non-essential to the process.
Bacteria are inevitably part of the scaffolding of dental calculus in
humans, since, as I mentioned in the beginning of this chapter, our
mouths are full of bacteria, and dental plaque is essentially built by
bacteria. Mineralisation does seem to start in the biofilm matrix
between microorganisms, but they are eventually also mineralised along
with the biofilm matrix
(\protect\hyperlink{ref-friskoppUltrastructureNondecalcified1983}{Friskopp,
1983}). There are pockets of living bacteria within dental calculus.
These pockets and the layer of plaque that covers the surface of dental
calculus are likely what cause the correlation between calculus presence
and periodontal disease
(\protect\hyperlink{ref-tanBacterialViability2004}{B. T. K. Tan, Mordan,
et al., 2004}). While the process can be explained by chemistry, the
conditions leading up to and surrounding the process are both chemical
and biological in nature, and certainly involve bacteria. The main
source of minerals in the oral cavity is saliva, which enters the mouth
through salivary glands. The three main paired glands are the parotid,
sublingual, and submandibular glands, located by the cheeks, under the
tongue, and under the lower jaw bone, respectively. Saliva contains
sodium (Na), potassium (K), calcium (Ca), chlorine (Cl), bicarbonate
(buffer), and inorganic phosphate (Pi)
(\protect\hyperlink{ref-dawesEffectsDiet1970}{Colin Dawes, 1970};
\protect\hyperlink{ref-doddsHealthBenefits2005}{Michael W. J. Dodds et
al., 2005}), and the locations of the glands contribute to the pattern
of dental calculus deposits within the mouth, which commonly grow on the
buccal portion of maxillary (upper) molars and the lingual portion of
mandibular (lower) incisors
(\protect\hyperlink{ref-jinSupragingivalCalculus2002}{Jin \& Yip, 2002};
\protect\hyperlink{ref-whiteDentalCalculus1997}{D. J. White, 1997}).
Salivary pH also affects saturation of salts, which in turn is
influenced by salivary flow rates. Increased flow rate of saliva will
increase salivary pH, which reduces dissolution and increases
precipitation of calcium and phosphate. This is an important mechanism
that protects our teeth against demineralisation of the enamel caused by
caries. Protection is provided by the exchange of calcium and phosphate
from saliva to enamel
(\protect\hyperlink{ref-dahlenMicrobiologicalStudy2010}{Dahlén et al.,
2010}). Saliva further acts as a buffer for the oral cavity, reducing
the impact of short-term drops in pH caused by metabolic byproducts of
acid-producing bacteria
(\protect\hyperlink{ref-doddsHealthBenefits2005}{Michael W. J. Dodds et
al., 2005}; \protect\hyperlink{ref-jinSupragingivalCalculus2002}{Jin \&
Yip, 2002}). Higher rates of salivary flow are also likely to contribute
to an increase in calcium and phosphate secretion in addition to pH, all
contributing to an environment favouring plaque mineralisation.
Metabolic byproducts produced by bacteria can also affect local pH, both
pushing towards alkaline conditions as well as acidic. A major cause of
acidic pH is metabolism of overabundant dietary sugars and starch,
especially the metabolic activity of \emph{Streptococcus mutans}, known
to be one of the main culprits behind dental caries
(\protect\hyperlink{ref-bowenOralBiofilms2018}{Bowen et al., 2018};
\protect\hyperlink{ref-duarteInfluencesStarch2008}{Duarte et al., 2008};
\protect\hyperlink{ref-extercateAAA2010}{Exterkate et al., 2010}).

Conversely, alkaline conditions can be generated by metabolism of
various products that can either be directly or indirectly linked to
diet. One such product is urea. Urea is present in saliva, and its
concentration depends on multiple factors. One of these factors is a
high-protein diet, which increases levels of urea in serum and saliva
(\protect\hyperlink{ref-lieverseDietAetiology1999}{Lieverse, 1999}).
Hydrolysis of urea produces ammonia and causes a rise in pH. Bacteria
possess the ability to produce ammonia from urea, which is further used
by ammonia-oxidising organisms and converted to nitrite
(\protect\hyperlink{ref-flemmingBiofilmsEmergent2016}{Flemming et al.,
2016}; \protect\hyperlink{ref-sissonsPHResponse1994}{Sissons et al.,
1994}; \protect\hyperlink{ref-wongCalciumPhosphate2002}{Wong et al.,
2002}). In a similar way, arginine can be broken down to ammonia and
increase in pH. Extended fluctuations in environmental conditions can
alter the composition of biofilms
(\protect\hyperlink{ref-huangFactorsAssociated2012}{X. Huang et al.,
2012}; \protect\hyperlink{ref-huangEffectArginine2017}{Xuelian Huang et
al., 2017}). Another pathway to alkalinity is through enzymatic
activity. Saliva contains proteases which specialise in breaking down
proteins into smaller components such as ammonia, and increased protease
activity in saliva may therefore cause an increase in calculus
production (\protect\hyperlink{ref-jinSupragingivalCalculus2002}{Jin \&
Yip, 2002}).

There are also a number of inhibitors and promoters of mineralisation
present in the oral cavity, originating both from saliva and bacteria.
Substances known to promote plaque mineralisation through hydroxyapatite
formation and deposition, calcium-phospholipid-phosphate complexes
(CPLX), are present in bacteria. \emph{Corynebacterium matruchotii}
(formerly \emph{Bacterionema matruchotii}) accumulates calcium within
its cell structure, and has therefore received a lot of attention in
biomineralisation studies Ennever \& Creamer
(\protect\hyperlink{ref-enneverMicrobiologicCalcification1967}{1967}).
Biomineralisation is not a feature unique to \emph{Corynebacterium
matruchotii}. Even species associated with caries may induce
calcification under the right conditions and after cell death
(\protect\hyperlink{ref-moorerCalcificationCariogenic1993}{Moorer et
al., 1993};
\protect\hyperlink{ref-sidawayMicrobiologicalStudy1978a}{Sidaway,
1978}). Inhibitors of biomineralisation include salivary proline-rich
polypeptides, small amino acids important for the immune system; and
statherin, a protein that controls the precipitation of calcium
phosphate in saliva
(\protect\hyperlink{ref-jinSupragingivalCalculus2002}{Jin \& Yip,
2002}).

It's likely that multiple biomineralisation events occur under various
conditions, resulting in a heterogenous calculus composition with
crystals of various stages of growth
(\protect\hyperlink{ref-friskoppUltrastructureNondecalcified1983}{Friskopp,
1983}; \protect\hyperlink{ref-friskoppComparativeScanning1980}{Friskopp
\& Hammarström, 1980}). The differing susceptibility of bacteria to
calcification is also a contributor the heterogenous composition.
Overall, plaque mineralisation is a complex interaction between
conditions in the local environment, availability of minerals, the
equilibrium between precipitation and dissolution, balance between
nucleation promoters and inhibitors.

\hypertarget{oral-biofilm-models}{%
\section{Oral biofilm models}\label{oral-biofilm-models}}

Biofilm models are a way of studying the growth and development of
biofilms. By creating models that replicate the conditions and
complexity (to some extent) of biofilms in a lab, models allow
researchers to conduct various experiments to test the efficacy of
treatments on the growth and pathogenicity of biofilms. There are many
choices to be made when growing a biofilm, such as the composition of
the initial oral microbial community, nutrient content and availability,
and the makeup of the atmosphere {[}surrounding{]} the model. As such,
biofilm models can differ widely in their complexity and ability to
mimic conditions in a human mouth. A choice of model can be made based
on the end-goals of the research, or in some cases the choice is made
for you based on (a lack of) available equipment and financial
constraints. All models must have a defined biome containing a
substratum and nutrients. The substratum is a surface on which the
biofilm is intended to form and grow. For oral biofilm models the
environment is the oral cavity and the substrata are the teeth, tongue,
mucosa, or whatever the model isthe biofilm supposed to be mimicing. The
simplest models generally involve multiwell plates (e.g., 6-, 24-, and
98-well plates) with a substratum, usually glass cover-slips or
hydroxyapatite discs, placed at the bottom of the well. Similar models
suspend the substrata from a lid to promote active attachment of
bacteria to the substrata
(\protect\hyperlink{ref-extercateAAA2010}{Exterkate et al., 2010}). When
the substrata are attached to a lid instead of the multiwell plates, it
allows samples to be periodically transferred between solutions/media if
necessary, adding more flexibility to the experimental setup.

Next, an inoculate is chosen. This can be anything from a single species
of bacterium (pure culture), to multiple select species (defined
consortium), to all organisms occurring naturally within a system
(microcosm) (\protect\hyperlink{ref-mcbainBiofilmModels2009}{McBain,
2009}). The purpose of the incoulate is to initiate biofilm formation by
allowing the bacteria to adsorb to the substrata, ideally in the
presence of a conditioning film, such as saliva. For pure cultures and
defined consortia, the inoculate may come from saliva or another oral
site, such as dental plaque. The bacteria of interest are then isolated
using selective media, essentially providing ideal growing conditions to
certain types of bacteria, promoting their growth and eliminating others
(e.g. \protect\hyperlink{ref-bassonEstablishmentCommunity1996}{Basson \&
van Wyk, 1996}). Alternatively, the bacteria can be acquired directly
from companies like the American Type Culture Collection (ATCC). For
microcosms, the inoculate is often the saliva itself, or dental plaque,
in its (mostly) raw form. The inoculate is added to the wells to
initiate biofilm formation on the substrata as described
\protect\hyperlink{dental-plaque}{above}. As such, the content of the
inoculate influences the complexity of the biofilm microbiome as well as
the interactions between the communities within the biofilm
(\protect\hyperlink{ref-roderStudyingBacterial2016}{Røder et al.,
2016}). It's not always possible to use donated saliva as a growth
medium for the duration of the experiment. Especially if the experiment
lasts more than a few days. Media containing salivary components may be
added for extended experiments. There are many different recipes for
media floating around out there, but most of them are generally a
mixture containing mucin, proteins, minerals commonly found in saliva,
and a buffer to maintain pH
(\protect\hyperlink{ref-extercateAAA2010}{Exterkate et al., 2010};
\protect\hyperlink{ref-prattenVitroStudies1998}{Pratten et al., 1998};
\protect\hyperlink{ref-shellisSyntheticSaliva1978}{Shellis, 1978};
\protect\hyperlink{ref-sissonsMultistationPlaque1991}{Sissons et al.,
1991}; \protect\hyperlink{ref-tianUsingDGGE2010}{Tian et al., 2010}).

More complicated models make use of increasingly sophisticated equipment
to mimic the oral environment. Another level of model complexity can be
added by adjusting the rate at which nutrients are dispersed through the
system, and the overall nutrient supply. Nutrient distribtion can be
continuous, semi-continuous, or batch cultures, with the latter
providing a finite amount of nutrients in a closed system. An example of
a batch culture model is a biofilm grown on an agar plate, which has a
finite amount of resources
(\protect\hyperlink{ref-kearnsMasterRegulator2005}{Kearns et al.,
2005}). Once the nutrients in the agar have been depleted, that's it. At
the other end of the spectrum is a system with a pump attached to a
reservoir that can continuously supply the biofilm with growth medium,
similar to salivary flow. In between the former options is the
semi-continuous supply of nutrients. This can, for example, be the
multiwell plate model with a lid, where the samples can be periodically
transferred to new plates containing fresh growth medium
(\protect\hyperlink{ref-extercateAAA2010}{Exterkate et al., 2010}).
Other parameters that can be controlled to more closely simulate
conditions in the oral cavity are pH and gas phase, as can be done with
the multistation artificial mouth (MAM). The MAM gives researchers
control over a large number of parameters using multiple chambers with
complete control over the flow of treatment and/or nutrient
conditions---environmental conditions such as pH, temperature, and gas
phase---and access to real-time measurements
(\protect\hyperlink{ref-sissonsArtificialPlaque1997}{Sissons, 1997}).

The duration of an experiment depends on the scope of the study. If the
purpose is to learn more about initial biofilm formation and prevention,
it may only be necessary to grow the biofilms for a few hours to 48
hours (\protect\hyperlink{ref-dibdinDiffusionSugars1981}{Dibdin, 1981};
\protect\hyperlink{ref-extercateAAA2010}{Exterkate et al., 2010}). If,
instead, the goal is to learn more about biofilm maturation and
calcification, the experiments can run for days or even weeks
(\protect\hyperlink{ref-filocheFluorescenceAssay2007}{Sara K. Filoche et
al., 2007};
\protect\hyperlink{ref-sissonsMultistationPlaque1991}{Sissons et al.,
1991}; \protect\hyperlink{ref-wongCalciumPhosphate2002}{Wong et al.,
2002}).

Models developed for studying oral biofilms include, in increasing
complexity, the ACTA active attachment (AAA) model
(\protect\hyperlink{ref-extercateAAA2010}{Exterkate et al., 2010}),
Calgary biofilm device
(\protect\hyperlink{ref-ceriCalgaryBiofilm1999}{Ceri et al., 1999}),
modified Robbins device (MRD)
(\protect\hyperlink{ref-honraetModifiedRobbins2006}{Honraet \& Nelis,
2006}), constant depth film-fermenter (CDFF)
(\protect\hyperlink{ref-petersConstantDepth1988}{Peters \& Wimpenny,
1988}), and the multistation artificial mouth (MAM)
(\protect\hyperlink{ref-sissonsMultistationPlaque1991}{Sissons et al.,
1991}) representing the upper echelon of complexity. Summaries of
biofilm models, including benefits and limitations of the various types,
can be found in reviews by McBain -McBain
(\protect\hyperlink{ref-mcbainBiofilmModels2009}{2009}), Tan and
colleagues -C. H. Tan et al.
(\protect\hyperlink{ref-tanAllTogether2017}{2017}), and Røder and
colleagues -Røder et al.
(\protect\hyperlink{ref-roderStudyingBacterial2016}{2016}).

It might be tempting to think that the goal should always be to mimic
the oral environment as closely as possible. However, there are benefits
to more simplistic models, as well as limitations to the more
sophisticated models. Benefits of pure cultures and defined consortia
are reproducibility between experiments and more control over
physiological and factors and making it easier to take various
measurements. Microcosms have the benefit of more closely mimicking the
complexity of the organisms' natural environment
(\protect\hyperlink{ref-mcbainBiofilmModels2009}{McBain, 2009}).
However, even microcosms can be limited in their ability to recreate the
complexity and diversity of the oral microbiome
(\protect\hyperlink{ref-tianUsingDGGE2010}{Tian et al., 2010}).
Alternatives to \emph{in vitro} models are \emph{in situ} models which
usually involve growing plaque on a removable surface inside a the mouth
of a willing participant. These models add a level of realism, as they
are grown inside an actual oral cavity, and can reflect biogeographical
differences in biofilm composition caused by differing conditions across
the oral cavity. They also come with additional difficulties and reduced
control over experimental parameters
(\protect\hyperlink{ref-marshRoleMicrobiology1995}{P. D. Marsh, 1995};
\protect\hyperlink{ref-zeroSituCaries1995}{Zero, 1995}).

\hypertarget{references-1}{%
\section*{References}\label{references-1}}
\addcontentsline{toc}{section}{References}

\markright{References}

\hypertarget{refs-2}{}
\begin{CSLReferences}{1}{0}
\leavevmode\vadjust pre{\hypertarget{ref-adamsHitchhikersGuide2002}{}}%
Adams, D. (2002). \emph{The {Hitchhiker}'s {Guide} to the {Galaxy}}.
{Picador}.

\leavevmode\vadjust pre{\hypertarget{ref-bassonEstablishmentCommunity1996}{}}%
Basson, N. J., \& van Wyk, C. W. (1996). The establishment of a
community of oral bacteria that controls the growth of {Candida}
albicans in a chemostat. \emph{Oral Microbiology and Immunology},
\emph{11}(3), 199--202.
\url{https://doi.org/10.1111/j.1399-302X.1996.tb00358.x}

\leavevmode\vadjust pre{\hypertarget{ref-bosPhysicochemistryInitial1999}{}}%
Bos, R. (1999). Physico-chemistry of initial microbial adhesive
interactions -- its mechanisms and methods for study. \emph{FEMS
Microbiology Reviews}, \emph{23}(2), 179--229.
\url{https://doi.org/10.1016/S0168-6445(99)00004-2}

\leavevmode\vadjust pre{\hypertarget{ref-bowenOralBiofilms2018}{}}%
Bowen, W. H., Burne, R. A., Wu, H., \& Koo, H. (2018). Oral {Biofilms}:
{Pathogens}, {Matrix} and {Polymicrobial Interactions} in
{Microenvironments}. \emph{Trends in Microbiology}, \emph{26}(3),
229--242. \url{https://doi.org/10.1016/j.tim.2017.09.008}

\leavevmode\vadjust pre{\hypertarget{ref-boyan-salyersRelationshipProteolipids1980}{}}%
Boyan-Salyers, B. D., \& Boskey, A. L. (1980). Relationship between
proteolipids and calcium-phospholipid-phosphate complexes
{inBacterionema} matruchotii calcification. \emph{Calcified Tissue
International}, \emph{30}(1), 167--174.
\url{https://doi.org/10.1007/BF02408622}

\leavevmode\vadjust pre{\hypertarget{ref-ceriCalgaryBiofilm1999}{}}%
Ceri, H., Olson, M. E., Stremick, C., Read, R. R., Morck, D., \& Buret,
A. (1999). The {Calgary Biofilm Device}: {New Technology} for {Rapid
Determination} of {Antibiotic Susceptibilities} of {Bacterial Biofilms}.
\emph{Journal of Clinical Microbiology}, \emph{37}(6), 1771--1776.
\url{https://doi.org/10.1128/JCM.37.6.1771-1776.1999}

\leavevmode\vadjust pre{\hypertarget{ref-chenSpecificGenes1999}{}}%
Chen, P., Qi, F., Novak, J., \& Caufield, P. W. (1999). The {Specific
Genes} for {Lantibiotic Mutacin II Biosynthesis} in {Streptococcus}
mutans {T8 Are Clustered} and {Can Be} {Transferred En Bloc}.
\emph{Applied and Environmental Microbiology}, \emph{65}(3), 1356--1360.
\url{https://www.ncbi.nlm.nih.gov/pmc/articles/PMC91190/}

\leavevmode\vadjust pre{\hypertarget{ref-costertonBacterialBiofilms1987}{}}%
Costerton, J. W., Cheng, K. J., Geesey, G. G., Ladd, T. I., Nickel, J.
C., Dasgupta, M., \& Marrie, T. J. (1987). Bacterial {Biofilms} in
{Nature} and {Disease}. \emph{Annual Review of Microbiology},
\emph{41}(1), 435--464.
\url{https://doi.org/10.1146/annurev.mi.41.100187.002251}

\leavevmode\vadjust pre{\hypertarget{ref-costertonMicrobialBiofilms1995}{}}%
Costerton, J. William, Lewandowski, Z., Caldwell, D. E., Korber, D. R.,
\& Lappin-Scott, H. M. (1995). Microbial {Biofilms}. \emph{Annual Review
of Microbiology}, \emph{49}(1), 711--745.
\url{https://doi.org/10.1146/annurev.mi.49.100195.003431}

\leavevmode\vadjust pre{\hypertarget{ref-curtisRoleMicrobiota2020}{}}%
Curtis, M. A., Diaz, P. I., \& Dyke, T. E. V. (2020). The role of the
microbiota in periodontal disease. \emph{Periodontology 2000},
\emph{83}(1), 14--25. \url{https://doi.org/10.1111/prd.12296}

\leavevmode\vadjust pre{\hypertarget{ref-dahlenMicrobiologicalStudy2010}{}}%
Dahlén, G., Konradsson, K., Eriksson, S., Teanpaisan, R., Piwat, S., \&
Carlén, A. (2010). A microbiological study in relation to the presence
of caries and calculus. \emph{Acta Odontologica Scandinavica},
\emph{68}(4), 199--206. \url{https://doi.org/10.3109/00016351003745514}

\leavevmode\vadjust pre{\hypertarget{ref-dawBacteriocinsNature1996}{}}%
Daw, M. A., \& Falkiner, F. R. (1996). Bacteriocins: Nature, function
and structure. \emph{Micron (Oxford, England: 1993)}, \emph{27}(6),
467--479. \url{https://doi.org/10.1016/s0968-4328(96)00028-5}

\leavevmode\vadjust pre{\hypertarget{ref-dawesEffectsDiet1970}{}}%
Dawes, C. (1970). Effects of {Diet} on {Salivary Secretion} and
{Composition}. \emph{Journal of Dental Research}, \emph{49}, 1263--1272.

\leavevmode\vadjust pre{\hypertarget{ref-dibdinDiffusionSugars1981}{}}%
Dibdin, G. H. (1981). Diffusion of sugars and carboxylic acids through
human dental plaque in vitro. \emph{Archives of Oral Biology},
\emph{26}(6), 515--523.
\url{https://doi.org/10.1016/0003-9969(81)90010-8}

\leavevmode\vadjust pre{\hypertarget{ref-doddsHealthBenefits2005}{}}%
Dodds, M. W. J., Johnson, D. A., \& Yeh, C.-K. (2005). Health benefits
of saliva: A review. \emph{Journal of Dentistry}, \emph{33}(3),
223--233. \url{https://doi.org/10.1016/j.jdent.2004.10.009}

\leavevmode\vadjust pre{\hypertarget{ref-duarteInfluencesStarch2008}{}}%
Duarte, S., Klein, M. I., Aires, C. P., Cury, J. A., Bowen, W. H., \&
Koo, H. (2008). Influences of starch and sucrose on {Streptococcus}
mutans biofilms. \emph{Oral Microbiology and Immunology}, \emph{23}(3),
206--212. \url{https://doi.org/10.1111/j.1399-302X.2007.00412.x}

\leavevmode\vadjust pre{\hypertarget{ref-edlundUncoveringComplex2018}{}}%
Edlund, A., Yang, Y., Yooseph, S., He, X., Shi, W., \& McLean, J. S.
(2018). Uncovering complex microbiome activities via metatranscriptomics
during 24 hours of oral biofilm assembly and maturation.
\emph{Microbiome}, \emph{6}(1), 217.
\url{https://doi.org/10.1186/s40168-018-0591-4}

\leavevmode\vadjust pre{\hypertarget{ref-enneverIntracellularCalcification1960}{}}%
Ennever, J. (1960). Intracellular {Calcification} by {Oral Filamentous
Microorganisms}. \emph{The Journal of Periodontology}, \emph{31}(4),
304--307. \url{https://doi.org/10.1902/jop.1960.31.4.304}

\leavevmode\vadjust pre{\hypertarget{ref-enneverMicrobiologicCalcification1967}{}}%
Ennever, J., \& Creamer, H. (1967). Microbiologic calcification: {Bone}
mineral and bacteria. \emph{Calcified Tissue Research}, \emph{1}(1),
87--93. \url{https://doi.org/10.1007/BF02008078}

\leavevmode\vadjust pre{\hypertarget{ref-extercateAAA2010}{}}%
Exterkate, R. A. M., Crielaard, W., \& Ten Cate, J. M. (2010). Different
{Response} to {Amine Fluoride} by {Streptococcus} mutans and
{Polymicrobial Biofilms} in a {Novel High-Throughput Active Attachment
Model}. \emph{Caries Research}, \emph{44}(4), 372--379.
\url{https://doi.org/10.1159/000316541}

\leavevmode\vadjust pre{\hypertarget{ref-filocheFluorescenceAssay2007}{}}%
Filoche, S. K., Coleman, M. J., Angker, L., \& Sissons, C. H. (2007). A
fluorescence assay to determine the viable biomass of microcosm dental
plaque biofilms. \emph{Journal of Microbiological Methods},
\emph{69}(3), 489--496.
\url{https://doi.org/10.1016/j.mimet.2007.02.015}

\leavevmode\vadjust pre{\hypertarget{ref-flemmingBiofilmsEmergent2016}{}}%
Flemming, H.-C., Wingender, J., Szewzyk, U., Steinberg, P., Rice, S. A.,
\& Kjelleberg, S. (2016). Biofilms: An emergent form of bacterial life.
\emph{Nature Reviews Microbiology}, \emph{14}(9), 563--575.
\url{https://doi.org/10.1038/nrmicro.2016.94}

\leavevmode\vadjust pre{\hypertarget{ref-friskoppUltrastructureNondecalcified1983}{}}%
Friskopp, J. (1983). Ultrastructure of {Nondecalcified Supragingival}
and {Subgingival Calculus}. \emph{Journal of Periodontology},
\emph{54}(9), 542--550. \url{https://doi.org/10.1902/jop.1983.54.9.542}

\leavevmode\vadjust pre{\hypertarget{ref-friskoppComparativeScanning1980}{}}%
Friskopp, J., \& Hammarström, L. (1980). A {Comparative}, {Scanning
Electron Microscopic Study} of {Supragingival} and {Subgingival
Calculus}. \emph{Journal of Periodontology}, \emph{51}(10), 553--562.
\url{https://doi.org/10.1902/jop.1980.51.10.553}

\leavevmode\vadjust pre{\hypertarget{ref-glasBiophysicalStudies1962}{}}%
Glas, J.-E., \& Krasse, B. (1962). Biophysical {Studies} on {Dental
Calculus} from {Germfree} and {Conventional Rats}. \emph{Acta
Odontologica Scandinavica}, \emph{20}(2), 127--134.
\url{https://doi.org/10.3109/00016356209026100}

\leavevmode\vadjust pre{\hypertarget{ref-grahamEnterococcusFaecalis2017}{}}%
Graham, C. E., Cruz, M. R., Garsin, D. A., \& Lorenz, M. C. (2017).
Enterococcus faecalis bacteriocin {EntV} inhibits hyphal morphogenesis,
biofilm formation, and virulence of {Candida} albicans.
\emph{Proceedings of the National Academy of Sciences}, \emph{114}(17),
4507--4512. \url{https://doi.org/10.1073/pnas.1620432114}

\leavevmode\vadjust pre{\hypertarget{ref-honraetModifiedRobbins2006}{}}%
Honraet, K., \& Nelis, H. J. (2006). Use of the modified robbins device
and fluorescent staining to screen plant extracts for the inhibition of
{S}. Mutans biofilm formation. \emph{Journal of Microbiological
Methods}, \emph{64}(2), 217--224.
\url{https://doi.org/10.1016/j.mimet.2005.05.005}

\leavevmode\vadjust pre{\hypertarget{ref-huangFactorsAssociated2012}{}}%
Huang, X., Exterkate, R. A. M., \& ten Cate, J. M. (2012). Factors
{Associated} with {Alkali Production} from {Arginine} in {Dental
Biofilms}. \emph{Journal of Dental Research}, \emph{91}(12), 1130--1134.
\url{https://doi.org/10.1177/0022034512461652}

\leavevmode\vadjust pre{\hypertarget{ref-huangEffectArginine2017}{}}%
Huang, Xuelian, Zhang, K., Deng, M., Exterkate, R. A. M., Liu, C., Zhou,
X., Cheng, L., \& ten Cate, J. M. (2017). Effect of arginine on the
growth and biofilm formation of oral bacteria. \emph{Archives of Oral
Biology}, \emph{82}, 256--262.
\url{https://doi.org/10.1016/j.archoralbio.2017.06.026}

\leavevmode\vadjust pre{\hypertarget{ref-jinSupragingivalCalculus2002}{}}%
Jin, Y., \& Yip, H.-K. (2002). Supragingival {Calculus}: {Formation} and
{Control}. \emph{Critical Reviews in Oral Biology \& Medicine}.
\url{https://doi.org/10.1177/154411130201300506}

\leavevmode\vadjust pre{\hypertarget{ref-kearnsMasterRegulator2005}{}}%
Kearns, D. B., Chu, F., Branda, S. S., Kolter, R., \& Losick, R. (2005).
A master regulator for biofilm formation by {Bacillus} subtilis.
\emph{Molecular Microbiology}, \emph{55}(3), 739--749.
\url{https://doi.org/10.1111/j.1365-2958.2004.04440.x}

\leavevmode\vadjust pre{\hypertarget{ref-kolenbranderAdhereToday1993}{}}%
Kolenbrander, P. E., \& London, J. (1993). Adhere today, here tomorrow:
Oral bacterial adherence. \emph{Journal of Bacteriology},
\emph{175}(11), 3247--3252.
\url{https://doi.org/10.1128/jb.175.11.3247-3252.1993}

\leavevmode\vadjust pre{\hypertarget{ref-kolenbranderOralMultispecies2010}{}}%
Kolenbrander, P. E., Palmer, R. J., Periasamy, S., \& Jakubovics, N. S.
(2010). Oral multispecies biofilm development and the key role of
cell--cell distance. \emph{Nature Reviews Microbiology}, \emph{8}(7),
471--480. \url{https://doi.org/10.1038/nrmicro2381}

\leavevmode\vadjust pre{\hypertarget{ref-lieverseDietAetiology1999}{}}%
Lieverse, A. R. (1999). Diet and the aetiology of dental calculus.
\emph{International Journal of Osteoarchaeology}, \emph{9}(4), 219--232.
\url{https://doi.org/10.1002/(SICI)1099-1212(199907/08)9:4\%3C219::AID-OA475\%3E3.0.CO;2-V}

\leavevmode\vadjust pre{\hypertarget{ref-marshRoleMicrobiology1995}{}}%
Marsh, P. D. (1995). The {Role} of {Microbiology} in {Models} of {Dental
Caries}. \emph{Advances in Dental Research}, \emph{9}(3), 244--254.
\url{https://doi.org/10.1177/08959374950090030901}

\leavevmode\vadjust pre{\hypertarget{ref-marshDentalPlaque2005}{}}%
Marsh, Philip D. (2005). Dental plaque: Biological significance of a
biofilm and community life-style. \emph{Journal of Clinical
Periodontology}, \emph{32}(s6), 7--15.
\url{https://doi.org/10.1111/j.1600-051X.2005.00790.x}

\leavevmode\vadjust pre{\hypertarget{ref-marshDentalPlaque2006}{}}%
Marsh, Philip D. (2006). Dental plaque as a biofilm and a microbial
community -- implications for health and disease. \emph{BMC Oral
Health}, \emph{6}(S1), S14.
\url{https://doi.org/10.1186/1472-6831-6-S1-S14}

\leavevmode\vadjust pre{\hypertarget{ref-marshMicrobiologyDental2010}{}}%
Marsh, Philip D. (2010). Microbiology of {Dental Plaque Biofilms} and
{Their Role} in {Oral Health} and {Caries}. \emph{Dental Clinics of
North America}, \emph{54}(3), 441--454.
\url{https://doi.org/10.1016/j.cden.2010.03.002}

\leavevmode\vadjust pre{\hypertarget{ref-marshPhysiologicalApproaches1997}{}}%
Marsh, Philip D., \& Bradshaw, D. J. (1997). Physiological {Approaches}
to the {Control} of {Oral Biofilms}. \emph{Advances in Dental Research},
\emph{11}(1), 176--185.
\url{https://doi.org/10.1177/08959374970110010901}

\leavevmode\vadjust pre{\hypertarget{ref-marshDentalPlaque2016}{}}%
Marsh, Philip D., Lewis, M. A. O., Rogers, H., Williams, D. W., \&
Wilson, M. (2016). Dental {Plaque}. In \emph{Marsh and {Martin}'s {Oral
Microbiology}} (6th Edition, pp. 81--111). {Elsevier Health Sciences}.

\leavevmode\vadjust pre{\hypertarget{ref-mcbainBiofilmModels2009}{}}%
McBain, A. J. (2009). In {Vitro Biofilm Models}: {An Overview}. In
\emph{Advances in {Applied Microbiology}} (Vol. 69, pp. 99--132).
{Academic Press}. \url{https://doi.org/10.1016/S0065-2164(09)69004-3}

\leavevmode\vadjust pre{\hypertarget{ref-moorerCalcificationCariogenic1993}{}}%
Moorer, W. R., Ten Cate, J. M., \& Buijs, J. F. (1993). Calcification of
a {Cariogenic Streptococcus} and of {Corynebacterium} ({Bacterionema})
matruchotii. \emph{Journal of Dental Research}, \emph{72}(6),
1021--1026. \url{https://doi.org/10.1177/00220345930720060501}

\leavevmode\vadjust pre{\hypertarget{ref-nikitkovaStarchBiofilms2013}{}}%
Nikitkova, A. E., Haase, E. M., \& Scannapieco, F. A. (2013). Taking the
{Starch} out of {Oral Biofilm Formation}: {Molecular Basis} and
{Functional Significance} of {Salivary} α-{Amylase Binding} to {Oral
Streptococci}. \emph{Applied and Environmental Microbiology},
\emph{79}(2), 416--423. \url{https://doi.org/10.1128/AEM.02581-12}

\leavevmode\vadjust pre{\hypertarget{ref-nobbsStreptococcusAdherence2009}{}}%
Nobbs, A. H., Lamont, R. J., \& Jenkinson, H. F. (2009). Streptococcus
{Adherence} and {Colonization}. \emph{Microbiology and Molecular Biology
Reviews}, \emph{73}(3), 407--450.
\url{https://doi.org/10.1128/MMBR.00014-09}

\leavevmode\vadjust pre{\hypertarget{ref-omelonReviewPhosphate2013}{}}%
Omelon, S., Ariganello, M., Bonucci, E., Grynpas, M., \& Nanci, A.
(2013). A {Review} of {Phosphate Mineral Nucleation} in {Biology} and
{Geobiology}. \emph{Calcified Tissue International}, \emph{93}(4),
382--396. \url{https://doi.org/10.1007/s00223-013-9784-9}

\leavevmode\vadjust pre{\hypertarget{ref-palmerCoaggregationInteractions2003}{}}%
Palmer, Robert J., Jr., Gordon, S. M., Cisar, J. O., \& Kolenbrander, P.
E. (2003). Coaggregation-{Mediated Interactions} of {Streptococci} and
{Actinomyces Detected} in {Initial Human Dental Plaque}. \emph{Journal
of Bacteriology}, \emph{185}(11), 3400--3409.
\url{https://doi.org/10.1128/JB.185.11.3400-3409.2003}

\leavevmode\vadjust pre{\hypertarget{ref-palmerInterbacterialAdhesion2017}{}}%
Palmer, Robert J., Shah, N., Valm, A., Paster, B., Dewhirst, F., Inui,
T., \& Cisar, J. O. (2017). Interbacterial {Adhesion Networks} within
{Early Oral Biofilms} of {Single Human Hosts}. \emph{Applied and
Environmental Microbiology}, \emph{83}(11), e00407--17.
\url{https://doi.org/10.1128/AEM.00407-17}

\leavevmode\vadjust pre{\hypertarget{ref-petersConstantDepth1988}{}}%
Peters, A., \& Wimpenny, J. W. T. (1988). A {Constant-Depth Laboratory
Model Film Fermenter}. In \emph{{CRC Handbook} of {Laboratory Model
Systems} for {Microbial Ecosystems}}. {CRC Press}.

\leavevmode\vadjust pre{\hypertarget{ref-petersonViscoelasticityBiofilms2015}{}}%
Peterson, B. W., He, Y., Ren, Y., Zerdoum, A., Libera, M. R., Sharma, P.
K., van Winkelhoff, A.-J., Neut, D., Stoodley, P., van der Mei, H. C.,
\& Busscher, H. J. (2015). Viscoelasticity of biofilms and their
recalcitrance to mechanical and chemical challenges. \emph{FEMS
Microbiology Reviews}, \emph{39}(2), 234--245.
\url{https://doi.org/10.1093/femsre/fuu008}

\leavevmode\vadjust pre{\hypertarget{ref-petrovaEscapingBiofilm2016}{}}%
Petrova, O. E., \& Sauer, K. (2016). Escaping the biofilm in more than
one way: Desorption, detachment or dispersion. \emph{Current Opinion in
Microbiology}, \emph{30}, 67--78.
\url{https://doi.org/10.1016/j.mib.2016.01.004}

\leavevmode\vadjust pre{\hypertarget{ref-prattenVitroStudies1998}{}}%
Pratten, Wills, Barnett, \& Wilson. (1998). In vitro studies of the
effect of antiseptic-containing mouthwashes on the formation and
viability of {Streptococcus} sanguis biofilms. \emph{Journal of Applied
Microbiology}, \emph{84}(6), 1149--1155.
\url{https://doi.org/10.1046/j.1365-2672.1998.00462.x}

\leavevmode\vadjust pre{\hypertarget{ref-renduelesMechanismsCompetition2015}{}}%
Rendueles, O., \& Ghigo, J.-M. (2015). Mechanisms of {Competition} in
{Biofilm Communities}. \emph{Microbiology Spectrum}, \emph{3}(3),
3.3.28. \url{https://doi.org/10.1128/microbiolspec.MB-0009-2014}

\leavevmode\vadjust pre{\hypertarget{ref-rennerPhysicochemicalRegulation2011}{}}%
Renner, L. D., \& Weibel, D. B. (2011). Physicochemical regulation of
biofilm formation. \emph{MRS Bulletin}, \emph{36}(5), 347--355.
\url{https://doi.org/10.1557/mrs.2011.65}

\leavevmode\vadjust pre{\hypertarget{ref-roderStudyingBacterial2016}{}}%
Røder, H. L., Sørensen, S. J., \& Burmølle, M. (2016). Studying
{Bacterial Multispecies Biofilms}: {Where} to {Start}? \emph{Trends in
Microbiology}, \emph{24}(6), 503--513.
\url{https://doi.org/10.1016/j.tim.2016.02.019}

\leavevmode\vadjust pre{\hypertarget{ref-shellisSyntheticSaliva1978}{}}%
Shellis, R. P. (1978). A synthetic saliva for cultural studies of dental
plaque. \emph{Archives of Oral Biology}, \emph{23}(6), 485--489.
\url{https://doi.org/10.1016/0003-9969(78)90081-X}

\leavevmode\vadjust pre{\hypertarget{ref-sidawayMicrobiologicalStudy1978a}{}}%
Sidaway, D. A. (1978). A microbiological study of dental calculus.
\emph{Journal of Periodontal Research}, \emph{13}(4), 360--366.
\url{https://doi.org/10.1111/j.1600-0765.1978.tb00190.x}

\leavevmode\vadjust pre{\hypertarget{ref-sissonsArtificialPlaque1997}{}}%
Sissons, C. H. (1997). Artificial {Dental Plaque Biofilm Model Systems}.
\emph{Advances in Dental Research}, \emph{11}(1), 110--126.
\url{https://doi.org/10.1177/08959374970110010201}

\leavevmode\vadjust pre{\hypertarget{ref-sissonsMultistationPlaque1991}{}}%
Sissons, C. H., Cutress, T. W., Hoffman, M. P., \& Wakefield, J. S. J.
(1991). A {Multi-station Dental Plaque Microcosm} ({Artificial Mouth})
for the {Study} of {Plaque Growth}, {Metabolism}, {pH}, and
{Mineralization}: \emph{Journal of Dental Research}.
\url{https://doi.org/10.1177/00220345910700110301}

\leavevmode\vadjust pre{\hypertarget{ref-sissonsPHResponse1994}{}}%
Sissons, C. H., Wong, L., Hancock, E. M., \& Cutress, T. W. (1994). The
{pH} response to urea and the effect of liquid flow in {``artificial
mouth''} microcosm plaques. \emph{Archives of Oral Biology},
\emph{39}(6), 497--505.
\url{https://doi.org/10.1016/0003-9969(94)90146-5}

\leavevmode\vadjust pre{\hypertarget{ref-songEffectsMaterial2015}{}}%
Song, F., Koo, H., \& Ren, D. (2015). Effects of {Material Properties}
on {Bacterial Adhesion} and {Biofilm Formation}. \emph{Journal of Dental
Research}, \emph{94}(8), 1027--1034.
\url{https://doi.org/10.1177/0022034515587690}

\leavevmode\vadjust pre{\hypertarget{ref-takazoeCalciumHydroxyapatite1970}{}}%
Takazoe, I., Vogel, J., \& Ennever, J. (1970). Calcium {Hydroxyapatite
Nucleation} by {Lipid Extract} of {Bacterionema} matruchotii.
\emph{Journal of Dental Research}, \emph{49}(2), 395--398.
\url{https://doi.org/10.1177/00220345700490023301}

\leavevmode\vadjust pre{\hypertarget{ref-tanBacterialViability2004}{}}%
Tan, B. T. K., Mordan, N. J., Embleton, J., Pratten, J., \& Galgut, P.
N. (2004). Study of {Bacterial Viability} within {Human Supragingival
Dental Calculus}. \emph{Journal of Periodontology}, \emph{75}(1),
23--29. \url{https://doi.org/10.1902/jop.2004.75.1.23}

\leavevmode\vadjust pre{\hypertarget{ref-tanAllTogether2017}{}}%
Tan, C. H., Lee, K. W. K., Burmølle, M., Kjelleberg, S., \& Rice, S. A.
(2017). All together now: Experimental multispecies biofilm model
systems. \emph{Environmental Microbiology}, \emph{19}(1), 42--53.
\url{https://doi.org/10.1111/1462-2920.13594}

\leavevmode\vadjust pre{\hypertarget{ref-theiladeGermfreeCalculus1964}{}}%
Theilade, J., Fitzgerald, R. J., Scott, D. B., \& Nylen, M. U. (1964).
Electron microscopic observations of dental calculus in germfree and
conventional rats. \emph{Archives of Oral Biology}, \emph{9}(1),
97--IN17. \url{https://doi.org/10.1016/0003-9969(64)90051-2}

\leavevmode\vadjust pre{\hypertarget{ref-tianUsingDGGE2010}{}}%
Tian, Y., He, X., Torralba, M., Yooseph, S., Nelson, K. e., Lux, R.,
McLean, J. s., Yu, G., \& Shi, W. (2010). Using {DGGE} profiling to
develop a novel culture medium suitable for oral microbial communities.
\emph{Molecular Oral Microbiology}, \emph{25}(5), 357--367.
\url{https://doi.org/10.1111/j.2041-1014.2010.00585.x}

\leavevmode\vadjust pre{\hypertarget{ref-uzelMicrobialShifts2011}{}}%
Uzel, N. G., Teles, F. R., Teles, R. P., Song, X. Q., Torresyap, G.,
Socransky, S. S., \& Haffajee, A. D. (2011). Microbial shifts during
dental biofilm re-development in the absence of oral hygiene in
periodontal health and disease. \emph{Journal of Clinical
Periodontology}, \emph{38}(7), 612--620.
\url{https://doi.org/10.1111/j.1600-051X.2011.01730.x}

\leavevmode\vadjust pre{\hypertarget{ref-vigeantReversibleIrreversible2002}{}}%
Vigeant, M. A.-S., Ford, R. M., Wagner, M., \& Tamm, L. K. (2002).
Reversible and {Irreversible Adhesion} of {Motile Escherichia} coli
{Cells Analyzed} by {Total Internal Reflection Aqueous Fluorescence
Microscopy}. \emph{Applied and Environmental Microbiology},
\emph{68}(6), 2794--2801.
\url{https://doi.org/10.1128/AEM.68.6.2794-2801.2002}

\leavevmode\vadjust pre{\hypertarget{ref-whiteDentalCalculus1997}{}}%
White, D. J. (1997). Dental calculus: Recent insights into occurrence,
formation, prevention, removal and oral health effects of supragingival
and subgingival deposits. \emph{European Journal of Oral Sciences},
\emph{105}(5), 508--522.
\url{https://doi.org/10.1111/j.1600-0722.1997.tb00238.x}

\leavevmode\vadjust pre{\hypertarget{ref-wongCalciumPhosphate2002}{}}%
Wong, L., Sissons, C. H., Pearce, E. I. F., \& Cutress, T. W. (2002).
Calcium phosphate deposition in human dental plaque microcosm biofilms
induced by a ureolytic {pH-rise} procedure. \emph{Archives of Oral
Biology}, \emph{47}(11), 779--790.
\url{https://doi.org/10.1016/S0003-9969(02)00114-0}

\leavevmode\vadjust pre{\hypertarget{ref-yaoIdentificationProtein2003}{}}%
Yao, Y., Berg, E. A., Costello, C. E., Troxler, R. F., \& Oppenheim, F.
G. (2003). Identification of protein components in human acquired enamel
pellicle and whole saliva using novel proteomics approaches. \emph{J
Biol Chem}, \emph{278}(7), 5300--5308.
\url{https://doi.org/10.1074/jbc.M206333200}

\leavevmode\vadjust pre{\hypertarget{ref-zeroSituCaries1995}{}}%
Zero, D. T. (1995). In {Situ Caries Models}. \emph{Advances in Dental
Research}, \emph{9}(3), 214--230.
\url{https://doi.org/10.1177/08959374950090030501}

\leavevmode\vadjust pre{\hypertarget{ref-zijngeBiofilmArchitecture2010}{}}%
Zijnge, V., van Leeuwen, M. B. M., Degener, J. E., Abbas, F., Thurnheer,
T., Gmür, R., \& M. Harmsen, H. J. (2010). Oral {Biofilm Architecture}
on {Natural Teeth}. \emph{PLoS ONE}, \emph{5}(2), e9321.
\url{https://doi.org/10.1371/journal.pone.0009321}

\end{CSLReferences}

\bookmarksetup{startatroot}

\hypertarget{byoc-valid}{%
\chapter{Assessing the validity of a calcifying oral biofilm model as a
suitable proxy for dental calculus}\label{byoc-valid}}

\hypertarget{introduction}{%
\section{Introduction}\label{introduction}}

Dental calculus is becoming an increasingly popular substance for
exploring health and diet in past populations
(\protect\hyperlink{ref-warinnerNewEra2015}{Warinner et al., 2015}).
During life, dental plaque undergoes periodic mineralisation, trapping
biomolecules and microfossils that are embedded within the dental plaque
biofilm in the newly-formed dental calculus. This process is repeated as
new plaque is deposited and subsequently mineralises, resulting in a
layered structure representing a temporal record of biofilm growth and
development (\protect\hyperlink{ref-warinnerPathogensHost2014}{Warinner,
Rodrigues, et al., 2014}). The calculus serves as a protective casing
for the entrapped biomolecules and microfossils, preserving them for
thousands of years after death and burial
(\protect\hyperlink{ref-yatesOralMicrobiome2021}{Fellows Yates et al.,
2021}). Studies using archaeological dental calculus span a wide range
of topics in different regions and time periods. These include
characterisation of the oral microbiome and its evolution in past
populations (\protect\hyperlink{ref-adlerSequencingAncient2013}{Adler et
al., 2013}; \protect\hyperlink{ref-yatesOralMicrobiome2021}{Fellows
Yates et al., 2021};
\protect\hyperlink{ref-kazarinaPostmedievalMicrobial2021}{Kazarina et
al., 2021};
\protect\hyperlink{ref-velskoMicrobialDifferences2019}{Velsko et al.,
2019}; \protect\hyperlink{ref-warinnerPathogensHost2014}{Warinner,
Rodrigues, et al., 2014}), as well as extraction of microbotanical
remains (\protect\hyperlink{ref-hardyStarchGranules2009}{Hardy et al.,
2009}; \protect\hyperlink{ref-henryCalculusSyria2008}{Henry \& Piperno,
2008}; \protect\hyperlink{ref-maHumanDiet2022}{Z. Ma et al., 2022};
\protect\hyperlink{ref-mickleburghNewInsights2012}{Mickleburgh \&
Pagán-Jiménez, 2012}) and other residues to infer dietary patterns and
nicotine use
(\protect\hyperlink{ref-bartholdyMultiproxyAnalysis2023}{Bartholdy,
Hasselstrøm, et al., 2023};
\protect\hyperlink{ref-buckleyDentalCalculus2014}{Buckley et al., 2014};
\protect\hyperlink{ref-eerkensDentalCalculus2018}{Eerkens et al., 2018};
\protect\hyperlink{ref-hendyProteomicCalculus2018}{Hendy et al., 2018};
\protect\hyperlink{ref-velskoDentalCalculus2017}{Velsko, Overmyer, et
al., 2017}). Dental calculus has already provided a unique and valuable
insight into the past, but the exact mechanism of the incorporation,
retention, and preservation of microfossils and biomolecules exogenous
to the microbial biofilm is largely unknown; even the process of plaque
mineralisation is not fully understood
(\protect\hyperlink{ref-jinSupragingivalCalculus2002}{Jin \& Yip, 2002};
\protect\hyperlink{ref-omelonReviewPhosphate2013}{Omelon et al., 2013}).
This means that there may be hidden biases affecting our interpretations
of dietary/activity patterns extrapolated from ancient dental calculus.
These biases have been explored archaeologically
(\protect\hyperlink{ref-fagernasMicrobialBiogeography2022}{Fagernäs et
al., 2022}; \protect\hyperlink{ref-trompEDTACalculus2017}{Tromp et al.,
2017}) as well as in contemporary humans
(\protect\hyperlink{ref-leonardPlantMicroremains2015}{Leonard et al.,
2015}) and non-human primates
(\protect\hyperlink{ref-powerChimpCalculus2015}{R. C. Power et al.,
2015}), but not experimentally.

Dental plaque is an oral biofilm and is part of the normal state of the
oral cavity. However, when left unchecked, plaque can lead to
infections, such as dental caries and periodontitis, and/or
mineralisation (\protect\hyperlink{ref-marshDentalPlaque2006}{Philip D.
Marsh, 2006}). The dental plaque biofilm grows in a well-characterized
manner before mineralisation, in a process that repeats regularly to
build up dental calculus. Shortly after teeth are cleaned (whether
mechanically or otherwise), salivary components adsorb to the crown or
root and form the acquired dental pellicle. The pellicle provides a
viable surface for bacteria to attach, especially early-coloniser
species within the genera \emph{Streptococcus} and \emph{Actinomyces}
(\protect\hyperlink{ref-marshDentalPlaque2006}{Philip D. Marsh, 2006}).
Once the tooth surface has been populated by specialists in
surface-attachment, other species of bacteria can attach to the adherent
cells, increasing the biofilm density and diversity. The bacterial
species secrete polysaccharides, proteins, lipids, and nucleic acids,
into their immediate environment to form a matrix that provides
structural support, nutrition, and allows for environmental niche
partitioning
(\protect\hyperlink{ref-flemmingBiofilmsEmergent2016}{Flemming et al.,
2016}).

Biofilms can become susceptible to calcification under certain
microenvironmental conditions, including an increased concentration of
salts and a decrease in statherin and proline-rich proteins in saliva,
rises in local plaque pH, and increased hydrolysis of urea
(\protect\hyperlink{ref-whiteDentalCalculus1997}{D. J. White, 1997};
\protect\hyperlink{ref-wongCalciumPhosphate2002}{Wong et al., 2002}).
These conditions can cause increased precipitation and decreased
dissolution of calcium phosphate salts within saliva and the plaque
biofilm. The resulting supersaturation of calcium phosphate salts is the
main driver of biofilm mineralisation
(\protect\hyperlink{ref-jinSupragingivalCalculus2002}{Jin \& Yip,
2002}). The primary minerals in dental calculus are hydroxyapatite,
octacalcium phosphate, whitlockite, and brushite. During initial
mineralisation the main mineral component is brushite, which shifts to
hydroxyapatite in more mature dental calculus
(\protect\hyperlink{ref-hayashizakiSiteSpecific2008}{Hayashizaki et al.,
2008}; \protect\hyperlink{ref-jinSupragingivalCalculus2002}{Jin \& Yip,
2002}). The exact elemental composition of dental calculus varies among
individuals due to various factors, including diet
(\protect\hyperlink{ref-hayashizakiSiteSpecific2008}{Hayashizaki et al.,
2008}; \protect\hyperlink{ref-jiFluorideMagnesium2000}{Ji et al.,
2000}).

Dental plaque can also be grown \emph{in vitro}, and these oral biofilm
models are commonly used in dental research to assess the efficacy of
certain treatments on dental pathogens
(\protect\hyperlink{ref-extercateAAA2010}{Exterkate et al., 2010};
\protect\hyperlink{ref-filochePlaqueMicrocosm2007}{S. K. Filoche et al.,
2007}) without the ethical issues of inducing plaque accumulation in
study participants and the complexity of access and sampling in humans
or animals. Oral biofilm models are often short-term models grown over a
few days, but longer term models also exist (up to six weeks) which are
used to develop mature plaque or dental calculus
(\protect\hyperlink{ref-middletonVitroCalculus1965}{J. D. Middleton,
1965}; \protect\hyperlink{ref-sissonsMultistationPlaque1991}{Sissons et
al., 1991};
\protect\hyperlink{ref-velskoConsistentReproducible2018}{Velsko \&
Shaddox, 2018}; \protect\hyperlink{ref-wongCalciumPhosphate2002}{Wong et
al., 2002}). A well-known limitation of biofilm models is the difficulty
in capturing the diversity and complexity of bacterial communities and
metabolic dependencies, micro-environments, nutrient availability, and
host immune-responses in the natural oral biome
(\protect\hyperlink{ref-bjarnsholtVivoBiofilm2013}{Bjarnsholt et al.,
2013}; \protect\hyperlink{ref-edlundUncoveringComplex2018}{Edlund et
al., 2018}; \protect\hyperlink{ref-velskoCytokineResponse2017}{Velsko,
Cruz-Almeida, et al., 2017};
\protect\hyperlink{ref-velskoConsistentReproducible2018}{Velsko \&
Shaddox, 2018}). These limitations can be overcome by complex
experimental setups, but at the cost of lower throughput and increased
requirements for laboratory facilities.

Despite the limitations, oral biofilm models have many benefits over
\emph{in situ} research. There are many variables involved in dental
calculus formation, such as intra- and inter-individual variation in
salivary flow, oral pH, and amylase activity, which can be hard to tease
apart \emph{in situ}. Oral biofilm models provide a controlled
environment to explore the effect of selected variables on the growth of
calculus and the retention of dietary components in the biofilm, as well
as a means to identify how the methods used in archaeology may
inadvertently bias the interpretations. This type of research has, so
far, been limited, but has the potential to greatly benefit
archaeological research on past diet
(\protect\hyperlink{ref-radiniDirtyTeeth2022}{Radini \& Nikita, 2022}).

We present an oral biofilm model that can serve as a viable proxy for
dental calculus for archaeology-oriented research questions. It is a
multispecies biofilm using whole saliva as the inoculate, with a simple
multiwell plate setup that is accessible even to smaller lab budgets and
those with limited facilities for microbiology work. Here, we used
next-generation sequencing and metagenomic classification to
characterise the bacterial composition of our model dental calculus and
compare it to oral reference samples, including saliva, buccal mucosa,
plaque, and modern human dental calculus. This was done to ensure that
the model microbiome is predominantly oral and not overgrown by
environmental contaminants. We then determined the mineral composition
of the model dental calculus using Fourier transform infrared (FTIR)
spectroscopy to verify the presence of calculus-specific mineral phases
and functional groups, and perform a qualitative comparison with modern
and archaeological reference calculus. Overall the model calculus is
chemically similar to natural calculus, and has a predominantly oral
microbiome. The microbial diversity and richness within the model
samples were lower than oral reference samples, suggesting that the
model samples do not contain identical species composition and
abundances as the natural samples. The mineral composition closely
resembles modern and archaeological reference calculus, predominantly
comprised of carbonate hydroxyapatite with a similar level of
crystallinity and order. As such, the model dental calculus presented
here is a viable proxy to natural dental calculus and can be used to
explore many of the currently unexplained processes we see in the
archaeological material, when working within the limitations of an oral
biofilm model.

\hypertarget{materials-and-methods}{%
\section{Materials and methods}\label{materials-and-methods}}

Our biofilm setup consists of whole saliva as the inoculate to
approximate natural microbial communities within the human oral cavity,
and a 24-well plate to generate multiple replicated conditions in a
single experimental run (see Figure~\ref{fig-protocol} for an overview
of the protocol). The biofilm is grown for 25 days to allow time for
growth of larger deposits and mineralisation. Raw potato and wheat
starch solutions were added during the biofilm growth to explore the
biases involved in their incorporation and extraction from dental
calculus. These results are presented in a separate article
(\protect\hyperlink{ref-bartholdyInvestigatingBiases2022}{Bartholdy \&
Henry, 2022}).

\begin{figure}

{\centering \includegraphics{figures/Exp_protocol.png}

}

\caption{\label{fig-protocol}Overview of the protocol for biofilm
growth. The samples for metagenomic analysis were grown in a separate
experimental plate than the FTIR samples under the same experimental
conditions. Biofilm (B) and calculus (C) samples were used for FTIR
spectroscopy, and saliva (S), artificial saliva (M), and calculus
samples were used for metagenomic analysis.}

\end{figure}

To determine the composition of microbial communities, we sampled the
medium from the biofilm wells over the course of the experiment. We
sequenced the DNA to identify species that are present in the model, and
assess whether these mimic natural oral communities. During a separate
experimental run, under the same conditions, we directly sampled the
biofilms on multiple days and determined the mineral composition using
FTIR, and compared the spectra to those of natural dental calculus, both
modern and archaeological. Samples were taken from both controls and
starch treatments, but differences between these samples were not
explored in this study.

\hypertarget{biofilm-growth}{%
\section{Biofilm growth}\label{biofilm-growth}}

We employ a multispecies oral biofilm model following a modified
protocol from Sissons and colleagues
(\protect\hyperlink{ref-sissonsMultistationPlaque1991}{1991}) and
Shellis (\protect\hyperlink{ref-shellisSyntheticSaliva1978}{1978}). The
setup comprises a polypropylene 24 deepwell PCR plate (KingFisher
97003510) with a lid containing 24 pegs (substrata), which are
autoclaved at 120\(^{\circ}\)C, 1 bar overpressure, for 20 mins.

The artificial saliva (hereafter referred to as medium) is a modified
version of the basal medium mucin (BMM) described by Sissons and
colleagues
(\protect\hyperlink{ref-sissonsMultistationPlaque1991}{1991}). It is a
complex medium containing 2.5 g/l partially purified mucin from porcine
stomach (Type III, Sigma M1778), 5 g/l trypticase peptone (Roth 2363.1),
10 g/l proteose peptone (Oxoid LP0085), 5 g/l yeast extract (BD 211921),
2.5 g/l KCl, 0.35 g/l NaCl, 1.8 mmol/l CaCl\textsubscript{2}, 5.2 mmol/l
Na\textsubscript{2}HPO\textsubscript{4}
(\protect\hyperlink{ref-sissonsMultistationPlaque1991}{Sissons et al.,
1991}), 6.4 mmol/l NaHCO\textsubscript{3}
(\protect\hyperlink{ref-shellisSyntheticSaliva1978}{Shellis, 1978}), 2.5
mg/l haemin. This is subsequently adjusted to pH 7 with NaOH pellets and
stirring, autoclaved (15 min, 120\(^{\circ}\)C, 1 bar overpressure), and
supplemented with 5.8 (mu)mol/l menadione, 5 mmol/l urea, and 1 mmol/l
arginine (\protect\hyperlink{ref-sissonsMultistationPlaque1991}{Sissons
et al., 1991}).

Fresh whole saliva (WS) for inoculation was provided by a 31-year-old
male donor with no history of caries, who abstained from oral hygiene
for 24 hours, and no food was consumed two hours prior to donation. No
antibiotics were taken up to six months prior to donation. Saliva was
stimulated by chewing on parafilm, then filtered through a
bleach-sterilised nylon cloth to remove particulates. Substrata were
inoculated with 1 ml/well of a two-fold dilution of WS in sterilised
20\% glycerine for four hours at 36\(^{\circ}\)C, to allow attachment of
the salivary pellicle and plaque-forming bacteria. After initial
inoculation, the substrata were transferred to a new plate containing 1
ml/well medium and incubated at 36\(^{\circ}\)C, with gentle motion at
30 rpm. The inoculation process was repeated on days 3 and 5 by
transferring the samples to a new plate with inoculate. Medium was
partially refreshed once per day, by topping up the wells to the
original volume with more medium, and fully refreshed every three days,
throughout the experiment, by transferring the substrata to a new plate
containing medium. To feed the bacteria, the substrata were transferred
to a new plate, containing 5\% (w/v) sucrose, for six minutes twice
daily, except on inoculation days (days 0, 3, and 5), where the samples
only received one sucrose treatment after inoculation.

On day 9, starch treatments were introduced, replacing sucrose
treatments (except for control sample). As with the sucrose treatments,
starch treatments occurred twice per day for six minutes, and involved
transferring the substrata to a new plate containing a 0.25\% (w/v)
starch from potato (Roth 9441.1) solution, a 0.25\% (w/v) starch from
wheat (Sigma S5127) solution, and a 0.5\% (w/v) mixture of equal
concentrations (w/v) wheat and potato. All starch solutions were created
in a 5\% (w/v) sucrose solution. Before transferring biofilm samples to
the starch treatments, the starch plates were agitated to keep the
starches in suspension in the solutions, and during treatments, the rpm
was increased to 60. The purpose of starch treatments was to explore the
incorporation of starch granules into the model calculus. Starch
treatments were initiated on day 9 (Figure~\ref{fig-protocol}) to avoid
starch granule counts being affected by \(\alpha\)-amylase hydrolysis
from the inoculation saliva. An \(\alpha\)-amylase assay conducted on
samples from days 3, 6, 8, 9, 10, 12, and 14 also showed that there was
no host salivary \(\alpha\)-amylase activity in the system. The results
of the starch incorporation and \(\alpha\)-amylase activity assay have
been reported in a separate article
(\protect\hyperlink{ref-bartholdyInvestigatingBiases2022}{Bartholdy \&
Henry, 2022}).

After 15 days, mineralisation was encouraged with a calcium phosphate
monofluorophosphate urea (CPMU) solution containing 20 mmol/l
CaCl\textsubscript{2}, 12 mmol/l
NaH\textsubscript{2}PO\textsubscript{4}, 5 mmol/l
Na\textsubscript{2}PO\textsubscript{3}F, 500 mmol/l urea
(\protect\hyperlink{ref-pearceConcomitantDeposition1987}{Pearce \&
Sissons, 1987};
\protect\hyperlink{ref-sissonsMultistationPlaque1991}{Sissons et al.,
1991}), and 0.04 g/l MgCl. The substrata were submerged in 1 ml/well
CPMU five times daily, every two hours, for six minutes, at 30 rpm.
During the mineralisation period, starch treatments were reduced to once
per day, two hours after the last CPMU treatment. This cycle was
repeated for 10 days until the end of the experiment on day 24
(Figure~\ref{fig-protocol}). More detailed protocols are available at
\url{https://dx.doi.org/10.17504/protocols.io.dm6gpj9rdgzp/v1}.

All laboratory work was conducted in sterile conditions under a laminar
flow hood to prevent starch and bacterial contamination. Starch-free
control samples that were only fed sucrose were included to detect
starch contamination.

\hypertarget{metagenomics}{%
\section{Metagenomics}\label{metagenomics}}

\hypertarget{tbl-dna-samples}{}
\begin{longtable}[]{@{}lrr@{}}
\caption{\label{tbl-dna-samples}Number of samples taken during the
experiment, separated by sampling day and sample type.}\tabularnewline
\toprule\noalign{}
Sample type & Sampling day & n \\
\midrule\noalign{}
\endfirsthead
\toprule\noalign{}
Sample type & Sampling day & n \\
\midrule\noalign{}
\endhead
\bottomrule\noalign{}
\endlastfoot
saliva & 0 & 1 \\
saliva & 3 & 1 \\
saliva & 5 & 1 \\
medium & 5 & 2 \\
medium & 7 & 2 \\
medium & 9 & 2 \\
medium & 12 & 2 \\
medium & 15 & 2 \\
medium & 18 & 2 \\
medium & 21 & 2 \\
medium & 24 & 2 \\
model\_calculus & 24 & 16 \\
\end{longtable}

A total of 35 samples were taken during the experiment from the donated
saliva, artificial saliva, and from the biofilm end-product on day 24
(Table~\ref{tbl-dna-samples}). DNA extraction was performed at the Max
Planck Institute for the Science of Human History (Jena, Germany), using
the DNeasy PowerSoil Kit from QIAGEN. C2 inhibitor removal step skipped,
going directly to C3 step.

The DNA was sheared to 500bp through sonication with a Covaris M220
Focused-ultrasonicator. Double-stranded libraries were prepared
(\protect\hyperlink{ref-aronHalfUDG2020}{Aron et al., 2020}) and dual
indexed (\protect\hyperlink{ref-stahlDoublestrandedIndexing2019}{Stahl
et al., 2019}), with the indexing protocol being adapted for longer DNA
fragments. Briefly, the modifications consisted of adding 3 μl of DMSO
to the indexing reaction, and extending the amplification cycles to
95\(^{\circ}\)C for 60 s, 58\(^{\circ}\)C for 60 s, and 72\(^{\circ}\)C
for 90 s. The libraries were paired-end sequenced on a NextSeq 500 to
150bp, and demultiplexed by an in-house script.

\hypertarget{preprocessing}{%
\subsection{Preprocessing}\label{preprocessing}}

The raw DNA reads were preprocessed using the nf-core/eager, v2.4.4
pipeline (\protect\hyperlink{ref-yatesEAGER2020}{Fellows Yates et al.,
2020}). The pipeline included adapter removal and read merging using
AdapterRemoval, v2.3.2
(\protect\hyperlink{ref-AdapterRemovalv2}{Schubert et al., 2016}).
Merged reads were mapped to the human reference genome (GRCh38) using
BWA, v0.7.17-r1188 (\protect\hyperlink{ref-BWA}{H. Li \& Durbin, 2009})
(-n 0.01; -l 32), and unmapped reads were extracted using Samtools,
v1.12. The final step of the pipeline, metagenomic classification, was
conducted in kraken, v2.1.2 (\protect\hyperlink{ref-kraken2}{Wood et
al., 2019}) using the Standard 60GB database
(\url{https://genome-idx.s3.amazonaws.com/kraken/k2_standard_20220926.tar.gz}).

Environmental reference samples were downloaded directly from ENA and
from NCBI using the SRA Toolkit. Oral reference samples were downloaded
from the Human Metagenome Project (HMP), and modern calculus samples
from Velsko et al.
(\protect\hyperlink{ref-velskoDentalCalculus2017}{2017}). From the HMP
data, only paired reads were processed, singletons were removed.
\emph{In vitro} biofilm model samples from Edlund et al.
(\protect\hyperlink{ref-edlundUncoveringComplex2018}{2018}) were used as
a reference. Links to the specific sequences are included in the
metadata. Human-filtered reads produced in this study were uploaded to
ENA under accession number PRJEB61886.

\hypertarget{authentication}{%
\subsection{Authentication}\label{authentication}}

Species with lower than 0.001\% relative abundance across all samples
were removed from the species table. SourceTracker2
(\protect\hyperlink{ref-knightsSourceTracker2011}{Knights et al., 2011})
was used to estimate source composition of the abundance-filtered oral
biofilm model samples using a Bayesian framework, and samples falling
below 70\% oral source were removed from downstream analyses.
Well-preserved abundance-filtered samples were compared to oral and
environmental controls to detect potential external contamination. The R
package decontam v1.20.0 (\protect\hyperlink{ref-Rdecontam}{Davis et
al., 2018}) was used to identify potential contaminants in the
abundance-filtered table using DNA concentrations with a probability
threshold of 0.95 and negative controls with a probability threshold of
0.05. Putative contaminant species were filtered out of the OTU tables
for all downstream analyses.

\hypertarget{community-composition}{%
\subsection{Community composition}\label{community-composition}}

Relative abundances of communities were calculated at the species- and
genus-level, as recommended for compositional data
(\protect\hyperlink{ref-gloorMicrobiomeDatasets2017}{Gloor et al.,
2017}). Shannon index and Pileou's evenness index were calculated on
species-level OTU tables of all model and oral reference samples using
the vegan v2.6.4 R package (\protect\hyperlink{ref-Rvegan}{Oksanen et
al., 2022}). Shannon index was calculated for all experimental samples
to see if there is an overall loss or gain in diversity and richness
across the experiment. Sparse principal component analysis (sPCA) was
performed on model biofilm samples to assess differences in microbial
composition between samples within the experiment, and a separate sPCA
analysis was performed on model calculus and oral reference samples. The
sPCA analysis was conducted using the mixOmics v 6.24.0 R package
(\protect\hyperlink{ref-RmixOmics}{Rohart et al., 2017}).

The core microbiome was calculated by taking the mean genus-level
relative abundance within each sample type for model calculus, modern
reference calculus, sub- and supragingival plaque. Genera present at
lower than 5\% relative abundance were grouped into the category
`other'. Information on the oxygen tolerance of bacterial species was
downloaded from BacDive
(\protect\hyperlink{ref-reimerBacDive2022}{Reimer et al., 2022}) and all
variations of the major categories anaerobe, facultative anaerobe, and
aerobe were combined into the appropriate major category. At the time of
writing, 55.7\% species were missing aerotolerance values. This was
mitigated by aggregating genus-level tolerances to species with missing
values, and may have some errors (although unlikely to make any
significant difference).

\hypertarget{differential-abundance}{%
\subsection{Differential abundance}\label{differential-abundance}}

Differential abundance of species was calculated using the Analysis of
Compositions of Microbiomes with Bias Correction (ANCOM-BC) method from
the ANCOMBC R package v2.2.0 (\protect\hyperlink{ref-linANCOMBC2020}{Lin
\& Peddada, 2020}), with a species-level OTU table as input. Results are
presented as the log fold change of species between paired sample types
with 95\% confidence intervals. P-values are adjusted using the false
discovery rate (FDR) method. Samples are grouped by sample type
(i.e.~saliva, plaque, modern calculus, model calculus). To supplement
the sPCA analyses, we visualised the log-fold change of the top 30
species in each of principal components 1 and 2, allowing us to see
which species are enriched in the different samples and causing
clustering in the sPCA.

\hypertarget{ftir}{%
\section{FTIR}\label{ftir}}

To determine the mineral composition and level of crystallisation of the
model dental calculus samples, we used Fourier Transform Infrared (FTIR)
spectroscopy. We compared the spectra of model dental calculus with
spectra of archaeological and modern dental calculus and used a built-in
Omnic search library for mineral identification
(\protect\hyperlink{ref-mentzerDistributionAuthigenic2014}{Mentzer et
al., 2014};
\protect\hyperlink{ref-weinerInfraredSpectroscopy2010}{Weiner, 2010b}).
The archaeological dental calculus was sampled from an isolated
permanent tooth from Middenbeemster, a rural, 19th century Dutch site
(\protect\hyperlink{ref-lemmersMiddenbeemster2013}{Lemmers et al.,
2013}). Samples were analysed at the Laboratory for Sedimentary
Archaeology, Haifa University. The analysis was conducted with a Thermo
Scientific Nicolet is5 spectrometer in transmission, at 4 cm\(^{-1}\)
resolution, with an average of 32 scans between 4000 and 400 cm\(^{-1}\)
wavenumbers.

\hypertarget{tbl-ftir-byoc}{}
\begin{longtable}[]{@{}lrrr@{}}
\caption{\label{tbl-ftir-byoc}Summary of samples used in FTIR analysis,
including type of sample, sampling day, number of samples (n), and mean
weight in mg.}\tabularnewline
\toprule\noalign{}
Sample type & Sampling day & n & Weight (mg) \\
\midrule\noalign{}
\endfirsthead
\toprule\noalign{}
Sample type & Sampling day & n & Weight (mg) \\
\midrule\noalign{}
\endhead
\bottomrule\noalign{}
\endlastfoot
biofilm & 7 & 2 & 0.79 \\
biofilm & 12 & 3 & 1.01 \\
biofilm & 16 & 7 & 2.00 \\
biofilm & 20 & 6 & 3.50 \\
model\_calculus & 24 & 8 & 3.87 \\
\end{longtable}

Analysis was conducted on 26 model calculus samples from days 7, 12, 16,
20, and 24 (Table~\ref{tbl-ftir-byoc}). Some samples from the same
sampling day had to be combined to provide enough material for analysis.
Samples analysed with FTIR were grown during a separate experimental run
from the samples sequenced for DNA, but following the same setup and
protocol (as described above). Samples were analysed following the
method presented in Asscher, Regev, et al.
(\protect\hyperlink{ref-asscherAtomicDisorder2011}{2011}) and Asscher,
Weiner, et al.
(\protect\hyperlink{ref-asscherVariationsAtomic2011}{2011}). A few
\(\mu\)g of each sample were repeatedly ground together with KBr and
pressed in a 7 mm die under two tons of pressure using a Specac
mini-pellet press (Specac Ltd., GS01152). Repeated measurements of the
splitting factor (SF) of the absorbance bands at 605 and 567 cm−1
wavenumbers were taken after each grind, and a grind curve was produced
following Asscher, Regev, et al.
(\protect\hyperlink{ref-asscherAtomicDisorder2011}{2011}) to try and
detect changes in the hydroxyapatite crystallinity over time. Samples
were ground and analysed up to six times (sample suffix a-f) for the
grinding curve. Grinding curves were prepared for samples from days 16,
20, and 24. No grind curves were produced for samples from days 7 and
12. These were largely composed of organics and proteins, and did not
form enough mineral (hydroxyapatite) for analysis. The splitting factor
of carbonate hydroxyapatite was calculated using a macro script,
following Weiner \& Bar-Yosef
(\protect\hyperlink{ref-weinerStatesPreservation1990}{1990}). The
calculation involves dividing the sum of the height of the absorptions
at 603 cm\(^{-1}\) and 567 cm\(^{-1}\) by the height of the valley
between them. Following Asscher, Regev, et al.
(\protect\hyperlink{ref-asscherAtomicDisorder2011}{2011}) and Asscher,
Weiner, et al.
(\protect\hyperlink{ref-asscherVariationsAtomic2011}{2011}), we plotted
the splitting factor against the full width at half maximum (FWHM) of
the main absorption at 1035-1043 cm\(^{-1}\) to explore crystallinity
(crystal size) and the order and disorder of hydroxyapatite. We then
compared our grinding curve slopes and FWHM to the ones produced by
Asscher, Weiner, et al.
(\protect\hyperlink{ref-asscherVariationsAtomic2011}{2011}). Asscher,
Weiner, et al.
(\protect\hyperlink{ref-asscherVariationsAtomic2011}{2011}) and Asscher,
Regev, et al. (\protect\hyperlink{ref-asscherAtomicDisorder2011}{2011})
demonstrated that while the decrease in FWHM of each grinding in the
curve reflects a decrease in particle size due to grinding, the location
of the curves within a plot of the FWHM against the splitting factor
expresses the disorder effect. Thus the curves with steeper slopes,
higher splitting factor, and lower FWHM represent lower levels of
disorder in the mineral (Figure 2 in
\protect\hyperlink{ref-asscherVariationsAtomic2011}{Asscher, Weiner, et
al., 2011}).

\hypertarget{statistics}{%
\section{Statistics}\label{statistics}}

Statistical analysis was conducted in R version 4.3.1 (2023-06-16)
(Beagle Scouts) (\protect\hyperlink{ref-Rbase}{R Core Team, 2020}). Data
cleaning and wrangling performed with packages from tidyverse
(\protect\hyperlink{ref-tidyverse2019}{Hadley Wickham et al., 2019}).
Plots were created using ggplot2 v3.4.2
(\protect\hyperlink{ref-ggplot2}{H. Wickham, 2016}).

\hypertarget{results}{%
\section{Results}\label{results}}

\hypertarget{metagenomic-analysis}{%
\section{Metagenomic analysis}\label{metagenomic-analysis}}

\hypertarget{sample-authentication}{%
\subsection{Sample authentication}\label{sample-authentication}}

To determine the extent of contamination in our samples, we performed a
source-tracking analysis using SourceTracker2
(\protect\hyperlink{ref-knightsSourceTracker2011}{Knights et al.,
2011}). Results suggest that the majority of taxa across samples have an
oral microbial signature, and therefore our samples are minimally
affected by external contamination (Figure S1). We compared
SourceTracker2 results to a database of oral taxa from the cuperdec
v1.1.0 R package
(\protect\hyperlink{ref-yatesOralMicrobiome2021}{Fellows Yates et al.,
2021}) to prevent removal of samples where oral taxa were assigned to a
non-oral source (Figure S2), as some taxa with a signature from multiple
sources are often classified as ``Unknown''
(\protect\hyperlink{ref-velskoMicrobialDifferences2019}{Velsko et al.,
2019}). We included several oral sources, which may increase the risk of
this occurring. Samples containing a large proportion (\textgreater70\%)
of environmental contamination were removed. The removed samples were
predominantly medium samples from later in the experiment, and a few
model calculus samples. After contaminated samples were removed,
suspected contaminant-species were removed from the remaining samples
using the decontam R package (\protect\hyperlink{ref-Rdecontam}{Davis et
al., 2018}). After contamination removal, samples consisted of between
88 and 284 species with a mean of 182.

\hypertarget{decrease-in-community-diversity-across-experiment}{%
\subsection{Decrease in community diversity across
experiment}\label{decrease-in-community-diversity-across-experiment}}

\begin{figure}

{\centering \includegraphics{figures/fig-byoc-valid-fig-diversity-byoc-1.pdf}

}

\caption{\label{fig-diversity-byoc}Plot of Pielou Evenness Index, number
of species, and Shannon Index across experiment samples grouped by
sampling time. inoc = samples from days 0-5; treatm = samples from days
6-23; model = model calculus samples from day 24.}

\end{figure}

To monitor the development of microbial communities over the course of
the experiment, we used the Shannon Index to assess the species
diversity and richness at various stages of our protocol. Samples were
grouped into sampling categories due to low sample sizes on sampling
days (inoc = days 0, 3, 5; treatm = days 7, 9, 12, 15; model = day 24).
There was a slight decrease in mean Shannon Index between inoculation
and treatment samples, followed by a slight increase to model calculus
samples, as well as a decrease in variance within samples types. The
Pielou Evenness Index showed a similar pattern while the number of
species increased between the treatment period and the final model
calculus (Figure~\ref{fig-diversity-byoc}).

\hypertarget{medium-and-model-calculus-samples-are-distinct-from-the-inoculate}{%
\subsection{Medium and model calculus samples are distinct from the
inoculate}\label{medium-and-model-calculus-samples-are-distinct-from-the-inoculate}}

\begin{figure}

{\centering \includegraphics{figures/fig-byoc-valid-fig-spca-byoc-1.pdf}

}

\caption{\label{fig-spca-byoc}sPCA on species-level counts and oxygen
tolerance in samples from this study only. Figure shows the main sPCA
plot (A), species loadings on PC2 (B), and species loadings on PC1 (C).}

\end{figure}

We next examined whether there is a change in the species composition
over time in our samples by assessing the beta-diversity in a PCA. The
species profiles of the saliva inoculate used in our experiment were
distinct from both medium and model calculus samples. Most of the
separation of saliva from model calculus is on PC1 of the sPCA, where
most of the positive sample loadings are driven by anaerobic species
(model calculus), especially \emph{Selenomonas} spp, and negative
loadings are predominantly facultative anaerobes and some aerobes, such
as \emph{Rothia} and \emph{Neisseria} spp (saliva). Medium and saliva
are separated mostly on PC2, with medium samples located between saliva
and model calculus samples. Model calculus samples also cluster
separately from the medium samples on PC2, with some overlap between the
more mature medium samples and model calculus. Most of the negative
loadings separating saliva and model calculus from medium samples are
dominated by \emph{Actinomyces} spp., while positive species loadings
are more diverse, and seemingly unrelated to aerotolerance
(Figure~\ref{fig-spca-byoc}).

\begin{figure}

{\centering \includegraphics{figures/fig-byoc-valid-fig-diffabund-byoc-1.pdf}

}

\caption{\label{fig-diffabund-byoc}Log-fold changes between sample
types. Circles are species enriched in the model calculus, squares are
enriched in saliva, and triangles in medium. Lines are standard error.
Plot shows the top 30 absolute log-fold changes between model calculus
and saliva.}

\end{figure}

We determined whether there are species that are differentially abundant
between our sample types using the ANCOMBC R package
(\protect\hyperlink{ref-linANCOMBC2020}{Lin \& Peddada, 2020}), giving
us an idea of how the biofilm develops under our experimental
conditions. Species enriched in saliva compared to model calculus are
largely aerobic or facultatively anaerobic, while species enriched in
model calculus compared to saliva are mainly anaerobes. The differences
between saliva and calculus are more pronounced than between medium and
model calculus, which is expected (Figure~\ref{fig-diffabund-byoc}).

\hypertarget{lower-diversity-in-artificial-samples-than-oral-references}{%
\subsection{Lower diversity in artificial samples than oral
references}\label{lower-diversity-in-artificial-samples-than-oral-references}}

\begin{figure}

{\centering \includegraphics{figures/fig-byoc-valid-fig-shannon-compar-1.pdf}

}

\caption{\label{fig-shannon-compar}Shannon Index for model calculus and
medium samples, as well as oral reference samples and comparative
\emph{in vitro} study.}

\end{figure}

We used the Shannon Index to compare alpha-diversity in our model to
oral reference samples. The mean Shannon Index of model
samples---medium, model calculus, reference \emph{in vitro} biofilm were
consistently lower than the means of oral reference samples---mucosa,
modern reference dental calculus, saliva, and subgingival and
subgingival plaque. The Pielou species evenness index has a similar
distribution, although the comparative biofilm samples have a higher
mean than biofilm samples from this study. Saliva inoculate samples from
this study have a lower mean Shannon index than reference samples, which
may have contributed to the lower alpha-diversity in model samples
compared to reference samples. The number of species follows the same
trend.

\hypertarget{model-calculus-is-distinct-from-dental-calculus-and-other-oral-samples}{%
\subsection{Model calculus is distinct from dental calculus and other
oral
samples}\label{model-calculus-is-distinct-from-dental-calculus-and-other-oral-samples}}

\begin{figure}

{\centering \includegraphics{figures/fig-byoc-valid-fig-core-genera-1.pdf}

}

\caption{\label{fig-core-genera}Core genera within the different types
of samples represented as mean relative abundances at the genus level.
Other = other genera present in lower than 5\% relative abundance.}

\end{figure}

We calculated the mean relative abundances of the genera in each sample
to compare the core genera of model calculus with oral reference
samples. The most common genera (\textgreater5\% relative abundance) are
shown in Figure~\ref{fig-core-genera}. The main overlap between the
model calculus and oral reference samples is the high relative abundance
of \emph{Streptococcus}. Model calculus consists mostly of
\emph{Enterococcus} and \emph{Veillonella} spp., despite both having low
abundance in donor saliva. \emph{Enterococcus} are also known
environmental contaminants, and we cannot exclude environmental
contamination as a possible source for these species in our model. Oral
reference samples have a more balanced composition, as they are also
represented by fastidious early-coloniser species like
\emph{Capnocytophaga} and \emph{Neisseria} spp., which require an
environment with at least 5\% carbon dioxide to thrive
(\protect\hyperlink{ref-tonjumNeisseria2017}{Tønjum \& van Putten,
2017}).

\begin{figure}

{\centering \includegraphics{figures/fig-byoc-valid-fig-spca-compar-1.pdf}

}

\caption{\label{fig-spca-compar}sPCA on species-level counts from model
calculus and reference samples. Figure shows (A) the main sPCA plot, (B)
the species loadings from PC2, and (C) species loadings on PC1.}

\end{figure}

To directly compare the beta-diversity of our model calculus with oral
reference samples, including modern dental calculus, we used an sPCA
including only our model calculus and reference samples. Model calculus
samples are distinct from both the oral reference samples and the
biofilm model reference samples. They are separated from oral reference
samples mainly on PC1, and from biofilm model reference samples (and, to
some extent, oral samples) on PC2. The highest negative contributions
are a mix of all types of aerotolerance, while the positive
contributions are mostly (facultative) anaerobes, with
\emph{Enterococcus} spp. as the top three positive contributors to PC1.
Top negative contributors are \emph{Capnocytophaga} spp as well as the
aerobes \emph{Corynebacterium matruchotii} and \emph{Rothia
dentocariosa}. The top positive contributors to PC2 are all anaerobes,
mainly from the genus \emph{Selenomonas}. Top negative contributors to
PC2 are a mix of aerotolerances, with many \emph{Streptococcus} spp.

\begin{figure}

{\centering \includegraphics{figures/fig-byoc-valid-fig-diffabund-comp-1.pdf}

}

\caption{\label{fig-diffabund-comp}Log-fold changes between sample
types. Circles are species enriched in the model calculus, triangles in
modern calculus, diamonds are enriched in subgingival plaque, and
squares in supragingival plaque. Plot shows the top 30 loadings
(absolute value) in PC1 (A) and PC2 (B) between model calculus and other
sample types, ordered by decreasing log-fold change. Bars represent
standard error.}

\end{figure}

To investigate which species are enriched in different sample types, and
compare the final product of our model with naturally occurring plaque
and calculus samples, we performed differential abundance analysis on
our model calculus samples, modern dental calculus, and sub- and
supragingival plaque. Based on the differential abundance analysis the
main differences between model calculus and oral reference samples, when
looking at the top 30 contributors to PC1, are that the oral reference
samples are enriched with species with a diverse oxygen tolerance from a
wide range of genera, while the model calculus is enriched with
\emph{Enterococcus} spp. The largest differences occur in
\emph{Corynebacterium matruchotii}, \emph{Rothia dentocariosa}, and
\emph{Capnocytophaga gingivalis} (Figure~\ref{fig-diffabund-comp}A).
This is echoed when looking at the top 30 contributors to PC2, where
most of the species are enriched in model calculus, all of which are
anaerobes, and the largest differences occurring in
\emph{Cryptobacterium curtum}, \emph{Eggerthella lenta}, and
\emph{Mogibacterium diversum} (Figure~\ref{fig-diffabund-comp}B).

\hypertarget{samples-show-an-increased-mineralisation-over-the-course-of-the-experiment}{%
\section{Samples show an increased mineralisation over the course of the
experiment}\label{samples-show-an-increased-mineralisation-over-the-course-of-the-experiment}}

Comparing the development of the model calculus to the reference
samples, it is evident that between days 7 and 24 there is a decrease of
the protein components and increase of the inorganic mineral carbonate
hydroxyapatite. The model calculus samples from the end of the
experiment are similar to both the modern and archaeological reference
samples. The main difference is a lower organic component in reference
samples seen as a reduced amide I peak at around 1637 compared to the
carbonate peak at around 1420, and an absence of amide II and III.
Further, there is a reduction in CH3 bands at 3000-2900 cm\(^{-1}\)
(Figure~\ref{fig-ftir-spectra}A-D).

Sample spectra from days 7 and 12 are characterised by a high content of
proteins as evident by the strong amide I absorbance band at 1650, a
less pronounced amide II band at 1545 cm\(^{-1}\), and the small amide
III band at 1237 cm\(^{-1}\). Related to the organic component of the
samples are also the three marked CH\textsubscript{3} and
CH\textsubscript{2} stretching vibrations at 2960, 2920, and 2850
cm\(^{-1}\) wavenumbers. The presence of mineral component is evident
from the presence of C--O\(^{2-}_3\) absorbance bands at 1450 and 1400
cm\(^{-1}\) wavenumbers typical of carbonates, and P--O\(^{3-}_4\)
absorbance band at 1080 and 1056 cm\(^{-1}\) which are related to
phosphate minerals. There is a large variation between the spectra,
possibly indicating different formation rates of the different
components in the samples (Figure~\ref{fig-ftir-spectra}A and B).

In spectra from days 16 to 24, the ratio of amides to
PO\textsubscript{4} has shifted, with the main peak shifting to the
PO\textsubscript{4} v\textsubscript{3} absorbance band at 1039--1040
cm\(^{-1}\), indicating that the main component of the samples is
carbonate hydroxyapatite. A well-defined PO\textsubscript{4} doublet at
600 and 560 is present. Small CO\(_3^{2-}\) asymmetric stretching at
1450 cm\(^{-1}\) and 1415 cm\(^{-1}\), and stretching vibrations at
875-870 cm\(^{-1}\) indicate that the carbonate minerals component is
also becoming more crystallised. There is a decreased variability
between the spectra, with most spectra exhibiting a higher
phosphate-to-protein/lipid ratio (Figure~\ref{fig-ftir-spectra}C and D).

\hypertarget{model-calculus-has-a-similar-mineral-composition-to-natural-calculus}{%
\section{Model calculus has a similar mineral composition to natural
calculus}\label{model-calculus-has-a-similar-mineral-composition-to-natural-calculus}}

To determine whether the model dental calculus is comparable to natural
dental calculus, both modern and archaeological dental calculus were
analysed with FTIR spectroscopy to ascertain their composition. The
archaeological and modern reference spectra are largely
indistinguishable and consist of a broad O--H absorbance band (3400
cm\(^{-1}\)) related to amid a and hydroxyl group, weak CH3 bands
(3000--2900 cm\(^{-1}\)), amide I band (1650 cm\(^{-1}\)) which is
related to the protein content, carbonate (1420, 1458-1450, 875-870
cm\(^{-1}\)), and phosphates (1036-1040, 602-4, 563-566 cm\(^{-1}\))
(Figure~\ref{fig-ftir-spectra}E) which, together with the hydroxyl and
the carbonate, can be identified as derived from carbonate
hydroxyapatite, the main mineral found in mature dental calculus
(\protect\hyperlink{ref-hayashizakiSiteSpecific2008}{Hayashizaki et al.,
2008}; \protect\hyperlink{ref-jinSupragingivalCalculus2002}{Jin \& Yip,
2002}).

\begin{figure}

{\centering \includegraphics{figures/fig-byoc-valid-fig-ftir-spectra-1.pdf}

}

\caption{\label{fig-ftir-spectra}Select spectra from all experiment
sampling days; (A) day 7, (B) day 12, (C) day 16, and (D) day 24.
Absorbance bands in stretching mode around 3400 cm−1 typical of the
hydroxyl group. Analysis ID for model samples is constructed as: F{[}day
sampled{]}.{[}well sampled{]}\_{[}grind sample{]}.}

\end{figure}

\hypertarget{samples-show-similar-crystallinity-and-order-to-reference-calculus}{%
\section{Samples show similar crystallinity and order to reference
calculus}\label{samples-show-similar-crystallinity-and-order-to-reference-calculus}}

We determined the level of crystallinity and order of the carbonate
hydroxyapatite in our samples as an indication for its maturity by using
the grinding curves method presented by Asscher, Regev, et al.
(\protect\hyperlink{ref-asscherAtomicDisorder2011}{2011}) and Asscher,
Weiner, et al.
(\protect\hyperlink{ref-asscherVariationsAtomic2011}{2011}).\\
Samples were compared to published trendlines for archaeological and
modern enamel
(\protect\hyperlink{ref-asscherAtomicDisorder2011}{Asscher, Regev, et
al., 2011}). We see no appreciable differences between days 16, 20, and
24. The archaeological dental calculus shows a slightly increased slope
compared to model calculus from the three sampling days used in the
grind curve (Figure~\ref{fig-grind-curve}), possibly indicating larger
crystal size due to more complete crystalisation. The steeper slope of
enamel samples is consistent with a more ordered structure in enamel
compared to dental calculus.

\begin{figure}

{\centering \includegraphics{figures/fig-byoc-valid-fig-grind-curve-1.pdf}

}

\caption{\label{fig-grind-curve}Grinding curves of our biofilm and model
calculus compared to published trendlines (dashed light grey lines) for
archaeological (dotted line) and modern (dashed line) enamel.}

\end{figure}

\hypertarget{discussion}{%
\section{Discussion}\label{discussion}}

In this study we present a calcifying oral biofilm model to produce
artificial dental calculus. Our proposed use of the model is to address
a variety of research questions related to dietary information extracted
from dental calculus, in both modern and archaeological samples. For
that to be feasible, the model needs to serve as a viable proxy to
dental calculus grown under natural conditions, i.e., in the human oral
cavity. It needs, as much as possible, to mimic the diversity and
complexity of the natural oral microbiome, while also offering control
over factors such as dietary input, growth conditions, and replicability
within and between experiments. Here, we assessed the viability of our
model as a proxy for dental calculus using metagenomic classification
and FTIR analysis to explore the bacterial and mineral composition, and
compare with oral reference samples.

\hypertarget{microbiome}{%
\section{Microbiome}\label{microbiome}}

Model calculus has lower species diversity than inocula saliva and oral
reference samples, which is a common limitation in biofilm models
(\protect\hyperlink{ref-bjarnsholtVivoBiofilm2013}{Bjarnsholt et al.,
2013}; \protect\hyperlink{ref-edlundBiofilmModel2013}{Edlund et al.,
2013}). The donated saliva for the experiment had a lower diversity than
the reference saliva samples, and may have contributed to a lower
diversity in experimental samples. Consequently, there is also a lower
diversity and richness when compared to other modern oral reference
samples, including oral mucosa, saliva, plaque, and calculus. Samples of
the medium from early in the experiment have similar species profiles to
the donated saliva, but gradually diverge over the course of the
experiment. This may be caused by experimental setup not sufficiently
mimicking the oral environment, allowing species to thrive that do not
normally thrive in the natural oral environment.

Oral reference samples have a relative abundance of streptococci similar
to our model, but a more diverse representation from other genera and an
overall higher species diversity and richness than our model. Reference
samples also had a more diverse aerotolerance profile than our model,
which primarily consisted of (faculatative) anaerobes. Species within
predominantly aerobic genera, are deficient in the model, suggesting a
shift from a largely aerotolerant profile to an anaerobic profile during
the experiment. While our model is not set up as an anaerobic system,
the anaerobes seem to have outcompeted aerobes and, to some extent,
facultative anaerobes. This is likely a result of communities of
bacteria within the biofilm creating favourable microenvironments
facilitated by the protective properties of the biofilm matrix
(\protect\hyperlink{ref-edlundUncoveringComplex2018}{Edlund et al.,
2018}; \protect\hyperlink{ref-flemmingBiofilmsEmergent2016}{Flemming et
al., 2016}).

Overall, the majority of model calculus samples contained a distinctly
oral signature, providing a promising starting point for the use of the
model as a viable proxy to dental calculus. The main differences between
model and oral reference samples may be due to human variation, as there
can be large differences in the oral microbiome of two individuals at
the species level due to variations in age, sex, and other demographic
factors, as well as how and when saliva samples were collected
(\protect\hyperlink{ref-burchamPatternsOral2020}{Burcham et al., 2020};
\protect\hyperlink{ref-nearingAssessingVariation2020}{Nearing et al.,
2020}). Whether or not distinct microbial profiles, and the
extracellular matrix they produce, will affect the retention of dietary
particles in plaque remains to be seen, but is an important question to
address in the future.

\hypertarget{mineralisation}{%
\section{Mineralisation}\label{mineralisation}}

FTIR analysis allowed us to address the mineralisation process of the
model, which showed an increasing mineral composition over the course of
the experiment. As the model biofilm matured, the predominantly organic
content of early samples was replaced by inorganic content in the form
of carbonated hydroxyapatite, consistent with a shift from a high
presence of bacterial cells in a matrix of extracellular polysaccharides
(\protect\hyperlink{ref-jainIsolationCharacterization2013}{Jain et al.,
2013}; \protect\hyperlink{ref-sutherlandBiofilmMatrix2001}{Sutherland,
2001}; \protect\hyperlink{ref-zhangMeasurementPolysaccharides1998}{X.
Zhang et al., 1998}) to a predominantly mineral content.

The model calculus samples resemble both the modern reference calculus
and the archaeological calculus in mineral composition and
crystallinity. The steeper slope in the grind curve plots of the
archaeological sample suggests that the crystals in archaeological
samples are larger, and hence more ordered than in model calculus. A
possible explanation is that the inorganic crystals within
archaeological calculus have had more time to grow into the space left
by degraded organic matter
(\protect\hyperlink{ref-weinerBiologicalMaterials2010}{Weiner, 2010a});
however, we only analysed one archaeological sample and cannot
definitively address this. The short duration of model calculus growth
may also have affected the results, compared to the longer-term growth
and mineralisation of natural calculus. The constant disruptions in
growth of \emph{in vivo} dental plaque/calculus, due to oral hygiene and
other external pressures on biofilm growth, may lead to multiple stages
of calcium phosphates, whereas our model has more stable growth
conditions.

One of the most well-known biomineralisers, \emph{Corynebacterium
matruchotii}
(\protect\hyperlink{ref-enneverCharacterizationBacterionema1978}{Ennever
et al., 1978};
\protect\hyperlink{ref-takazoeCalciumHydroxyapatite1970}{Takazoe et al.,
1970}), exhibited a lower abundance in our model calculus compared to
modern reference calculus. However, the mineral composition of the end
results were similar, reinforcing the idea that, under the right
circumstances, biofilms with a range of microbial profiles can
facilitate mineralisation
(\protect\hyperlink{ref-moorerCalcificationCariogenic1993}{Moorer et
al., 1993}). Bacteria and their ability to secrete an extracellular
matrix are integral in the formation of dental calculus, and inevitably
serve as part of the structure that dental calculus is built upon
(\protect\hyperlink{ref-rohanizadehUltrastructuralStudy2005}{Rohanizadeh
\& LeGeros, 2005}), while the exact species composition of the biofilm
communities may be less important.

\hypertarget{replicability}{%
\section{Replicability}\label{replicability}}

Model calculus displayed similar species diversity and microbial
profiles across all samples, indicating a high level of replicability
between samples in the experimental run. It remains to be seen whether
the replicability within the experiment also scales up to
between-experiment replicability in our model, though others have
already shown that replicability in long-term models is possible when
using the same inocula
(\protect\hyperlink{ref-velskoConsistentReproducible2018}{Velsko \&
Shaddox, 2018}). The variation in mineral composition in our model was
initially high, but samples from day 24 were largely similar in
composition as observed in the FTIR spectra. The use of a simple
multiwell plate setup allows us to submit many samples to the same
conditions, increasing replicability between samples
(\protect\hyperlink{ref-extercateAAA2010}{Exterkate et al., 2010}).

\hypertarget{limitations}{%
\section{Limitations}\label{limitations}}

While our in vitro model calculus system provides reproducible and
consistent artificial dental calculus for archaeological research, as
demonstrated by the species composition and the mineralisation
properties, we recognise the model has several limitations. Our
single-donor approach may have affected the diversity of the model. The
donated saliva from our study had a lower mean Shannon Index than other
saliva samples. The lower diversity may be caused by only using one
donor instead of pooling saliva from multiple individuals. However,
having a single inoculum donor allows us to maintain the integrity of a
native oral microbiome which may be lost when samples are pooled
(\protect\hyperlink{ref-edlundBiofilmModel2013}{Edlund et al., 2013}).
It is also possible that the diversity was affected by the collection
and storage methods we used. This has been shown to have minimal effect
on microbial profiles at the genus level
(\protect\hyperlink{ref-limSalivaMicrobiome2017}{Lim et al., 2017}), but
some effect on beta diversity calculations
(\protect\hyperlink{ref-omoriComparativeEvaluation2021}{Omori et al.,
2021}).

Some samples were grown with starch-sucrose solutions as nutrients,
while controls were grown with sucrose only. Due to the financial cost,
we did not sequence enough samples of each nutrient treatment to assess
the influence of starch on the microbial community or mineral
composition. Biofilms were grown in a standard shaking bacterial growth
incubator, rather than an incubator specific to cell cultures. The lack
of complex environmental control may cause the model to deviate from its
natural growth over the 25 days that the experiment is run, due to a
lack of precise control over conditions such as pH and salivary flow
rates.

There is also the possibility that contamination was introduced into the
model during the experiment. While the CPMU solution was prepared under
sterile conditions, the solution itself was not autoclaved or
filter-sterilised. In the species composition metagenomic analysis, all
medium samples collected after the introduction of CPMU on day 14 were
removed by the authentication step because the majority of species
appeared to derive from environmental sources indicating external
contamination. Going forward we recommend filter-sterilising solutions
that are not autoclaved.

To avoid disturbing the growth and development of our biofilm, we took
samples of media from the bottom of the wells after three days without
full media replacement, careful not to disturb other plate-bound
biofilms. The samples may therefore not fully reflect the composition of
the biofilm itself. Going forward we recommend sampling from the actual
biofilm, as this is the sample type under investigation.

\hypertarget{future-work}{%
\section{Future work}\label{future-work}}

Further protocol optimisation will also be necessary to address some of
the limitations of our current model, such as reducing the frequency of
medium replacement (currently every three days) to help promote the
growth of slow-growing fastidious organisms and limit generalists such
as enterococci, and supplementing it with serum to provide additional
nutrients and biofilm stability
(\protect\hyperlink{ref-ammannZurichBiofilm2012}{Ammann et al., 2012};
\protect\hyperlink{ref-tianUsingDGGE2010}{Tian et al., 2010}). More
infrequent medium replacement would facilitate slow-growing bacteria in
establishing their metabolic relationships, allowing the byproducts of
some species to become abundant enough for others that depend on these
to grow (\protect\hyperlink{ref-marshDentalPlaque2005}{Philip D. Marsh,
2005}).

Our goals for additional validation measures involve functional profiles
of bacteria, to see if metabolic behaviour of bacteria is consistent
with \emph{in vivo} conditions, and whether this is affected by the
presence/absence of amylase and starch treatments. The absence of host
salivary \(\alpha\)-amylase activity in our model (as shown in Bartholdy
\& Henry
(\protect\hyperlink{ref-bartholdyInvestigatingBiases2022}{2022}))
provides an opportunity to explore the effect of various amylase levels
on biofilm growth and composition, as well as the incorporation of
dietary compounds in dental calculus.

The model can also be used to explore limitations and biases of methods
used to reconstruct past dietary patterns from dental calculus. To this
end, sucrose and raw starch treatments can be replaced with other
dietary components of interest, such as cooked starches, whole plant
extracts, and various proteins.

\hypertarget{refs-3}{}
\begin{CSLReferences}{1}{0}
\leavevmode\vadjust pre{\hypertarget{ref-adlerSequencingAncient2013}{}}%
Adler, C. J., Dobney, K., Weyrich, L. S., Kaidonis, J., Walker, A. W.,
Haak, W., Bradshaw, C. J., Townsend, G., Sołtysiak, A., Alt, K. W.,
Parkhill, J., \& Cooper, A. (2013). Sequencing ancient calcified dental
plaque shows changes in oral microbiota with dietary shifts of the
{Neolithic} and {Industrial} revolutions. \emph{Nature Genetics},
\emph{45}(4), 450--455, 455e1. \url{https://doi.org/10.1038/ng.2536}

\leavevmode\vadjust pre{\hypertarget{ref-ammannZurichBiofilm2012}{}}%
Ammann, T. W., Gmür, R., \& Thurnheer, T. (2012). Advancement of the
10-species subgingival {Zurich} biofilm model by examining different
nutritional conditions and defining the structure of the in vitro
biofilms. \emph{BMC Microbiology}, \emph{12}, 227.
\url{https://doi.org/10.1186/1471-2180-12-227}

\leavevmode\vadjust pre{\hypertarget{ref-aronHalfUDG2020}{}}%
Aron, F., Neumann, G., \& Brandt, G. (2020). Half-{UDG} treated
double-stranded ancient {DNA} library preparation for illumina
sequencing v1 {[}{Data} set{]}. \emph{Protocols. Io}.

\leavevmode\vadjust pre{\hypertarget{ref-asscherAtomicDisorder2011}{}}%
Asscher, Y., Regev, L., Weiner, S., \& Boaretto, E. (2011). Atomic
{Disorder} in {Fossil Tooth} and {Bone Mineral}: {An FTIR Study Using}
the {Grinding Curve Method}. \emph{ArcheoSciences. Revue
d'archéométrie}, \emph{35, 35}, 135--141.
\url{https://doi.org/10.4000/archeosciences.3062}

\leavevmode\vadjust pre{\hypertarget{ref-asscherVariationsAtomic2011}{}}%
Asscher, Y., Weiner, S., \& Boaretto, E. (2011). Variations in {Atomic
Disorder} in {Biogenic Carbonate Hydroxyapatite Using} the {Infrared
Spectrum Grinding Curve Method}. \emph{Advanced Functional Materials},
\emph{21}(17), 3308--3313. \url{https://doi.org/10.1002/adfm.201100266}

\leavevmode\vadjust pre{\hypertarget{ref-bartholdyMultiproxyAnalysis2023}{}}%
Bartholdy, B. P., Hasselstrøm, J. B., Sørensen, L. K., Casna, M.,
Hoogland, M., Beemster, H. G., \& Henry, A. G. (2023). \emph{Multiproxy
analysis exploring patterns of diet and disease in dental calculus and
skeletal remains from a 19th century {Dutch} population}. {Zenodo}.
\url{https://doi.org/10.5281/zenodo.7649151}

\leavevmode\vadjust pre{\hypertarget{ref-bartholdyInvestigatingBiases2022}{}}%
Bartholdy, B. P., \& Henry, A. G. (2022). Investigating {Biases
Associated With Dietary Starch Incorporation} and {Retention With} an
{Oral Biofilm Model}. \emph{Frontiers in Earth Science}, \emph{10}.
\url{https://www.frontiersin.org/articles/10.3389/feart.2022.886512}

\leavevmode\vadjust pre{\hypertarget{ref-bjarnsholtVivoBiofilm2013}{}}%
Bjarnsholt, T., Alhede, M., Alhede, M., Eickhardt-Sørensen, S. R.,
Moser, C., Kühl, M., Jensen, P. Ø., \& Høiby, N. (2013). The in vivo
biofilm. \emph{Trends in Microbiology}, \emph{21}(9), 466--474.
\url{https://doi.org/10.1016/j.tim.2013.06.002}

\leavevmode\vadjust pre{\hypertarget{ref-buckleyDentalCalculus2014}{}}%
Buckley, S., Usai, D., Jakob, T., Radini, A., \& Hardy, K. (2014).
Dental {Calculus Reveals Unique Insights} into {Food Items}, {Cooking}
and {Plant Processing} in {Prehistoric Central Sudan}. \emph{PLOS ONE},
\emph{9}(7), e100808. \url{https://doi.org/10.1371/journal.pone.0100808}

\leavevmode\vadjust pre{\hypertarget{ref-burchamPatternsOral2020}{}}%
Burcham, Z. M., Garneau, N. L., Comstock, S. S., Tucker, R. M., Knight,
R., Metcalf, J. L., Genetics of Taste Lab Citizen Scientists, Miranda,
A., Reinhart, B., Meyers, D., Woltkamp, D., Boxer, E., Hutchens, J.,
Kim, K., Archer, M., McAteer, M., Huss, P., Defonseka, R., Stahle, S.,
\ldots{} Reusser, W. (2020). Patterns of {Oral Microbiota Diversity} in
{Adults} and {Children}: {A Crowdsourced Population Study}.
\emph{Scientific Reports}, \emph{10}(1), 2133.
\url{https://doi.org/10.1038/s41598-020-59016-0}

\leavevmode\vadjust pre{\hypertarget{ref-Rdecontam}{}}%
Davis, N. M., Proctor, D. M., Holmes, S. P., Relman, D. A., \& Callahan,
B. J. (2018). Simple statistical identification and removal of
contaminant sequences in marker-gene and metagenomics data.
\emph{Microbiome}, \emph{6}(1), 226.
\url{https://doi.org/10.1186/s40168-018-0605-2}

\leavevmode\vadjust pre{\hypertarget{ref-edlundBiofilmModel2013}{}}%
Edlund, A., Yang, Y., Hall, A. P., Guo, L., Lux, R., He, X., Nelson, K.
E., Nealson, K. H., Yooseph, S., Shi, W., \& McLean, J. S. (2013). An in
vitrobiofilm model system maintaining a highly reproducible species and
metabolic diversity approaching that of the human oral microbiome.
\emph{Microbiome}, \emph{1}(1), 25.
\url{https://doi.org/10.1186/2049-2618-1-25}

\leavevmode\vadjust pre{\hypertarget{ref-edlundUncoveringComplex2018}{}}%
Edlund, A., Yang, Y., Yooseph, S., He, X., Shi, W., \& McLean, J. S.
(2018). Uncovering complex microbiome activities via metatranscriptomics
during 24 hours of oral biofilm assembly and maturation.
\emph{Microbiome}, \emph{6}(1), 217.
\url{https://doi.org/10.1186/s40168-018-0591-4}

\leavevmode\vadjust pre{\hypertarget{ref-eerkensDentalCalculus2018}{}}%
Eerkens, J. W., Tushingham, S., Brownstein, K. J., Garibay, R., Perez,
K., Murga, E., Kaijankoski, P., Rosenthal, J. S., \& Gang, D. R. (2018).
Dental calculus as a source of ancient alkaloids: {Detection} of
nicotine by {LC-MS} in calculus samples from the {Americas}.
\emph{Journal of Archaeological Science: Reports}, \emph{18}, 509--515.
\url{https://doi.org/10.1016/j.jasrep.2018.02.004}

\leavevmode\vadjust pre{\hypertarget{ref-enneverCharacterizationBacterionema1978}{}}%
Ennever, J., Riggan, L. J., Vogel, J. J., \& Boyan-Salyers, B. (1978).
Characterization of {Bacterionema} matruchotii {Calcification
Nucleator}. \emph{Journal of Dental Research}, \emph{57}(4), 637--642.
\url{https://doi.org/10.1177/00220345780570041901}

\leavevmode\vadjust pre{\hypertarget{ref-extercateAAA2010}{}}%
Exterkate, R. A. M., Crielaard, W., \& Ten Cate, J. M. (2010). Different
{Response} to {Amine Fluoride} by {Streptococcus} mutans and
{Polymicrobial Biofilms} in a {Novel High-Throughput Active Attachment
Model}. \emph{Caries Research}, \emph{44}(4), 372--379.
\url{https://doi.org/10.1159/000316541}

\leavevmode\vadjust pre{\hypertarget{ref-fagernasMicrobialBiogeography2022}{}}%
Fagernäs, Z., Salazar-García, D. C., Haber Uriarte, M., Avilés
Fernández, A., Henry, A. G., Lomba Maurandi, J., Ozga, A. T., Velsko, I.
M., \& Warinner, C. (2022). Understanding the microbial biogeography of
ancient human dentitions to guide study design and interpretation.
\emph{FEMS Microbes}, \emph{3}, xtac006.
\url{https://doi.org/10.1093/femsmc/xtac006}

\leavevmode\vadjust pre{\hypertarget{ref-yatesEAGER2020}{}}%
Fellows Yates, J. A., Lamnidis, T. C., Borry, M., Valtueña, A. A.,
Fagernäs, Z., Clayton, S., Garcia, M. U., Neukamm, J., \& Peltzer, A.
(2020). Reproducible, portable, and efficient ancient genome
reconstruction with nf-core/eager. \emph{bioRxiv}, 2020.06.11.145615.
\url{https://doi.org/10.1101/2020.06.11.145615}

\leavevmode\vadjust pre{\hypertarget{ref-yatesOralMicrobiome2021}{}}%
Fellows Yates, J. A., Velsko, I. M., Aron, F., Posth, C., Hofman, C. A.,
Austin, R. M., Parker, C. E., Mann, A. E., Nägele, K., Arthur, K. W.,
Arthur, J. W., Bauer, C. C., Crevecoeur, I., Cupillard, C., Curtis, M.
C., Dalén, L., Bonilla, M. D.-Z., Fernández-Lomana, J. C. D., Drucker,
D. G., \ldots{} Warinner, C. (2021). The evolution and changing ecology
of the {African} hominid oral microbiome. \emph{Proceedings of the
National Academy of Sciences}, \emph{118}(20).
\url{https://doi.org/10.1073/pnas.2021655118}

\leavevmode\vadjust pre{\hypertarget{ref-filochePlaqueMicrocosm2007}{}}%
Filoche, S. K., Soma, K. J., \& Sissons, C. H. (2007). Caries-related
plaque microcosm biofilms developed in microplates. \emph{Oral
Microbiology and Immunology}, \emph{22}(2), 73--79.
\url{https://doi.org/10.1111/j.1399-302X.2007.00323.x}

\leavevmode\vadjust pre{\hypertarget{ref-flemmingBiofilmsEmergent2016}{}}%
Flemming, H.-C., Wingender, J., Szewzyk, U., Steinberg, P., Rice, S. A.,
\& Kjelleberg, S. (2016). Biofilms: An emergent form of bacterial life.
\emph{Nature Reviews Microbiology}, \emph{14}(9), 563--575.
\url{https://doi.org/10.1038/nrmicro.2016.94}

\leavevmode\vadjust pre{\hypertarget{ref-gloorMicrobiomeDatasets2017}{}}%
Gloor, G. B., Macklaim, J. M., Pawlowsky-Glahn, V., \& Egozcue, J. J.
(2017). Microbiome {Datasets Are Compositional}: {And This Is Not
Optional}. \emph{Frontiers in Microbiology}, \emph{8}, 2224.
\url{https://doi.org/10.3389/fmicb.2017.02224}

\leavevmode\vadjust pre{\hypertarget{ref-hardyStarchGranules2009}{}}%
Hardy, K., Blakeney, T., Copeland, L., Kirkham, J., Wrangham, R., \&
Collins, M. (2009). Starch granules, dental calculus and new
perspectives on ancient diet. \emph{Journal of Archaeological Science},
\emph{36}(2), 248--255. \url{https://doi.org/10.1016/j.jas.2008.09.015}

\leavevmode\vadjust pre{\hypertarget{ref-hayashizakiSiteSpecific2008}{}}%
Hayashizaki, J., Ban, S., Nakagaki, H., Okumura, A., Yoshii, S., \&
Robinson, C. (2008). Site specific mineral composition and
microstructure of human supra-gingival dental calculus. \emph{Archives
of Oral Biology}, \emph{53}(2), 168--174.
\url{https://doi.org/10.1016/j.archoralbio.2007.09.003}

\leavevmode\vadjust pre{\hypertarget{ref-hendyProteomicCalculus2018}{}}%
Hendy, J., Warinner, C., Bouwman, A., Collins, M. J., Fiddyment, S.,
Fischer, R., Hagan, R., Hofman, C. A., Holst, M., Chaves, E., Klaus, L.,
Larson, G., Mackie, M., McGrath, K., Mundorff, A. Z., Radini, A., Rao,
H., Trachsel, C., Velsko, I. M., \& Speller, C. F. (2018). Proteomic
evidence of dietary sources in ancient dental calculus.
\emph{Proceedings. Biological Sciences}, \emph{285}(1883), 20180977.
\url{https://doi.org/10.1098/rspb.2018.0977}

\leavevmode\vadjust pre{\hypertarget{ref-henryCalculusSyria2008}{}}%
Henry, A. G., \& Piperno, D. R. (2008). Using plant microfossils from
dental calculus to recover human diet: A case study from {Tell}
al-{Raqā}'i, {Syria}. \emph{Journal of Archaeological Science},
\emph{35}(7), 1943--1950.
\url{https://doi.org/10.1016/j.jas.2007.12.005}

\leavevmode\vadjust pre{\hypertarget{ref-jainIsolationCharacterization2013}{}}%
Jain, K., Parida, S., Mangwani, N., Dash, H. R., \& Das, S. (2013).
Isolation and characterization of biofilm-forming bacteria and
associated extracellular polymeric substances from oral cavity.
\emph{Annals of Microbiology}, \emph{63}(4), 1553--1562.
\url{https://doi.org/10.1007/s13213-013-0618-9}

\leavevmode\vadjust pre{\hypertarget{ref-jiFluorideMagnesium2000}{}}%
Ji, H., Nakagaki, H., Hayashizaki, J., Tsuboi, S., Kato, K., Toyama, A.,
Arai, K., Thuy, T. T., Ha, N. T. T., Kameyama, Y., Kirkham, J., \&
Robinson, C. (2000). Fluoride and magnesium concentrations in human
dental calculus obtained from {Japanese} and {Chinese} patients.
\emph{Archives of Oral Biology}, \emph{45}(7), 611--615.
\url{https://doi.org/10.1016/S0003-9969(00)00021-2}

\leavevmode\vadjust pre{\hypertarget{ref-jinSupragingivalCalculus2002}{}}%
Jin, Y., \& Yip, H.-K. (2002). Supragingival {Calculus}: {Formation} and
{Control}. \emph{Critical Reviews in Oral Biology \& Medicine}.
\url{https://doi.org/10.1177/154411130201300506}

\leavevmode\vadjust pre{\hypertarget{ref-kazarinaPostmedievalMicrobial2021}{}}%
Kazarina, A., Petersone-Gordina, E., Kimsis, J., Kuzmicka, J., Zayakin,
P., Griškjans, Ž., Gerhards, G., \& Ranka, R. (2021). The {Postmedieval
Latvian Oral Microbiome} in the {Context} of {Modern Dental Calculus}
and {Modern Dental Plaque Microbial Profiles}. \emph{Genes},
\emph{12}(2), 309. \url{https://doi.org/10.3390/genes12020309}

\leavevmode\vadjust pre{\hypertarget{ref-knightsSourceTracker2011}{}}%
Knights, D., Kuczynski, J., Charlson, E. S., Zaneveld, J., Mozer, M. C.,
Collman, R. G., Bushman, F. D., Knight, R., \& Kelley, S. T. (2011).
Bayesian community-wide culture-independent microbial source tracking.
\emph{Nature Methods}, \emph{8}(9), 761--763.
\url{https://doi.org/10.1038/nmeth.1650}

\leavevmode\vadjust pre{\hypertarget{ref-lemmersMiddenbeemster2013}{}}%
Lemmers, S. A. M., Schats, R., Hoogland, M. L. P., \& Waters-Rist, A.
(2013). Fysisch antropologische analyse Middenbeemster. In \emph{De
begravingen bij de Keyserkerk te Middenbeemster} (pp. 35--60).

\leavevmode\vadjust pre{\hypertarget{ref-leonardPlantMicroremains2015}{}}%
Leonard, C., Vashro, L., O'Connell, J. F., \& Henry, A. G. (2015). Plant
microremains in dental calculus as a record of plant consumption: {A}
test with {Twe} forager-horticulturalists. \emph{Journal of
Archaeological Science: Reports}, \emph{2}, 449--457.
\url{https://doi.org/10.1016/j.jasrep.2015.03.009}

\leavevmode\vadjust pre{\hypertarget{ref-BWA}{}}%
Li, H., \& Durbin, R. (2009). Fast and accurate short read alignment
with {Burrows}--{Wheeler} transform. \emph{Bioinformatics},
\emph{25}(14), 1754--1760.
\url{https://doi.org/10.1093/bioinformatics/btp324}

\leavevmode\vadjust pre{\hypertarget{ref-limSalivaMicrobiome2017}{}}%
Lim, Y., Totsika, M., Morrison, M., \& Punyadeera, C. (2017). The saliva
microbiome profiles are minimally affected by collection method or {DNA}
extraction protocols. \emph{Scientific Reports}, \emph{7}(1, 1), 8523.
\url{https://doi.org/10.1038/s41598-017-07885-3}

\leavevmode\vadjust pre{\hypertarget{ref-linANCOMBC2020}{}}%
Lin, H., \& Peddada, S. D. (2020). Analysis of compositions of
microbiomes with bias correction. \emph{Nature Communications},
\emph{11}(1, 1), 3514. \url{https://doi.org/10.1038/s41467-020-17041-7}

\leavevmode\vadjust pre{\hypertarget{ref-maHumanDiet2022}{}}%
Ma, Z., Liu, S., Li, Z., Ye, M., \& Huan, X. (2022). Human {Diet
Patterns During} the {Qijia Cultural Period}: {Integrated Evidence} of
{Stable Isotopes} and {Plant Micro-remains From} the {Lajia Site},
{Northwest China}. \emph{Frontiers in Earth Science}, \emph{10}.
\url{https://www.frontiersin.org/articles/10.3389/feart.2022.884856}

\leavevmode\vadjust pre{\hypertarget{ref-marshDentalPlaque2005}{}}%
Marsh, P. D. (2005). Dental plaque: Biological significance of a biofilm
and community life-style. \emph{Journal of Clinical Periodontology},
\emph{32}(s6), 7--15.
\url{https://doi.org/10.1111/j.1600-051X.2005.00790.x}

\leavevmode\vadjust pre{\hypertarget{ref-marshDentalPlaque2006}{}}%
Marsh, P. D. (2006). Dental plaque as a biofilm and a microbial
community -- implications for health and disease. \emph{BMC Oral
Health}, \emph{6}(S1), S14.
\url{https://doi.org/10.1186/1472-6831-6-S1-S14}

\leavevmode\vadjust pre{\hypertarget{ref-mentzerDistributionAuthigenic2014}{}}%
Mentzer, S. M., Miller, C. E., Kloos, P., Wadley, L., \& Conard, N. J.
(2014). The distribution of authigenic minerals in the {Middle Stone
Age} deposits of {Sibudu} ({South Africa}), and implications for the
preservation of archaeological features. \emph{European Society for the
Study of Human Evolution, {4thAnnual} Meeting, Florence, Italy}.

\leavevmode\vadjust pre{\hypertarget{ref-mickleburghNewInsights2012}{}}%
Mickleburgh, H. L., \& Pagán-Jiménez, J. R. (2012). New insights into
the consumption of maize and other food plants in the pre-{Columbian
Caribbean} from starch grains trapped in human dental calculus.
\emph{Journal of Archaeological Science}, \emph{39}(7), 2468--2478.
\url{https://doi.org/10.1016/j.jas.2012.02.020}

\leavevmode\vadjust pre{\hypertarget{ref-middletonVitroCalculus1965}{}}%
Middleton, J. D. (1965). Human salivary proteins and artificial calculus
formation in vitro. \emph{Archives of Oral Biology}, \emph{10}(2),
227--235. \url{https://doi.org/10.1016/0003-9969(65)90024-5}

\leavevmode\vadjust pre{\hypertarget{ref-moorerCalcificationCariogenic1993}{}}%
Moorer, W. R., Ten Cate, J. M., \& Buijs, J. F. (1993). Calcification of
a {Cariogenic Streptococcus} and of {Corynebacterium} ({Bacterionema})
matruchotii. \emph{Journal of Dental Research}, \emph{72}(6),
1021--1026. \url{https://doi.org/10.1177/00220345930720060501}

\leavevmode\vadjust pre{\hypertarget{ref-nearingAssessingVariation2020}{}}%
Nearing, J. T., DeClercq, V., Van Limbergen, J., \& Langille, M. G. I.
(2020). Assessing the {Variation} within the {Oral Microbiome} of
{Healthy Adults}. \emph{mSphere}, \emph{5}(5), e00451--20.
\url{https://doi.org/10.1128/mSphere.00451-20}

\leavevmode\vadjust pre{\hypertarget{ref-Rvegan}{}}%
Oksanen, J., Simpson, G. L., Blanchet, F. G., Kindt, R., Legendre, P.,
Minchin, P. R., O'Hara, R. B., Solymos, P., Stevens, M. H. H., Szoecs,
E., Wagner, H., Barbour, M., Bedward, M., Bolker, B., Borcard, D.,
Carvalho, G., Chirico, M., De Caceres, M., Durand, S., \ldots{} Weedon,
J. (2022). \emph{Vegan: {Community} ecology package} {[}Manual{]}.
\url{https://CRAN.R-project.org/package=vegan}

\leavevmode\vadjust pre{\hypertarget{ref-omelonReviewPhosphate2013}{}}%
Omelon, S., Ariganello, M., Bonucci, E., Grynpas, M., \& Nanci, A.
(2013). A {Review} of {Phosphate Mineral Nucleation} in {Biology} and
{Geobiology}. \emph{Calcified Tissue International}, \emph{93}(4),
382--396. \url{https://doi.org/10.1007/s00223-013-9784-9}

\leavevmode\vadjust pre{\hypertarget{ref-omoriComparativeEvaluation2021}{}}%
Omori, M., Kato-Kogoe, N., Sakaguchi, S., Fukui, N., Yamamoto, K.,
Nakajima, Y., Inoue, K., Nakano, H., Motooka, D., Nakano, T., Nakamura,
S., \& Ueno, T. (2021). Comparative evaluation of microbial profiles of
oral samples obtained at different collection time points and using
different methods. \emph{Clinical Oral Investigations}, \emph{25}(5),
2779--2789. \url{https://doi.org/10.1007/s00784-020-03592-y}

\leavevmode\vadjust pre{\hypertarget{ref-pearceConcomitantDeposition1987}{}}%
Pearce, E. I. F., \& Sissons, C. H. (1987). The {Concomitant Deposition}
of {Strontium} and {Fluoride} in {Dental Plaque}. \emph{Journal of
Dental Research}, \emph{66}(10), 1518--1522.
\url{https://doi.org/10.1177/00220345870660100101}

\leavevmode\vadjust pre{\hypertarget{ref-powerChimpCalculus2015}{}}%
Power, R. C., Salazar-Garcia, D. C., Wittig, R. M., Freiberg, M., \&
Henry, A. G. (2015). Dental calculus evidence of {Tai Forest Chimpanzee}
plant consumption and life history transitions. \emph{Scientific
Reports}, \emph{5}, 15161. \url{https://doi.org/10.1038/srep15161}

\leavevmode\vadjust pre{\hypertarget{ref-Rbase}{}}%
R Core Team. (2020). \emph{R: {A} language and environment for
statistical computing} {[}Manual{]}. {R Foundation for Statistical
Computing}; {R Foundation for Statistical Computing}.
\url{https://www.R-project.org/}

\leavevmode\vadjust pre{\hypertarget{ref-radiniDirtyTeeth2022}{}}%
Radini, A., \& Nikita, E. (2022). Beyond dirty teeth: {Integrating}
dental calculus studies with osteoarchaeological parameters.
\emph{Quaternary International}.
\url{https://doi.org/10.1016/j.quaint.2022.03.003}

\leavevmode\vadjust pre{\hypertarget{ref-reimerBacDive2022}{}}%
Reimer, L. C., Sardà Carbasse, J., Koblitz, J., Ebeling, C., Podstawka,
A., \& Overmann, J. (2022). {BacDive} in 2022: The knowledge base for
standardized bacterial and archaeal data. \emph{Nucleic Acids Research},
\emph{50}(D1), D741--D746. \url{https://doi.org/10.1093/nar/gkab961}

\leavevmode\vadjust pre{\hypertarget{ref-rohanizadehUltrastructuralStudy2005}{}}%
Rohanizadeh, R., \& LeGeros, R. Z. (2005). Ultrastructural study of
calculus--enamel and calculus--root interfaces. \emph{Archives of Oral
Biology}, \emph{50}(1), 89--96.
\url{https://doi.org/10.1016/j.archoralbio.2004.07.001}

\leavevmode\vadjust pre{\hypertarget{ref-RmixOmics}{}}%
Rohart, F., Gautier, B., Singh, A., \& Le Cao, K.-A. (2017). {mixOmics}:
{An R} package for 'omics feature selection and multiple data
integration. \emph{PLoS Computational Biology}, \emph{13}(11), e1005752.
\url{http://www.mixOmics.org}

\leavevmode\vadjust pre{\hypertarget{ref-AdapterRemovalv2}{}}%
Schubert, M., Lindgreen, S., \& Orlando, L. (2016). {AdapterRemoval} v2:
Rapid adapter trimming, identification, and read merging. \emph{BMC
Research Notes}, \emph{9}, 88.
\url{https://doi.org/10.1186/s13104-016-1900-2}

\leavevmode\vadjust pre{\hypertarget{ref-shellisSyntheticSaliva1978}{}}%
Shellis, R. P. (1978). A synthetic saliva for cultural studies of dental
plaque. \emph{Archives of Oral Biology}, \emph{23}(6), 485--489.
\url{https://doi.org/10.1016/0003-9969(78)90081-X}

\leavevmode\vadjust pre{\hypertarget{ref-sissonsMultistationPlaque1991}{}}%
Sissons, C. H., Cutress, T. W., Hoffman, M. P., \& Wakefield, J. S. J.
(1991). A {Multi-station Dental Plaque Microcosm} ({Artificial Mouth})
for the {Study} of {Plaque Growth}, {Metabolism}, {pH}, and
{Mineralization}: \emph{Journal of Dental Research}.
\url{https://doi.org/10.1177/00220345910700110301}

\leavevmode\vadjust pre{\hypertarget{ref-stahlDoublestrandedIndexing2019}{}}%
Stahl, R., Warinner, C., Velsko, I., Orfanou, E., Aron, F., \& Brandt,
G. (2019). Illumina double-stranded {DNA} dual indexing for ancient
{DNA} v1 {[}{Data} set{]}. \emph{Protocols. Io}.

\leavevmode\vadjust pre{\hypertarget{ref-sutherlandBiofilmMatrix2001}{}}%
Sutherland, I. W. (2001). The biofilm matrix -- an immobilized but
dynamic microbial environment. \emph{Trends in Microbiology},
\emph{9}(5), 222--227.
\url{https://doi.org/10.1016/S0966-842X(01)02012-1}

\leavevmode\vadjust pre{\hypertarget{ref-takazoeCalciumHydroxyapatite1970}{}}%
Takazoe, I., Vogel, J., \& Ennever, J. (1970). Calcium {Hydroxyapatite
Nucleation} by {Lipid Extract} of {Bacterionema} matruchotii.
\emph{Journal of Dental Research}, \emph{49}(2), 395--398.
\url{https://doi.org/10.1177/00220345700490023301}

\leavevmode\vadjust pre{\hypertarget{ref-tianUsingDGGE2010}{}}%
Tian, Y., He, X., Torralba, M., Yooseph, S., Nelson, K. e., Lux, R.,
McLean, J. s., Yu, G., \& Shi, W. (2010). Using {DGGE} profiling to
develop a novel culture medium suitable for oral microbial communities.
\emph{Molecular Oral Microbiology}, \emph{25}(5), 357--367.
\url{https://doi.org/10.1111/j.2041-1014.2010.00585.x}

\leavevmode\vadjust pre{\hypertarget{ref-tonjumNeisseria2017}{}}%
Tønjum, T., \& van Putten, J. (2017). 179 - {Neisseria}. In J. Cohen, W.
G. Powderly, \& S. M. Opal (Eds.), \emph{Infectious {Diseases} ({Fourth
Edition})} (pp. 1553--1564.e1). {Elsevier}.
\url{https://doi.org/10.1016/B978-0-7020-6285-8.00179-9}

\leavevmode\vadjust pre{\hypertarget{ref-trompEDTACalculus2017}{}}%
Tromp, M., Buckley, H., Geber, J., \& Matisoo-Smith, E. (2017). {EDTA}
decalcification of dental calculus as an alternate means of
microparticle extraction from archaeological samples. \emph{Journal of
Archaeological Science: Reports}, \emph{14}, 461--466.
\url{https://doi.org/10.1016/j.jasrep.2017.06.035}

\leavevmode\vadjust pre{\hypertarget{ref-velskoCytokineResponse2017}{}}%
Velsko, I. M., Cruz-Almeida, Y., Huang, H., Wallet, S. M., \& Shaddox,
L. M. (2017). Cytokine response patterns to complex biofilms by
mononuclear cells discriminate patient disease status and biofilm
dysbiosis. \emph{Journal of Oral Microbiology}, \emph{9}(1), 1330645.
\url{https://doi.org/10.1080/20002297.2017.1330645}

\leavevmode\vadjust pre{\hypertarget{ref-velskoMicrobialDifferences2019}{}}%
Velsko, I. M., Fellows Yates, J. A., Aron, F., Hagan, R. W., Frantz, L.
A. F., Loe, L., Martinez, J. B. R., Chaves, E., Gosden, C., Larson, G.,
\& Warinner, C. (2019). Microbial differences between dental plaque and
historic dental calculus are related to oral biofilm maturation stage.
\emph{Microbiome}, \emph{7}(1), 102.
\url{https://doi.org/10.1186/s40168-019-0717-3}

\leavevmode\vadjust pre{\hypertarget{ref-velskoDentalCalculus2017}{}}%
Velsko, I. M., Overmyer, K. A., Speller, C., Klaus, L., Collins, M. J.,
Loe, L., Frantz, L. A. F., Sankaranarayanan, K., Lewis, C. M., Martinez,
J. B. R., Chaves, E., Coon, J. J., Larson, G., \& Warinner, C. (2017).
The dental calculus metabolome in modern and historic samples.
\emph{Metabolomics}, \emph{13}(11), 134.
\url{https://doi.org/10.1007/s11306-017-1270-3}

\leavevmode\vadjust pre{\hypertarget{ref-velskoConsistentReproducible2018}{}}%
Velsko, I. M., \& Shaddox, L. M. (2018). Consistent and reproducible
long-term in vitro growth of health and disease-associated oral
subgingival biofilms. \emph{BMC Microbiology}, \emph{18}(1), 70.
\url{https://doi.org/10.1186/s12866-018-1212-x}

\leavevmode\vadjust pre{\hypertarget{ref-warinnerPathogensHost2014}{}}%
Warinner, C., Rodrigues, J. F., Vyas, R., Trachsel, C., Shved, N.,
Grossmann, J., Radini, A., Hancock, Y., Tito, R. Y., Fiddyment, S.,
Speller, C., Hendy, J., Charlton, S., Luder, H. U., Salazar-Garcia, D.
C., Eppler, E., Seiler, R., Hansen, L. H., Castruita, J. A., \ldots{}
Cappellini, E. (2014). Pathogens and host immunity in the ancient human
oral cavity. \emph{Nature Genetics}, \emph{46}(4), 336--344.
\url{https://doi.org/10.1038/ng.2906}

\leavevmode\vadjust pre{\hypertarget{ref-warinnerNewEra2015}{}}%
Warinner, C., Speller, C., \& Collins, M. J. (2015). A new era in
palaeomicrobiology: Prospects for ancient dental calculus as a long-term
record of the human oral microbiome. \emph{Philosophical Transactions of
the Royal Society B: Biological Sciences}, \emph{370}(1660), 20130376.
\url{https://doi.org/10.1098/rstb.2013.0376}

\leavevmode\vadjust pre{\hypertarget{ref-weinerBiologicalMaterials2010}{}}%
Weiner, S. (2010a). Biological {Materials}: {Bones} and {Teeth}. In
\emph{Microarchaeology: {Beyond} the {Visible Archaeological Record}}
(pp. 99--134). {Cambridge University Press}.

\leavevmode\vadjust pre{\hypertarget{ref-weinerInfraredSpectroscopy2010}{}}%
Weiner, S. (2010b). Infrared {Spectroscopy} in {Archaeology}. In
\emph{Microarchaeology: {Beyond} the {Visible Archaeological Record}}
(1st ed., pp. 275--316). {Cambridge University Press}.
\url{https://doi.org/10.1017/CBO9780511811210}

\leavevmode\vadjust pre{\hypertarget{ref-weinerStatesPreservation1990}{}}%
Weiner, S., \& Bar-Yosef, O. (1990). States of preservation of bones
from prehistoric sites in the {Near East}: {A} survey. \emph{Journal of
Archaeological Science}, \emph{17}(2), 187--196.
\url{https://doi.org/10.1016/0305-4403(90)90058-D}

\leavevmode\vadjust pre{\hypertarget{ref-whiteDentalCalculus1997}{}}%
White, D. J. (1997). Dental calculus: Recent insights into occurrence,
formation, prevention, removal and oral health effects of supragingival
and subgingival deposits. \emph{European Journal of Oral Sciences},
\emph{105}(5), 508--522.
\url{https://doi.org/10.1111/j.1600-0722.1997.tb00238.x}

\leavevmode\vadjust pre{\hypertarget{ref-ggplot2}{}}%
Wickham, H. (2016). \emph{Ggplot2: {Elegant Graphics} for {Data
Analysis}}. {Springer-Verlag}. \url{https://ggplot2.tidyverse.org}

\leavevmode\vadjust pre{\hypertarget{ref-tidyverse2019}{}}%
Wickham, Hadley, Averick, M., Bryan, J., Chang, W., McGowan, L. D.,
François, R., Grolemund, G., Hayes, A., Henry, L., Hester, J., Kuhn, M.,
Pedersen, T. L., Miller, E., Bache, S. M., Müller, K., Ooms, J.,
Robinson, D., Seidel, D. P., Spinu, V., \ldots{} Yutani, H. (2019).
Welcome to the {tidyverse}. \emph{Journal of Open Source Software},
\emph{4}(43), 1686. \url{https://doi.org/10.21105/joss.01686}

\leavevmode\vadjust pre{\hypertarget{ref-wongCalciumPhosphate2002}{}}%
Wong, L., Sissons, C. H., Pearce, E. I. F., \& Cutress, T. W. (2002).
Calcium phosphate deposition in human dental plaque microcosm biofilms
induced by a ureolytic {pH-rise} procedure. \emph{Archives of Oral
Biology}, \emph{47}(11), 779--790.
\url{https://doi.org/10.1016/S0003-9969(02)00114-0}

\leavevmode\vadjust pre{\hypertarget{ref-kraken2}{}}%
Wood, D. E., Lu, J., \& Langmead, B. (2019). Improved metagenomic
analysis with {Kraken} 2. \emph{Genome Biology}, \emph{20}(1), 257.
\url{https://doi.org/10.1186/s13059-019-1891-0}

\leavevmode\vadjust pre{\hypertarget{ref-zhangMeasurementPolysaccharides1998}{}}%
Zhang, X., Bishop, P. L., \& Kupferle, M. J. (1998). Measurement of
polysaccharides and proteins in biofilm extracellular polymers.
\emph{Water Science and Technology}, \emph{37}(4), 345--348.
\url{https://doi.org/10.1016/S0273-1223(98)00127-9}

\end{CSLReferences}

\bookmarksetup{startatroot}

\hypertarget{conclusions}{%
\chapter{Conclusions}\label{conclusions}}

The bacterial profile of our model calculus is not an exact match to the
natural modern or archaeological reference calculus, but species
richness and diversity falls within a similar range as the reference
\emph{in vitro} model, and the core genera are predominantly oral. Our
model calculus had a distinct microbial profile from modern reference
calculus, but a similar mineral composition to modern and archaeological
reference calculus, consisting of carbonate hydroxyapatite and similar
levels of crystallinity and order, with a slightly higher organic phase.

Our model has many potential benefits within archaeological research,
especially since the setup does not require highly specialised
equipment, making it accessible to many labs within the archaeological
sciences. It can be used to test many fundamental aspects of the process
of incorporation, retention, and subsequent extraction of various
dietary components from archaeological dental calculus. Using an oral
biofilm model in a controlled environment with known dietary input, we
can learn more about how different methods of food processing in the
past may affect results of dental calculus analyses, and how the methods
we use may further distort this picture. Our method can be used to test
methods (e.g.~DNA, proteomics, etc.), decontamination protocols, as well
as training on these methods and protocols without depleting limited
archaeological resources. The purpose of our model is not to replace
studies conducted on archaeological and natural dental calculus, but
rather to balance limitations of each method and serve as a
complementary approach to expand our toolkit.

\bookmarksetup{startatroot}

\hypertarget{investigating-biases-associated-with-dietary-starch-incorporation-and-retention-with-an-oral-biofilm-model}{%
\chapter{Investigating Biases Associated with Dietary Starch
Incorporation and Retention with an Oral Biofilm
Model}\label{investigating-biases-associated-with-dietary-starch-incorporation-and-retention-with-an-oral-biofilm-model}}

\bookmarksetup{startatroot}

\hypertarget{byocstarch-intro}{%
\chapter{Introduction}\label{byocstarch-intro}}

Dental calculus has proven to contain a wealth of dietary information in
the form of plant microfossils
(\protect\hyperlink{ref-hardyStarchGranules2009}{Hardy et al., 2009};
\protect\hyperlink{ref-henryCalculusSyria2008}{Henry \& Piperno, 2008}),
proteins (\protect\hyperlink{ref-hendyProteomicCalculus2018}{Hendy et
al., 2018}; \protect\hyperlink{ref-warinnerEvidenceMilk2014}{Warinner,
Hendy, et al., 2014}), and other organic residues
(\protect\hyperlink{ref-buckleyDentalCalculus2014}{Buckley et al.,
2014}). This dietary information can be preserved within the mineralised
dental plaque over many millennia, providing a unique window into the
food-related behaviours of past populations
(\protect\hyperlink{ref-henryCalculusSyria2008}{Henry \& Piperno, 2008};
\protect\hyperlink{ref-jovanovicNeolithicCalculus2021}{Jovanović et al.,
2021}; \protect\hyperlink{ref-taoWheatCalculus2020}{Tao et al., 2020})
and extinct species
(\protect\hyperlink{ref-hardyNeanderthalMedics2012}{Hardy et al., 2012};
\protect\hyperlink{ref-henryNeanderthalCalculus2014}{Henry et al.,
2014}).

Until recently, only a few studies directly investigated the presence of
plant microremains in the dental calculus of archaeological remains. The
ability to extract phytoliths from the dental calculus of archaeological
fauna to investigate diet was first noted by Armitage
(\protect\hyperlink{ref-armitageExtractionIdentification1975}{1975}),
and later by Middleton and Rovner
(\protect\hyperlink{ref-middletonOpalPhytoliths1994}{1994}), and Fox and
colleagues (\protect\hyperlink{ref-foxPhytolithCalculus1996}{1996}).
Starches and phytoliths were extracted from human dental calculus by
Cummings and Magennis
(\protect\hyperlink{ref-cummingsMayanCalculus1997}{1997}).\\
In more recent years, the study of dental calculus has increased
exponentially, and the wealth of information contained within the
mineralised matrix has largely been acknowledged. The use of dental
calculus spans a wide variety of archaeological research areas, such as
oral microbiome characterisation (including pathogens) through the
analysis of DNA and proteins
(\protect\hyperlink{ref-adlerSequencingAncient2013}{Adler et al., 2013};
\protect\hyperlink{ref-warinnerPathogensHost2014}{Warinner, Rodrigues,
et al., 2014}), microbotanical remains
(\protect\hyperlink{ref-hardyStarchGranules2009}{Hardy et al., 2009};
\protect\hyperlink{ref-henryCalculusSyria2008}{Henry \& Piperno, 2008};
\protect\hyperlink{ref-mickleburghNewInsights2012}{Mickleburgh \&
Pagán-Jiménez, 2012}), other organic residues and proteins from dietary
compounds (\protect\hyperlink{ref-buckleyDentalCalculus2014}{Buckley et
al., 2014}; \protect\hyperlink{ref-hendyProteomicCalculus2018}{Hendy et
al., 2018}), and nicotine use
(\protect\hyperlink{ref-eerkensDentalCalculus2018}{Eerkens et al.,
2018}). Especially the extraction of starch granules has become a rich
source of dietary information, as starch granules have proven to
preserve well within dental calculus over a variety of geographical and
temporal ranges
(\protect\hyperlink{ref-henryNeanderthalCalculus2014}{Henry et al.,
2014}; \protect\hyperlink{ref-jovanovicNeolithicCalculus2021}{Jovanović
et al., 2021}; \protect\hyperlink{ref-pipernoStarchGrains2008}{Piperno
\& Dillehay, 2008}; \protect\hyperlink{ref-taoWheatCalculus2020}{Tao et
al., 2020}).

Despite this, our knowledge of dental calculus and the incorporation
pathways of the various markers is limited
(\protect\hyperlink{ref-radiniFoodPathways2017}{Radini et al., 2017}),
as is our knowledge of information-loss caused by these pathways.
Additionally, the methods we use to extract and analyse dental calculus,
and make inferences on past diets represent another potential source of
bias. Studies on both archaeological and modern individuals have
explored these biases in more detail. Extraction methods were tested by
Tromp and colleagues
(\protect\hyperlink{ref-trompEDTACalculus2017}{2017}), specifically
regarding decalcification using HCl or EDTA. The authors found
significantly more starches with the EDTA extraction method than the HCl
extraction method; however, as noted by the authors, comparisons
involving archaeological calculus are problematic due to variability
between and within individuals. Studies conducted on modern humans
(\protect\hyperlink{ref-leonardPlantMicroremains2015}{Leonard et al.,
2015}) and non-human primates
(\protect\hyperlink{ref-powerChimpCalculus2015}{R. C. Power et al.,
2015}; \protect\hyperlink{ref-powerRepresentativenessDental2021}{Robert
C. Power et al., 2021}) have explored how well microremains (phytoliths
and starches) extracted from dental calculus represent the actual
dietary intake. These studies are justifiably limited, despite
meticulous documentation and observation, due to unknown variables and
uncertainty involved in this kind of \emph{in vivo} research. Dental
calculus is a complex oral biofilm with a multifactorial aetiology and
variable formation rates both within and between individuals
(\protect\hyperlink{ref-haffajeeBiofilmPosition2009}{Haffajee et al.,
2009}; \protect\hyperlink{ref-jepsenCalculusRemoval2011}{Jepsen et al.,
2011}), contributing to the stochasticity of starch representation being
observed in numerous studies. Additionally, the concentration of oral
\(\alpha\)-amylase differs both between and within individuals
(\protect\hyperlink{ref-froehlichEffectOral1987}{Froehlich et al.,
1987}; \protect\hyperlink{ref-naterHumanAmylase2005}{Nater et al.,
2005}), causing different rates of hydrolysis of the starch granules
present in the oral cavity. Add to this the effects of the many
different methods of starch processing
(\protect\hyperlink{ref-hardyRecoveringInformation2018}{Hardy et al.,
2018}), as well as post-depositional processes that are still being
explored
(\protect\hyperlink{ref-graneroStarchTaphonomy2020}{García-Granero,
2020};
\protect\hyperlink{ref-mercaderExaggeratedExpectations2018}{Mercader et
al., 2018}), and it becomes clear that using dental calculus to
reconstruct diet is a highly unpredictable process.

In this exploratory study, we use an oral biofilm model to investigate
the retention of starch granules within dental calculus in a controlled
laboratory setting, allowing us full control over dietary input. Our
main questions concern the representation of granules extracted from the
calculus compared to the actual intake. How much of the original diet is
incorporated into the calculus, and how much is recovered? Is there
differential loss of information from specific dietary markers that
affects the obtained dietary information, and how does this affect the
representation of diet from extracted microremains?\\
We find that, despite the absence of \(\alpha\)-amylase in the model, a
limited proportion of the starch input is actually retained in the
calculus. We also observed a shift in the size ratios of individual
starch granules that are incorporated into the calculus, and that the
number of incorporated starch granules increases as the size of the
calculus deposit increases.

\hypertarget{refs-6}{}
\begin{CSLReferences}{1}{0}
\leavevmode\vadjust pre{\hypertarget{ref-adlerSequencingAncient2013}{}}%
Adler, C. J., Dobney, K., Weyrich, L. S., Kaidonis, J., Walker, A. W.,
Haak, W., Bradshaw, C. J., Townsend, G., Sołtysiak, A., Alt, K. W.,
Parkhill, J., \& Cooper, A. (2013). Sequencing ancient calcified dental
plaque shows changes in oral microbiota with dietary shifts of the
{Neolithic} and {Industrial} revolutions. \emph{Nature Genetics},
\emph{45}(4), 450--455, 455e1. \url{https://doi.org/10.1038/ng.2536}

\leavevmode\vadjust pre{\hypertarget{ref-armitageExtractionIdentification1975}{}}%
Armitage, P. L. (1975). The {Extraction} and {Identification} of {Opal
Phytoliths} from the {Teeth} of {Ungulates}. \emph{Journal of
Archaeological Science}, \emph{2}, 187--197.

\leavevmode\vadjust pre{\hypertarget{ref-buckleyDentalCalculus2014}{}}%
Buckley, S., Usai, D., Jakob, T., Radini, A., \& Hardy, K. (2014).
Dental {Calculus Reveals Unique Insights} into {Food Items}, {Cooking}
and {Plant Processing} in {Prehistoric Central Sudan}. \emph{PLOS ONE},
\emph{9}(7), e100808. \url{https://doi.org/10.1371/journal.pone.0100808}

\leavevmode\vadjust pre{\hypertarget{ref-eerkensDentalCalculus2018}{}}%
Eerkens, J. W., Tushingham, S., Brownstein, K. J., Garibay, R., Perez,
K., Murga, E., Kaijankoski, P., Rosenthal, J. S., \& Gang, D. R. (2018).
Dental calculus as a source of ancient alkaloids: {Detection} of
nicotine by {LC-MS} in calculus samples from the {Americas}.
\emph{Journal of Archaeological Science: Reports}, \emph{18}, 509--515.
\url{https://doi.org/10.1016/j.jasrep.2018.02.004}

\leavevmode\vadjust pre{\hypertarget{ref-foxPhytolithCalculus1996}{}}%
Fox, C. L., Juan, J., \& Albert, R. M. (1996). Phytolith analysis on
dental calculus, enamel surface, and burial soil: {Information} about
diet and paleoenvironment. \emph{American Journal of Physical
Anthropology}, \emph{101}(1), 101--113.
\url{https://doi.org/10.1002/(SICI)1096-8644(199609)101:1\%3C101::AID-AJPA7\%3E3.0.CO;2-Y}

\leavevmode\vadjust pre{\hypertarget{ref-froehlichEffectOral1987}{}}%
Froehlich, D. A., Pangborn, R. M., \& Whitaker, J. R. (1987). The effect
of oral stimulation on human parotid salivary flow rate and
alpha-amylase secretion. \emph{Physiology \& Behavior}, \emph{41}(3),
209--217. \url{https://doi.org/10.1016/0031-9384(87)90355-6}

\leavevmode\vadjust pre{\hypertarget{ref-graneroStarchTaphonomy2020}{}}%
García-Granero, J. J. (2020). Starch taphonomy, equifinality and the
importance of context: {Some} notes on the identification of food
processing through starch grain analysis. \emph{Journal of
Archaeological Science}, \emph{124}, 105267.
\url{https://doi.org/10.1016/j.jas.2020.105267}

\leavevmode\vadjust pre{\hypertarget{ref-haffajeeBiofilmPosition2009}{}}%
Haffajee, A. D., Teles, R. P., Patel, M. R., Song, X., Yaskell, T., \&
Socransky, S. S. (2009). Factors affecting human supragingival biofilm
composition. {II}. {Tooth} position. \emph{Journal of Periodontal
Research}, \emph{44}(4), 520--528.
\url{https://doi.org/10.1111/j.1600-0765.2008.01155.x}

\leavevmode\vadjust pre{\hypertarget{ref-hardyStarchGranules2009}{}}%
Hardy, K., Blakeney, T., Copeland, L., Kirkham, J., Wrangham, R., \&
Collins, M. (2009). Starch granules, dental calculus and new
perspectives on ancient diet. \emph{Journal of Archaeological Science},
\emph{36}(2), 248--255. \url{https://doi.org/10.1016/j.jas.2008.09.015}

\leavevmode\vadjust pre{\hypertarget{ref-hardyNeanderthalMedics2012}{}}%
Hardy, K., Buckley, S., Collins, M. J., Estalrrich, A., Brothwell, D.,
Copeland, L., García-Tabernero, A., García-Vargas, S., de la Rasilla,
M., Lalueza-Fox, C., Huguet, R., Bastir, M., Santamaría, D., Madella,
M., Wilson, J., Cortés, Á. F., \& Rosas, A. (2012). Neanderthal medics?
{Evidence} for food, cooking, and medicinal plants entrapped in dental
calculus. \emph{Naturwissenschaften}, \emph{99}(8), 617--626.
\url{https://doi.org/10.1007/s00114-012-0942-0}

\leavevmode\vadjust pre{\hypertarget{ref-hardyRecoveringInformation2018}{}}%
Hardy, K., Buckley, S., \& Copeland, L. (2018). Pleistocene dental
calculus: {Recovering} information on {Paleolithic} food items,
medicines, paleoenvironment and microbes. \emph{Evolutionary
Anthropology: Issues, News, and Reviews}, \emph{27}(5), 234--246.
\url{https://doi.org/10.1002/evan.21718}

\leavevmode\vadjust pre{\hypertarget{ref-hendyProteomicCalculus2018}{}}%
Hendy, J., Warinner, C., Bouwman, A., Collins, M. J., Fiddyment, S.,
Fischer, R., Hagan, R., Hofman, C. A., Holst, M., Chaves, E., Klaus, L.,
Larson, G., Mackie, M., McGrath, K., Mundorff, A. Z., Radini, A., Rao,
H., Trachsel, C., Velsko, I. M., \& Speller, C. F. (2018). Proteomic
evidence of dietary sources in ancient dental calculus.
\emph{Proceedings. Biological Sciences}, \emph{285}(1883), 20180977.
\url{https://doi.org/10.1098/rspb.2018.0977}

\leavevmode\vadjust pre{\hypertarget{ref-henryNeanderthalCalculus2014}{}}%
Henry, A. G., Brooks, A. S., \& Piperno, D. R. (2014). Plant foods and
the dietary ecology of {Neanderthals} and early modern humans.
\emph{Journal of Human Evolution}, \emph{69}, 44--54.
\url{https://doi.org/10.1016/j.jhevol.2013.12.014}

\leavevmode\vadjust pre{\hypertarget{ref-henryCalculusSyria2008}{}}%
Henry, A. G., \& Piperno, D. R. (2008). Using plant microfossils from
dental calculus to recover human diet: A case study from {Tell}
al-{Raqā}'i, {Syria}. \emph{Journal of Archaeological Science},
\emph{35}(7), 1943--1950.
\url{https://doi.org/10.1016/j.jas.2007.12.005}

\leavevmode\vadjust pre{\hypertarget{ref-jepsenCalculusRemoval2011}{}}%
Jepsen, S., Deschner, J., Braun, A., Schwarz, F., \& Eberhard, J.
(2011). Calculus removal and the prevention of its formation.
\emph{Periodontology 2000}, \emph{55}(1), 167--188.
\url{https://doi.org/10.1111/j.1600-0757.2010.00382.x}

\leavevmode\vadjust pre{\hypertarget{ref-jovanovicNeolithicCalculus2021}{}}%
Jovanović, J., Power, R. C., de Becdelièvre, C., Goude, G., \&
Stefanović, S. (2021). Microbotanical evidence for the spread of cereal
use during the {Mesolithic-Neolithic} transition in the {Southeastern
Europe} ({Danube Gorges}): {Data} from dental calculus analysis.
\emph{Journal of Archaeological Science}, \emph{125}, 105288.
\url{https://doi.org/10.1016/j.jas.2020.105288}

\leavevmode\vadjust pre{\hypertarget{ref-leonardPlantMicroremains2015}{}}%
Leonard, C., Vashro, L., O'Connell, J. F., \& Henry, A. G. (2015). Plant
microremains in dental calculus as a record of plant consumption: {A}
test with {Twe} forager-horticulturalists. \emph{Journal of
Archaeological Science: Reports}, \emph{2}, 449--457.
\url{https://doi.org/10.1016/j.jasrep.2015.03.009}

\leavevmode\vadjust pre{\hypertarget{ref-mercaderExaggeratedExpectations2018}{}}%
Mercader, J., Akeju, T., Brown, M., Bundala, M., Collins, M. J.,
Copeland, L., Crowther, A., Dunfield, P., Henry, A., Inwood, J., Itambu,
M., Kim, J.-J., Larter, S., Longo, L., Oldenburg, T., Patalano, R.,
Sammynaiken, R., Soto, M., Tyler, R., \& Xhauflair, H. (2018).
Exaggerated expectations in ancient starch research and the need for new
taphonomic and authenticity criteria. \emph{FACETS}, \emph{3}(1),
777--798. \url{https://doi.org/10.1139/facets-2017-0126}

\leavevmode\vadjust pre{\hypertarget{ref-mickleburghNewInsights2012}{}}%
Mickleburgh, H. L., \& Pagán-Jiménez, J. R. (2012). New insights into
the consumption of maize and other food plants in the pre-{Columbian
Caribbean} from starch grains trapped in human dental calculus.
\emph{Journal of Archaeological Science}, \emph{39}(7), 2468--2478.
\url{https://doi.org/10.1016/j.jas.2012.02.020}

\leavevmode\vadjust pre{\hypertarget{ref-middletonOpalPhytoliths1994}{}}%
Middleton, W. D., \& Rovner, I. (1994). Extraction of {Opal Phytoliths}
from {Herbivore Dental Calculus}. \emph{Journal of Archaeological
Science}, \emph{21}(4), 469--473.
\url{https://doi.org/10.1006/jasc.1994.1046}

\leavevmode\vadjust pre{\hypertarget{ref-naterHumanAmylase2005}{}}%
Nater, U. M., Rohleder, N., Gaab, J., Berger, S., Jud, A., Kirschbaum,
C., \& Ehlert, U. (2005). Human salivary alpha-amylase reactivity in a
psychosocial stress paradigm. \emph{International Journal of
Psychophysiology}, \emph{55}(3), 333--342.
\url{https://doi.org/10.1016/j.ijpsycho.2004.09.009}

\leavevmode\vadjust pre{\hypertarget{ref-pipernoStarchGrains2008}{}}%
Piperno, D. R., \& Dillehay, T. D. (2008). Starch grains on human teeth
reveal early broad crop diet in northern {Peru}. \emph{Proceedings of
the National Academy of Sciences}, \emph{105}(50), 19622--19627.
\url{https://doi.org/10.1073/pnas.0808752105}

\leavevmode\vadjust pre{\hypertarget{ref-powerChimpCalculus2015}{}}%
Power, R. C., Salazar-Garcia, D. C., Wittig, R. M., Freiberg, M., \&
Henry, A. G. (2015). Dental calculus evidence of {Tai Forest Chimpanzee}
plant consumption and life history transitions. \emph{Scientific
Reports}, \emph{5}, 15161. \url{https://doi.org/10.1038/srep15161}

\leavevmode\vadjust pre{\hypertarget{ref-powerRepresentativenessDental2021}{}}%
Power, Robert C., Wittig, R. M., Stone, J. R., Kupczik, K., \&
Schulz-Kornas, E. (2021). The representativeness of the dental calculus
dietary record: Insights from {Taï} chimpanzee faecal phytoliths.
\emph{Archaeological and Anthropological Sciences}, \emph{13}(6), 104.
\url{https://doi.org/10.1007/s12520-021-01342-z}

\leavevmode\vadjust pre{\hypertarget{ref-radiniFoodPathways2017}{}}%
Radini, A., Nikita, E., Buckley, S., Copeland, L., \& Hardy, K. (2017).
Beyond food: {The} multiple pathways for inclusion of materials into
ancient dental calculus. \emph{American Journal of Physical
Anthropology}, \emph{162}, 71--83.
\url{https://doi.org/10.1002/ajpa.23147}

\leavevmode\vadjust pre{\hypertarget{ref-cummingsMayanCalculus1997}{}}%
Scott Cummings, L., \& Magennis, A. (1997). A phytolith and starch
record of food and grit in {Mayan} human tooth tartar. In A. Pinilla, J.
Juan-Tresserras, \& M. J. Machado (Eds.), \emph{The {State-of-the-Art}
of {Phytoliths} in {Soils} and {Plants}}. {CSIC Press}.
\url{https://books.google.com?id=j66CDVfVhwEC}

\leavevmode\vadjust pre{\hypertarget{ref-taoWheatCalculus2020}{}}%
Tao, D., Zhang, G., Zhou, Y., \& Zhao, H. (2020). Investigating wheat
consumption based on multiple evidences: {Stable} isotope analysis on
human bone and starch grain analysis on dental calculus of humans from
the {Laodaojing} cemetery, {Central Plains}, {China}.
\emph{International Journal of Osteoarchaeology}, \emph{30}(5),
594--606. \url{https://doi.org/10.1002/oa.2884}

\leavevmode\vadjust pre{\hypertarget{ref-trompEDTACalculus2017}{}}%
Tromp, M., Buckley, H., Geber, J., \& Matisoo-Smith, E. (2017). {EDTA}
decalcification of dental calculus as an alternate means of
microparticle extraction from archaeological samples. \emph{Journal of
Archaeological Science: Reports}, \emph{14}, 461--466.
\url{https://doi.org/10.1016/j.jasrep.2017.06.035}

\leavevmode\vadjust pre{\hypertarget{ref-warinnerEvidenceMilk2014}{}}%
Warinner, C., Hendy, J., Speller, C., Cappellini, E., Fischer, R.,
Trachsel, C., Arneborg, J., Lynnerup, N., Craig, O. E., Swallow, D. M.,
Fotakis, A., Christensen, R. J., Olsen, J. V., Liebert, A., Montalva,
N., Fiddyment, S., Charlton, S., Mackie, M., Canci, A., \ldots{}
Collins, M. J. (2014). Direct evidence of milk consumption from ancient
human dental calculus. \emph{Scientific Reports}, \emph{4}, 7104.
\url{https://doi.org/10.1038/srep07104}

\leavevmode\vadjust pre{\hypertarget{ref-warinnerPathogensHost2014}{}}%
Warinner, C., Rodrigues, J. F., Vyas, R., Trachsel, C., Shved, N.,
Grossmann, J., Radini, A., Hancock, Y., Tito, R. Y., Fiddyment, S.,
Speller, C., Hendy, J., Charlton, S., Luder, H. U., Salazar-Garcia, D.
C., Eppler, E., Seiler, R., Hansen, L. H., Castruita, J. A., \ldots{}
Cappellini, E. (2014). Pathogens and host immunity in the ancient human
oral cavity. \emph{Nature Genetics}, \emph{46}(4), 336--344.
\url{https://doi.org/10.1038/ng.2906}

\end{CSLReferences}

\bookmarksetup{startatroot}

\hypertarget{materials-and-methods-1}{%
\chapter{Materials and Methods}\label{materials-and-methods-1}}

\hypertarget{biofilm-formation}{%
\section{Biofilm formation}\label{biofilm-formation}}

In this study we employ a multispecies oral biofilm model following a
modified protocol from Sissons and colleagues
(\protect\hyperlink{ref-sissonsMultistationPlaque1991}{1991}) and
Shellis (\protect\hyperlink{ref-shellisSyntheticSaliva1978}{1978}). In
brief, a biofilm inoculated with whole saliva was grown on a substrate
suspended in artificial saliva, and fed with sugar (sucrose). After
several days of growth, the biofilm was exposed to starch solutions.
Mineralisation of the biofilm was aided by exposure to a calcium
phosphate solution. After 25 days of growth, the mineralised biofilm was
collected for further analysis. The setup comprises a polypropylene 24
deepwell PCR plate (KingFisher 97003510) with a lid containing 24 pegs,
which is autoclaved at 120°C, 1 bar overpressure, for 20 mins. The
individual pegs were the substrata on which the calculus grew. Using
this system allowed for easy transfer of the growing biofilm between
saliva, feeding solutions, and mineral solutions.

The artificial saliva (AS) is a modified version of the basal medium
mucin (BMM) described by Sissons and colleagues
(\protect\hyperlink{ref-sissonsMultistationPlaque1991}{1991}). It
contains 2.5 g/l partially purified mucin from porcine stomach (Type
III, Sigma M1778), 5 g/l trypticase peptone (Roth 2363.1), 10 g/l
proteose peptone (Oxoid LP0085), 5 g/l yeast extract (BD 211921), 2.5
g/l KCl, 0.35 g/l NaCl, 1.8 mmol/l CaCl\textsubscript{2}, 5.2 mmol/l
Na\textsubscript{2}HPO\textsubscript{4}
(\protect\hyperlink{ref-sissonsMultistationPlaque1991}{Sissons et al.,
1991}), 6.4 mmol/l NaHCO\textsubscript{3}
(\protect\hyperlink{ref-shellisSyntheticSaliva1978}{Shellis, 1978}), 2.5
mg/l haemin. This is subsequently adjusted to pH 7 with NaOH pellets and
stirring, autoclaved (15 min, 120°C, 1 bar overpressure), and
supplemented with 5.8 \(\mu\)mol/l menadione, 5 mmol/l urea, and 1
mmol/l arginine
(\protect\hyperlink{ref-sissonsMultistationPlaque1991}{Sissons et al.,
1991}).

Fresh whole saliva (WS) for inoculation was provided by a 31-year-old
male donor with no history of caries, who abstained from oral hygiene
for 24 hours. No food was consumed two hours prior to donation and no
antibiotics were taken up to six months prior to donation. The saliva
was filtered through a sterilised (with sodium hypochlorite, 10--15\%
active chlorine) nylon cloth to remove particulates. Substrata were
inoculated with 1 ml/well of a two-fold dilution of WS in sterilised
20\% (v/v) glycerine for four hours at 36°C, to allow attachment of the
salivary pellicle and plaque-forming bacteria. After initial
inoculation, the substrata were transferred to a new plate containing 1
ml/well AS and incubated in a shaking incubator (Infors HT Ecotron) at
36°C, 30 rpm. The inoculation process was repeated on days 3 and 5. AS
was partially refreshed once per day and fully refreshed every three
days, throughout the experiment, by transferring the substrata to a new
plate containing stock AS. To feed the bacteria, the substrata were
transferred to a new plate, containing 5\% (w/v) sucrose, for six
minutes twice daily, except on inoculation days (days 0, 3, and 5),
where the samples only received one sucrose treatment after inoculation.

Starch treatments were initiated on day 9 to avoid starch granule counts
being affected by \(\alpha\)-amylase hydrolysis from saliva inoculation
days. An \(\alpha\)-amylase (EC 3.2.1.1) activity assay was conducted to
confirm that no amylase was present in the model before starch
treatments started. Starch treatments replaced sucrose treatments,
occurring twice per day for six minutes. The starch treatments involved
transferring the substrata to a new plate containing a 0.25\% (w/v)
starch from potato (Roth 9441.1) solution, a 0.25\% (w/v) starch from
wheat (Sigma S5127) solution, and a 0.5\% (w/v) mixture of equal
concentrations (w/v) wheat and potato. All starch treatments were
created in dH\textsubscript{2}O with 5\% (w/v) sucrose. Before
transferring biofilm samples to the starch treatment plate, the plates
were agitated to keep the starches in suspension in the solutions.
During treatments, the rpm was increased to 60 to facilitate contact
between starch granules and biofilms.

After 15 days, mineralisation was encouraged with a calcium phosphate
monofluorophosphate urea (CPMU) solution containing 20 mmol/l
CaCl\textsubscript{2}, 12 mmol/l
NaH\textsubscript{2}PO\textsubscript{4}, 5 mmol/l
Na\textsubscript{2}PO\textsubscript{3}F, 500 mmol/l urea, and (0.04 g/l
MgCl) (\protect\hyperlink{ref-pearceConcomitantDeposition1987}{Pearce \&
Sissons, 1987};
\protect\hyperlink{ref-sissonsMultistationPlaque1991}{Sissons et al.,
1991}). The substrata were submerged in 1 ml/well CPMU for six minutes,
five times daily, in a two-hour cycle. During the mineralisation period,
starch treatments were reduced to once per day after the five CPMU
treatments. This process was repeated for 10 days until the end of the
experiment on day 24 (see Figure @ref(fig:protocol-fig) for an overview
of the protocol).

All laboratory work was conducted in sterile conditions under a laminar
flow hood to prevent starch and bacterial contamination. Control samples
that only received sucrose as a treatment were included to detect starch
contamination from the environment or cross-contamination from other
wells in the same plate.

\hypertarget{amylase-activity-detection}{%
\section{Amylase activity detection}\label{amylase-activity-detection}}

An \(\alpha\)-amylase (EC 3.2.1.1) activity assay was conducted on
artificial saliva samples collected from the plate wells on days 3, 6,
8, 9, 10, 12, and 14. Whole saliva samples were collected on days 0, 3,
and 5 as positive controls. Collected samples were stored at 4°C until
the assay was conducted on day 18. All samples and standard curves were
run in triplicates on two separate plates. Positive control saliva
samples were compared against a standard curve containing
H\textsubscript{2}O, while artificial saliva samples were compared
against a standard curve containing stock AS (due to the colour of
artificial saliva). Two photometric readings were conducted for each
plate with a 540 nm filter on a Multiskan FC Microplate Photometer
(Thermo Scientific 51119000). The protocol is a modified version of an
Enzymatic Assay of \(\alpha\)-Amylase
(\url{https://www.sigmaaldrich.com/NL/en/technical-documents/protocol/protein-biology/enzyme-activity-assays/enzymatic-assay-of-a-amylase})
(\protect\hyperlink{ref-bernfeldAmylase1955}{Bernfeld, 1955}), which
measures the amount of maltose released from starch by
\(\alpha\)-amylase activity. Results are reported in units (U) per mL
enzyme, where 1 U releases 1 \(\mu\)mole of maltose in 6 minutes. The
detailed protocol can be found here:
\url{https://www.protocols.io/view/amylase-activity-bw8jphun}.

\hypertarget{treatment-solutions}{%
\section{Treatment solutions}\label{treatment-solutions}}

A 1 ml aliquot of each starch solution was taken, from which 10 \(\mu\)l
was mounted on a microscope slide with an 18 x 18 mm coverslip, and
counted under a light microscope (Zeiss Axioscope A1). For wheat and
mixed treatment samples, we counted three slide transects (at ca. 1/4,
1/2, and 3/4 of the slide), and the sample counts were extrapolated to
the total number of granules exposed to the samples over 16 days of
treatments (see Supplementary Material for more details). For potato
treatment samples, the whole slide was counted.

\hypertarget{extraction-method}{%
\section{Extraction method}\label{extraction-method}}

Extraction of starches from the calculus samples was performed by
dissolving the calculus in 0.5 \(\tiny{M}\) ethylenediaminetetraacetic
acid (EDTA) (\protect\hyperlink{ref-lemoyneCalculusPretreatments2021}{Le
Moyne \& Crowther, 2021};
\protect\hyperlink{ref-modiCalculusMethodologies2020}{Modi et al.,
2020}; \protect\hyperlink{ref-trompEDTACalculus2017}{Tromp et al.,
2017}), and vortexing for 3 days until the sample was completely
dissolved. Twenty \(\mu\)l of sample was mounted onto a slide with an
18x18 mm coverslip. When transferring the sample to the slide, the
sample was homogenised using the pipette to ensure that the counted
transects were representative of the whole slide. The count from the
slide was extrapolated to the whole sample (see Supplementary Material
for more detail).

Both wheat and potato granules were divided into three size categories:
small (\textless10 \(\mu\)m), medium (10 -- 20 \(\mu\)m), and large
(\textgreater20 \(\mu\)m).

\hypertarget{statistical-analysis}{%
\section{Statistical analysis}\label{statistical-analysis}}

Statistical analysis was conducted in R version 4.3.1 (2023-06-16)
(\protect\hyperlink{ref-Rbase}{R Core Team, 2020}) and the following
packages: tidyverse (\protect\hyperlink{ref-tidyverse2019}{Hadley
Wickham et al., 2019}), broom (\protect\hyperlink{ref-Rbroom}{Robinson
et al., 2021}), here (\protect\hyperlink{ref-Rhere}{Müller, 2020}), and
patchwork (\protect\hyperlink{ref-Rpatchwork}{Pedersen, 2020}).

To see if biofilm growth was differently affected by starch treatments,
a one-way ANOVA with sample weight as the dependent variable (DV) and
treatment as the grouping variable (GV) was conducted. To analyse
granule counts and calculate size proportions, mean counts for each
treatment were taken across both experimental plates, resulting in a
mean count for each granule size category within each treatment.

Pearson's \emph{r} was conducted on sample weight and total starch
count, as well as sample weight and starch count per mg calculus. The
total count for each sample within a treatment was standardised by
z-score to account for the differences in magnitude between the potato
and wheat counts. This was applied to total biofilm weight and starch
count per mg calculus (also z-score standardised) to account for
differences in starch concentration in the calculus (as per
\protect\hyperlink{ref-wesolowskiEvaluatingMicrofossil2010}{Wesolowski
et al., 2010}).

\hypertarget{refs-7}{}
\begin{CSLReferences}{1}{0}
\leavevmode\vadjust pre{\hypertarget{ref-bernfeldAmylase1955}{}}%
Bernfeld, P. (1955). Amylases, α and β. In \emph{Methods in
{Enzymology}} (Vol. 1, pp. 149--158). {Academic Press}.
\url{https://doi.org/10.1016/0076-6879(55)01021-5}

\leavevmode\vadjust pre{\hypertarget{ref-lemoyneCalculusPretreatments2021}{}}%
Le Moyne, C., \& Crowther, A. (2021). Effects of chemical pre-treatments
on modified starch granules: {Recommendations} for dental calculus
decalcification for ancient starch research. \emph{Journal of
Archaeological Science: Reports}, \emph{35}, 102762.
\url{https://doi.org/10.1016/j.jasrep.2020.102762}

\leavevmode\vadjust pre{\hypertarget{ref-modiCalculusMethodologies2020}{}}%
Modi, A., Pisaneschi, L., Zaro, V., Vai, S., Vergata, C., Casalone, E.,
Caramelli, D., Moggi-Cecchi, J., Mariotti Lippi, M., \& Lari, M. (2020).
Combined methodologies for gaining much information from ancient dental
calculus: Testing experimental strategies for simultaneously analysing
{DNA} and food residues. \emph{Archaeological and Anthropological
Sciences}, \emph{12}(1), 10.
\url{https://doi.org/10.1007/s12520-019-00983-5}

\leavevmode\vadjust pre{\hypertarget{ref-Rhere}{}}%
Müller, K. (2020). \emph{Here: {A} simpler way to find your files}
{[}Manual{]}. \url{https://CRAN.R-project.org/package=here}

\leavevmode\vadjust pre{\hypertarget{ref-pearceConcomitantDeposition1987}{}}%
Pearce, E. I. F., \& Sissons, C. H. (1987). The {Concomitant Deposition}
of {Strontium} and {Fluoride} in {Dental Plaque}. \emph{Journal of
Dental Research}, \emph{66}(10), 1518--1522.
\url{https://doi.org/10.1177/00220345870660100101}

\leavevmode\vadjust pre{\hypertarget{ref-Rpatchwork}{}}%
Pedersen, T. L. (2020). \emph{Patchwork: {The} composer of plots}
{[}Manual{]}. \url{https://CRAN.R-project.org/package=patchwork}

\leavevmode\vadjust pre{\hypertarget{ref-Rbase}{}}%
R Core Team. (2020). \emph{R: {A} language and environment for
statistical computing} {[}Manual{]}. {R Foundation for Statistical
Computing}; {R Foundation for Statistical Computing}.
\url{https://www.R-project.org/}

\leavevmode\vadjust pre{\hypertarget{ref-Rbroom}{}}%
Robinson, D., Hayes, A., \& Couch, S. (2021). \emph{Broom: {Convert}
statistical objects into tidy tibbles} {[}Manual{]}.
\url{https://CRAN.R-project.org/package=broom}

\leavevmode\vadjust pre{\hypertarget{ref-shellisSyntheticSaliva1978}{}}%
Shellis, R. P. (1978). A synthetic saliva for cultural studies of dental
plaque. \emph{Archives of Oral Biology}, \emph{23}(6), 485--489.
\url{https://doi.org/10.1016/0003-9969(78)90081-X}

\leavevmode\vadjust pre{\hypertarget{ref-sissonsMultistationPlaque1991}{}}%
Sissons, C. H., Cutress, T. W., Hoffman, M. P., \& Wakefield, J. S. J.
(1991). A {Multi-station Dental Plaque Microcosm} ({Artificial Mouth})
for the {Study} of {Plaque Growth}, {Metabolism}, {pH}, and
{Mineralization}: \emph{Journal of Dental Research}.
\url{https://doi.org/10.1177/00220345910700110301}

\leavevmode\vadjust pre{\hypertarget{ref-trompEDTACalculus2017}{}}%
Tromp, M., Buckley, H., Geber, J., \& Matisoo-Smith, E. (2017). {EDTA}
decalcification of dental calculus as an alternate means of
microparticle extraction from archaeological samples. \emph{Journal of
Archaeological Science: Reports}, \emph{14}, 461--466.
\url{https://doi.org/10.1016/j.jasrep.2017.06.035}

\leavevmode\vadjust pre{\hypertarget{ref-wesolowskiEvaluatingMicrofossil2010}{}}%
Wesolowski, V., Ferraz Mendonça de Souza, S. M., Reinhard, K. J., \&
Ceccantini, G. (2010). Evaluating microfossil content of dental calculus
from {Brazilian} sambaquis. \emph{Journal of Archaeological Science},
\emph{37}(6), 1326--1338.
\url{https://doi.org/10.1016/j.jas.2009.12.037}

\leavevmode\vadjust pre{\hypertarget{ref-tidyverse2019}{}}%
Wickham, H., Averick, M., Bryan, J., Chang, W., McGowan, L. D.,
François, R., Grolemund, G., Hayes, A., Henry, L., Hester, J., Kuhn, M.,
Pedersen, T. L., Miller, E., Bache, S. M., Müller, K., Ooms, J.,
Robinson, D., Seidel, D. P., Spinu, V., \ldots{} Yutani, H. (2019).
Welcome to the {tidyverse}. \emph{Journal of Open Source Software},
\emph{4}(43), 1686. \url{https://doi.org/10.21105/joss.01686}

\end{CSLReferences}

\bookmarksetup{startatroot}

\hypertarget{results-1}{%
\chapter{Results}\label{results-1}}

All samples yielded sufficient biofilm growth and starch incorporation
to be included in the analysis (Figure @ref(fig:microscope-fig)),
resulting in a total of 48 biofilm samples (two plates of 24), 45 of
which were used for analysis (three samples were set aside for later
analysis). Most control samples contained no starch granules, while some
contained negligible quantities (see Supplementary Material).

\hypertarget{no-amylase-activity-detected-in-the-model}{%
\section{No amylase activity detected in the
model}\label{no-amylase-activity-detected-in-the-model}}

No \(\alpha\)-amylase activity was detected in any of the artificial
saliva samples from any of the days that were sampled. Only positive
controls (saliva) contained amylase activity that could be detected in
the assay, ranging from 9.93 to 30.2 U/mL enzyme (full results can be
found in the Supplementary Material). The results are not comparable to
other studies presenting \(\alpha\)-amylase activity levels in humans,
as the unit definition may differ; however, they are sufficient to show
that there is no activity in the model.

\hypertarget{treatment-type-had-minimal-effect-on-biofilm-growth}{%
\section{Treatment type had minimal effect on biofilm
growth}\label{treatment-type-had-minimal-effect-on-biofilm-growth}}

A one-way ANOVA suggests that the type of starch used during the biofilm
growth period had a minimal effect on the growth of the biofilm
(expressed as total dry weight of the sample), F(3, 43) = 1.16, p =
0.335. A summary of sample weights is available in Table
@ref(tab:anova-tbl).

\hypertarget{starch-counts}{%
\section{Starch counts}\label{starch-counts}}

It was not possible to differentiate between potato and wheat starches
smaller than ca. 10 \(\mu\)m. These were counted as wheat, as we assumed
that the majority of the small granules were wheat. We make this
assumption based on the counts of small starches in the wheat-only and
potato-only solutions. Of the combined amount of small starches in these
two solutions, 99.2\% are from wheat.

The separate wheat and potato solutions were made with a 0.25\% (w/v)
starch concentration, while the mixed-starch solution was made with
0.25\% (w/v) of each starch, with a total concentration of 0.50\% (w/v).
The mixed treatment had the highest absolute count of starch granules in
solution (mean = \ensuremath{2.9\times 10^{7}}), while the biofilms
exposed to the wheat solution preserved the greatest number of granules
(mean = \ensuremath{2.77\times 10^{4}}). The potato treatment had the
lowest absolute counts in both the solution
(\ensuremath{3.02\times 10^{6}}) and in the biofilm samples (4850)
(Tables @ref(tab:solution-count-tbl) and @ref(tab:sample-count-tbl)).

\hypertarget{proportion-of-available-starches-incorporated-in-samples}{%
\subsection{Proportion of available starches incorporated in
samples}\label{proportion-of-available-starches-incorporated-in-samples}}

The proportion of total starches from the solutions that were
incorporated into the samples ranged from 0.06\% to 0.16\%, with potato
granules being more readily incorporated than wheat in both the
separated- and mixed-treatment samples (Table
@ref(tab:sample-prop-tbl)). There is an inverse relationship between the
absolute starch count in the solutions and the proportional
incorporation of starches in the biofilm samples, i.e., potato had the
lowest absolute count in solutions, but the highest proportional
incorporation, and vice versa for the mixed treatment.

Wheat incorporation was most affected in the mixed-treatment samples,
with only 0.06\% of the total available starches being incorporated into
the sample, compared to 0.16\% in the separated wheat treatment.

\hypertarget{size-ratios-differ-between-solutions-and-samples}{%
\subsection{Size ratios differ between solutions and
samples}\label{size-ratios-differ-between-solutions-and-samples}}

Overall, medium starch granules had a higher mean rate of incorporation
(0.171\%) than small (0.120\%) and large (0.066\%) starch granules
across all treatments, while large potato starches had the lowest rate
of incorporation across all treatments.

The difference in incorporation between the size categories resulted in
a change in size ratios between the original starch solutions and the
extracted samples. Large potato granules (\textgreater{} 20 \(\mu\)m)
were most affected, with a 32.3\% decrease in relative abundance in the
potato-only treatment, and a 26.5\% decrease in mixed treatments. Medium
granules increased in relative abundance across all samples, while small
granules decreased in wheat treatments and increased in potato
treatments (Figure @ref(fig:ratio-plots)).

\hypertarget{biofilm-weight-correlated-positively-with-extracted-starch-counts}{%
\subsection{Biofilm weight correlated positively with extracted starch
counts}\label{biofilm-weight-correlated-positively-with-extracted-starch-counts}}

Pearson's \emph{r} suggests a strong positive correlation between the
total weight of the biofilms and the total starch count (standardised by
z-score) extracted from the samples across treatments, \emph{r} = 0.659,
90\%CI{[}0.463, 0.794{]}, p \textless{} 0.001 (Figure
@ref(fig:cor-plot)A).

The same test was applied to total biofilm weight and starch count per
mg calculus (also standardised by z-score), resulting in a weak positive
correlation, \emph{r} = 0.3, 90\%CI{[}0.0618, 0.506{]}, p = 0.0403
(Figure @ref(fig:cor-plot)B).

\bookmarksetup{startatroot}

\hypertarget{discussion-1}{%
\chapter{Discussion}\label{discussion-1}}

Here, we have provided a method for exploring the incorporation of
dietary starches into the mineral matrix of a dental calculus biofilm
model. Our results show that a very low proportion of the starches
exposed to the biofilm during growth are retained in the mineral matrix,
and that the size of the starch granules may affect the likelihood of
incorporation. The proportions of starch granules (of all sizes) present
in the extracted samples were similar across all treatments (0.06\% to
0.16\%), despite large differences in absolute granule counts between
wheat (mean = 25,404,000) and potato (mean = 3,016,000) solutions.\\
The absolute counts, however, differed more visibly between treatments
and was proportional with the total count of granules in the treatment
solutions. Wheat and mixed solutions had the highest absolute mean count
of starch granules, and also had the highest absolute mean count of
starch granules extracted from the dental calculus (Tables
@ref(tab:solution-count-tbl) and @ref(tab:sample-count-tbl)). This
suggests that the starches that are more frequently consumed will be
present in higher quantities in the dental calculus, at least prior to
inhumation and degradation in the burial environment. Despite the low
proportion of granules recovered from the model calculus (0.06\% to
0.16\%), the absolute counts were still substantially greater than
counts recovered from archaeological remains
(\protect\hyperlink{ref-trompEDTACalculus2017}{Tromp et al., 2017};
\protect\hyperlink{ref-trompDietaryNondietary2015}{Tromp \& Dudgeon,
2015};
\protect\hyperlink{ref-wesolowskiEvaluatingMicrofossil2010}{Wesolowski
et al., 2010}), which could in part be due to the lack of oral amylase
activity in our model. Previous research conducted on dental calculus
from contemporary humans and non-human primates suggest a high level of
stochasticity involved in the retention of starch granules in dental
calculus, and that starch granules extracted from dental calculus are
underrepresented with regard to actual starch intake, which is
consistent with our findings (illustrated by high standard deviations
and low proportional incorporation). Leonard and colleagues
(\protect\hyperlink{ref-leonardPlantMicroremains2015}{2015}) found
individual calculus samples to be a poor predictor of diet in a
population, as many of the consumed plants were missing from some
individual samples, but were present in others.\\
Power and colleagues
(\protect\hyperlink{ref-powerChimpCalculus2015}{2015}) presented similar
findings in non-human primates, where phytoliths were more
representative of individual diets than starch granules. The size bias
is also consistent with the findings by Power and colleagues
(\protect\hyperlink{ref-powerChimpCalculus2015}{2015}), who found that
plants producing starches 10--20 \(\mu\)m in size were over-represented;
however, the representation of granules larger than 20 \(\mu\)m in their
study is unclear.

We have also shown that the size of the starch granules influences the
likelihood of incorporation into the calculus. Starch granules larger
than 20 \(\mu\)m in maximum length were underrepresented in the calculus
samples compared to the original starch solutions, an effect that was
consistent across all three treatments. Medium granules (10--20
\(\mu\)m) were often over-represented (Table @ref(tab:sample-prop-tbl),
and Figure @ref(fig:ratio-plots)). Large potato granules were most
affected, potentially because of the greater size-range. They can reach
up to 100 \(\mu\)m in maximum length, whereas wheat granules generally
only reach up to 35 \(\mu\)m
(\protect\hyperlink{ref-gismondiStarchGranules2019}{Gismondi et al.,
2019}; \protect\hyperlink{ref-haslamDecompositionStarch2004}{Haslam,
2004}; \protect\hyperlink{ref-seidemannStarchAtlas1966}{Seidemann, 1966,
pp. 174--176}). Granule morphology may also play a role. Large wheat
granules are lenticular and have a larger surface area compared to
volume, whereas large potato granules are ovoid and have a larger volume
compared to surface area
(\protect\hyperlink{ref-janeAnthologyStarch1994}{Jane et al., 1994};
\protect\hyperlink{ref-reichertStarchBible1913b}{Reichert, 1913, pp.
364--365}; \protect\hyperlink{ref-seidemannStarchAtlas1966}{Seidemann,
1966, pp. 174--176};
\protect\hyperlink{ref-vandeveldeStarchMorphology2002}{van de Velde et
al., 2002}). Another potentially important factor from our results is
the size of the calculus deposit. We found a strong positive correlation
between size of biofilm deposit and retained starch granules (Figure
@ref(fig:cor-plot)A), meaning larger calculus deposits contain a higher
quantity of granules; a result that contradicts findings from
archaeological contexts
(\protect\hyperlink{ref-dudgeonDietGeography2014}{Dudgeon \& Tromp,
2014};
\protect\hyperlink{ref-wesolowskiEvaluatingMicrofossil2010}{Wesolowski
et al., 2010}). When the concentration of starch granules per mg
calculus is considered, the correlation is weaker, but still present
(Figure @ref(fig:cor-plot)B). While the larger deposits contain a higher
absolute count, our findings also suggest that they contain a slightly
higher concentration of starches. This may also explain the lower mean
retention of starch granules in mixed treatments compared to wheat
treatments. Wheat treatment samples (mean = 5.53 mg) were on average
larger than mixed treatment samples (mean = 4.28 mg) (Table
@ref(tab:anova-tbl)); and while mixed treatment solutions contained the
highest mean overall granule counts, wheat treatment samples had the
highest mean starch retention. Further research is needed to determine
why this differs from previous archaeological findings.

The mechanism by which starch granules are incorporated into plaque and
calculus remains largely unknown, and few studies have directly
investigated potential mechanisms. We know that a proportion of the
starch granules entering the mouth can become trapped in the
plaque/calculus, and can be recovered from archaeological samples of
considerable age
(\protect\hyperlink{ref-buckleyDentalCalculus2014}{Buckley et al.,
2014}; \protect\hyperlink{ref-henryNeanderthalCalculus2014}{Henry et
al., 2014}; \protect\hyperlink{ref-wuDietEarliest2021}{Wu et al.,
2021}). Studies have also shown that not all starch granules come from a
dietary source. Other pathways include cross-contamination from plant
interactions in soil, such as palm phytoliths adhering to the skin of
sweet potatoes (\protect\hyperlink{ref-trompDietaryNondietary2015}{Tromp
\& Dudgeon, 2015}), or accidental ingestion not related to food
consumption (\protect\hyperlink{ref-radiniFoodPathways2017}{Radini et
al., 2017}, \protect\hyperlink{ref-radiniMedievalWomen2019}{2019}).\\
When starch granules enter the mouth, whether through ingestion of food
or accidental intake, they immediately encounter multiple obstacles. It
is likely that the bulk of starch granules are swallowed along with the
food, and are only briefly present in the oral cavity. Other granules
that are broken off during mastication may be retained in the dentition
through attachment to tooth surfaces (including plaque and dental
calculus) and mucous membranes
(\protect\hyperlink{ref-doddsCarbohydrateRetention1988}{M. W. J. Dodds
\& Edgar, 1988}; \protect\hyperlink{ref-kashketFoodRetention1991}{S.
Kashket et al., 1991}). Bacteria also have the ability to adhere to
starch granules
(\protect\hyperlink{ref-toppingResistantStarch2003}{Topping et al.,
2003}), which would allow starches to attach to bacterial communities
within the biofilm. These granules are then susceptible to mechanical
removal by the tongue, salivary clearance, and hydrolysis by
\(\alpha\)-amylase (\protect\hyperlink{ref-kashketFoodParticles1996}{S.
Kashket et al., 1996}). The susceptibility of granules to hydrolysis
depends on the crystallinity and size of the starch granule, as well as
the mode of processing. Smaller and pre-processed (e.g., cooked) starch
granules are more susceptible to enzymatic degradation, while dehydrated
starches will have a reduced susceptibility
(\protect\hyperlink{ref-bjorckStarchProcessing1984}{Björck et al.,
1984}; \protect\hyperlink{ref-francoStarchDegradation1992}{Franco et
al., 1992};
\protect\hyperlink{ref-haslamDecompositionStarch2004}{Haslam, 2004};
\protect\hyperlink{ref-henryCookingStarch2009}{Henry et al., 2009};
\protect\hyperlink{ref-lingstromStarchyFood1994}{Lingstrom et al.,
1994}). Cracks on the surface of the dental calculus, as well as
unmineralised islands and channels may also be able to contain starch
granules (\protect\hyperlink{ref-charlierSEMCalculus2010}{Charlier et
al., 2010}; \protect\hyperlink{ref-powerSEMCalculus2014}{R. C. Power et
al., 2014}; \protect\hyperlink{ref-tanCalculusUltrastructure2004}{B. T.
K. Tan, Gillam, et al., 2004}). Starch granules that are trapped in
these pockets are (at least to some extent) protected from
aforementioned clearance mechanisms, especially once a new layer of
plaque has covered the surface of the plaque/calculus. The size bias
against large granules (\textgreater20 \(\mu\)m) from both wheat and
potato (Table @ref(tab:sample-prop-tbl)) may give further credence to
this incorporation pathway, as the smaller starch granules have an
advantage over larger granules, and can be stored in larger quantities.
This was also suggested by Power and colleagues
(\protect\hyperlink{ref-powerSEMCalculus2014}{2014}), who observed
clusters of starches within dental calculus, rather than an even
distribution across the surface of the dental calculus. Granules trapped
in plaque/calculus may still be susceptible to hydrolysis, as
\(\alpha\)-amylase has the ability to bind to both tooth enamel and
bacteria within a biofilm and retain a portion of its hydrolytic
activity (\protect\hyperlink{ref-nikitkovaStarchBiofilms2013}{Nikitkova
et al., 2013};
\protect\hyperlink{ref-scannapiecoSalivaryAmylase1993}{Scannapieco et
al., 1993}; \protect\hyperlink{ref-tanBacterialViability2004}{B. T. K.
Tan, Mordan, et al., 2004};
\protect\hyperlink{ref-tanCalculusUltrastructure2004}{B. T. K. Tan,
Gillam, et al., 2004}). After the death of an individual, starches
within dental calculus are susceptible to further degradation by
post-depositional processes, depending on burial environment (pH,
temperature, moisture content, microorganisms)
(\protect\hyperlink{ref-francoStarchDegradation1992}{Franco et al.,
1992};
\protect\hyperlink{ref-graneroStarchTaphonomy2020}{García-Granero,
2020}; \protect\hyperlink{ref-haslamDecompositionStarch2004}{Haslam,
2004}; \protect\hyperlink{ref-henryCookingStarch2009}{Henry et al.,
2009}). Future study should explore how burial affects the recovery of
starch from the biofilm model.

The absence of \(\alpha\)-amylase in the model is a limitation of this
study, as the total granule counts were not subject to hydrolysis. This
would likely have reduced and affected the size ratios, as smaller
starches may be more susceptible to hydrolysis
(\protect\hyperlink{ref-francoStarchDegradation1992}{Franco et al.,
1992}; \protect\hyperlink{ref-haslamDecompositionStarch2004}{Haslam,
2004}). The absence may also affect biofilm growth due to the lack of
amylase-bacterium interactions
(\protect\hyperlink{ref-nikitkovaStarchBiofilms2013}{Nikitkova et al.,
2013}). Conversely, the model may benefit from the absence of
\(\alpha\)-amylase, because it can allow us to directly explore its
effect on starch counts in future experiments, where \(\alpha\)-amylase
can be added to the model in concentrations similar to those found in
the oral cavity
(\protect\hyperlink{ref-scannapiecoSalivaryAmylase1993}{Scannapieco et
al., 1993}). We are able to show how absolute counts in the treatments
cause a difference in incorporation. However, this was merely a
side-effect of the difference in the number of granules in potato and
wheat solutions of the same concentration (w/v). Further research should
test multiple differing concentrations of the same starch type. The use
of EDTA may also have affected counts. While previous studies have shown
negligible morphological changes caused by exposure to EDTA
(\protect\hyperlink{ref-lemoyneCalculusPretreatments2021}{Le Moyne \&
Crowther, 2021};
\protect\hyperlink{ref-modiCalculusMethodologies2020}{Modi et al.,
2020}; \protect\hyperlink{ref-trompEDTACalculus2017}{Tromp et al.,
2017}), these studies have not considered changes to separate size
categories within starch types, and whether shifts in size ratios occur
due to exposure to the pre-treatment chemicals. The total number of
granules on a slide often exceeded a number that was feasible to count
in a reasonable time period, so we calculated the total counts by
extrapolating from three slide transects. Thus, we reasonably assume
that the three transects are a good representation of the entire slide,
and that the distribution of all granules on the slide is relatively
homogeneous.\\
Finally, we only used native starches in the experimental procedure and
the results will likely differ for processed starches
(\protect\hyperlink{ref-graneroStarchTaphonomy2020}{García-Granero,
2020}). Based on the comparatively low counts obtained by Leonard and
colleagues (\protect\hyperlink{ref-leonardPlantMicroremains2015}{2015},
Supplement 2), processing and amylase may have a substantial effect on
starch granule retention in the oral cavity.

While we are unable to sufficiently address the mechanism(s) of starch
incorporation with the data obtained in this study, the dental calculus
model presented here is uniquely suited to explore these questions and
may improve interpretations of dietary practices in past populations.
Further analyses using this model can address the call for more baseline
testing of biases associated with dietary research conducted on dental
calculus (\protect\hyperlink{ref-lemoyneCalculusPretreatments2021}{Le
Moyne \& Crowther, 2021}). Our high-throughput experimental setup allows
us a higher degree of control over the factors that influence starch
incorporation and retention, such as dietary intake, differential
survivability of starches, and inter- and intra-individual variation in
plaque accumulation and mineralisation. The latter is especially
difficult to control \emph{in vivo} as it is influenced by numerous
factors including genetics, diet, salivary flow, and tooth position and
morphology
(\protect\hyperlink{ref-fagernasMicrobialBiogeography2021}{Fagernäs et
al., 2021}; \protect\hyperlink{ref-haffajeeBiofilmPosition2009}{Haffajee
et al., 2009}; \protect\hyperlink{ref-jepsenCalculusRemoval2011}{Jepsen
et al., 2011};
\protect\hyperlink{ref-proctorSpatialGradient2018}{Proctor et al.,
2018}; \protect\hyperlink{ref-simonsoroOralGeography2013}{Simón-Soro et
al., 2013}), as well as evolutionary differences
(\protect\hyperlink{ref-yatesOralMicrobiome2021}{Fellows Yates et al.,
2021}). The set of limitations for our model differ from \emph{in vivo}
methods and, as such, we expect our model to complement the results and
interpretations of existing and new \emph{in vivo} studies. It can also
facilitate training of students and researchers on methods of dental
calculus analysis, such as starch and phytolith extraction and
identification, where it can replace the use of finite archaeological
resources.

\hypertarget{refs-9}{}
\begin{CSLReferences}{1}{0}
\leavevmode\vadjust pre{\hypertarget{ref-bjorckStarchProcessing1984}{}}%
Björck, I., Asp, N.-G., Birkhed, D., Eliasson, A.-C., Sjöberg, L.-B., \&
Lundquist, I. (1984). Effects of processing on starch availability {In}
vitro and {In} vivo. {II}. {Drum-drying} of wheat flour. \emph{Journal
of Cereal Science}, \emph{2}(3), 165--178.
\url{https://doi.org/10.1016/S0733-5210(84)80030-2}

\leavevmode\vadjust pre{\hypertarget{ref-buckleyDentalCalculus2014}{}}%
Buckley, S., Usai, D., Jakob, T., Radini, A., \& Hardy, K. (2014).
Dental {Calculus Reveals Unique Insights} into {Food Items}, {Cooking}
and {Plant Processing} in {Prehistoric Central Sudan}. \emph{PLOS ONE},
\emph{9}(7), e100808. \url{https://doi.org/10.1371/journal.pone.0100808}

\leavevmode\vadjust pre{\hypertarget{ref-charlierSEMCalculus2010}{}}%
Charlier, P., Huynh-Charlier, I., Munoz, O., Billard, M., Brun, L., \&
Grandmaison, G. L. de la. (2010). The microscopic (optical and {SEM})
examination of dental calculus deposits ({DCD}). {Potential} interest in
forensic anthropology of a bio-archaeological method. \emph{Legal
Medicine}, \emph{12}(4), 163--171.
\url{https://doi.org/10.1016/j.legalmed.2010.03.003}

\leavevmode\vadjust pre{\hypertarget{ref-doddsCarbohydrateRetention1988}{}}%
Dodds, M. W. J., \& Edgar, W. M. (1988). The {Relationship Between
Plaque pH}, {Plaque Acid Anion Profiles}, and {Oral Carbohydrate
Retention After Ingestion} of {Several} '{Reference Foods}' by {Human
Subjects}. \emph{Journal of Dental Research}, \emph{67}(5), 861--865.
\url{https://doi.org/10.1177/00220345880670051301}

\leavevmode\vadjust pre{\hypertarget{ref-dudgeonDietGeography2014}{}}%
Dudgeon, J. V., \& Tromp, M. (2014). Diet, {Geography} and {Drinking
Water} in {Polynesia}: {Microfossil Research} from {Archaeological Human
Dental Calculus}, {Rapa Nui} ({Easter Island}). \emph{International
Journal of Osteoarchaeology}, \emph{24}(5), 634--648.
\url{https://doi.org/10.1002/oa.2249}

\leavevmode\vadjust pre{\hypertarget{ref-fagernasMicrobialBiogeography2021}{}}%
Fagernäs, Z., Salazar-García, D. C., Avilés, A., Haber, M., Henry, A.,
Maurandi, J. L., Ozga, A., Velsko, I. M., \& Warinner, C. (2021).
Understanding the microbial biogeography of ancient human dentitions to
guide study design and interpretation. \emph{bioRxiv},
2021.08.16.456492. \url{https://doi.org/10.1101/2021.08.16.456492}

\leavevmode\vadjust pre{\hypertarget{ref-yatesOralMicrobiome2021}{}}%
Fellows Yates, J. A., Velsko, I. M., Aron, F., Posth, C., Hofman, C. A.,
Austin, R. M., Parker, C. E., Mann, A. E., Nägele, K., Arthur, K. W.,
Arthur, J. W., Bauer, C. C., Crevecoeur, I., Cupillard, C., Curtis, M.
C., Dalén, L., Bonilla, M. D.-Z., Fernández-Lomana, J. C. D., Drucker,
D. G., \ldots{} Warinner, C. (2021). The evolution and changing ecology
of the {African} hominid oral microbiome. \emph{Proceedings of the
National Academy of Sciences}, \emph{118}(20).
\url{https://doi.org/10.1073/pnas.2021655118}

\leavevmode\vadjust pre{\hypertarget{ref-francoStarchDegradation1992}{}}%
Franco, C. M. L., Preto, S. J. do R., \& Ciacco, C. F. (1992). Factors
that {Affect} the {Enzymatic Degradation} of {Natural Starch Granules}
-{Effect} of the {Size} of the {Granules}. \emph{Starch - Stärke},
\emph{44}(11), 422--426. \url{https://doi.org/10.1002/star.19920441106}

\leavevmode\vadjust pre{\hypertarget{ref-graneroStarchTaphonomy2020}{}}%
García-Granero, J. J. (2020). Starch taphonomy, equifinality and the
importance of context: {Some} notes on the identification of food
processing through starch grain analysis. \emph{Journal of
Archaeological Science}, \emph{124}, 105267.
\url{https://doi.org/10.1016/j.jas.2020.105267}

\leavevmode\vadjust pre{\hypertarget{ref-gismondiStarchGranules2019}{}}%
Gismondi, A., D'Agostino, A., Canuti, L., Di Marco, G., Basoli, F., \&
Canini, A. (2019). Starch granules: A data collection of 40 food
species. \emph{Plant Biosystems - An International Journal Dealing with
All Aspects of Plant Biology}, \emph{153}(2), 273--279.
\url{https://doi.org/10.1080/11263504.2018.1473523}

\leavevmode\vadjust pre{\hypertarget{ref-haffajeeBiofilmPosition2009}{}}%
Haffajee, A. D., Teles, R. P., Patel, M. R., Song, X., Yaskell, T., \&
Socransky, S. S. (2009). Factors affecting human supragingival biofilm
composition. {II}. {Tooth} position. \emph{Journal of Periodontal
Research}, \emph{44}(4), 520--528.
\url{https://doi.org/10.1111/j.1600-0765.2008.01155.x}

\leavevmode\vadjust pre{\hypertarget{ref-haslamDecompositionStarch2004}{}}%
Haslam, M. (2004). The decomposition of starch grains in soils:
Implications for archaeological residue analyses. \emph{Journal of
Archaeological Science}, \emph{31}(12), 1715--1734.
\url{https://doi.org/10.1016/j.jas.2004.05.006}

\leavevmode\vadjust pre{\hypertarget{ref-henryNeanderthalCalculus2014}{}}%
Henry, A. G., Brooks, A. S., \& Piperno, D. R. (2014). Plant foods and
the dietary ecology of {Neanderthals} and early modern humans.
\emph{Journal of Human Evolution}, \emph{69}, 44--54.
\url{https://doi.org/10.1016/j.jhevol.2013.12.014}

\leavevmode\vadjust pre{\hypertarget{ref-henryCookingStarch2009}{}}%
Henry, A. G., Hudson, H. F., \& Piperno, D. R. (2009). Changes in starch
grain morphologies from cooking. \emph{Journal of Archaeological
Science}, \emph{36}(3), 915--922.
\url{https://doi.org/10.1016/j.jas.2008.11.008}

\leavevmode\vadjust pre{\hypertarget{ref-janeAnthologyStarch1994}{}}%
Jane, J.-L., Kasemsuwan, T., Leas, S., Zobel, H., \& Robyt, J. F.
(1994). Anthology of {Starch Granule Morphology} by {Scanning Electron
Microscopy}. \emph{Starch - Stärke}, \emph{46}(4), 121--129.
\url{https://doi.org/10.1002/star.19940460402}

\leavevmode\vadjust pre{\hypertarget{ref-jepsenCalculusRemoval2011}{}}%
Jepsen, S., Deschner, J., Braun, A., Schwarz, F., \& Eberhard, J.
(2011). Calculus removal and the prevention of its formation.
\emph{Periodontology 2000}, \emph{55}(1), 167--188.
\url{https://doi.org/10.1111/j.1600-0757.2010.00382.x}

\leavevmode\vadjust pre{\hypertarget{ref-kashketFoodRetention1991}{}}%
Kashket, S., Van Houte, J., Lopez, L. R., \& Stocks, S. (1991). Lack of
{Correlation Between Food Retention} on the {Human Dentition} and
{Consumer Perception} of {Food Stickiness}. \emph{Journal of Dental
Research}, \emph{70}(10), 1314--1319.
\url{https://doi.org/10.1177/00220345910700100101}

\leavevmode\vadjust pre{\hypertarget{ref-kashketFoodParticles1996}{}}%
Kashket, S., Zhang, J., \& Houte, J. V. (1996). Accumulation of
{Fermentable Sugars} and {Metabolic Acids} in {Food Particles} that
{Become Entrapped} on the {Dentition}. \emph{Journal of Dental
Research}, 8.

\leavevmode\vadjust pre{\hypertarget{ref-lemoyneCalculusPretreatments2021}{}}%
Le Moyne, C., \& Crowther, A. (2021). Effects of chemical pre-treatments
on modified starch granules: {Recommendations} for dental calculus
decalcification for ancient starch research. \emph{Journal of
Archaeological Science: Reports}, \emph{35}, 102762.
\url{https://doi.org/10.1016/j.jasrep.2020.102762}

\leavevmode\vadjust pre{\hypertarget{ref-leonardPlantMicroremains2015}{}}%
Leonard, C., Vashro, L., O'Connell, J. F., \& Henry, A. G. (2015). Plant
microremains in dental calculus as a record of plant consumption: {A}
test with {Twe} forager-horticulturalists. \emph{Journal of
Archaeological Science: Reports}, \emph{2}, 449--457.
\url{https://doi.org/10.1016/j.jasrep.2015.03.009}

\leavevmode\vadjust pre{\hypertarget{ref-lingstromStarchyFood1994}{}}%
Lingstrom, P., Birkhed, D., Ruben, J., \& Arends, J. (1994). Effect of
{Frequent Consumption} of {Starchy Food Items} on {Enamel} and {Dentin
Demineralization} and on {Plaque pH} in situ. \emph{Journal of Dental
Research}, \emph{73}(3), 652--660.
\url{https://doi.org/10.1177/00220345940730031101}

\leavevmode\vadjust pre{\hypertarget{ref-modiCalculusMethodologies2020}{}}%
Modi, A., Pisaneschi, L., Zaro, V., Vai, S., Vergata, C., Casalone, E.,
Caramelli, D., Moggi-Cecchi, J., Mariotti Lippi, M., \& Lari, M. (2020).
Combined methodologies for gaining much information from ancient dental
calculus: Testing experimental strategies for simultaneously analysing
{DNA} and food residues. \emph{Archaeological and Anthropological
Sciences}, \emph{12}(1), 10.
\url{https://doi.org/10.1007/s12520-019-00983-5}

\leavevmode\vadjust pre{\hypertarget{ref-nikitkovaStarchBiofilms2013}{}}%
Nikitkova, A. E., Haase, E. M., \& Scannapieco, F. A. (2013). Taking the
{Starch} out of {Oral Biofilm Formation}: {Molecular Basis} and
{Functional Significance} of {Salivary} α-{Amylase Binding} to {Oral
Streptococci}. \emph{Applied and Environmental Microbiology},
\emph{79}(2), 416--423. \url{https://doi.org/10.1128/AEM.02581-12}

\leavevmode\vadjust pre{\hypertarget{ref-powerChimpCalculus2015}{}}%
Power, R. C., Salazar-Garcia, D. C., Wittig, R. M., Freiberg, M., \&
Henry, A. G. (2015). Dental calculus evidence of {Tai Forest Chimpanzee}
plant consumption and life history transitions. \emph{Scientific
Reports}, \emph{5}, 15161. \url{https://doi.org/10.1038/srep15161}

\leavevmode\vadjust pre{\hypertarget{ref-powerSEMCalculus2014}{}}%
Power, R. C., Salazar-García, D. C., Wittig, R. M., \& Henry, A. G.
(2014). Assessing use and suitability of scanning electron microscopy in
the analysis of micro remains in dental calculus. \emph{Journal of
Archaeological Science}, \emph{49}, 160--169.
\url{https://doi.org/10.1016/j.jas.2014.04.016}

\leavevmode\vadjust pre{\hypertarget{ref-proctorSpatialGradient2018}{}}%
Proctor, D. M., Fukuyama, J. A., Loomer, P. M., Armitage, G. C., Lee, S.
A., Davis, N. M., Ryder, M. I., Holmes, S. P., \& Relman, D. A. (2018).
A spatial gradient of bacterial diversity in the human oral cavity
shaped by salivary flow. \emph{Nature Communications}, \emph{9}(1), 681.
\url{https://doi.org/10.1038/s41467-018-02900-1}

\leavevmode\vadjust pre{\hypertarget{ref-radiniFoodPathways2017}{}}%
Radini, A., Nikita, E., Buckley, S., Copeland, L., \& Hardy, K. (2017).
Beyond food: {The} multiple pathways for inclusion of materials into
ancient dental calculus. \emph{American Journal of Physical
Anthropology}, \emph{162}, 71--83.
\url{https://doi.org/10.1002/ajpa.23147}

\leavevmode\vadjust pre{\hypertarget{ref-radiniMedievalWomen2019}{}}%
Radini, A., Tromp, M., Beach, A., Tong, E., Speller, C., McCormick, M.,
Dudgeon, J. V., Collins, M. J., Rühli, F., Kröger, R., \& Warinner, C.
(2019). Medieval women's early involvement in manuscript production
suggested by lapis lazuli identification in dental calculus.
\emph{Science Advances}, \emph{5}(1), eaau7126.
\url{https://doi.org/10.1126/sciadv.aau7126}

\leavevmode\vadjust pre{\hypertarget{ref-reichertStarchBible1913b}{}}%
Reichert, E. T. (1913). \emph{The differentiation and specificity of
starches in relation to genera, species, etc: Stereochemistry applied to
protoplasmic processes and products, and as a strictly scientific basis
for the classification of plants and animals} (Vol. 2). {Carnegie
institution of Washington}.

\leavevmode\vadjust pre{\hypertarget{ref-scannapiecoSalivaryAmylase1993}{}}%
Scannapieco, F. A., Torres, G., \& Levine, M. J. (1993). Salivary
α-amylase: Role in dental plaque and caries formation. \emph{Critical
Reviews in Oral Biology \& Medicine}, \emph{4}(3), 301--307.

\leavevmode\vadjust pre{\hypertarget{ref-seidemannStarchAtlas1966}{}}%
Seidemann, J. (1966). \emph{St\{\textbackslash"a\}rke-{Atlas}:
{Grundlagen} der {St}\{\textbackslash"a\}rke-{Mikroskopie} und
{Beschreibung} der wichtigsten {St}\{\textbackslash"a\}rkearten}.
{Parey}.

\leavevmode\vadjust pre{\hypertarget{ref-simonsoroOralGeography2013}{}}%
Simón-Soro, A., Tomás, I., Cabrera-Rubio, R., Catalan, M. D., Nyvad, B.,
\& Mira, A. (2013). Microbial geography of the oral cavity.
\emph{Journal of Dental Research}, \emph{92}(7), 616--621.
\url{https://doi.org/10.1177/0022034513488119}

\leavevmode\vadjust pre{\hypertarget{ref-tanCalculusUltrastructure2004}{}}%
Tan, B. T. K., Gillam, D. G., Mordan, N. J., \& Galgut, P. N. (2004). A
preliminary investigation into the ultrastructure of dental calculus and
associated bacteria. \emph{Journal of Clinical Periodontology},
\emph{31}(5), 364--369.
\url{https://doi.org/10.1111/j.1600-051X.2004.00484.x}

\leavevmode\vadjust pre{\hypertarget{ref-tanBacterialViability2004}{}}%
Tan, B. T. K., Mordan, N. J., Embleton, J., Pratten, J., \& Galgut, P.
N. (2004). Study of {Bacterial Viability} within {Human Supragingival
Dental Calculus}. \emph{Journal of Periodontology}, \emph{75}(1),
23--29. \url{https://doi.org/10.1902/jop.2004.75.1.23}

\leavevmode\vadjust pre{\hypertarget{ref-toppingResistantStarch2003}{}}%
Topping, D. L., Fukushima, M., \& Bird, A. R. (2003). Resistant starch
as a prebiotic and synbiotic: State of the art. \emph{Proceedings of the
Nutrition Society}, \emph{62}(1), 171--176.
\url{https://doi.org/10.1079/PNS2002224}

\leavevmode\vadjust pre{\hypertarget{ref-trompEDTACalculus2017}{}}%
Tromp, M., Buckley, H., Geber, J., \& Matisoo-Smith, E. (2017). {EDTA}
decalcification of dental calculus as an alternate means of
microparticle extraction from archaeological samples. \emph{Journal of
Archaeological Science: Reports}, \emph{14}, 461--466.
\url{https://doi.org/10.1016/j.jasrep.2017.06.035}

\leavevmode\vadjust pre{\hypertarget{ref-trompDietaryNondietary2015}{}}%
Tromp, M., \& Dudgeon, J. V. (2015). Differentiating dietary and
non-dietary microfossils extracted from human dental calculus: The
importance of sweet potato to ancient diet on {Rapa Nui}. \emph{Journal
of Archaeological Science}, \emph{54}, 54--63.
\url{https://doi.org/10.1016/j.jas.2014.11.024}

\leavevmode\vadjust pre{\hypertarget{ref-vandeveldeStarchMorphology2002}{}}%
van de Velde, F., van Riel, J., \& Tromp, R. H. (2002). Visualisation of
starch granule morphologies using confocal scanning laser microscopy
({CSLM}). \emph{Journal of the Science of Food and Agriculture},
\emph{82}(13), 1528--1536. \url{https://doi.org/10.1002/jsfa.1165}

\leavevmode\vadjust pre{\hypertarget{ref-wesolowskiEvaluatingMicrofossil2010}{}}%
Wesolowski, V., Ferraz Mendonça de Souza, S. M., Reinhard, K. J., \&
Ceccantini, G. (2010). Evaluating microfossil content of dental calculus
from {Brazilian} sambaquis. \emph{Journal of Archaeological Science},
\emph{37}(6), 1326--1338.
\url{https://doi.org/10.1016/j.jas.2009.12.037}

\leavevmode\vadjust pre{\hypertarget{ref-wuDietEarliest2021}{}}%
Wu, Y., Tao, D., Wu, X., \& Liu, W. (2021). \emph{Diet of the earliest
modern humans in {East Asia}} {[}Preprint{]}. {In Review}.
\url{https://doi.org/10.21203/rs.3.rs-442096/v1}

\end{CSLReferences}

\bookmarksetup{startatroot}

\hypertarget{conclusion}{%
\chapter{Conclusion}\label{conclusion}}

This preliminary study shows that a very small proportion of the input
starch granules are retained in a dental calculus model. This and
previous studies have shown that calculus has a low capacity for
retention of starch granules, an effect that is compounded by diagenetic
effects in archaeological remains, resulting in low overall counts of
extracted granules. The proportion of starches consumed will in many
cases be reflected in the quantity of starches extracted from the dental
calculus---i.e., the more starch granules entering the oral cavity, the
more will be recovered from extraction---at least in modern calculus
samples unaffected by diagenesis and hydrolysis. Whether or not this
also applies to archaeological samples remains to be tested.
Additionally, we have shown that the size of granules will influence the
likelihood of incorporation, as large (\textgreater20 \(\mu\)m) starches
have a decreased incorporation rate, medium (10--20 \(\mu\)m) starches
an increased rate, and small (\textless10 \(\mu\)m) granules remained
somewhat constant. The size of calculus deposit also seems to influence
the capacity of granule incorporation; as the size of the deposit
increases, so does the absolute count of incorporated granules.\\
While we have shown multiple factors that influence the likelihood of
incorporation, the process still appears to be somewhat stochastic.
Further research is needed to make sense of the contributing factors,
and to explore the mechanisms of intra-oral starch incorporation and
retention in dental calculus. The oral biofilm model described in this
study provides a method to explore the incorporation and extraction of
dietary compounds from dental calculus in a controlled laboratory
setting. We do not expect our model to replace \emph{in vivo} methods;
instead, it can provide a complementary means to address the limitations
of \emph{in vivo} studies, and unearth the potential biases associated
with dietary research conducted on archaeological dental calculus.

\bookmarksetup{startatroot}

\hypertarget{multiproxy-analysis-exploring-patterns-of-diet-and-disease-in-dental-calculus-and-skeletal-remains-from-a-19th-century-dutch-population}{%
\chapter{Multiproxy analysis exploring patterns of diet and disease in
dental calculus and skeletal remains from a 19th century Dutch
population}\label{multiproxy-analysis-exploring-patterns-of-diet-and-disease-in-dental-calculus-and-skeletal-remains-from-a-19th-century-dutch-population}}

\hypertarget{mb11CalculusPilot}{%
\section{Introduction}\label{mb11CalculusPilot}}

Dental calculus has proven to be an excellent source of a wide variety
of information about our past. The increased accessibility and
advancement of methods in aDNA, paleoproteomics, and mass spectrometry,
has expanded our ability to identify biomarkers of diet and disease on
an increasingly large scale
(\protect\hyperlink{ref-gismondiMultidisciplinaryApproach2020}{Gismondi
et al., 2020}; \protect\hyperlink{ref-velskoDentalCalculus2017}{Velsko,
Overmyer, et al., 2017};
\protect\hyperlink{ref-warinnerEvidenceMilk2014}{Warinner, Hendy, et
al., 2014}).

One such collection of biomarkers is alkaloids, a plant-derived group of
compounds. Many alkaloids have important medicinal and psychoactive
effects in humans, and their direct detection, or detection of their
metabolites, is of great interest to archaeologists. Previous studies
have successfully recovered alkaloids in archaeological contexts,
including ceramics
(\protect\hyperlink{ref-smithDetectionOpium2018}{Smith et al., 2018}),
pipes (\protect\hyperlink{ref-raffertyCurrentResearch2012}{Rafferty et
al., 2012}), human hair
(\protect\hyperlink{ref-echeverriaNicotineHair2013}{Echeverría \&
Niemeyer, 2013};
\protect\hyperlink{ref-ogaldeIdentificationPsychoactive2009}{Ogalde et
al., 2009}), and even dental calculus employing both targeted
(\protect\hyperlink{ref-eerkensDentalCalculus2018}{Eerkens et al.,
2018}) and untargeted approaches
(\protect\hyperlink{ref-buckleyDentalCalculus2014}{Buckley et al.,
2014};
\protect\hyperlink{ref-gismondiMultidisciplinaryApproach2020}{Gismondi
et al., 2020}). Especially nicotine, the principal alkaloid in tobacco
leaves, has been widely studied in the archaeological record due to its
apparent stability and ability to survive over long periods of time
(\protect\hyperlink{ref-eerkensDentalCalculus2018}{Eerkens et al.,
2018}; \protect\hyperlink{ref-raffertyCurrentResearch2012}{Rafferty et
al., 2012};
\protect\hyperlink{ref-tushinghamHuntergathererTobacco2013}{Tushingham
et al., 2013}).

Alkaloids may enter the oral cavity via two pathways: (1) direct
incorporation through oral consumption of alkaloid-containing plants,
whether deliberate or accidental; and (2) passive diffusion as alkaloids
and other compounds are transferred from plasma to saliva, and then into
the oral cavity through the salivary glands in the hours to days
following consumption
(\protect\hyperlink{ref-coneInterpretationOral2007}{Cone \& Huestis,
2007}). The relation to plasma is why there is often a close correlation
between presence (not concentration) of drugs in oral fluid and blood
(\protect\hyperlink{ref-coneInterpretationOral2007}{Cone \& Huestis,
2007}; \protect\hyperlink{ref-milmanOralFluid2011}{Milman et al., 2011};
\protect\hyperlink{ref-willeRelationshipOral2009}{Wille et al., 2009}).
The second pathway allows the identification of parent compounds that
are not consumed orally, as long as they, or their metabolites, are
excreted through saliva. .

Many of the components involved in the formation and growth of dental
calculus originate from oral fluid. Proteins, bacteria, salts and other
compounds are transferred from saliva to biofilms on the tooth surface
(\protect\hyperlink{ref-jinSupragingivalCalculus2002}{Jin \& Yip, 2002};
\protect\hyperlink{ref-whiteDentalCalculus1997}{D. J. White, 1997}).
This may also allow various alkaloids of dietary and medicinal origin to
become incorporated in dental plaque. Dental plaque undergoes frequent
mineralisation events, ultimately causing the entrapped alkaloids and
their metabolites to become preserved within the dental calculus.
Barring intentional or accidental removal of the calculus during life,
burial, excavation, and post-excavation cleaning, the alkaloids can then
be detected by various methods to show a record of consumption during
life.

In this study we use a ultra-high-performance liquid
chromatography-tandem mass spectrometry (UHPLC-MS/MS) method that was
developed in a previous study on dental calculus from cadavers and
validated by comparing the results to compounds detected in the blood of
the same individuals
(\protect\hyperlink{ref-sorensenDrugsCalculus2021}{Sørensen et al.,
2021}). All compounds that were detected in the blood were also detected
in dental calculus, with additional compounds present in dental calculus
that were not present in blood, suggesting that dental calculus
represents a comprehensive history of consumption over a long period of
time (\protect\hyperlink{ref-sorensenDrugsCalculus2021}{Sørensen et al.,
2021}). We were able to detect both parent compounds and metabolites,
including caffeine, nicotine, theophylline, and cotinine, in the dental
calculus of individuals from a 19th century Dutch population from
Middenbeemster. By detecting these compounds we are able to show the
consumption of tea and coffee and smoking of tobacco on an individual
scale, which is also confirmed by historic documentation and
identification of pipe notches in the dentition.

\hypertarget{mb11CalculusPilot-mat}{%
\section{Materials}\label{mb11CalculusPilot-mat}}

The sample consists of 41 individuals from Middenbeemster, a 19th
century rural Dutch site. The village of Middenbeemster and the
surrounding Beemsterpolder was established in the beginning of the 17th
century, when the Beemster lake was drained to create more farmland,
mainly for the cultivation of cole seeds (de Vries 1978). In 1615, a
decision was made to build a church, and construction started in 1618
(Hakvoort 2013). The excavated cemetery is associated with the
Keyserkerk church, where the inhabitants of the Middenbeemster village
and the surrounding Beemsterpolder were buried between AD 1615 and 1866
(\protect\hyperlink{ref-lemmersMiddenbeemster2013}{Lemmers et al.,
2013}). Archival documents are available for those buried between AD
1829 and 1866, when the majority of individuals were interred
(\protect\hyperlink{ref-palmerActivityReconstruction2016}{J. L. A.
Palmer et al., 2016}). The main occupation of the inhabitants was dairy
farming, consisting largely of manual labour prior to the industrial
revolution (\protect\hyperlink{ref-aten400Jaar2012}{Aten et al., 2012};
\protect\hyperlink{ref-palmerActivityReconstruction2016}{J. L. A. Palmer
et al., 2016}).

To reduce the number of potentially confounding factors to account for
in the analysis, we preferentially selected males from the middle adult
age category (35-49 years). The sample consists of 27 males, 11 probable
males, 2 probable females, and 1 female
(Figure~\ref{fig-sample-demography}). We selected males due to a higher
occurrence of pipe notches and dental calculus deposits than females
(unpublished observation).

\begin{figure}

{\centering \includegraphics{05-article_files/figure-pdf/fig-sample-demography-1.pdf}

}

\caption{\label{fig-sample-demography}Overview of sample demography.
Left plot is the first batch and right plot is the replication batch
with 29 of the individuals from the first batch. eya = early young adult
(18-24 years); lya = late young adult (25-34 years); ma = middle adult
(35-49 years); old = old adult (50+ years). Male? = probable male;
Female? = probable female.}

\end{figure}

\hypertarget{methods}{%
\section{Methods}\label{methods}}

\hypertarget{skeletal-analysis}{%
\subsection{Skeletal analysis}\label{skeletal-analysis}}

Demographic and pathological analyses were conducted in the Laboratory
for Human Osteoarchaeology at Leiden University. Sex was estimated using
cranial and pelvic morphological traits
(\protect\hyperlink{ref-Standards1994}{Buikstra \& Ubelaker, 1994}).
Age-at-death was estimated using dental wear, auricular and pubic
surface appearance, cranial suture closure, and epiphyseal fusion
(\protect\hyperlink{ref-SucheyBrooks1990}{Brooks \& Suchey, 1990};
\protect\hyperlink{ref-buckberryAuricular2002}{Buckberry \& Chamberlain,
2002}; \protect\hyperlink{ref-Standards1994}{Buikstra \& Ubelaker,
1994}; \protect\hyperlink{ref-lovejoyAuricular1985}{Lovejoy et al.,
1985}; \protect\hyperlink{ref-meindlSutureClosure1985}{Meindl \&
Lovejoy, 1985}), and divided into the following categories: early young
adult (18-24 years), late young adult (25-34 years), middle adult (
35-49 years), old adult (50+ years).

\hypertarget{paleopathology}{%
\subsubsection{Paleopathology}\label{paleopathology}}

Pathological conditions and lesions that occur frequently in the
population were included in the analysis. Data were dichotomised to
presence/absence to allow statistical analysis. Osteoarthritis was
considered present in cases where eburnation was visible on one or more
joint surfaces. Vertebral osteophytosis is identified by marginal
lipping and/or osteophyte formation on the margin of the superior and
inferior surfaces of the vertebral body. Cribra orbitalia was diagnosed
based on the presence of pitting on the superior surface of the orbit.
No distinction was made between active or healing lesions. Degenerative
disc disease, or spondylosis, is identified as a large diffuse
depression of the superior and/or inferior surfaces of the vertebral
body (\protect\hyperlink{ref-rogersPalaeopathologyJoint2000}{J. Rogers,
2000}). Schmorl's nodes are identified as any cortical depressions on
the surface of the vertebral body. Data on chronic maxillary sinusitis
from Casna et al.
(\protect\hyperlink{ref-casnaUrbanizationRespiratory2021}{2021}) were
included in this study to assess the relationship between upper
respiratory diseases with environmental factors (i.e.~tobacco smoke,
caffeine consumption). Lesions associated with chronic maxillary
sinusitis as defined by Boocock et al.
(\protect\hyperlink{ref-boocockMaxillarySinusitis1995}{1995}) were
recorded for each individual and classified as ``pitting'',
``spicule-type bone formation'', ``remodeled spicules'', or ``white
pitted bone''. chronic maxillary sinusitis was scored as absent when the
sinus presented smooth surfaces with little or no associated pitting.

\hypertarget{dental-pathology}{%
\subsubsection{Dental pathology}\label{dental-pathology}}

Caries ratios were calculated by dividing the number of lesions by the
number of teeth scored, resulting in a single caries ratio per
individual. If the surface where the lesion originated is not visible,
i.e.~if the lesion covered multiple surfaces, this was scored as
``crown''. Calculus indices were calculated according to Greene and
colleagues
(\protect\hyperlink{ref-greeneQuantifyingCalculus2005}{2005}). Calculus
was scored with a four-stage scoring system (0-3) to score absent,
slight, moderate, and heavy calculus deposits
(\protect\hyperlink{ref-brothwellDiggingBones1981}{Brothwell, 1981}) on
the lingual, buccal (and labial), and interproximal surfaces of each
tooth. Only one score was used for the combined interproximal surfaces,
resulting in three scores per tooth (when surfaces are intact), and four
calculus indices per individual; upper anterior, upper posterior, lower
anterior, lower posterior. Each index was calculated by dividing the sum
of calculus scores for each surface by the total number of surfaces
scored in each quadrant. If a tooth could not be scored on all three
surfaces, the tooth was not included
(\protect\hyperlink{ref-greeneQuantifyingCalculus2005}{T. R. Greene et
al., 2005}). Periodontitis was scored on a visual four-stage (0-3)
scoring system according to distance from cemento-enamel junction of
each tooth to alveolar bone
(\protect\hyperlink{ref-maatManualPhysical2005}{Maat \& Mastwijk,
2005}).

\hypertarget{calculus-sampling}{%
\subsection{Calculus sampling}\label{calculus-sampling}}

Where possible, we used material that had already been sampled for a
previous study to prevent unnecessary repeated sampling of individuals.
Calculus from the previous study was sampled in a dedicated ancient DNA
laboratory at the Laboratories of Molecular Anthropology and Microbiome
Research in Norman, Oklahoma, U.S.A, using established ancient DNA
protocols. More details on the methods can be found in the published
articles (\protect\hyperlink{ref-ziesemer16SChallenges2015}{Ziesemer et
al., 2015}, \protect\hyperlink{ref-ziesemerGenomeCalculus2018}{2018}).
Of the 41 individuals that were originally included in our sample, 29
were replicated in a separate analysis only using calculus from the
previous study.\\
New dental calculus samples were taken under sterile conditions in a
positive pressure laminar flow hood in a dedicated dental calculus lab
at Leiden University. The surface of the tooth was lightly brushed with
a sterile, disposable toothbrush to get rid of surface contaminants. A
sterile dental curette was then used to scrape calculus from the tooth
onto weighing paper, which was transferred to 1.5 ml Eppendorf tubes.
All calculus samples were sent to the Department of Forensic Medicine at
Aarhus University for ultra-high-performance liquid
chromatography-tandem mass spectrometry (UHPLC-MS/MS) analysis.

\hypertarget{uhplc-msms}{%
\subsection{UHPLC-MS/MS}\label{uhplc-msms}}

The list of targeted compounds included both naturally occurring
compounds known to have been used in the past, as well as synthetic
modern drugs that did not exist at the time (e.g.~Fentanyl, MDMA,
Amphetamine). These were part of the toxicology screening for the
original method
(\protect\hyperlink{ref-sorensenDrugsCalculus2021}{Sørensen et al.,
2021}), developed on cadavers. In our study they serve as an
authentication step, as their presence in archaeological samples could
only be the result of contamination.

Briefly, samples of dental calculus were washed three times each with
one mL of methanol (MeOH), to remove surface contaminants. The wash
solutions were collected separately. The solvent was evaporated and the
residues were dissolved in 50 µL 30\% MeOH. The washed calculus was
homogenized in presence of 0.5 M citric acid using a lysing tube with
stainless steel beads. Following one hour of incubation the dissolution
extract was cleaned by weak and strong cation-exchange. After
evaporation of the elution solvent the residue was dissolved in 50 µL
30\% MeOH. The final extracts obtained from washing and dissolution of
the dental calculus were analysed by UHPLC-MS/MS using a reversed-phase
biphenyl column for chromatography. To obtain quantitative results,
isotope dilution was applied. For more details about the method and
validation, see the original study by Sørensen and colleagues
(\protect\hyperlink{ref-sorensenDrugsCalculus2021}{2021}).

\hypertarget{statistical-analysis-1}{%
\subsection{Statistical analysis}\label{statistical-analysis-1}}

All compounds and pathological conditions/lesions were converted to a
presence/absence score. Pearson product-moment correlation was applied
to the dichotomised pathological lesions (point-biserial correlation),
compound concentrations, calculus indices, and caries ratios to explore
relationships paired continuous-continuous variables and paired
continuous-binary variables. Compound concentrations were then
dichotomised to presence/absence, and the caries ratio and calculus
index for each individual were converted to an ordinal score from 0 to 4
by using quartiles. Polychoric correlation was applied to the paired
dichotomous variables and dichotomous-ordinal variables.

All statistical analysis was conducted in R version 4.3.1 (2023-06-16),
Beagle Scouts, (\protect\hyperlink{ref-Rbase}{R Core Team, 2020}). Data
wrangling was conducted with the \textbf{tidyverse}
(\protect\hyperlink{ref-tidyverse2019}{Hadley Wickham et al., 2019}) and
visualisations were created using \textbf{ggplot2}
(\protect\hyperlink{ref-ggplot2}{H. Wickham, 2016}). Polychoric
correlations were calculated with the \textbf{psych} package
(\protect\hyperlink{ref-Rpsych}{Revelle, 2022}).

\hypertarget{results-2}{%
\section{Results}\label{results-2}}

Multiple compounds were detected in the dental calculus samples.
Compounds detected at a lower concentration than the lower limit of
quantitation (LLOQ) were considered not present. Not all the compounds
detected in the first batch could be replicated in the second batch
(Table~\ref{tbl-compound-detect}). For a full list of targeted
compounds, see Supplementary Material.

\hypertarget{tbl-compound-detect}{}
\begin{longtable}[]{@{}lllr@{}}
\caption{\label{tbl-compound-detect}Target compound including whether it
was detected (TRUE) or not (FALSE) in each batch, as well as the lower
limit of quantitation (LLOQ) in ng. CBD = cannabidiol; CBN = cannabinol;
THC = tetrahydrocannabinol; THCA-A = tetrahydrocannabinolic acid A;
THCVA = tetrahydrocannabivarin acid.}\tabularnewline
\toprule\noalign{}
Compound & Batch 1 & Batch 2 & LLOQ \\
\midrule\noalign{}
\endfirsthead
\toprule\noalign{}
Compound & Batch 1 & Batch 2 & LLOQ \\
\midrule\noalign{}
\endhead
\bottomrule\noalign{}
\endlastfoot
CBD & TRUE & FALSE & 0.050 \\
CBN & TRUE & FALSE & 0.050 \\
Caffeine & TRUE & TRUE & 0.050 \\
Cocaine & TRUE & FALSE & 0.025 \\
Cotinine & TRUE & TRUE & 0.050 \\
Nicotine & TRUE & TRUE & 0.100 \\
Salicylic acid & TRUE & TRUE & 0.500 \\
THC & TRUE & FALSE & 0.100 \\
THCA-A & TRUE & FALSE & 0.025 \\
THCVA & TRUE & FALSE & 0.010 \\
Theophylline & TRUE & TRUE & 0.010 \\
\end{longtable}

The pattern we expect to see in authentic compounds representing
compounds trapped within the dental calculus, is a reduction in the
quantity from wash 1 to wash 3 as potential surface contaminants are
washed off, and then a spike in the final extraction when entrapped
compounds are released and detected.

Most plots show a large increase in extracted mass of a compound between
the calculus wash extracts (wash 1-3) and the dissolved calculus (calc).
Most samples containing theophylline and caffeine had the largest
quantity of the compound extracted from the first wash, then decreasing
in washes 2 and 3. There is an increase between wash 3 and the dissolved
calculus in all samples. The patterns are consistent across batches 1
and 2. Nicotine and cotinine have the same relative quantities in the
samples, i.e., the sample with the highest extracted quantity of
nicotine also had the highest extracted quantity of cotinine
Figure~\ref{fig-auth-plot-batch2}.

\begin{figure}

{\centering \includegraphics{05-article_files/figure-pdf/fig-auth-plot-batch2-1.pdf}

}

\caption{\label{fig-auth-plot-batch2}(A) Number of samples in which each
compound was detected in the first and second batch. (B) Quantity (ng)
of each compound extracted from each sample in batch 2. The plot
displays the extracted quantity across the three washes and final
calculus extraction (calc). Each coloured line represents a different
calculus sample. CBD = cannabidiol; CBN = cannabinol; THC =
tetrahydrocannabinol; THCA-A = tetrahydrocannabinolic acid A; THCVA =
tetrahydrocannabivarin acid.}

\end{figure}

To see if preservation of the skeletal remains had any effect on the
detection of compounds, we compare extracted quantities of compounds to
the various levels of skeletal preservation. Our results from batch 2
suggest that detection of a compound may be linked to the preservation
of the skeleton, with better preservation leading to increased
extraction quantity (Figure~\ref{fig-detection-preservation}A). We also
find a weak positive correlation between the weight of the calculus
sample and the quantity of compound extracted from the calculus
(Figure~\ref{fig-detection-preservation}B).

\begin{figure}

{\centering \includegraphics{05-article_files/figure-pdf/fig-detection-preservation-1.pdf}

}

\caption{\label{fig-detection-preservation}(A) Violin plot with overlaid
box plots depicting the distribution of extracted quantities of each
compound from batch 2 separated by state of preservation of the
skeleton. (B) Extracted quantity (ng) of compound plotted against
weights of the calculus samples from batch 2.}

\end{figure}

The presence of pipe notch(es) in an individual and concurrent detection
of nicotine and/or cotinine is used as a crude indicator of the accuracy
of the method. Only males were used in accuracy calculations, as pipe
notches are ubiquitous in males, but not in females. In batch 2, the
method was able to detect some form of tobacco in 14 of 25 individuals
with a pipe notch (56.0\%). When also considering correct the absence of
a tobacco alkaloid together with the absence of a pipe notch, the
accuracy of the method is 59.3\%. Accuracy in the old adult age category
is 100.0\%, but with only 2 individuals.

One individual---an old adult, probable female---was positive for both
nicotine and cotinine, and had no signs of a pipe notch.

\hypertarget{correlations-between-detected-alkaloids-and-diseases}{%
\subsection{Correlations between detected alkaloids and
diseases}\label{correlations-between-detected-alkaloids-and-diseases}}

For further statistical analyses, only the UHPLC-MS/MS results from
batch 2 were used, as batch 1 had multiple compounds that were not
detected in batch 2 and may have been contaminated.

\hypertarget{tbl-pearson}{}
\begin{longtable}[]{@{}
  >{\raggedright\arraybackslash}p{(\columnwidth - 16\tabcolsep) * \real{0.1566}}
  >{\raggedright\arraybackslash}p{(\columnwidth - 16\tabcolsep) * \real{0.0843}}
  >{\raggedright\arraybackslash}p{(\columnwidth - 16\tabcolsep) * \real{0.1084}}
  >{\raggedright\arraybackslash}p{(\columnwidth - 16\tabcolsep) * \real{0.0843}}
  >{\raggedright\arraybackslash}p{(\columnwidth - 16\tabcolsep) * \real{0.1084}}
  >{\raggedright\arraybackslash}p{(\columnwidth - 16\tabcolsep) * \real{0.0843}}
  >{\raggedright\arraybackslash}p{(\columnwidth - 16\tabcolsep) * \real{0.1566}}
  >{\raggedright\arraybackslash}p{(\columnwidth - 16\tabcolsep) * \real{0.1084}}
  >{\raggedright\arraybackslash}p{(\columnwidth - 16\tabcolsep) * \real{0.1084}}@{}}
\caption{\label{tbl-pearson}Pearson correlation (\emph{r}) on
dichotomous skeletal lesions and compound concentrations (ng/mg) from
the second batch. Correlations between pairs of dichotomous variables
are removed due to incompatibility with a Pearson correlation. OA =
osteoarthritis; VOP = vertebral osteophytosis; SN = Schmorl's nodes; DDD
= degenerative disc disease; CO = cribra orbitalia; CMS = chronic
maxillary sinusitis; SA = salicylic acid; PN = pipe
notches.}\tabularnewline
\toprule\noalign{}
\begin{minipage}[b]{\linewidth}\raggedright
\end{minipage} & \begin{minipage}[b]{\linewidth}\raggedright
Caries
\end{minipage} & \begin{minipage}[b]{\linewidth}\raggedright
Nicotine
\end{minipage} & \begin{minipage}[b]{\linewidth}\raggedright
SA
\end{minipage} & \begin{minipage}[b]{\linewidth}\raggedright
Calculus
\end{minipage} & \begin{minipage}[b]{\linewidth}\raggedright
PN
\end{minipage} & \begin{minipage}[b]{\linewidth}\raggedright
Theophylline
\end{minipage} & \begin{minipage}[b]{\linewidth}\raggedright
Caffeine
\end{minipage} & \begin{minipage}[b]{\linewidth}\raggedright
Cotinine
\end{minipage} \\
\midrule\noalign{}
\endfirsthead
\toprule\noalign{}
\begin{minipage}[b]{\linewidth}\raggedright
\end{minipage} & \begin{minipage}[b]{\linewidth}\raggedright
Caries
\end{minipage} & \begin{minipage}[b]{\linewidth}\raggedright
Nicotine
\end{minipage} & \begin{minipage}[b]{\linewidth}\raggedright
SA
\end{minipage} & \begin{minipage}[b]{\linewidth}\raggedright
Calculus
\end{minipage} & \begin{minipage}[b]{\linewidth}\raggedright
PN
\end{minipage} & \begin{minipage}[b]{\linewidth}\raggedright
Theophylline
\end{minipage} & \begin{minipage}[b]{\linewidth}\raggedright
Caffeine
\end{minipage} & \begin{minipage}[b]{\linewidth}\raggedright
Cotinine
\end{minipage} \\
\midrule\noalign{}
\endhead
\bottomrule\noalign{}
\endlastfoot
OA & -0.19 & -0.074 & 0.21 & 0.07 & 0.14 & 0.28 & 0.00098 & -0.067 \\
VOP & -0.061 & -0.16 & 0.34 & 0.061 & 0.25 & -0.06 & 0.013 & -0.13 \\
SN & -0.22 & 0.16 & 0.095 & 0.089 & 0.17 & 0.24 & 0.16 & 0.093 \\
DDD & 0.032 & 0.0037 & 0.19 & -0.39 & -0.077 & 0.31 & 0.06 & -0.0086 \\
CO & 0.14 & -0.051 & 0.2 & 0.14 & -0.2 & -0.11 & 0.19 & -0.065 \\
CMS & -0.18 & 0.28 & 0.0017 & -0.27 & 0.032 & 0.19 & 0.36 & 0.22 \\
Caries & & -0.13 & -0.27 & -0.19 & -0.037 & -0.16 & 0.079 & -0.16 \\
Nicotine & & & -0.21 & 0.01 & -0.014 & 0.43 & 0.14 & 0.98 \\
SA & & & & 0.14 & 0.37 & 0.038 & 0.17 & -0.17 \\
Calculus & & & & & 0.13 & -0.15 & -0.13 & 0.031 \\
PN & & & & & & -0.16 & 0.18 & -0.0068 \\
Theophylline & & & & & & & 0.51 & 0.36 \\
Caffeine & & & & & & & & 0.078 \\
\end{longtable}

Point-biserial correlation was conducted on paired continuous and
dichotomous variables, to see if any relationships exist between
extracted concentrations and other variables. The strongest
point-biserial (Pearson) correlation correlations were a near-perfect
positive correlation between cotinine and nicotine (0.982), and moderate
correlations between theophylline and nicotine (0.432), caffeine and
theophylline (0.507) (Table~\ref{tbl-pearson}).

Polychoric correlation was conducted on the dichotomised compounds and
pathological conditions, as well as the discretised dental diseases.
Salicylic acid was removed due to its ubiquitous presence in the sample,
and is likely to cause spurious correlations. Strong correlations were
found between cotinine and nicotine (0.851). Moderate correlations were
found between OA and DDD (0.484), VOP and periodontitis (0.491), SN and
cotinine (0.558), DDD and calculus (-0.421), CMS and caffeine (0.528),
caries and periodontitis (0.486), caries and theophylline (-0.491),
periodontitis and caries (0.486), periodontitis and caffeine (0.515),
nicotine and CMS (0.496), calculus and caries (0.502), age-at-death and
theophylline (-0.447), theophylline and age-at-death (-0.447), caffeine
and periodontitis (0.515), cotinine and CMS (0.425). Remaining
correlations were weak or absent (Figure~\ref{fig-polycorr}).
Correlations with age will be depressed because age was largely
controlled for in the sample selection.

\begin{figure}

{\centering \includegraphics{05-article_files/figure-pdf/fig-polycorr-1.pdf}

}

\caption{\label{fig-polycorr}Plot of the polychoric correlations
(\emph{rho}). Larger circles and increased opacity indicates a stronger
correlation coefficient. OA = osteoarthritis; VOP = vertebral
osteophytosis; SN = Schmorl's nodes; DDD = degenerative disc disease; CO
= cribra orbitalia; CMS = chronic maxillary sinusitis; SA = salicylic
acid.}

\end{figure}

\hypertarget{discussion-2}{%
\section{Discussion}\label{discussion-2}}

In this study we were able to extract and identify multiple alkaloids
and salicylic acid from the dental calculus of individuals from
Middenbeemster, a 19th century Dutch archaeological site. We applied
ultra-high-performance liquid chromatography-tandem mass spectrometry
(UHPLC-MS/MS), a method that was validated by co-occurrence of drugs and
metabolites in dental calculus and blood
(\protect\hyperlink{ref-sorensenDrugsCalculus2021}{Sørensen et al.,
2021}). Here we have shown that the method can also be successfully
applied to archaeological dental calculus. We extend findings from
previous studies on alkaloids in archaeological samples by extracting
multiple different alkaloids from dental calculus, including nicotine,
cotinine, caffeine, theophylline, and salicylic acid in multiple
individuals. The detection of these compounds was solidified in a
replication analysis on different samples from the same individuals.
Cocaine and multiple cannabinoids were also detected during the first
analysis, but were not replicated. We discuss the implications of these
findings in light of historical and archaeological evidence for the
consumption of these drugs.

Nicotine and its principal/main metabolite, cotinine, were strongly
positively correlated, both in concentration and presence/absence in
individuals (Table~\ref{tbl-pearson} and Figure~\ref{fig-polycorr}). The
detection of nicotine and cotinine is not surprising, as pipe-smoking in
the Beemsterpolder is well-documented in the literature
(\protect\hyperlink{ref-aten400Jaar2012}{Aten et al., 2012};
\protect\hyperlink{ref-boumanBegravenis2017}{Bouman, 2017}), and visible
on the skeletal remains as pipe notches
(\protect\hyperlink{ref-lemmersMiddenbeemster2013}{Lemmers et al.,
2013}). There is also documented medicinal use of nicotine in the
Beemsterpolder, where a tobacco-smoke enema was used for headaches,
respiratory problems, colds, and drowsiness from around 1780 to 1830
(\protect\hyperlink{ref-aten400Jaar2012}{Aten et al., 2012}). In our
sample, we also detected nicotine and cotinine (replicated) in an old
adult, probable female individual. In this particular case it is
unlikely that the compounds entered the dental calculus through
pipe-smoking, as the individual had no visible pipe notches; more likely
the tobacco entered through an alternate mode of consumption, secondhand
smoke, or the aforementioned tobacco-smoke enema.

Theophylline and caffeine were positively correlated in our samples,
though to a lesser extent than nicotine and cotinine, so we are unable
to determine if they originated from the same source
(Table~\ref{tbl-pearson} and Figure~\ref{fig-polycorr}). Caffeine and
theophylline have very similar chemical structures, so we expect they
would experience similar rates of incorporation and degradation,
allowing us to interpret the ratio and correlations between the
compounds. Caffeine is present in coffee, tea, and cocoa beans, with
concentrations slightly higher in coffee
(\protect\hyperlink{ref-bispoSimultaneousDetermination2002}{Bispo et
al., 2002}; \protect\hyperlink{ref-chinCaffeineContent2008}{Chin et al.,
2008}; \protect\hyperlink{ref-srdjenovicSimultaneousHPLC2008}{Srdjenovic
et al., 2008};
\protect\hyperlink{ref-stavricVariabilityCaffeine1988}{Stavric et al.,
1988}). Theophylline is present in both coffee beans and tea leaves, but
in negligible quantities
(\protect\hyperlink{ref-stavricVariabilityCaffeine1988}{Stavric et al.,
1988}). It is also a primary metabolite of caffeine produced by the
liver. Given the low correlation, there are likely multiple sources of
caffeine and theophylline in the population, with tea and coffee being
the most obvious.\\
Tea consumption had become widespread in the Netherlands by 1820,
reaching all parts of society
(\protect\hyperlink{ref-nierstraszTeaTrade2015}{Nierstrasz, 2015, p.
91}). Historically, we also know that both tea and coffee were consumed
in the Beemsterpolder during the 19th century. `Theegasten' (teatime)
was a special occasion occurring from 15.00-20.00 hours, where tea was
served along with the evening bread
(\protect\hyperlink{ref-schuijtemakerTeTheegasten2011}{Schuijtemaker,
2011}). Many households also owned at least one coffee pot and tea pot
(\protect\hyperlink{ref-boumanBegravenis2017}{Bouman, 2017}).
Distinguishing between tea, coffee, and chocolate may be possible by
also including theobromine and comparing ratios of the compounds, as
theobromine is present in higher quantities in chocolate compared to
caffeine and theophylline
(\protect\hyperlink{ref-alanonAssessmentFlavanol2016}{Alañón et al.,
2016}; \protect\hyperlink{ref-bispoSimultaneousDetermination2002}{Bispo
et al., 2002};
\protect\hyperlink{ref-stavricVariabilityCaffeine1988}{Stavric et al.,
1988}). However, In addition to oral factors affecting alkaloid uptake
in dental calculus, there is some indication that theobromine does not
preserve well in the archaeological record
(\protect\hyperlink{ref-velskoDentalCalculus2017}{Velsko, Overmyer, et
al., 2017}), and frequent consumption of all three items would be
difficult to parse.

Salicylic acid was found in all but one individual in our sample. It can
be extracted from the bark of willow trees, \emph{Salix alba}, and has
long been used for its pain-relieving properties
(\protect\hyperlink{ref-bruinsmaBijdragenTot1872}{Bruinsma, 1872, p.
119}). It is also present in many plant-based foods
(\protect\hyperlink{ref-duthieNaturalSalicylates2011}{Duthie \& Wood,
2011}; \protect\hyperlink{ref-malakarNaturallyOccurring2017}{Malakar et
al., 2017}), including potatoes, which were a staple of the
Beemsterpolder diet (\protect\hyperlink{ref-aten400Jaar2012}{Aten et
al., 2012}). The extracted quantity from our samples decreased over the
three washes, followed by a sharp increase in the final calculus
extraction, which is what we would expect to see if the salicylic acid
was incorporated during life Figure~\ref{fig-auth-plot-batch2}. However,
it has been shown that salicilyc acid is a very mobile organic acid and
the ubiquitous presence may be due to environmental contamination, which
would also explain the high quantity in the washes
(\protect\hyperlink{ref-badriRegulationFunction2009}{Badri \& Vivanco,
2009}; \protect\hyperlink{ref-chenCa2Dependent2001}{H. Chen et al.,
2001}). Given the multiple plausible sources of this residue, it will be
necessary to explore the extent to which salicylic acid can leach into
the dental calculus from the soil, and what the rate of degradation is
for salicylic acid when trapped in dental calculus.

Cannabinoids---specifically THC, THCA-A, THCVA, CBD, CBN---were found in
the first batch, but none were replicated in the second batch. Medicinal
use of cannabinoids has been well-established in Europe since
Medieval-times, and it was also grown in the Netherlands
(\protect\hyperlink{ref-bruinsmaBijdragenTot1872}{Bruinsma, 1872}).
Administration was most common in the form of concoctions containing
various portions of the cannabis plant for ingestion; not until the late
19th century did it become recommended to smoke it for more immediate
effects (\protect\hyperlink{ref-clarkeCannabisEvolution2013}{Clarke,
2013}). A Dutch medicinal use of hemp involved an emulsion prepared from
the seeds of the plants to treat pain and various stomach ailments.
Another preparation involving the roots of the plants was used for
inflammation, gout, and joint pains
(\protect\hyperlink{ref-clarkeCannabisEvolution2013}{Clarke, 2013}). The
ability to detect cannabinoids in calculus may be limited by their
reduced ability to diffuse from serum to salivary glands due to an
affinity for protein-binding,
(\protect\hyperlink{ref-coneInterpretationOral2007}{Cone \& Huestis,
2007}), meaning detection would rely on oral consumption. Even then, the
overall instability of some cannabinoids could also affect detection
(\protect\hyperlink{ref-lindholstLongTerm2010}{Lindholst, 2010};
\protect\hyperlink{ref-sorensenEffectAntioxidants2018}{Sørensen \&
Hasselstrøm, 2018}). However, given the lack of replication, we cannot
with security confirm that cannabis was used by the Beemster population.

Despite many of our sampled individuals having lived during the height
of the opium era in the Netherlands
(\protect\hyperlink{ref-machtHistoryOpium1915}{Macht, 1915}), none of
the targeted opioids (morphine, codeine, thebaine, papaverine,
norcodeine, noscapine) were detected. The absence of opioids could be a
result of the people ascribing more to the ``traditional'' rather than
``scientific'' medicine, although laudanum and another opium containing
concoction was part of the ``traditional'' medicine in the Netherlands
(\protect\hyperlink{ref-leuwProhibitionLegalization1994}{Leuw \&
Marshall, 1994}), including Middenbeemster
(\protect\hyperlink{ref-aten400Jaar2012}{Aten et al., 2012}). It was
also generally considered a drug of the upper class
(\protect\hyperlink{ref-scheltemaOpiumTrade1907}{Scheltema, 1907}), and
may have been more common in urban centers. The absence could also be
attributed to postmortem degradation. It has been shown that, while
abundant in opium, morphine degrades rapidly, while thebaine and
papaverine are more resistant to various ageing processes
(\protect\hyperlink{ref-chovanecOpiumMasses2012}{Chovanec et al.,
2012}). The latter were also absent from our samples.

The only strictly modern compound (at least in a European context)
detected in the sample was cocaine, which was detected in the first
batch of samples. Our sample is derived from an early--mid 19th century
population, and cocaine was isolated in 1860 by Albert Niemann, and
entered popular medical practice in 1884. Coca arrived in Europe as
early as 1771, but as botanical specimens rather than for consumption,
and there were also issues importing enough viable specimens of coca for
cocaine extraction (\protect\hyperlink{ref-abucaCocaTrade2019}{Abduca,
2019, p. 108}; \protect\hyperlink{ref-mortimerHistoryCoca1901}{Mortimer,
1901, p. 179}). We considered it possible that it would be present in a
sample with most individuals originating from the early- to mid-19th
century. If corroborated, this would have been the first case of
coca-leaf-consumption in Europe. In our replication batch, we included
all of the individuals who had been cocaine-positive in the first batch.
We were unable to replicate any of the cocaine results, and we were
unable to detect the principal metabolite, benzoylecgonine, in either
batch. We suspect that the original detection of cocaine was a result of
lab contamination during analysis.

We explored the relationship between detected compounds and various
skeletal indicators, such as pathological and dental lesions,
preservation, and pipe notches. We found some evidence to suggest that
preservation of the skeleton influences the recovery of compounds from
the dental calculus, with well-preserved skeletons potentially serving
as a better target for sampling.\\
We found a positive correlation between CMS and nicotine, which may be
indicative of the impact tobacco smoking had on the respiratory health
of the Beemster inhabitants. Tobacco smoke may play a significant role
in diseases of the upper respiratory tract, including chronic maxillary
sinusities (\protect\hyperlink{ref-rehImpactTobacco2012}{Reh et al.,
2012}). Although the mechanisms by which smoking increases the risk of
infections is not fully understood, solid evidence has been presented
linking tobacco smoke to increased mucosal permeability and impairment
of mucociliary clearance
(\protect\hyperlink{ref-arcaviCigaretteSmoking2004}{Arcavi \& Benowitz,
2004}). Such changes, together with an altered immunologic response, are
thought to predispose to the development of chronic maxillary sinusitis
(\protect\hyperlink{ref-slavinDiagnosisManagement2005}{Slavin et al.,
2005}).\\
We also observed a moderate positive correlation between chronic
maxillary sinusitis and caffeine which contradicts previous research
linking chronic coffee consumption with a positive effect on the
respiratory system, suggesting a preventive association between caffeine
intake and pneumonia (e.g.
\protect\hyperlink{ref-alfaroChronicCoffee2018}{Alfaro et al., 2018};
\protect\hyperlink{ref-kondoAssociationCoffee2021}{Kondo et al., 2021}).
However, while the lower respiratory tract seems to benefit from chronic
coffee consumption, it is possible that elevated caffeine intake impacts
mucosal moisture due to its dehydrating effect
(\protect\hyperlink{ref-maughanCaffeineIngestion2003}{Maughan \&
Griffin, 2003}), thereby exposing individuals to greater risk of sinus
infection.

The detection of nicotine in dental calculus has previously been
presented by Eerkens and colleagues
(\protect\hyperlink{ref-eerkensDentalCalculus2018}{2018}) in two
individuals from pre-contact California. They also targeted caffeine,
cotinine, and theophylline in their samples, but were unable to detect
any of them. It remains to be seen whether this is due to differences in
methods used, or due to our samples being more recent. They also suggest
that the choice of tooth for sampling may impact the detection of
certain compounds, as the incorporation in dental calculus may depend on
the mode of consumption. Tobacco smokers may have more nicotine present
in calculus on incisors, whereas tobacco chewers may have more on molars
(\protect\hyperlink{ref-eerkensDentalCalculus2018}{Eerkens et al.,
2018}). However, sampling may not be limited to mode of consumption. The
presence of cotinine suggests that the excretion of a compound after
being metabolised in the body is also a source of deposition, and that
deposition of alkaloids in dental calculus can occur both on the way
into the body, i.e.~during consumption, and on the way out,
i.e.~disposal of waste products via saliva secretion into the mouth.
Especially mucin-rich saliva from the sublingual and submandibular
glands preferentially binds toxins
(\protect\hyperlink{ref-doddsHealthBenefits2005}{Michael W. J. Dodds et
al., 2005}), and since these glands are located closest to the lower
incisors, they may be the most effective target for these studies. This
has yet to be systematically tested in archaeological dental calculus.
Because we homogenised samples from multiple teeth of an individual, we
were unable to test the effect of oral biogeography. It is also possible
that resident microflora within biofilms contribute to alkaloid
breakdown and that the presence of caffeine and nicotine metabolites
following direct ingestion can be explained by this pathway. However,
the literature on biofilm biodegradation of alkaloids is limited, and
\emph{in vitro} studies have only found minimal contributions by certain
oral bacteria in isolation
(\protect\hyperlink{ref-cogoVitroEvaluation2008}{Cogo et al., 2008};
\protect\hyperlink{ref-sunMetabolomicsEvaluation2016}{Sun et al.,
2016}); it is possible that a larger role is played by oral bacteria
within larger, more metabolically active communities, e.g.~biofilms
(\protect\hyperlink{ref-takahashiOralMicrobiome2015}{Takahashi, 2015}).

Because we targeted individuals with moderate-to-large calculus
deposits, it is likely a biased sample. The presence of calculus may
increase the risk of premature death
(\protect\hyperlink{ref-yaussyCalculusSurvivorship2019}{Yaussy \&
DeWitte, 2019}), and periodontal disease (which may or may not be
associated with dental calculus build-up) is a risk-factor for
respiratory diseases, if periodontal and respiratory pathogens enter the
bloodstream
(\protect\hyperlink{ref-azarpazhoohSystematicReview2006}{Azarpazhooh \&
Leake, 2006};
\protect\hyperlink{ref-scannapiecoRoleOral1999}{Scannapieco, 1999};
\protect\hyperlink{ref-scannapiecoPotentialAssociations2001}{Scannapieco
\& Ho, 2001}). In our sample, the percentage of chronic maxillary
sinusitis (37.0\%) is lower than in another (more representative) male
sample (44.1\%)
(\protect\hyperlink{ref-casnaUrbanizationRespiratory2021}{Casna et al.,
2021}), and the caries percentage is similarly lower in our sample
(12.7\%) than a more representative sample (22.9\%)
(\protect\hyperlink{ref-lemmersMiddenbeemster2013}{Lemmers et al.,
2013}).\\
We used the presence/absence of a pipe notch and concurrent detection of
tobacco as a crude estimate of the accuracy of the method, which we
found to be around 59.3\%. This is a very rough estimate, as the
presence of a pipe notch is likely not a perfect indicator of whether or
not someone consumed tobacco. Dental calculus is also more transient
than for example bone, as it can be mechanically removed, intentionally
or unintentionally, during life, eliminating all trace of the alkaloids
consumed prior to its removal.\\
Quantitation of the detected compounds may have limited value in
archaeological samples due to degradation, and will greatly affect our
correlations related to concentration. Following burial, compound
stability over time will play a large role, as will microbial
degradation of compounds by bacteria and fungi in soil
(\protect\hyperlink{ref-liuNicotinedegradingMicroorganisms2015}{Liu et
al., 2015}), as well as the soil environment, such as temperature, pH,
and oxygen availability
(\protect\hyperlink{ref-lindholstLongTerm2010}{Lindholst, 2010};
\protect\hyperlink{ref-mackiePreservationMetaproteome2017}{Mackie et
al., 2017}).\\
The detected quantity of a compound will also depend on the quantity in
dental calculus during life, which is largely controlled by quantity of
consumption, how often the calculus was disrupted/removed, metabolic
breakdown of the compound, and inter- and intra-individual factors
related to stages of biofilm formation, maturation, and mineralisation
(\protect\hyperlink{ref-lustmannScanningElectron1976}{Lustmann et al.,
1976}; \protect\hyperlink{ref-velskoMicrobialDifferences2019}{Velsko et
al., 2019}; \protect\hyperlink{ref-zijngeBiofilmArchitecture2010}{Zijnge
et al., 2010}). In short, this means it is not really possible to detect
the absence of a compound. The absence of a compound is not evidence of
absence of consumption. This complicates the interpretation of our
results. We have attempted to minimise errors occurring due to this
limitation by including a relatively large sample of individuals and
replicating our analysis. Although given the relatively low detection
rate seen in tobacco, this remains a major limitation, and will likely
be compounded by increasing antiquity of the samples.

Future studies should explore how sampling from various types of teeth
and their position in the mouth affects the probability of a compound
becoming entrapped in dental calculus. This may also be related to
properties within the oral cavity, as well as chemical properties of the
compounds, which facilitate or reduce the incorporation-potential, and
which incorporation pathways are more likely for a given compound.\\
We only targeted drugs that were included in the forensic toxicological
screenings, and therefore only covered a limited number of the potential
compounds that could be of interest for exploring past diets and
medicinal treatments. The list of targeted compounds can be expanded as
we discover more potential targets based on which specific
compounds/metabolites are more likely to be incorporated and preserved
in dental calculus.\\
There is an increasing interest in using oral fluid as a means of
detecting alkaloids in living individuals due to the non-invasive nature
of the testing compared to blood and urine sampling
(\protect\hyperlink{ref-coneSalivaTesting1993}{Cone, 1993};
\protect\hyperlink{ref-valenDetermination212017}{Valen et al., 2017}).
These \emph{in vivo} studies are a valuable source of method validation
and can help determine the feasibility of detecting certain alkaloids in
oral fluid and, subsequently, dental calculus. Archaeologists, though,
will likely be responsible for exploring dental calculus specific
incorporation and retention of alkaloids, as well as their long-term
preservation in the burial environment.

While a major limitation is the uncertainty surrounding whether or not a
compound is actually absent, the power of the method lies in the ability
to detect dietary and other compounds that were incorporated via
multiple consumption pathways that are not detected by other methods.
Taking tobacco consumption as an example; while pipe notches are a
useful way to identify tobacco consumption, pipe smoking was not the
only mode of tobacco consumption, with others including chewing,
drinking, cigars, and snuff
(\protect\hyperlink{ref-goodmanTobaccoHistory1994}{Goodman, 1994, p.
67}). Pipe-smoking was mainly practised by males
(\protect\hyperlink{ref-eerkensDentalCalculus2018}{Eerkens et al.,
2018}; \protect\hyperlink{ref-lemmersMiddenbeemster2013}{Lemmers et al.,
2013}), so methods like the one presented here are suitable for
exploring tobacco consumption in an entire society, rather than a
trivial subset of past populations. Combined with other methods, it can
also give us a more complete picture of dietary patterns and
medicinal/recreational plant-use in the past by capturing multiple
possible incorporation pathways of dietary (and other) compounds.

\hypertarget{conclusion-1}{%
\section{Conclusion}\label{conclusion-1}}

This preliminary study outlines the benefits of using calculus to target
a variety of compounds that could have been consumed as medicine or
diet. This method allows us to directly address specific individuals,
which can be especially useful in individuals that are not always
well-documented in historic documentation, such as rural communities,
children and women. We also show that there are many limitations that
will need to be addressed going forward with this type of analysis, and
stress the need for more systematic research on the consumption of
alkaloid-containing items and their subsequent concentration and
preservation in dental calculus, in addition to how mode of consumption
may affect concentrations on different parts of the dentition. Another
limitation of dental calculus as a medium is the inter- and
intra-individual variability of its formation and the many factors that
can influence incorporation and retention of molecules and particles;
however, in the absence of hair and serum (quite uncommon in
archaeology), dental calculus represents an impressive long-term
reservoir of information regarding the consumption of various alkaloids,
whether dietary, medicinal, recreational, or otherwise.

\hypertarget{references-2}{%
\section*{References}\label{references-2}}
\addcontentsline{toc}{section}{References}

\markright{References}

\hypertarget{refs-11}{}
\begin{CSLReferences}{1}{0}
\leavevmode\vadjust pre{\hypertarget{ref-abucaCocaTrade2019}{}}%
Abduca, R. (2019). Coca leaf transfers to {Europe}. {Effects} on the
consumption of coca in {North-western Argentina}. In M. Kaller \& F.
Jacob (Eds.), \emph{Transatlantic {Trade} and {Global Cultural Transfers
Since} 1492: {More} than {Commodities}}. {Routledge}.
\url{https://books.google.com?id=13imDwAAQBAJ}

\leavevmode\vadjust pre{\hypertarget{ref-alanonAssessmentFlavanol2016}{}}%
Alañón, M. E., Castle, S. M., Siswanto, P. J., Cifuentes-Gómez, T., \&
Spencer, J. P. E. (2016). Assessment of flavanol stereoisomers and
caffeine and theobromine content in commercial chocolates. \emph{Food
Chemistry}, \emph{208}, 177--184.
\url{https://doi.org/10.1016/j.foodchem.2016.03.116}

\leavevmode\vadjust pre{\hypertarget{ref-alfaroChronicCoffee2018}{}}%
Alfaro, T. M., Monteiro, R. A., Cunha, R. A., \& Cordeiro, C. R. (2018).
Chronic coffee consumption and respiratory disease: {A} systematic
review. \emph{The Clinical Respiratory Journal}, \emph{12}(3),
1283--1294. \url{https://doi.org/10.1111/crj.12662}

\leavevmode\vadjust pre{\hypertarget{ref-arcaviCigaretteSmoking2004}{}}%
Arcavi, L., \& Benowitz, N. L. (2004). Cigarette {Smoking} and
{Infection}. \emph{Archives of Internal Medicine}, \emph{164}(20),
2206--2216. \url{https://doi.org/10.1001/archinte.164.20.2206}

\leavevmode\vadjust pre{\hypertarget{ref-aten400Jaar2012}{}}%
Aten, D., Bossaers, K. W. J. M., \& Misset, C. (2012). \emph{400 jaar
Beemster: 1612-2012}. {Stichting Uitgeverij Noord-Holland}.

\leavevmode\vadjust pre{\hypertarget{ref-azarpazhoohSystematicReview2006}{}}%
Azarpazhooh, A., \& Leake, J. L. (2006). Systematic {Review} of the
{Association Between Respiratory Diseases} and {Oral Health}.
\emph{Journal of Periodontology}, \emph{77}(9), 1465--1482.
\url{https://doi.org/10.1902/jop.2006.060010}

\leavevmode\vadjust pre{\hypertarget{ref-badriRegulationFunction2009}{}}%
Badri, D. V., \& Vivanco, J. M. (2009). Regulation and function of root
exudates. \emph{Plant, Cell \& Environment}, \emph{32}(6), 666--681.
\url{https://doi.org/10.1111/j.1365-3040.2009.01926.x}

\leavevmode\vadjust pre{\hypertarget{ref-bispoSimultaneousDetermination2002}{}}%
Bispo, M. S., Veloso, M. C. C., Pinheiro, H. L. C., De Oliveira, R. F.
S., Reis, J. O. N., \& De Andrade, J. B. (2002). Simultaneous
{Determination} of {Caffeine}, {Theobromine}, and {Theophylline} by
{High-Performance Liquid Chromatography}. \emph{Journal of
Chromatographic Science}, \emph{40}(1), 45--48.
\url{https://doi.org/10.1093/chromsci/40.1.45}

\leavevmode\vadjust pre{\hypertarget{ref-boocockMaxillarySinusitis1995}{}}%
Boocock, P., Roberts, C. A., \& Manchester, K. (1995). Maxillary
sinusitis in {Medieval Chichester}, {England}. \emph{American Journal of
Physical Anthropology}, \emph{98}(4), 483--495.
\url{https://doi.org/10.1002/ajpa.1330980408}

\leavevmode\vadjust pre{\hypertarget{ref-boumanBegravenis2017}{}}%
Bouman, J. (2017). De Begravenis. \emph{De Nieuwe Schouwschuit},
\emph{15}, 11--15.
\url{https://www.historischgenootschapbeemster.nl/wp-content/uploads/De_Nieuwe_Schouwschuit_15e_jaargang_november_2017.pdf}

\leavevmode\vadjust pre{\hypertarget{ref-SucheyBrooks1990}{}}%
Brooks, S., \& Suchey, J. M. (1990). Skeletal age determination based on
the os pubis: {A} comparison of the {Acsádi-Nemeskéri} and
{Suchey-Brooks} methods. \emph{Human Evolution}, \emph{5}(3), 227--238.
\url{https://doi.org/10.1007/BF02437238}

\leavevmode\vadjust pre{\hypertarget{ref-brothwellDiggingBones1981}{}}%
Brothwell, D. (1981). \emph{Digging up {Bones}: {The} excavation,
treatment and study of human skeletal remains} (3rd ed.). {British
Museum (Natural History)}.

\leavevmode\vadjust pre{\hypertarget{ref-bruinsmaBijdragenTot1872}{}}%
Bruinsma, J. J. (1872). \emph{Bijdragen tot de {Geneeskundige
Plaatsbeschrijving} van {Nederland}}. {Van Weelden en Mingelen}.
\url{https://dlcs.io/pdf/wellcome/pdf-item/b24874140/0}

\leavevmode\vadjust pre{\hypertarget{ref-buckberryAuricular2002}{}}%
Buckberry, J. L., \& Chamberlain, A. T. (2002). Age estimation from the
auricular surface of the ilium: A revised method. \emph{American Journal
of Physical Anthropology}, \emph{119}(3), 231--239.
\url{https://doi.org/10.1002/ajpa.10130}

\leavevmode\vadjust pre{\hypertarget{ref-buckleyDentalCalculus2014}{}}%
Buckley, S., Usai, D., Jakob, T., Radini, A., \& Hardy, K. (2014).
Dental {Calculus Reveals Unique Insights} into {Food Items}, {Cooking}
and {Plant Processing} in {Prehistoric Central Sudan}. \emph{PLOS ONE},
\emph{9}(7), e100808. \url{https://doi.org/10.1371/journal.pone.0100808}

\leavevmode\vadjust pre{\hypertarget{ref-Standards1994}{}}%
Buikstra, J. E., \& Ubelaker, D. H. (1994). Standards for data
collection from human skeletal remains: {Proceedings} of a seminar at
the {Field Museum} of {Natural History} ({Arkansas Archaeology Research
Series} 44). \emph{Fayetteville Arkansas Archaeological Survey}.

\leavevmode\vadjust pre{\hypertarget{ref-casnaUrbanizationRespiratory2021}{}}%
Casna, M., Burrell, C. L., Schats, R., Hoogland, M. L. P., \& Schrader,
S. A. (2021). Urbanization and respiratory stress in the {Northern Low
Countries}: {A} comparative study of chronic maxillary sinusitis in two
early modern sites from the {Netherlands} ({AD} 1626--1866).
\emph{International Journal of Osteoarchaeology}, \emph{31}(5),
891--901. \url{https://doi.org/10.1002/oa.3006}

\leavevmode\vadjust pre{\hypertarget{ref-chenCa2Dependent2001}{}}%
Chen, H., Hou, W., Kuć, J., \& Lin, Y. (2001). Ca2+‐dependent and
{Ca2}+‐independent excretion modes of salicylic acid in tobacco cell
suspension culture. \emph{Journal of Experimental Botany},
\emph{52}(359), 1219--1226.
\url{https://doi.org/10.1093/jexbot/52.359.1219}

\leavevmode\vadjust pre{\hypertarget{ref-chinCaffeineContent2008}{}}%
Chin, J. M., Merves, M. L., Goldberger, B. A., Sampson-Cone, A., \&
Cone, E. J. (2008). Caffeine {Content} of {Brewed Teas}. \emph{Journal
of Analytical Toxicology}, \emph{32}(8), 702--704.
\url{https://doi.org/10.1093/jat/32.8.702}

\leavevmode\vadjust pre{\hypertarget{ref-chovanecOpiumMasses2012}{}}%
Chovanec, Z., Rafferty, S., \& Swiny, S. (2012). Opium for the {Masses}.
\emph{Ethnoarchaeology}, \emph{4}(1), 5--36.
\url{https://doi.org/10.1179/eth.2012.4.1.5}

\leavevmode\vadjust pre{\hypertarget{ref-clarkeCannabisEvolution2013}{}}%
Clarke, R. (2013). \emph{Cannabis : {Evolution} and {Ethnobotany}}.
{University of California Press}.

\leavevmode\vadjust pre{\hypertarget{ref-cogoVitroEvaluation2008}{}}%
Cogo, K., Montan, M. F., Bergamaschi, C. de C., D. Andrade, E., Rosalen,
P. L., \& Groppo, F. C. (2008). In vitro evaluation of the effect of
nicotine, cotinine, and caffeine on oral microorganisms. \emph{Canadian
Journal of Microbiology}, \emph{54}(6), 501--508.
\url{https://doi.org/10.1139/W08-032}

\leavevmode\vadjust pre{\hypertarget{ref-coneSalivaTesting1993}{}}%
Cone, E. J. (1993). Saliva {Testing} for {Drugs} of {Abuse}.
\emph{Annals of the New York Academy of Sciences}, \emph{694}(1),
91--127. \url{https://doi.org/10.1111/j.1749-6632.1993.tb18346.x}

\leavevmode\vadjust pre{\hypertarget{ref-coneInterpretationOral2007}{}}%
Cone, E. J., \& Huestis, M. A. (2007). Interpretation of {Oral Fluid
Tests} for {Drugs} of {Abuse}. \emph{Annals of the New York Academy of
Sciences}, \emph{1098}, 51--103.
\url{https://doi.org/10.1196/annals.1384.037}

\leavevmode\vadjust pre{\hypertarget{ref-doddsHealthBenefits2005}{}}%
Dodds, M. W. J., Johnson, D. A., \& Yeh, C.-K. (2005). Health benefits
of saliva: A review. \emph{Journal of Dentistry}, \emph{33}(3),
223--233. \url{https://doi.org/10.1016/j.jdent.2004.10.009}

\leavevmode\vadjust pre{\hypertarget{ref-duthieNaturalSalicylates2011}{}}%
Duthie, G. G., \& Wood, A. D. (2011). Natural salicylates: Foods ,
functions and disease prevention. \emph{Food \& Function}, \emph{2}(9),
515--520. \url{https://doi.org/10.1039/C1FO10128E}

\leavevmode\vadjust pre{\hypertarget{ref-echeverriaNicotineHair2013}{}}%
Echeverría, J., \& Niemeyer, H. M. (2013). Nicotine in the hair of
mummies from {San Pedro} de {Atacama} ({Northern Chile}). \emph{Journal
of Archaeological Science}, \emph{40}(10), 3561--3568.
\url{https://doi.org/10.1016/j.jas.2013.04.030}

\leavevmode\vadjust pre{\hypertarget{ref-eerkensDentalCalculus2018}{}}%
Eerkens, J. W., Tushingham, S., Brownstein, K. J., Garibay, R., Perez,
K., Murga, E., Kaijankoski, P., Rosenthal, J. S., \& Gang, D. R. (2018).
Dental calculus as a source of ancient alkaloids: {Detection} of
nicotine by {LC-MS} in calculus samples from the {Americas}.
\emph{Journal of Archaeological Science: Reports}, \emph{18}, 509--515.
\url{https://doi.org/10.1016/j.jasrep.2018.02.004}

\leavevmode\vadjust pre{\hypertarget{ref-gismondiMultidisciplinaryApproach2020}{}}%
Gismondi, A., Baldoni, M., Gnes, M., Scorrano, G., D'Agostino, A.,
Marco, G. D., Calabria, G., Petrucci, M., Müldner, G., Tersch, M. V.,
Nardi, A., Enei, F., Canini, A., Rickards, O., Alexander, M., \&
Martínez-Labarga, C. (2020). A multidisciplinary approach for
investigating dietary and medicinal habits of the {Medieval} population
of {Santa Severa} (7th-15th centuries, {Rome}, {Italy}). \emph{PLOS
ONE}, \emph{15}(1), e0227433.
\url{https://doi.org/10.1371/journal.pone.0227433}

\leavevmode\vadjust pre{\hypertarget{ref-goodmanTobaccoHistory1994}{}}%
Goodman, J. (1994). \emph{Tobacco in history: The cultures of
dependence}. {Routledge}.

\leavevmode\vadjust pre{\hypertarget{ref-greeneQuantifyingCalculus2005}{}}%
Greene, T. R., Kuba, C. L., \& Irish, J. D. (2005). Quantifying
calculus: {A} suggested new approach for recording an important
indicator of diet and dental health. \emph{HOMO - Journal of Comparative
Human Biology}, \emph{56}(2), 119--132.
\url{https://doi.org/10.1016/j.jchb.2005.02.002}

\leavevmode\vadjust pre{\hypertarget{ref-jinSupragingivalCalculus2002}{}}%
Jin, Y., \& Yip, H.-K. (2002). Supragingival {Calculus}: {Formation} and
{Control}. \emph{Critical Reviews in Oral Biology \& Medicine}.
\url{https://doi.org/10.1177/154411130201300506}

\leavevmode\vadjust pre{\hypertarget{ref-kondoAssociationCoffee2021}{}}%
Kondo, K., Suzuki, K., Washio, M., Ohfuji, S., Adachi, S., Kan, S.,
Imai, S., Yoshimura, K., Miyashita, N., Fujisawa, N., Maeda, A.,
Fukushima, W., \& Hirota, Y. (2021). Association between coffee and
green tea intake and pneumonia among the {Japanese} elderly: A
case-control study. \emph{Scientific Reports}, \emph{11}(1, 1), 5570.
\url{https://doi.org/10.1038/s41598-021-84348-w}

\leavevmode\vadjust pre{\hypertarget{ref-lemmersMiddenbeemster2013}{}}%
Lemmers, S. A. M., Schats, R., Hoogland, M. L. P., \& Waters-Rist, A.
(2013). Fysisch antropologische analyse Middenbeemster. In \emph{De
begravingen bij de Keyserkerk te Middenbeemster} (pp. 35--60).

\leavevmode\vadjust pre{\hypertarget{ref-leuwProhibitionLegalization1994}{}}%
Leuw, E., \& Marshall, I. H. (1994). \emph{Between {Prohibition} and
{Legalization}: {The Dutch Experiment} in {Drug Policy}}. {Kugler
Publications}. \url{https://books.google.com?id=2mAVkStNG5EC}

\leavevmode\vadjust pre{\hypertarget{ref-lindholstLongTerm2010}{}}%
Lindholst, C. (2010). Long term stability of cannabis resin and cannabis
extracts. \emph{Australian Journal of Forensic Sciences}, \emph{42}(3),
181--190. \url{https://doi.org/10.1080/00450610903258144}

\leavevmode\vadjust pre{\hypertarget{ref-liuNicotinedegradingMicroorganisms2015}{}}%
Liu, J., Ma, G., Chen, T., Hou, Y., Yang, S., Zhang, K.-Q., \& Yang, J.
(2015). Nicotine-degrading microorganisms and their potential
applications. \emph{Applied Microbiology and Biotechnology},
\emph{99}(9), 3775--3785.
\url{https://doi.org/10.1007/s00253-015-6525-1}

\leavevmode\vadjust pre{\hypertarget{ref-lovejoyAuricular1985}{}}%
Lovejoy, C. O., Meindl, R. S., Pryzbeck, T. R., \& Mensforth, R. P.
(1985). Chronological metamorphosis of the auricular surface of the
ilium: {A} new method for the determination of adult skeletal age at
death. \emph{American Journal of Physical Anthropology}, \emph{68}(1),
15--28. \url{https://doi.org/10.1002/ajpa.1330680103}

\leavevmode\vadjust pre{\hypertarget{ref-lustmannScanningElectron1976}{}}%
Lustmann, J., Lewin-Epstein, J., \& Shteyer, A. (1976). Scanning
electron microscopy of dental calculus. \emph{Calcified Tissue
Research}, \emph{21}(1), 47--55.
\url{https://doi.org/10.1007/BF02547382}

\leavevmode\vadjust pre{\hypertarget{ref-maatManualPhysical2005}{}}%
Maat, G., \& Mastwijk, R. (2005). Manual for the physical
anthropological report. \emph{Barge's Anthropologica}, \emph{6}.

\leavevmode\vadjust pre{\hypertarget{ref-machtHistoryOpium1915}{}}%
Macht, D. I. (1915). The history of opium and some of its preparations
and alkaloids. \emph{The Journal of the American Medical Association},
\emph{LXIV}(6), 5.

\leavevmode\vadjust pre{\hypertarget{ref-mackiePreservationMetaproteome2017}{}}%
Mackie, M., Hendy, J., Lowe, A. D., Sperduti, A., Holst, M., Collins, M.
J., \& Speller, C. F. (2017). Preservation of the metaproteome:
Variability of protein preservation in ancient dental calculus.
\emph{STAR: Science \& Technology of Archaeological Research},
\emph{3}(1), 58--70. \url{https://doi.org/10.1080/20548923.2017.1361629}

\leavevmode\vadjust pre{\hypertarget{ref-malakarNaturallyOccurring2017}{}}%
Malakar, S., Gibson, P. R., Barrett, J. S., \& Muir, J. G. (2017).
Naturally occurring dietary salicylates: {A} closer look at common
{Australian} foods. \emph{Journal of Food Composition and Analysis},
\emph{57}, 31--39. \url{https://doi.org/10.1016/j.jfca.2016.12.008}

\leavevmode\vadjust pre{\hypertarget{ref-maughanCaffeineIngestion2003}{}}%
Maughan, R. J., \& Griffin, J. (2003). Caffeine ingestion and fluid
balance: A review. \emph{Journal of Human Nutrition and Dietetics},
\emph{16}(6), 411--420.
\url{https://doi.org/10.1046/j.1365-277X.2003.00477.x}

\leavevmode\vadjust pre{\hypertarget{ref-meindlSutureClosure1985}{}}%
Meindl, R. S., \& Lovejoy, C. O. (1985). Ectocranial suture closure: {A}
revised method for the determination of skeletal age at death based on
the lateral-anterior sutures. \emph{American Journal of Physical
Anthropology}, \emph{68}(1), 57--66.
\url{https://doi.org/10.1002/ajpa.1330680106}

\leavevmode\vadjust pre{\hypertarget{ref-milmanOralFluid2011}{}}%
Milman, G., Schwope, D. M., Schwilke, E. W., Darwin, W. D., Kelly, D.
L., Goodwin, R. S., Gorelick, D. A., \& Huestis, M. A. (2011). Oral
{Fluid} and {Plasma Cannabinoid Ratios} after {Around-the-Clock
Controlled Oral Δ9-Tetrahydrocannabinol Administration}. \emph{Clinical
Chemistry}, \emph{57}(11), 1597--1606.
\url{https://doi.org/10.1373/clinchem.2011.169490}

\leavevmode\vadjust pre{\hypertarget{ref-mortimerHistoryCoca1901}{}}%
Mortimer, W. G. (1901). \emph{Peru. {History} of coca, "the divine
plant" of the {Incas}; with an introductory account of the {Incas}, and
of the {Andean Indians} of to-day}. {New York, J. H. Vail \& Company}.
\url{http://archive.org/details/peruhistoryofcoc00mortrich}

\leavevmode\vadjust pre{\hypertarget{ref-nierstraszTeaTrade2015}{}}%
Nierstrasz, C. (2015). \emph{Rivalry for {Trade} in {Tea} and
{Textiles}: {The English} and {Dutch East India} companies
(1700--1800)}. {Springer}.
\url{https://books.google.com?id=uwtaCwAAQBAJ}

\leavevmode\vadjust pre{\hypertarget{ref-ogaldeIdentificationPsychoactive2009}{}}%
Ogalde, J. P., Arriaza, B. T., \& Soto, E. C. (2009). Identification of
psychoactive alkaloids in ancient {Andean} human hair by gas
chromatography/mass spectrometry. \emph{Journal of Archaeological
Science}, \emph{36}(2), 467--472.
\url{https://doi.org/10.1016/j.jas.2008.09.036}

\leavevmode\vadjust pre{\hypertarget{ref-palmerActivityReconstruction2016}{}}%
Palmer, J. L. A., Hoogland, M. H. L., \& Waters‐Rist, A. L. (2016).
Activity {Reconstruction} of {Post}‐{Medieval Dutch Rural Villagers}
from {Upper Limb Osteoarthritis} and {Entheseal Changes}.
\emph{International Journal of Osteoarchaeology}, \emph{26}(1), 78--92.
\url{https://doi.org/10.1002/oa.2397}

\leavevmode\vadjust pre{\hypertarget{ref-Rbase}{}}%
R Core Team. (2020). \emph{R: {A} language and environment for
statistical computing} {[}Manual{]}. {R Foundation for Statistical
Computing}; {R Foundation for Statistical Computing}.
\url{https://www.R-project.org/}

\leavevmode\vadjust pre{\hypertarget{ref-raffertyCurrentResearch2012}{}}%
Rafferty, S. M., Lednev, I., Virkler, K., \& Chovanec, Z. (2012).
Current research on smoking pipe residues. \emph{Journal of
Archaeological Science}, \emph{39}(7), 1951--1959.
\url{https://doi.org/10.1016/j.jas.2012.02.001}

\leavevmode\vadjust pre{\hypertarget{ref-rehImpactTobacco2012}{}}%
Reh, D. D., Higgins, T. S., \& Smith, T. L. (2012). Impact of {Tobacco
Smoke} on {Chronic Rhinosinusitis} -- {A Review} of the {Literature}.
\emph{International Forum of Allergy \& Rhinology}, \emph{2}(5), 362.
\url{https://doi.org/10.1002/alr.21054}

\leavevmode\vadjust pre{\hypertarget{ref-Rpsych}{}}%
Revelle, W. (2022). \emph{Psych: {Procedures} for psychological,
psychometric, and personality research} {[}Manual{]}. {Northwestern
University}. \url{https://CRAN.R-project.org/package=psych}

\leavevmode\vadjust pre{\hypertarget{ref-rogersPalaeopathologyJoint2000}{}}%
Rogers, J. (2000). The palaeopathology of joint disease. In M. Cox \& S.
Mays (Eds.), \emph{Human osteology : {In} archaeology and forensic
science.} (1st ed, pp. 163--182). {Cambridge University Press}.
\url{https://login.ezproxy.leidenuniv.nl:2443/login?URL=https://search.ebscohost.com/login.aspx?direct=true\&db=e000xww\&AN=40641\&site=ehost-live}

\leavevmode\vadjust pre{\hypertarget{ref-scannapiecoRoleOral1999}{}}%
Scannapieco, F. A. (1999). Role of {Oral Bacteria} in {Respiratory
Infection}. \emph{Journal of Periodontology}, \emph{70}(7), 793--802.
\url{https://doi.org/10.1902/jop.1999.70.7.793}

\leavevmode\vadjust pre{\hypertarget{ref-scannapiecoPotentialAssociations2001}{}}%
Scannapieco, F. A., \& Ho, A. W. (2001). Potential {Associations Between
Chronic Respiratory Disease} and {Periodontal Disease}: {Analysis} of
{National Health} and {Nutrition Examination Survey III}. \emph{Journal
of Periodontology}, \emph{72}(1), 50--56.
\url{https://doi.org/10.1902/jop.2001.72.1.50}

\leavevmode\vadjust pre{\hypertarget{ref-scheltemaOpiumTrade1907}{}}%
Scheltema, J. F. (1907). The {Opium Trade} in the {Dutch East Indies}.
{I}. \emph{American Journal of Sociology}, \emph{13}(1), 79--112.

\leavevmode\vadjust pre{\hypertarget{ref-schuijtemakerTeTheegasten2011}{}}%
Schuijtemaker, D. (2011). Te Theegasten. \emph{De Nieuwe Schouwschuit},
\emph{9}, 16--17.

\leavevmode\vadjust pre{\hypertarget{ref-slavinDiagnosisManagement2005}{}}%
Slavin, R. G., Spector, S. L., Bernstein, I. L., Slavin, R. G., Kaliner,
M. A., Kennedy, D. W., Virant, F. S., Wald, E. R., Khan, D. A.,
Blessing-Moore, J., Lang, D. M., Nicklas, R. A., Oppenheimer, J. J.,
Portnoy, J. M., Schuller, D. E., Tilles, S. A., Borish, L., Nathan, R.
A., Smart, B. A., \& Vandewalker, M. L. (2005). The diagnosis and
management of sinusitis: {A} practice parameter update. \emph{Journal of
Allergy and Clinical Immunology}, \emph{116}, S13--S47.
\url{https://doi.org/10.1016/j.jaci.2005.09.048}

\leavevmode\vadjust pre{\hypertarget{ref-smithDetectionOpium2018}{}}%
Smith, R. K., Stacey, R. J., Bergström, E., \& Thomas-Oates, J. (2018).
Detection of opium alkaloids in a {Cypriot} base-ring juglet.
\emph{Analyst}, \emph{143}(21), 5127--5136.
\url{https://doi.org/10.1039/C8AN01040D}

\leavevmode\vadjust pre{\hypertarget{ref-sorensenEffectAntioxidants2018}{}}%
Sørensen, L. K., \& Hasselstrøm, J. B. (2018). The effect of
antioxidants on the long-term stability of {THC} and related
cannabinoids in sampled whole blood. \emph{Drug Testing and Analysis},
\emph{10}(2), 301--309. \url{https://doi.org/10.1002/dta.2221}

\leavevmode\vadjust pre{\hypertarget{ref-sorensenDrugsCalculus2021}{}}%
Sørensen, L. K., Hasselstrøm, J. B., Larsen, L. S., \& Bindslev, D. A.
(2021). Entrapment of drugs in dental calculus -- {Detection} validation
based on test results from post-mortem investigations. \emph{Forensic
Science International}, \emph{319}, 110647.
\url{https://doi.org/10.1016/j.forsciint.2020.110647}

\leavevmode\vadjust pre{\hypertarget{ref-srdjenovicSimultaneousHPLC2008}{}}%
Srdjenovic, B., Djordjevic-Milic, V., Grujic, N., Injac, R., \&
Lepojevic, Z. (2008). Simultaneous {HPLC Determination} of {Caffeine},
{Theobromine}, and {Theophylline} in {Food}, {Drinks}, and {Herbal
Products}. \emph{Journal of Chromatographic Science}, \emph{46}(2),
144--149. \url{https://doi.org/10.1093/chromsci/46.2.144}

\leavevmode\vadjust pre{\hypertarget{ref-stavricVariabilityCaffeine1988}{}}%
Stavric, B., Klassen, R., Watkinson, B., Karpinski, K., Stapley, R., \&
Fried, P. (1988). Variability in caffeine consumption from coffee and
tea: {Possible} significance for epidemiological studies. \emph{Food and
Chemical Toxicology}, \emph{26}(2), 111--118.
\url{https://doi.org/10.1016/0278-6915(88)90107-X}

\leavevmode\vadjust pre{\hypertarget{ref-sunMetabolomicsEvaluation2016}{}}%
Sun, J., Jin, J., Beger, R. D., Cerniglia, C. E., Yang, M., \& Chen, H.
(2016). Metabolomics evaluation of the impact of smokeless tobacco
exposure on the oral bacterium {Capnocytophaga} sputigena.
\emph{Toxicology in Vitro}, \emph{36}, 133--141.
\url{https://doi.org/10.1016/j.tiv.2016.07.020}

\leavevmode\vadjust pre{\hypertarget{ref-takahashiOralMicrobiome2015}{}}%
Takahashi, N. (2015). Oral {Microbiome Metabolism}: {From} {``{Who Are
They}?''} To {``{What Are They Doing}?''} \emph{Journal of Dental
Research}, \emph{94}(12), 1628--1637.
\url{https://doi.org/10.1177/0022034515606045}

\leavevmode\vadjust pre{\hypertarget{ref-tushinghamHuntergathererTobacco2013}{}}%
Tushingham, S., Ardura, D., Eerkens, J. W., Palazoglu, M., Shahbaz, S.,
\& Fiehn, O. (2013). Hunter-gatherer tobacco smoking: Earliest evidence
from the {Pacific Northwest Coast} of {North America}. \emph{Journal of
Archaeological Science}, \emph{40}(2), 1397--1407.
\url{https://doi.org/10.1016/j.jas.2012.09.019}

\leavevmode\vadjust pre{\hypertarget{ref-valenDetermination212017}{}}%
Valen, A., Leere Øiestad, Å. M., Strand, D. H., Skari, R., \& Berg, T.
(2017). Determination of 21 drugs in oral fluid using fully automated
supported liquid extraction and {UHPLC-MS}/{MS}. \emph{Drug Testing and
Analysis}, \emph{9}(5), 808--823. \url{https://doi.org/10.1002/dta.2045}

\leavevmode\vadjust pre{\hypertarget{ref-velskoMicrobialDifferences2019}{}}%
Velsko, I. M., Fellows Yates, J. A., Aron, F., Hagan, R. W., Frantz, L.
A. F., Loe, L., Martinez, J. B. R., Chaves, E., Gosden, C., Larson, G.,
\& Warinner, C. (2019). Microbial differences between dental plaque and
historic dental calculus are related to oral biofilm maturation stage.
\emph{Microbiome}, \emph{7}(1), 102.
\url{https://doi.org/10.1186/s40168-019-0717-3}

\leavevmode\vadjust pre{\hypertarget{ref-velskoDentalCalculus2017}{}}%
Velsko, I. M., Overmyer, K. A., Speller, C., Klaus, L., Collins, M. J.,
Loe, L., Frantz, L. A. F., Sankaranarayanan, K., Lewis, C. M., Martinez,
J. B. R., Chaves, E., Coon, J. J., Larson, G., \& Warinner, C. (2017).
The dental calculus metabolome in modern and historic samples.
\emph{Metabolomics}, \emph{13}(11), 134.
\url{https://doi.org/10.1007/s11306-017-1270-3}

\leavevmode\vadjust pre{\hypertarget{ref-warinnerEvidenceMilk2014}{}}%
Warinner, C., Hendy, J., Speller, C., Cappellini, E., Fischer, R.,
Trachsel, C., Arneborg, J., Lynnerup, N., Craig, O. E., Swallow, D. M.,
Fotakis, A., Christensen, R. J., Olsen, J. V., Liebert, A., Montalva,
N., Fiddyment, S., Charlton, S., Mackie, M., Canci, A., \ldots{}
Collins, M. J. (2014). Direct evidence of milk consumption from ancient
human dental calculus. \emph{Scientific Reports}, \emph{4}, 7104.
\url{https://doi.org/10.1038/srep07104}

\leavevmode\vadjust pre{\hypertarget{ref-whiteDentalCalculus1997}{}}%
White, D. J. (1997). Dental calculus: Recent insights into occurrence,
formation, prevention, removal and oral health effects of supragingival
and subgingival deposits. \emph{European Journal of Oral Sciences},
\emph{105}(5), 508--522.
\url{https://doi.org/10.1111/j.1600-0722.1997.tb00238.x}

\leavevmode\vadjust pre{\hypertarget{ref-ggplot2}{}}%
Wickham, H. (2016). \emph{Ggplot2: {Elegant Graphics} for {Data
Analysis}}. {Springer-Verlag}. \url{https://ggplot2.tidyverse.org}

\leavevmode\vadjust pre{\hypertarget{ref-tidyverse2019}{}}%
Wickham, Hadley, Averick, M., Bryan, J., Chang, W., McGowan, L. D.,
François, R., Grolemund, G., Hayes, A., Henry, L., Hester, J., Kuhn, M.,
Pedersen, T. L., Miller, E., Bache, S. M., Müller, K., Ooms, J.,
Robinson, D., Seidel, D. P., Spinu, V., \ldots{} Yutani, H. (2019).
Welcome to the {tidyverse}. \emph{Journal of Open Source Software},
\emph{4}(43), 1686. \url{https://doi.org/10.21105/joss.01686}

\leavevmode\vadjust pre{\hypertarget{ref-willeRelationshipOral2009}{}}%
Wille, S. M. R., Raes, E., Lillsunde, P., Gunnar, T., Laloup, M., Samyn,
N., Christophersen, A. S., Moeller, M. R., Hammer, K. P., \& Verstraete,
A. G. (2009). Relationship {Between Oral Fluid} and {Blood
Concentrations} of {Drugs} of {Abuse} in {Drivers Suspected} of {Driving
Under} the {Influence} of {Drugs}. \emph{Therapeutic Drug Monitoring},
\emph{31}(4), 511. \url{https://doi.org/10.1097/FTD.0b013e3181ae46ea}

\leavevmode\vadjust pre{\hypertarget{ref-yaussyCalculusSurvivorship2019}{}}%
Yaussy, S. L., \& DeWitte, S. N. (2019). Calculus and survivorship in
medieval {London}: {The} association between dental disease and a
demographic measure of general health. \emph{American Journal of
Physical Anthropology}, \emph{168}(3), 552--565.
\url{https://doi.org/10.1002/ajpa.23772}

\leavevmode\vadjust pre{\hypertarget{ref-ziesemer16SChallenges2015}{}}%
Ziesemer, K. A., Mann, A. E., Sankaranarayanan, K., Schroeder, H., Ozga,
A. T., Brandt, B. W., Zaura, E., Waters-Rist, A., Hoogland, M.,
Salazar-Garcia, D. C., Aldenderfer, M., Speller, C., Hendy, J., Weston,
D. A., MacDonald, S. J., Thomas, G. H., Collins, M. J., Lewis, C. M.,
Hofman, C., \& Warinner, C. (2015). Intrinsic challenges in ancient
microbiome reconstruction using {16S rRNA} gene amplification. \emph{Sci
Rep}, \emph{5}, 16498. \url{https://doi.org/10.1038/srep16498}

\leavevmode\vadjust pre{\hypertarget{ref-ziesemerGenomeCalculus2018}{}}%
Ziesemer, K. A., Ramos‐Madrigal, J., Mann, A. E., Brandt, B. W.,
Sankaranarayanan, K., Ozga, A. T., Hoogland, M., Hofman, C. A.,
Salazar‐García, D. C., Frohlich, B., Milner, G. R., Stone, A. C.,
Aldenderfer, M., Lewis, C. M., Hofman, C. L., Warinner, C., \&
Schroeder, H. (2018). The efficacy of whole human genome capture on
ancient dental calculus and dentin. \emph{American Journal of Physical
Anthropology}. \url{https://doi.org/10.1002/ajpa.23763}

\leavevmode\vadjust pre{\hypertarget{ref-zijngeBiofilmArchitecture2010}{}}%
Zijnge, V., van Leeuwen, M. B. M., Degener, J. E., Abbas, F., Thurnheer,
T., Gmür, R., \& M. Harmsen, H. J. (2010). Oral {Biofilm Architecture}
on {Natural Teeth}. \emph{PLoS ONE}, \emph{5}(2), e9321.
\url{https://doi.org/10.1371/journal.pone.0009321}

\end{CSLReferences}

\bookmarksetup{startatroot}

\hypertarget{chap-discussion}{%
\chapter{Discussion}\label{chap-discussion}}

Archaeological researchers are presented with a unique challenge.
Because time eventually degrades everything, the archaeological record
will always be incomplete. Barring the invention of time travel---and
depending on your position on travelling back to a time before time
travel is invented---we are limited in our ability to fill these gaps in
our knowledge. Consider it a puzzle that needs to be put back together.
The only problem is that some pieces are permanently missing, while the
rest are mostly broken. Researchers will attempt to complete the puzzle
by fixing the broken pieces with scientific analyses, and recreate the
missing pieces based on what we can see from the broken pieces. To
further complicate things, the methods we use to recreate the broken
pieces may not be able to entirely accurately recreate the pieces, which
results in pieces that look like they fit, but are actually different
from the originals. Dental calculus is an example of a puzzle with many
missing and broken pieces. Even if we analysed dental calculus from a
living person, we would still not be able to completely recreate the
entirety of that persons diet by only looking at the food debris within
the dental calculus. For some reason, some of the things we eat will
leave traces on our teeth, while some will not. Now add to that a few
hundred or thousand years in the ground with physical and chemical
processes that are constantly degrading the organic material, and the
picture becomes even murkier. We can show something is there if we
detect it. But what about the things we don't detect? Were they not
there, or could we not detect them? If they weren't there, why weren't
they there? If the thing in question was consumed, but not entrapped in
the dental calculus; why is this the case?

As shown in \protect\hyperlink{fig-plot-and-wordclouds}{Chapter 1},
dental calculus has become a very popular substance within
archaeological research. One of its primary uses is to reconstruct the
diet of past populations. It's not surprising why this is the case. It
forms and grows inside our mouth over time, and it is in direct contact
with everything we put in our mouth. However, there is limited
systematic and fundamental research and experimentation being conducted
within the fields that make use of archaeological dental calculus. There
are of course exceptions
(\protect\hyperlink{ref-fagernasMicrobialBiogeography2021}{Fagernäs et
al., 2021}; \protect\hyperlink{ref-leonardPlantMicroremains2015}{Leonard
et al., 2015}; \protect\hyperlink{ref-powerChimpCalculus2015}{R. C.
Power et al., 2015};
\protect\hyperlink{ref-powerRepresentativenessDental2021}{Robert C.
Power et al., 2021};
\protect\hyperlink{ref-sotoCharacterizationDecontamination2019}{Soto et
al., 2019}; \protect\hyperlink{ref-trompEDTACalculus2017}{Tromp et al.,
2017}; \protect\hyperlink{ref-velskoMicrobialDifferences2019}{Velsko et
al., 2019}, \protect\hyperlink{ref-velskoHighConservation2023}{2023}),
but they have not addressed the full extent of dental calculus
limitations (nor should they). This type of research should aim to
validate aspects of our current analytical methods on synthetic
materials or through detailed observation and documentation of dietary
habits in living humans (or non-human primates), and critically evaluate
the patterns of information we extract. Methods-validation has also been
conducted on archaeological material
(\protect\hyperlink{ref-fagernasMicrobialBiogeography2021}{Fagernäs et
al., 2021}; \protect\hyperlink{ref-modiCalculusMethodologies2020}{Modi
et al., 2020}; \protect\hyperlink{ref-trompEDTACalculus2017}{Tromp et
al., 2017}), but these studies are limited by the fact that we have no
way of knowing what the original diet looked like. At least not at the
resolution necessary to really scrutinize the results of a method. All
we have are pieces of information from the, likely incomplete, dietary
remains that ended up in the calculus, and from contextual remains, such
as animal bones, food residues, and plant remains, both macro- and
microscopic. And even then we have no way of saying for certain whether
the materials were included in the diet, or just there because our
pathological need for oxygen means the oral cavity is not a closed
system (\protect\hyperlink{ref-radiniFoodPathways2017}{Radini et al.,
2017}).

In this dissertation, I have mainly focused on the development,
validation, and application of an oral biofilm model and its potential
for informing archaeological research. I have shown that it was possible
to develop a protocol for an oral biofilm model with a relatively simple
setup, and use it to grow artificial dental calculus, and that it can
serve as a reasonable proxy to natural dental calculus
{[}\protect\hyperlink{byoc-valid}{Chapter 3}; Bartholdy, Velsko, et al.
(\protect\hyperlink{ref-bartholdyAssessingValidity2023}{2023})). I
demonstrated how the oral biofilm model can answer questions and
identify hidden biases related to using dental calculus for paleodietary
reconstructions, specifically addressing the identification and
quantification of starch granules. The results from this study showed
that what goes in, doesn't necessarily come out. And the loss of
information is not evenly distributed across the different types of
starches, depending on size and morphology
{[}\protect\hyperlink{byoc-starch}{Chapter 4}; Bartholdy \& Henry
(\protect\hyperlink{ref-bartholdyInvestigatingBiases2022}{2022}){]}. In
\protect\hyperlink{mb11CalculusPilot}{Chapter 5} I present a study that
goes beyond the model and looks at archaeological dental calculus. This
is, after all, a dissertation in archaeology. We analysed dental
calculus samples from a rural Dutch archaeological site in
Middenbeemster, using ultra high performance liquid chromatography
tandem mass spectrometry (UHPLC-ESI-MS/MS). This allowed us to identify
a number of residues from plants that may have been consumed for
nutrition, medicine, recreation, or all of the above.

\hypertarget{the-dental-calculus-model}{%
\section{The dental calculus model}\label{the-dental-calculus-model}}

While the use of oral biofilm models in dental research is
well-established, even long-term calcifying models to produce dental
calculus, they never made it into archaeological research. At least not
to the extent that they were published. The oral biofilm model outlined
in this dissertation is by no means the ultimate solution to save us
from the limitations of archaeological dental calculus, but may provide
a small step towards understanding them a little better, and hopefully
promote further exploration through systematic fundamental research. The
goal of developing a dental calculus model was to explore core aspects
of how we use dental calculus in paleodietary research, with a
relatively simple setup that is accessible to most labs in
archaeological science. The idea is to take a step back and really
scrutinise our current methods for interpreting diet from dental
calculus. What the field has accomplished so far is undeniably
impressive, but there are many things we still don't understand. Some of
the things we don't understand are on a very basic level, such as how
plant microremains become trapped inside calculus, how much of what we
consume ends up inside calculus, and to what extent our current methods
are able to accurately extract that information.

The model we chose was a simple model using a shaking incubator and a 24
deepwell plate with the plastic lids as a substratum. The artificial
saliva we used was based on the basal modified medium used by Sissons
and colleagues
(\protect\hyperlink{ref-sissonsMultistationPlaque1991}{1991},
\protect\hyperlink{ref-sissonsPHResponse1994}{1994};
\protect\hyperlink{ref-sissonsArtificialPlaque1997}{1997}) to grow
dental calculus. We also made use of their calcifying solution, calcium
phosphate monofluorophosphate urea (CPMU) to speed up the mineralisation
process (natural dental calculus can take weeks, even months, to form).
To make sure the calculus we were growing in the lab was a good model
for calculus grown naturally, we sequenced the DNA of our model calculus
and compared it to samples from various sites inside the human mouth,
including dental plaque and calculus. The bacterial composition of our
model calculus samples had a strong oral signature, but was distinct
from other natural oral samples, including modern dental plaque and
calculus. The main difference between natural samples and model calculus
was that the natural samples were more heterogenous in composition. They
had a larger number and variety of microbes compared to the model
calculus. This was reflected in the aerotolerance of dominant microbes
in model calculus, which were largely anaerobes, while the most abundant
microbes in natural samples were aerobes and facultative anaerobes. The
natural samples also had a more diverse representation of bacteria from
all stages of biofilm development, including early- middle-, and
late-colonisers, while model calculus samples were predominantly
late-colonisers {[}\protect\hyperlink{byoc-valid}{Chapter 3}; Bartholdy,
Velsko, et al.
(\protect\hyperlink{ref-bartholdyAssessingValidity2023}{2023}){]}.
Results from our metagenomic analysis were similar to a comparable
\emph{in vitro} biofilm model. In their study, the authors also used a
24-well plate with pooled saliva as inoculate. The growth medium was
similar but also contained a sheep's-blood serum, and the samples were
only grown for 24 hours
(\protect\hyperlink{ref-edlundUncoveringComplex2018}{Edlund et al.,
2018}). As with our model, the comparison with natural oral samples
showed a lower overall richness and diversity, and a distinct microbial
profile {[}\protect\hyperlink{byoc-valid}{Chapter 3}; Bartholdy, Velsko,
et al.
(\protect\hyperlink{ref-bartholdyAssessingValidity2023}{2023}){]}. Given
that our results are similar to a short-term biofilm model, we may be
replacing the medium too often (every three days), and not allowing
communities to establish more complex metabolic pathways that are
normally present in mature biofilms. To resolve this and other issues,
our protocol will benefit from further refinement. Using serum in the
medium may help to establish thicker and more stable biofilms, and allow
slow-growing organisms to become more established
(\protect\hyperlink{ref-ammannZurichBiofilm2012}{Ammann et al., 2012}).
Filter-sterilising the heat-sensitive solutions that are not autoclaved,
such as CPMU and starch solutions, may prevent environmental
contamination from entering the biofilm during the setup, such as
members of the \emph{Enterococcus} genus. While these are commonly
present in oral samples, they were significantly more abundant in our
samples than the natural oral samples to which we compared them. Once
changes to the model setup, the model will have to be re-validated, as
the concentrations of nutrients, let alone the type of nutrients, will
impact the community composition of the biofilms
(\protect\hyperlink{ref-edlundBiofilmModel2013}{Edlund et al., 2013}).

We also used Fourier Transform Infrared (FTIR) spectroscopy to assess
the mineral content of our model and compare to natural dental calculus,
both modern and archaeological. Our analysis showed that, after 25 days
of growth, our biofilm model produced a substance that is chemically
very similar to both modern and archaeological calculus. It is
interesting that the mineral composition was so similar to natural
calculus given the unique microbial profile. It suggests that the
mineralisation occurs in a predictable manner regardless of the
microbial profile, if conditions are favourable. Even in the absence of
known mineraliser, \emph{Corynebacterium matruchotii}. The crystallinity
of the model calculus also matched the archaeological sample we used as
a comparison, though with a slightly less ordered structure. This may be
related to the age differences in model calculus compared to
archaeological calculus. Not only did the archaeological calculus spend
a few hundred years maturing in the ground, allowing crystals to expand
into the gaps created by degraded organic matter
(\protect\hyperlink{ref-weinerBiologicalMaterials2010}{Weiner, 2010a}),
but given the known lack of oral hygiene practices in the past, the
calculus was surely older than 25 days before being buried. We also only
analysed a single archaeological sample, so we don't know how
representative this sample is of archaeological samples in general.
Perhaps this was a particularly under- or over-mineralised sample. It
would be more appropriate to compare to the modern reference samples,
since we are actually trying to recreate something that mimics natural
modern calculus, not something that has been buried for hundreds of
years or more. Unfortunately we didn't have access to new modern samples
and couldn't produce modern calculus grind curves for this analysis.

\hypertarget{model-application}{%
\subsection{Model application}\label{model-application}}

After establishing that our model dental calculus mimics, at least to
some extent, the real deal, we assessed what biases may occur in starch
incorporation. It is a mistake to think you can solve any major problems
just with potatoes (\protect\hyperlink{ref-adamsLifeUniverse2002}{Adams,
2002a}), so we also included wheat starch in the model to cover a wider
range of granule shapes and sizes. Put simply, we added a known amount
of starch granules---well, to the extent we could estimate the large
quantities in our starch solutions without counting every single
granule---to our biofilm over the course of the 25-day experiment. Then
we dissolved the calculus and counted the number of starches that that
were inside. Those who are familiar with previous dietary research on
archaeological dental calculus will probably not be surprised that the
number of starches we extracted was nowhere near the amount we put in.
More interestingly, though, the size of the starch granules influenced
the outcome; fewer large starches were extracted than what was put in
the model during growth. This could be related to how starch granules
are trapped in biofilms in the first place, where size and/or surface
morphology of the starch granules could influence the likelihood of
being retained in the biofilm. We also found that a very, VERY, low
proportion of the starch granules that we `fed' our samples actually
made it into the dental calculus; only 0.06\% to 0.16\% of granules from
the treatment solutions were extracted from the dental calculus
{[}\protect\hyperlink{byoc-starch}{Chapter 4}; Bartholdy \& Henry
(\protect\hyperlink{ref-bartholdyInvestigatingBiases2022}{2022}){]}.
Given how few actually make it in, this may suggest that evidence for
dietary starches are the result of repeated exposure to a large quantity
of granule-containing foods.

\hypertarget{disc-model-limitations}{%
\subsection{Model limitations}\label{disc-model-limitations}}

So far I have covered what our biofilm model can do. It is equally
important to talk about what our model can't do. After all, we demand
rigidly defined areas of doubt and uncertainty
(\protect\hyperlink{ref-adamsHitchhikersGuide2002}{Adams, 2002c}). While
we have a high degree of control and reproducibility, especially when
compared to \emph{in vivo} models, there are certain conditions we
cannot regulate with our current setup. This includes environmental
conditions such as CO\textsubscript{2} and oxygen availability, which
rely on the conditions in the lab where the experiments take place. To
some extent, the bacterial communities within a biofilm can generate
favorable conditions in a local environment through metabolic
processes---one of the adaptive benefits from being part of a
biofilm---but these are still somewhat dependent on the extrinsic
environment in which they are situated. Biofilms on hard tissues will
differ in composition from those found on soft tissues. And biofilms
found closer to the front of the mouth will differ from those found
towards the back
(\protect\hyperlink{ref-kolenbranderOralMultispecies2010}{Kolenbrander
et al., 2010}; \protect\hyperlink{ref-marshDentalPlaque2005}{Philip D.
Marsh, 2005};
\protect\hyperlink{ref-palmerCoaggregationInteractions2003}{Robert J.
Palmer Jr. et al., 2003};
\protect\hyperlink{ref-proctorSpatialGradient2018}{Proctor et al.,
2018}). This difference is also something that is difficult to mimic in
a single experimental setup; as is the ability to control salivary flow
rates and circadian rhythms, both of which can influence the growth of
plaque (\protect\hyperlink{ref-dawesCircadianRhythms1972}{C. Dawes,
1972}; \protect\hyperlink{ref-proctorSpatialGradient2018}{Proctor et
al., 2018}).

The effect of circadian rhythms differences in microbiome between
individuals can influence replication of the microbial composition of
our model, which will be limited by our use of whole saliva as inoculum
rather than using a handful of select species. This means microbial
profiles of the biofilms may change between (or even within)
experiments, since the microbial composition of our saliva can vary
slightly throughout the day, and the formation and composition therefore
depends on the time of day the saliva is collected. It can also differ
between donors. We reduced these limitations in our experiments by
collecting samples from a single donor at the same time of day for each
inoculation, but this will still cause differences between experiments.

The absence of \(\alpha\)-amylase in our model may have affected the
microbial composition of our biofilms. Our model has no renewable source
for \(\alpha\)-amylase once the inoculations have been completed. There
are streptococcal species present in the model that are known for their
ability to bind amylase
(\protect\hyperlink{ref-haaseComparativeGenomics2017}{Haase et al.,
2017}; \protect\hyperlink{ref-nikitkovaStarchBiofilms2013}{Nikitkova et
al., 2013}); however, we did not investigate whether the strains present
in our model contain these genes. Starch solutions were only introduced
on day 9 of the experiment. Prior to this, all samples were treated with
the sucrose solution. The absence of starch during inoculation could
have suppressed bacterial production of amylase-binding proteins
(\protect\hyperlink{ref-nikitkovaEffectStarch2012}{Nikitkova et al.,
2012}). Frequent medium replacements may also be clearing out all of the
unbound host salivary amylase. We don't know exactly why
\(\alpha\)-amylase is absent, and need to look into this. In the
meantime, this absence opens up opportunities to examine its role in the
incorporation process of dietary materials (see
\protect\hyperlink{bfmodels-in-arch}{below}).

A well-known limitation of biofilm models in general is the difficulty
in capturing the diversity and complexity of the natural oral biome.
Diversity and complexity may be represented as interspecies communities
and complex metabolic dependencies between organisms within the
communitues, or as an environmental complexity determined by nutrient
availability, host immune-responses to biofilms, and fluctuating
microenvironments across the biofilm in response to these factors
(\protect\hyperlink{ref-bjarnsholtVivoBiofilm2013}{Bjarnsholt et al.,
2013}; \protect\hyperlink{ref-edlundUncoveringComplex2018}{Edlund et
al., 2018}). These limitations can be mitigated by complex experimental
setups, but at the cost of lower throughput and higher financial cost.
Increasing the number of species included in a model can approach the
diversity found in the natural microbiome, but still falls short of
capturing the complete diversity
(\protect\hyperlink{ref-edlundBiofilmModel2013}{Edlund et al., 2013}),
and the use of whole saliva introduces another set of limitations (as
discussed above).

Then of course there's the inevitable limitation that we're dealing with
a model. An attempt to recreate the real thing under controlled
conditions, allowing us to test a variety of {[}situations{]} and see
what the outcome might look like in the real world. These are
generalizations that may not be comparable to any specific case. The
very isolated and controlled model setup also deviates from the natural
conditions in our mouths. Many of the biofilm's natural predators are
not present in our setup. Plaque is constantly at risk of removal by the
tongue, salivary flow, and oral hygiene practices, processes which
likely help shape the biofilm (ironic since they are processes of
removal) (\protect\hyperlink{ref-shawCommonalityElastic2004}{Shaw et
al., 2004}).

\hypertarget{further-model-validation}{%
\subsection{Further model validation}\label{further-model-validation}}

Going forward, we aim to further assess the validity of our model, as
well as optimise the protocol. While we have established that our model
is capable of forming a mineral composite comprising a largely oral
microbiome, there are properties that we have yet to determine. Just
because the bacteria in our model are identified as oral, doesn't mean
they necessarily behave like communities of natural oral bacteria. By
determining the functional and metabolic profiles of the bacteria and
communities within our model, we hope to get further insights on
metabolic dependencies, production of metabolic by products, and gene
expression in our model. As a result we will be able to further optimise
the protocol to more closely mimic the natural oral biome.

There are also other conditions within our model that we need to
determine, such as monitoring physiological responses to changing
conditions. For example, after carbohydrates have been consumed, there
is a dip in the pH within the oral cavity as the carbohydrates are
consumed by bacteria, which release acidic by-products. This occurs
within the first few hours of consuming carbohydrates, after which the
saliva will work to balance the pH back to pre-carbohydrate levels, also
known as the `Stephan curve'
(\protect\hyperlink{ref-stephanStudiesChanges1947}{Stephan \& Hemmens,
1947}). By acting as a buffer and restoring the oral pH-level, saliva
can help prevent high levels of acid from demineralising the tooth
surface and causing caries. Since our model is fed both with sucrose and
starch, it is important to know that the pH levels don't permanently
drop to levels that are unfavourable to mineral supersaturation and
plaque mineralisation.

Since FTIR only addresses the overall mineral composition, we will need
to further investigate whether there are any other structural/chemical
differences between our model and natural calculus that may be caused by
microbial profiles, and microscopically examine the model to determine
the micro-architecture.

\hypertarget{bfmodels-in-arch}{%
\subsection{Potential biofilm model applications in
archaeology}\label{bfmodels-in-arch}}

Biofilm models are an untapped resource in archaeological research,
especially for dental calculus research. Coupled with existing
validation methods to address current dental calculus limitations, the
proverbial sky is the limit. This section describes some possible
archaeological applications for a biofilm model, but is certainly not
complete. It is mainly comprised of questions that arose during the
experiments I conducted, as well as during the analysis of
archaeological material, that I was unable to address in this
dissertation due to time constraints. Hopefully these questions can be
answered by myself or others in the future.

The main question that came up concerns the mechanism of incorporation
of dietary compounds in dental calculus. How does it happen? This
seemingly simple question is particularly challenging, and one that I
hadn't prepared for in my experimental design. Going forward it will be
an important question to answer, as it may influence the likelihood of
certain compounds to become trapped in dental calculus, and at what
point during the formation and mineralisation process this occurs. By
staggering the treatments during the experiment, we may be able to see
if the rate of incorporation varies during biofilm growth, and whether
or not particles can penetrate the surface of the calculus after it has
mineralised. If not, this could mean the layered structure would
indicative of chronological consumption events. If so, what is the size
limit? Can starches infiltrate dental calculus post-burial, or is this
limited to smaller molecules? And do the chemical/physical properties of
molecules and microremains (amylopectin content of starch granules,
polarity and hydrophobicity of molecules, etc) influence their ability
to become incorporated or penetrate the mineralised surface? The
surfaces of starch granules mainly contain polar phospholipids
(\protect\hyperlink{ref-cornejo-ramirezStructuralCharacteristics2018}{Cornejo-Ramírez
et al., 2018}), making the phospholipid bylayer compatible with, or even
attracted to, a biofilm consisting largely of water, whereas hydrophobic
molecules might be less likely to associate with a biofilm. Further
growth of the biofilm with new layers would result in the starch
molecules becoming trapped. Smaller molecules may be able to hitch a
ride through diffusion channels that transport nutrients into the
biofilm (\protect\hyperlink{ref-flemmingBiofilmMatrix2010}{Flemming \&
Wingender, 2010}). Physicochemical properties of the smaller molecules
will likely influence their likelihood of entering, since biofilms are
known for their ability to limit diffusion of some molecule, such as
antibiotics
(\protect\hyperlink{ref-stewartAntimicrobialTolerance2015}{Stewart,
2015}). Once trapped inside the biofilm, retention of the dietary
particles depends on the ability to avoid digestive enzymes that are
commonly used by the communities of bacteria to break down the
macromolecules into more manageable sizes. These are just simplified
examples. The real story is likely more complicated. Diffusion of
molecules has been explored clinically, but mainly focusing on
antibacterial agents (\protect\hyperlink{ref-maModelingDiffusion2010}{R.
Ma et al., 2010};
\protect\hyperlink{ref-stewartAntimicrobialTolerance2015}{Stewart,
2015};
\protect\hyperlink{ref-takenakaDiffusionMacromolecules2009}{Takenaka et
al., 2009}). So far nothing has been done to explore the dietary
perspective in which we're interested.

This question also came up during the analysis of archaeological dental
calculus in \protect\hyperlink{mb11CalculusPilot}{Chapter 5}
(\protect\hyperlink{ref-bartholdyMultiproxyAnalysis2023}{Bartholdy,
Hasselstrøm, et al., 2023}). Based on the presence of many metabolites,
it seems that this may not have been during consumption, but rather
during excretion through saliva, or, put more simply, when the molecules
are on their way out of the body again. This makes some sense, since
food actually spends relatively little time in our mouth while we're
eating, and significantly longer travelling through our body. This may
also explain the very low retention of starch granules we found in
\protect\hyperlink{byoc-starch}{Chapter 4}. It seems that most of the
starch granules are swallowed, while few become lodged in our
teeth/plaque and are eventually trapped in dental calculus. Without
looking into the mechanism by which starches and other food molecules
are incorporated into dental plaque, we are always going to be guessing
(albeit educated guesses) what is happening archaeologically.

An important question to address within the framework of incorporation
pathways, is what role bacteria play in the incorporation of dietary
material, and whether differing bacterial profiles have an impact on the
retention of dietary molecules and microremains. It is likely that they
will cause differential retention given that they make use of a lot of
the food that passes through our mouths with the help of digestive
enzymes (\protect\hyperlink{ref-rogersRoleStreptococcus2001}{J. D.
Rogers et al., 2001}). The important question to answer is how, and, to
what extent, they influence this process. A systematic approach would be
to set up multiple experiments with different sets of defined consortia
grown under the same conditions. On a related note, the absence of host
salivary \(\alpha\)-amylase activity in our model (as shown in
\protect\hyperlink{byoc-starch}{Chapter 4}, Bartholdy \& Henry
(\protect\hyperlink{ref-bartholdyInvestigatingBiases2022}{2022}))
provides an opportunity to explore the effect of various amylase levels
on the incorporation and retention of dietary compounds, especially
starches, in dental calculus. \(\alpha\)-amylase can be purchased from
most laboratory supply companies, and can therefore be added to the
model and explored as a controlled variable. Some bacteria have the
ability to bind \(\alpha\)-amylase in order to use the degradation
products of starches as nutrients
(\protect\hyperlink{ref-nikitkovaEffectStarch2012}{Nikitkova et al.,
2012}; \protect\hyperlink{ref-rogersRoleStreptococcus2001}{J. D. Rogers
et al., 2001}), so the abundance of these bacteria coupled with
\(\alpha\)-amylase activity will likely influence starch retention.

Finally, it's worth noting how important it is to be able to generate an
unlimited number of samples for validating current methods and
developing new ones. Archaeological dental calculus is a finite material
and should be treated as such. We should know exactly what we're doing
when we are analysing samples. If not, then model dental calculus would
be a great substance to try out new things, and even for training
researchers on the range of methods at our disposal.

\hypertarget{dental-calculus-in-archaeology-and-future-challenges}{%
\section{Dental calculus in archaeology and future
challenges}\label{dental-calculus-in-archaeology-and-future-challenges}}

Dental calculus has provided unique perspectives on multiple activities
of humans in the past, from dietary practices to the evolution of the
oral microbiome. Researchers continue to find innovative ways to extract
information from a material that was once discarded. It is uniquely
situated to address diet because of its direct interaction with
everything that enters (and exits) our mouth, some of which leaves clues
behind that are embedded within the calculus itself. There are, however,
still limitations to address to further unlock the potential of dental
calculus to reconstruct past dietary activities. Probably the main
challenge we face in archaeology, let alone studies of dental calculus,
is identifying contamination versus the authentic remains left behind
from the past. A challenge more specifically related to dental calculus,
is understanding why some things are retained in dental calculus, and
why others are not. Finally, we should continue to optimise our sampling
and analytical methods to make sure we are getting the most out of these
small deposits of minerals, bacteria, food debris, and whatever else
made its way into the mouth during life.

\hypertarget{incorporation-pathways}{%
\subsection{Incorporation pathways}\label{incorporation-pathways}}

Perhaps the main challenge of working with dental calculus is our lack
of understanding of incorporation pathways. We need a better
understanding of how exogenous material becomes trapped inside, and to
what extent the processes within the oral cavity cause damage to, or
completely eliminate, the dietary compounds.

Larger particles (relatively speaking), such as dietary starches and
phytoliths are likely incorporated during consumption of containing
foods. What exactly about their morphology or physicochemical properties
allows them to enter and become trapped is still unknown. Therefore we
also don't understand why the remains of some plant species are
overrepresented while others are underrepresented. This pathway will be
heavily influenced by mode of consumption. If someone was chewing
tobacco or storing coca in their cheeks, the most likely place to detect
nicotine or cocaine, the principal alkaloids of these plants, would be
in dental calculus deposits on the molars. However, mucous-rich saliva,
produced by the sublingual and submandibular glands (located in the
front of the mouth), preferentially binds toxins
(\protect\hyperlink{ref-doddsHealthBenefits2005}{Michael W. J. Dodds et
al., 2005}), making the anterior teeth a good hypothetical target for
detecting these compounds.

Another consideration is the presence of molecules in dental calculus as
a result of excretion from the body through the saliva. If you consider
the amount of time you spend with food (or other things) in your mouth,
it is relatively short. A few minutes at most? Whereas the time spent in
your body is much longer, as food molecules enter the bloodstream and
are distributed throughout the body. The molecules can then re-enter the
mouth through the saliva and spend significantly more time in the mouth
the second time around, as excretion may take days
(\protect\hyperlink{ref-leeOralFluid2011}{Lee et al., 2011}). At this
point the original compounds may have been broken down by, for example,
the liver or kidneys, in which case mainly the metabolites will be
present. The plausibility of finding molecules via this pathway depends
on the size of the molecules and the ability to diffuse from
serum/plasma to saliva and enter the oral cavity. Given this
incorporation pathway, the molecules are, hypothetically, more likely to
be secreted in higher concentrations through the serum-rich saliva of
the parotid glands, located next to the molars
(\protect\hyperlink{ref-doddsHealthBenefits2005}{Michael W. J. Dodds et
al., 2005}). Molecules originating from this pathway would mean that it,
unfortunately, wouldn't be possible to determine the mode of consumption
(e.g.~chewing vs.~smoking) based on the mass spectrometric results
alone, but would also require analysis of the dentition to identify
tooth staining and periodontal disease, and rely on contextual materials
found at the site (something which should be done anyway). Further
testing through systematic sampling of different parts of the dentition
is needed.

\hypertarget{identification-of-fragmented-remains}{%
\subsection{Identification of fragmented
remains}\label{identification-of-fragmented-remains}}

Identifying and quantifying plant microremains has a particular set of
challenges, even before the food has entered our mouth. Humans have
become reliant on processing foods to aid digestion and to maximise the
energy acquired from eating. Unfortunately, this also means that the
microremains are put through various damaging processes during
preparation
(\protect\hyperlink{ref-graneroStarchTaphonomy2020}{García-Granero,
2020}). Pre-cooking processing may already render starch granules
unidentifiable (\protect\hyperlink{ref-liInfluenceGrinding2020}{W. Li et
al., 2020}). During cooking, starch granules are, at best, modified and,
at worst, completely destroyed depending on the cooking method
(\protect\hyperlink{ref-henryCookingStarch2009}{Henry et al., 2009}).
The granules that survive the cooking process are then submitted to
further harm in the oral cavity by the act of chewing and the presence
of digestive enzymes. After death, the starch granules that are trapped
in dental calculus will have to resist degradation from the burial
environment, including bacteria, funghi, and water damage
(\protect\hyperlink{ref-graneroStarchTaphonomy2020}{García-Granero,
2020}). To add final insult to injury, further damage can occur during
excavation and processing of the dental calculus
(\protect\hyperlink{ref-trompEDTACalculus2017}{Tromp et al., 2017}), and
even during preparation for microscopic identification
(\protect\hyperlink{ref-graneroStarchTaphonomy2020}{García-Granero,
2020}). Through all this, there are still dietary molecules and
microremains that somehow survive hundreds-to-thousands of years inside
dental calculus, and remain identifiable. Our next challenge is to
determine how to interpret these remaining microremains. To date, most
experimental methods have addressed the damage and modifications
occurring to microremains present on tools and cooking utensils
(\protect\hyperlink{ref-langejansRemainsDay2010}{Langejans, 2010};
\protect\hyperlink{ref-liInfluenceGrinding2020}{W. Li et al., 2020};
\protect\hyperlink{ref-maMorphologicalChanges2019}{Z. Ma et al., 2019}),
and not in the context of dental calculus. Given the added processes
affecting the survival and morphology of microremains unique to the oral
cavity, this context is very important. Validation conducted on
archaeological remains will suffer from the same limitations as \emph{in
vivo} studies, namely the variability of dental calculus growth. The
variability can affect comparisons between two or more individuals, as
well as between dental calculus deposits within the oral cavity of a
single individual. The human oral cavity is home to many unique
environments causing differences in the chemical and bacterial makeup of
dental calculus
(\protect\hyperlink{ref-fagernasMicrobialBiogeography2022}{Fagernäs et
al., 2022};
\protect\hyperlink{ref-hayashizakiSiteSpecific2008}{Hayashizaki et al.,
2008}). Our best option to control these many factors and explore the
precise nature of their individual impact on the incorporation and
retention of dietary materials in dental calculus, is to conduct
controlled experiments in a lab.

More recently developed methods offer us the ability to make
identifications on a much smaller scale. The `omics' approaches can be
used to detect many compounds which are otherwise invisible to the
naked, microscopically-aided, eye. There are still limitations to these
methods. Ancient DNA (aDNA) and paleoproteomics are limited by the low
amount of diet-related genetic material present in dental calculus
compared to an overwhelming number of host-associated genomes related to
the millions of microbes inhabiting the oral cavity. Further
complicating the matter is the inability to assign damaged DNA sequences
to a single precise species designation, and instead relying on low
resolution estimates (\protect\hyperlink{ref-mannHaveSomething2023}{Mann
et al., 2023}). Similar issues are encountered in protein identification
(\protect\hyperlink{ref-hendyAncientProtein2021}{Hendy, 2021}).

Adding to the challenge is the fact that not all materials will degrade
in a similar manner. Some materials/molecules are more robust than
others. To what extent, then, can we interpret the difference between
the abundance, or even presence and absence, of materials detected
within and between individuals? We know that the stability of molecules
plays a role in what will ultimately be detectable by mass spectrometry.
The chances of finding principal pharmacologically active or
psychoactive constituents of plants, such as morphine or
tetrahydrocannabinol, are relatively slim since these molecules are
unstable and have a hard enough time surviving decades, let alone
(pre-)historic timescales
(\protect\hyperlink{ref-lindholstLongTerm2010}{Lindholst, 2010}).
Protein and bacterial abundances are also impacted by differential
degradation (\protect\hyperlink{ref-hendyAncientProtein2021}{Hendy,
2021}). This makes it hard to determine whether the quantities of
molecules are an accurate reflection of the quantities during life,
which in turn complicates interpretations we make on the health and diet
of individuals.

\hypertarget{contamination-and-lab-processing}{%
\subsection{Contamination and lab
processing}\label{contamination-and-lab-processing}}

It has been shown that dental calculus preserves well, and that little
external contamination enters the calculus after burial
(\protect\hyperlink{ref-warinnerPathogensHost2014}{Warinner, Rodrigues,
et al., 2014}). Dental calculus is a robust material. After all, it's
made from a lot of the same material as bone. It can clearly provide
good protection to the microremains and various molecules trapped
inside, and survive thousands of years
(\protect\hyperlink{ref-yatesOralMicrobiome2021}{Fellows Yates et al.,
2021}; \protect\hyperlink{ref-henryNeanderthalCalculus2014}{Henry et
al., 2014}). It is, however, not impenetrable. In fact, it can be quite
porous (\protect\hyperlink{ref-friskoppComparativeScanning1980}{Friskopp
\& Hammarström, 1980}). This means it's important to consider what may
have been originally trapped within the calculus during life, and what
could have entered post-mortem. The proportions of original to exogenous
material may also change with time, depending on the physicochemical
propoerties of the molecules. It seems that small hydrophilic molecules
are more often lost from dental calculus than larger hydrophobic
molecules, suggesting postmortem movement of water through the substrate
(\protect\hyperlink{ref-velskoDentalCalculus2017}{Velsko, Overmyer, et
al., 2017}). In addition, these molecules may also be present as
contamination in labs or in the burial environment. I cannot stress
enough how important it is to collect control samples from surrounding
soil and to replicate findings in separate labs, with clear
identification of potential contaminants
(\protect\hyperlink{ref-crowtherDocumentingContamination2014}{Crowther
et al., 2014}).

In the study from {[}\protect\hyperlink{mb11CalculusPilot}{Chapter 5};
Bartholdy, Hasselstrøm, et al.
(\protect\hyperlink{ref-bartholdyMultiproxyAnalysis2023}{2023}){]}, we
detected various compounds in dental calculus using UHPLC-MS/MS,
including salicylic acid, a phytohormone from willow trees (\emph{Salix
alba}, for example) with medicinal properties. Willow bark has long been
known for its medicinal properties, and is present in many common foods.
It is therefore not surprising that we found it in the dental calculus
of people from the 19th century. We also know, however, that salicylic
acid is abundant and very mobile in soil. With this in mind, how do we
interpret our findings? There are currently no standards for
authenticating results from GC/LC-MS/MS analyses on archaeological
samples. Research in aDNA uses, among other things, damage patterns from
the sequences to determine whether a sequence is old or not, and there
are many tools available to accomplish this, such as decontam
(\protect\hyperlink{ref-Rdecontam}{Davis et al., 2018}), PMD tools
(\protect\hyperlink{ref-skoglundSeparatingEndogenous2014}{Skoglund et
al., 2014}), HOPS
(\protect\hyperlink{ref-hublerHOPSAutomated2019}{Hübler et al., 2019}),
and cuperdec (\protect\hyperlink{ref-yatesOralMicrobiome2021}{Fellows
Yates et al., 2021}). Similarly paleoproteomic research can look at
markers of degradation, such as deamidation
(\protect\hyperlink{ref-ramsoeDeamiDATESitespecific2020}{Ramsøe et al.,
2020}). We attempted to provide a method to authenticate our finds by
plotting the quantity of compounds in three washes and comparing these
quantities with the quantity extracted from the calculus itself. We
expect to see a decrease in quantities over the three washes as surface
contaminants are removed, and a subsequent increase in quantity as the
calculus is dissolved and the compounds that were embedded within the
calculus are extracted
(\protect\hyperlink{ref-bartholdyMultiproxyAnalysis2023}{Bartholdy,
Hasselstrøm, et al., 2023}). This assumes that the embedded compounds
were incorporated during life, and does not in any way verify that the
molecules are actually old. So what does this mean for our
interpretations? Well, until we can find a way to separate external
contamination from authentic compounds from the past, and quantify the
extent of external contamination in dental calculus, we can say that
they most likely consumed plants containing salicylic acid, but that we
also cannot rule out contamination from the burial environment as a
source. It's most likely a combination of both.

We also included modern synthetic compounds that we know would not have
been present in the past. These included MDMA, Fentanyl, Amphetamine,
and others. We detected cocaine in nine individuals. Cocaine is not a
modern compound, since it has been used for millinea in the Americas
(\protect\hyperlink{ref-abucaCocaTrade2019}{Abduca, 2019};
\protect\hyperlink{ref-indriatiCocaPrehistoric2001}{Indriati \&
Buikstra, 2001};
\protect\hyperlink{ref-springfieldCocaineMetabolites1993}{Springfield et
al., 1993}), however, it didn't become known to Europeans until
colonisation in the late 15th century, and was only widely adopted in
the late 19th century after cocaine was isolated by Albert Niemann
(\protect\hyperlink{ref-abucaCocaTrade2019}{Abduca, 2019};
\protect\hyperlink{ref-marianiCoca1886}{Company, 1886}). This
complicated things. Cocaine is an alkaloid find naturally in the leaves
of various species of coca plants. While we wouldn't expect a rural
population from 19th century Netherlands to have access to coca leaves,
it wasn't impossible to imagine. It was commonly observed to prevent
fatigue and suppress appetite, potentially useful to farmers. There was
some Dutch presence in South America with the Dutch West Indies, and
they even established the \emph{Nederlandsche Cocainefabriek} in
Amsterdam in 1900 (\protect\hyperlink{ref-bosHistoryLicit2006}{A. Bos,
2006}). Given the possible impact of such a finding, we analysed new
samples from the same individuals in a separate lab on different
equipment. We were unable to detect cocaine in any of the replicated
individuals, and it was probably a case of some sort of lab
contamination that managed to slip past our blanks
(\protect\hyperlink{ref-bartholdyMultiproxyAnalysis2023}{Bartholdy,
Hasselstrøm, et al., 2023}). Upon further research, we were unable to
find historic evidence of coca leaf-use in Europe for anything other
than study, and the only small-scale botanical imports were recorded
prior to the late 19th century (the most recent individuals included in
our study were buried in the 1860s). Coca leaves are also susceptible to
decay during travel and may not have been viable for their intended use
once they arrived in Europe
(\protect\hyperlink{ref-abucaCocaTrade2019}{Abduca, 2019}).

Contamination is widely recognised as a risk in all aspects of
archaeological research, including paleobotany
(\protect\hyperlink{ref-crowtherDocumentingContamination2014}{Crowther
et al., 2014}) and aDNA
(\protect\hyperlink{ref-cooperAncientDNA2000}{Cooper \& Poinar, 2000};
\protect\hyperlink{ref-gilbertBiochemicalPhysical2005}{Gilbert, Rudbeck,
et al., 2005};
\protect\hyperlink{ref-gilbertAssessingAncient2005}{Gilbert, Bandelt, et
al., 2005}; \protect\hyperlink{ref-knappSettingStage2012}{Knapp et al.,
2012}; \protect\hyperlink{ref-llamasFieldLaboratory2017}{Llamas et al.,
2017}), often because of bold claims made in the past (no specifics will
be mentioned here). Protocols for dental calculus sampling include
various steps for decontaminating dental calculus, and range from
brushing the surface to UV-radiation and sonication. Although, the use
of liquids for decontamination may be problematic when there are plans
to do biomolecular analyses
(\protect\hyperlink{ref-velskoDentalCalculus2017}{Velsko, Overmyer, et
al., 2017}). Sodium hydroxide (NaOH) has been suggested as a better
decontamination solution based on testing on synthetic precipitates of
calcium phosphate (the principal component of dental calculus)
(\protect\hyperlink{ref-sotoCharacterizationDecontamination2019}{Soto et
al., 2019}). It's not clear how valid approach is since the synthetic
dental calculus was grown without bacteria, and they're generally
responsible for the channels (supplying nutrients) in dental calculus
that would allow a decontaminating agent to seep into the calculus and
affect the microremains. Nevertheless, it is a step in the right
direction.

After decontamination, the dental calculus is dissolved to extract the
remains trapped inside. The exact method for dissolving dental calculus
inside depends on the type of analysis being done. The most commonly
used chemicals for extracting starches from dental calculus are
hydrochloric acid (HCl) and ethylenediaminetetratetraacetic acid (EDTA).
HCl has long been the preferred method for decalcification of dental
calculus for extraction of plant microremains
(\protect\hyperlink{ref-hardyDentalCalculus2016}{Hardy et al., 2016},
\protect\hyperlink{ref-hardyRecoveringInformation2018}{2018}). Although,
there was no apparent testing on the original use of HCl
(\protect\hyperlink{ref-middletonImprovedMethod1990}{W. D. Middleton,
1990}), which was originally developed for extraction of phytoliths,
which are very resistant to chemical degradation
(\protect\hyperlink{ref-cabanesPhytolithAnalysis2020}{Cabanes, 2020}).
It has since become clear that dental calculus is also a rich source of
starch granules (\protect\hyperlink{ref-henryCalculusSyria2008}{Henry \&
Piperno, 2008}; \protect\hyperlink{ref-cummingsMayanCalculus1997}{Scott
Cummings \& Magennis, 1997}), though it's not entirely clear how
resistant starch granules are to degradation by acids. It was briefly
mentioned in Henry \& Piperno
(\protect\hyperlink{ref-henryCalculusSyria2008}{2008}) that weak
solutions of HCl would not affect starch granules, but more recent
research suggests that EDTA can recover more material from
archaeological dental calculus than HCl
(\protect\hyperlink{ref-trompEDTACalculus2017}{Tromp et al., 2017}) and
cause less damage to the starches
(\protect\hyperlink{ref-lemoyneCalculusPretreatments2021}{Le Moyne \&
Crowther, 2021}). Validation of methods on archaeological material is
difficult since we don't really know the starting point.

One way to explore the external contamination of calculus and how it may
affect already present compounds and microremains, is to set up an
experiment where model calculus samples containing known quantities of
compounds (and controls without anything) are buried for different
periods of time (within a reasonable timeframe). We originally attempted
this, but the model calculus protocol was not ready and the model
calculus samples were not sufficiently mineralised to survive in the
ground. The initial biofilm growth and burial are included in a blog
post
(https://www.leidenarchaeologyblog.nl/articles/spit-tartar-and-burial-an-experiments-diary),
but no further results were written up because of the aforementioned
issue with the protocol, and intrusion by a pandemic. This particular
failure motivated me to revise the protocol and properly validate the
grown model dental calculus (see \protect\hyperlink{byoc-valid}{Chapter
3} and Bartholdy, Velsko, et al.
(\protect\hyperlink{ref-bartholdyAssessingValidity2023}{2023})).

There is an art, or rather, a knack to decontamination and dissolution
of dental calculus. The knack lies in learning how to make sure all
contaminants are removed and authentic material is dislodged from the
minerals, and preventing further degradation of the authentic materials
crucial to our understanding of past dietary activities. To continue the
laboured analogy from the beginning of this chapter; we don't want to
cause any more damage to the already broken puzzle pieces. Since it's
clear that water can potentially clear out some of the original
molecules from dental calculus, we need to be careful with lab
processing methods, and more extensive research on the effects of
processing methods needs to be done.

\hypertarget{deliberate-and-efficient-sampling-and-analysis}{%
\subsection{Deliberate and efficient sampling and
analysis}\label{deliberate-and-efficient-sampling-and-analysis}}

Dental calculus has many advantages over other elements from skeletal
remains, especially when it comes to dietary reconstructions. With
dental calculus we can more reliably argue that the substances we find
within are the result of direct consumption. Dental calculus is, after
all, formed inside our mouth, which is, famously, used during the act of
eating. It would be hard to justify the presence of plant microremains
found on any other element of skeletal remains as a result of
consumption. Any starches found outside of dental calculus, even within
the enamel of teeth, would likely have gotten there after death as the
result of environmental contamination. This doesn't mean we can throw
caution to the wind and interpret everything in dental calculus as food
(\protect\hyperlink{ref-radiniFoodPathways2017}{Radini et al., 2017}),
but it is one of the likelier scenarios. Because the formation of dental
calculus is continuous throughout life, the information we extract about
diet more likely reflects a broader time frame, but given the potential
for many growth disruptions and removal, it probably reflects dietary
patterns closer to the individual's death (depending on the size of the
deposit). That being said, other skeletal elements also have advantages
over dental calculus that should be considered when studying diet. When
it comes to studying the childhood of adult individuals, dental calculus
would not be applicable. This is because of the aforementioned cycle of
potential mechanical disruptions, and the fact that dental calculus is
uncommon in younger individuals. Any calculus visible on an adult
skeleton is unlikely to have formed during childhood. Here, enamel
represents the most appropriate choice. Enamel is formed during
childhood and remains largely unchanged during life
(\protect\hyperlink{ref-hillsonDentalAnthropology1996}{Hillson, 1996}),
so any dietary influences from childhood during the time of enamel
formation, which spans around 28 weeks \emph{in utero} to around 16
years (\emph{ex utero}, of course)
(\protect\hyperlink{ref-hillsonDentalAnthropology1996}{Hillson, 1996}),
will be present in the enamel of the adult dentition. Similarly, bone
and dentine (depending on where you sample the dentine) have a slower
turnover, and represent a more stable source of dietary patterns. And
since they are generally not exposed to environmental contamination
during life (otherwise you're in trouble), they may, in some cases, be
more reliable. However, methods using these skeletal elements suffer
from a low resolution, since they can generally ``only'' (highly
exaggerated air quotes since it's still incredibly useful) offer
insights into very broad dietary trends
(\protect\hyperlink{ref-katzenbergStableIsotope2008}{Katzenberg, 2008}),
whereas methods used on dental calculus can be much more specific,
sometimes even incredibly so
(\protect\hyperlink{ref-hendyProteomicCalculus2018}{Hendy et al., 2018};
\protect\hyperlink{ref-scottExoticFoods2021}{A. Scott et al., 2021}).

If sheer quantity of DNA is what you're after then there really is no
better substance than dental calculus. It is estimated to contain up to
170 times more DNA in archaeological samples compared to dentine samples
from the same tooth. The main difference is the presence of microbial
DNA. For human host DNA, the abundance in dentine is typically higher,
though more variable. Dental calculus contains limited host DNA, which
may be difficult to capture given the lower relative abundance compared
than bacterial DNA, and it can be more fragmented
(\protect\hyperlink{ref-mannDifferentialPreservation2018}{Mann et al.,
2018}; \protect\hyperlink{ref-ziesemerGenomeCalculus2018}{Ziesemer et
al., 2018}). This difference is due to the nature of the two substances.
During life, plaque is primarily made up of bacteria, while dentine does
not contain any bacteria. The exception is in some cases of oral
disease, such as periodontitis, where the presence of bacteria is a
byproduct of the disease process. Since dental calculus is also a trap
for food debris, dental calculus can contain plant DNA and food proteins
(\protect\hyperlink{ref-fagernasMicrobialBiogeography2022}{Fagernäs et
al., 2022}; \protect\hyperlink{ref-hendyProteomicCalculus2018}{Hendy et
al., 2018}; \protect\hyperlink{ref-scottExoticFoods2021}{A. Scott et
al., 2021}; \protect\hyperlink{ref-warinnerEvidenceMilk2014}{Warinner,
Hendy, et al., 2014}). The problem with detecting dietary DNA in dental
calculus is the same as for human host DNA; there is very little of it,
and it may be highly damaged. This causes problems when trying to
identify the source of the DNA. If the DNA sequences are not long enough
to distinguish between multiple related sources (e.g.~mammals), then
interpretations can be made difficult
(\protect\hyperlink{ref-mannHaveSomething2023}{Mann et al., 2023}). That
being said, as our techniques develop and we accumulate more complete
reference databases that allow us to make more robust identifications on
smaller DNA fragments, dental calculus can become even more of a
treasure trove of information than it is already.

While they are abundant in the past, dental calculus deposits are quite
small, ranging from less than one to around a hundred milligrams. It is
therefore important to make our sampling as efficient as possible so we
can retain some of the material for future analyses and replication.
Many of the analytical methods used on dental calculus required
destruction of at least part of the sample. When deciding to perform
destructive analyses, it is important to consider the goal of the
research. Dental calculus may not be suitable for all purposes. Horses
for courses. There are likely better sites to sample for human DNA. And
while it has been preferentially targeted due to the fact that it's
technically considered an ectopic growth and is not given the same
ethical scrutiny as skeletal material, maybe it should. After all, it
does contain human DNA, and our microbiomes are unique.

If dental calculus is the best substance for the particular research
goal, then it's important to maximise the information extracted from the
samples, and minimise the amount of sample needed. Since dental calculus
has become the target for many different types of analyses and studies,
there have been attempts to unify extraction protocols for different
analyses to save on time and minimise destructive sampling, such as a
combined extraction protocol for aDNA and proteomics
(\protect\hyperlink{ref-fagernasUnifiedProtocol2020}{Fagernäs et al.,
2020}) and aDNA and plant microremains
(\protect\hyperlink{ref-modiCalculusMethodologies2020}{Modi et al.,
2020}). The sequence of analyses should also be considered, as some
`non-destructive' techniques may cause invisible damage to the samples.
For example, high-powered imaging techniques involving radiation may
affect the quantity and quality of extracted DNA
(\protect\hyperlink{ref-immelEffectXray2016}{Immel et al., 2016}). We
should continue to explore ways to minimise the amount of material
required to conduct our studies.

\hypertarget{thoughts-on-the-future}{%
\section{Thoughts on the future}\label{thoughts-on-the-future}}

It's hard to imagine the future of dental calculus to be anywhere else
than in the hands of biomolecular methods. Further refinement of our
methods will identify and address current weaknesses and improve our
interpretations. Such method validation should be performed on a model
with known input, to accurately assess the outcomes and biases of our
analytical methods. Something that cannot be achieved using
archaeological dental calculus. By validating what we see in an
artificial substrate with known input, we can accelerate our knowledge
and start to make bolder interpretations that are grounded in systematic
experimentation.

A model can provide insights on many of the challenges listed above,
including differential degradation of remains (starches, metabolites,
DNA, proteins, etc.), likelihood of incorporation and retention during
life. What does it mean when we find X number of potato starches and Y
number of grass phytoliths in dental calculus? What does it mean when we
detect certain ratios of metabolites and can we use that to identify a
source? Model calculus is potentially a useful material to test the
recovery rates of unified protocols compared to separating samples and
analyses. Using robust materials as a control, it would be possible to
track the process from incorporation to extraction and quantification
without worrying about what was lost to enzymatic and acidic damage. An
example of such a material is palynospheres, black ceramic spheres which
are used as marker grains because they are resistant to chemical and
mechanical degradation
(\protect\hyperlink{ref-kitabaBlackCeramic2017}{Kitaba \& Nakagawa,
2017}) similar to
(\protect\hyperlink{ref-boyadjianDentalWash2007}{Boyadjian et al.,
2007}).

The wide range of analytical methods that can provide important insights
on dental calculus require a similarly wide range of expertise.
Inter-disciplinary collaboration is an absolute must for analyses
involving a deep understanding of scientific methods, as well as
continuous communication between archaeologists and other fields to
understand the limitations and strengths of methods and interpretations
in an archaeological context. Lists of authors on archaeological papers
are growing; as they should. Paleoproteomics has already shown that it's
possible to detect very specific information about dietary molecules
present in dental calculus, down to the type of food, its source, and
method of processing
(\protect\hyperlink{ref-hendyProteomicCalculus2018}{Hendy et al.,
2018}). It also have the advantage over DNA in that proteins seem to
preserve for longer. Further development of reference databases and
analytical methods is continuously improving the fields of
paleoproteomics and (oral) metagenomics by increasing quantity of, and
confidence in, species identifications of dietary sources and improved
methods for authenticating truly ancient sources of materials. It will
be exciting to see where these fields can lead us as they mature.

Another area which may lead to exciting discoveries is accessing the
layered structure of dental calculus through high-powered imaging
techniques (e.g.
\protect\hyperlink{ref-powerSynchrotronRadiationbased2022}{Robert C.
Power et al., 2022}). Sequential analysis of dental calculus layers to
determine a sequence of incorporation events. Since we can't yet access
information about the age of occurrence of the seemingly haphazard
mineralisation events in dental plaque, it is difficult to envision a
scenario where we can talk about dietary activities and the age of
individuals. Until then, though, it will still be beneficial to be able
to generate a sequence of deposition events and talk about the dietary
material found in each layer.

Amidst a scientific revolution, it's important to remember that there
are things that can be said about dental calculus without using
biomolecular or microscopic methods. Not to mention, visually scoring
calculus deposits is cheaper and requires no specialised equipment. The
presence of dental calculus and the size of the deposit can be
meaningful. For example, Yaussy \& DeWitte
(\protect\hyperlink{ref-yaussyCalculusSurvivorship2019}{2019}) found an
decreased survivorship in individuals with dental calculus formations.
Therefore, it's crucial to record the deposit \emph{in situ} before
proceeding with destructive sampling. This means taking photos and
scoring the deposit using existing methods, such as
(\protect\hyperlink{ref-brothwellDiggingBones1981}{Brothwell, 1981}),
and recording detailed information allowing researchers to filter out
unnecessary information in downstream analyses rather than missing out
on something that was never recorded. Ideally, each surface of the tooth
should be scored separately to retain the most information for future
analyses, and allows calculating a dental calculus index
(\protect\hyperlink{ref-greeneQuantifyingCalculus2005}{T. R. Greene et
al., 2005}). Calculating an index with calculus scored on multiple
surfaces of the teeth allows us to reveal more patterns related to the
presence and absence of dental calculus, such as uneven distribution
within the dental arcade, allowing more fine-grained comparisons between
populations and within different groups in the same population. No
analyses or substances should be considered in isolation. The best
approach considers multiple angles and makes use of multiple lines of
evidence to reach robust interpretations. Not only multiomic approaches,
but studies that incorporate the entire spectrum of archaeological
analyses. --\textgreater{}

Contributing to the knowledge of dental calculus Opening our methods
will facilitate faster improvements to existing protocols, as well as
open up opportunities for researchers in smaller labs. Here I'm not
talking about vague, cryptic methods sections in papers, but detailed
protocols accessible to anyone with the necessary materials and
equipment. Platforms like protocols.io are a great solution (e.g.~
\href{dx.doi.org/10.17504/protocols.io.bvt9n6r6}{10.17504/protocols.io.bvt9n6r6}
and
\href{dx.doi.org/10.17504/protocols.io.dm6gpj9rdgzp/v1}{10.17504/protocols.io.dm6gpj9rdgzp/v1}.
Adopting more open research practices will also make it easier to
incorporate multiple proxies in research studies, as this will no longer
be limited to those with access to enough material and range of
materials to conduct large-scale analyses
(\protect\hyperlink{ref-yatesOralMicrobiome2021}{Fellows Yates et al.,
2021}). Ensuring that we publish our data in a manner that is Findable,
Accessible, Interoperable, and Reusable (FAIR) will promote
reproducibility and replication, two crucial aspects of scientific
research. Creating communities that can promote these practices within
specific fields and subfields can be effective in creating relevant
standards and fostering an environment that promotes equitable research
practices. This has been realised by the SPAAM community and Open
Phytoliths with
\href{https://zenodo.org/record/7789069}{AncientMetagenomeDir} and the
\href{https://zenodo.org/record/6435441}{FAIR Phytoliths Project},
respectively. Many of these initiatives fall on researchers early in
their careers out of a need for more resources or sheer enthusiasm for
what they do. Unfortunately, there are still very few incentives for
organising these resources and practicing Open Science, and instead
rewarding fast science and measures of impact that have somehow been
assigned as important. Out-dated reward systems are preventing the
widespread adoption of open practices and disproportionately impacting
young scholars and early career researchers.

\hypertarget{concluding-remarks}{%
\section{Concluding remarks}\label{concluding-remarks}}

In my dissertation I set out to put dental calculus under the
microscope, scrutinizing what we know about dental calculus, what we
think we know, and what we need to know. To do this I created a model
system that allowed me, and will allow myself and others, to address
fundamental processes involved in all aspects relevant to the dental
calculus analytical lifecycle. Formation and growth. Exposure to dietary
and non-dietary materials. Burial with subsequent degradation of
original materials and the colonisation of materials and molecules from
the burial environment. Decontamination and extraction of materials
trapped within the calculus. Interpretation of the results from applied
analytical methods. With the help of co-authors, the model dental
calculus was examined to ensure that the bacterial and mineral
compositions were sufficient to mimic an oral environment and closely
resembles natural dental calculus. We deemed this to be satisfactory,
but further validation is absolutely encouraged. The model calculus
system was put into action to see what it could contribute to the use of
methods to extract and quantify plant microremains from dental calculus.
It showed that there is more to the process than dietary input, and that
size, morphology, and physicochemical properties of granules may have an
impact on what we ultimately end up seeing in archaeological dental
calculus. We applied a new method, previously validated on cadavers, to
explore the use of dietary and non-dietary alkaloids and metabolites in
a rural Dutch population from the 19th century. Detection of mundane
everyday compounds, such as those present in tea and coffee, has never
been more exciting. Even the absence of compounds raises a number of
questions about why they were absent, and if they were ever there to
begin with. Contamination is omnipresent in archaeological studies,
especially those employing biomolecular methods. Ours was no exception,
with the possible, but unlikely, detection of cocaine. Overall there
were more questions generated during the various projects than I could
possibly hope to answer over the duration of a PhD program (plus a
little extra), and there is a clear need to address many challenges
going forward, some of which may be addressed with oral biofilm models.

I have no doubt that we have just scratched the surface of what dental
calculus can do to inform us about past activities, diet and otherwise.
Novel analyses and biomolecular techniques have already taken us beyond
what was likely imagined back when archaeological dental calculus was
discarded. Microscopy, metagenomics, and paleoproteomics have already
provided incredibly detailed insights into the dietary activities of
people in the past, and will undoubtedly continue to improve our
understanding. Before we can achieve any of these things, though, we
need to take a closer look at how dental calculus incorporates these
markers of diet, and what biases the mechanisms of incorporation may
cause. Advances in dental calculus and dietary reconstructions will
require a deeper understanding of the substance. How it behaves under
certain conditions and how it interacts with the material and
environments with which it comes into contact. This dissertation
provides one possible solution to the need for more fundamental research
required to understand these processes, adding to our toolkit of
method-validation, which already includes ethnographic research, and
experimental archaeology.

We already understand that we are limited in what we can say about diet
in the past from dental calculus, especially from a quantitative
perspective. It's not enough to identify the problems, but rather to
identify the causes of the problems and their implications. With more
systematic research answering more fundamental questions, maybe we can
move beyond these limitations and be a little bolder in our
interpretations. How can we possibly expect to understand diet from
archaeological dental calculus if we don't understand fundamental
processes that lead to dietary components ending up in dental calculus
in the first place? Basically, we need to ask more stupid questions.
They are probably not stupid; it's more likely that they point out
fundamental assumptions that we have made without actually going to the
trouble of testing them. ``You can't possibly be a scientist if you mind
people thinking that you're a fool''
(\protect\hyperlink{ref-adamsLongThanks2002}{Adams, 2002b}).

\hypertarget{references-3}{%
\section*{References}\label{references-3}}
\addcontentsline{toc}{section}{References}

\markright{References}

\hypertarget{refs-12}{}
\begin{CSLReferences}{1}{0}
\leavevmode\vadjust pre{\hypertarget{ref-abucaCocaTrade2019}{}}%
Abduca, R. (2019). Coca leaf transfers to {Europe}. {Effects} on the
consumption of coca in {North-western Argentina}. In M. Kaller \& F.
Jacob (Eds.), \emph{Transatlantic {Trade} and {Global Cultural Transfers
Since} 1492: {More} than {Commodities}}. {Routledge}.
\url{https://books.google.com?id=13imDwAAQBAJ}

\leavevmode\vadjust pre{\hypertarget{ref-adamsLifeUniverse2002}{}}%
Adams, D. (2002a). \emph{Life, the {Universe} and {Everything}}.
{Picador}.

\leavevmode\vadjust pre{\hypertarget{ref-adamsLongThanks2002}{}}%
Adams, D. (2002b). \emph{So {Long}, and {Thanks} for {All} the {Fish}}.
{Picador}.

\leavevmode\vadjust pre{\hypertarget{ref-adamsHitchhikersGuide2002}{}}%
Adams, D. (2002c). \emph{The {Hitchhiker}'s {Guide} to the {Galaxy}}.
{Picador}.

\leavevmode\vadjust pre{\hypertarget{ref-ammannZurichBiofilm2012}{}}%
Ammann, T. W., Gmür, R., \& Thurnheer, T. (2012). Advancement of the
10-species subgingival {Zurich} biofilm model by examining different
nutritional conditions and defining the structure of the in vitro
biofilms. \emph{BMC Microbiology}, \emph{12}, 227.
\url{https://doi.org/10.1186/1471-2180-12-227}

\leavevmode\vadjust pre{\hypertarget{ref-bartholdyMultiproxyAnalysis2023}{}}%
Bartholdy, B. P., Hasselstrøm, J. B., Sørensen, L. K., Casna, M.,
Hoogland, M., Beemster, H. G., \& Henry, A. G. (2023). \emph{Multiproxy
analysis exploring patterns of diet and disease in dental calculus and
skeletal remains from a 19th century {Dutch} population}. {Zenodo}.
\url{https://doi.org/10.5281/zenodo.7649151}

\leavevmode\vadjust pre{\hypertarget{ref-bartholdyInvestigatingBiases2022}{}}%
Bartholdy, B. P., \& Henry, A. G. (2022). Investigating {Biases
Associated With Dietary Starch Incorporation} and {Retention With} an
{Oral Biofilm Model}. \emph{Frontiers in Earth Science}, \emph{10}.
\url{https://www.frontiersin.org/articles/10.3389/feart.2022.886512}

\leavevmode\vadjust pre{\hypertarget{ref-bartholdyAssessingValidity2023}{}}%
Bartholdy, B. P., Velsko, I. M., Gur-Arieh, S., Fagernäs, Z., Warinner,
C., \& Henry, A. G. (2023, May 30). \emph{Assessing the validity of a
calcifying oral biofilm model as a suitable proxy for dental calculus}.
\url{https://doi.org/10.1101/2023.05.23.541904}

\leavevmode\vadjust pre{\hypertarget{ref-bjarnsholtVivoBiofilm2013}{}}%
Bjarnsholt, T., Alhede, M., Alhede, M., Eickhardt-Sørensen, S. R.,
Moser, C., Kühl, M., Jensen, P. Ø., \& Høiby, N. (2013). The in vivo
biofilm. \emph{Trends in Microbiology}, \emph{21}(9), 466--474.
\url{https://doi.org/10.1016/j.tim.2013.06.002}

\leavevmode\vadjust pre{\hypertarget{ref-bosHistoryLicit2006}{}}%
Bos, A. (2006). The {History} of {Licit Cocaine} in the {Netherlands}.
\emph{De Economist}, \emph{154}(4), 581--586.
\url{https://doi.org/10.1007/s10645-006-9031-0}

\leavevmode\vadjust pre{\hypertarget{ref-boyadjianDentalWash2007}{}}%
Boyadjian, C. H. C., Eggers, S., \& Reinhard, K. (2007). Dental wash: A
problematic method for extracting microfossils from teeth. \emph{Journal
of Archaeological Science}, \emph{34}(10), 1622--1628.
\url{https://doi.org/10.1016/j.jas.2006.12.012}

\leavevmode\vadjust pre{\hypertarget{ref-brothwellDiggingBones1981}{}}%
Brothwell, D. (1981). \emph{Digging up {Bones}: {The} excavation,
treatment and study of human skeletal remains} (3rd ed.). {British
Museum (Natural History)}.

\leavevmode\vadjust pre{\hypertarget{ref-cabanesPhytolithAnalysis2020}{}}%
Cabanes, D. (2020). Phytolith {Analysis} in {Paleoecology} and
{Archaeology}. In A. G. Henry (Ed.), \emph{Handbook for the {Analysis}
of {Micro-Particles} in {Archaeological Samples}} (pp. 255--288).
{Springer International Publishing}.
\url{https://doi.org/10.1007/978-3-030-42622-4_11}

\leavevmode\vadjust pre{\hypertarget{ref-marianiCoca1886}{}}%
Company, M. (1886). \emph{Coca {Erythroxylon}: {Its Uses} in the
{Treatment} of {Disease}} (4th ed.). {Mariani \& Co.}
\url{https://dlcs.io/pdf/wellcome/pdf-item/b21069360/0}

\leavevmode\vadjust pre{\hypertarget{ref-cooperAncientDNA2000}{}}%
Cooper, A., \& Poinar, H. N. (2000). Ancient {DNA}: {Do} it right or
{Not} at all. \emph{Science}, \emph{289}.
\url{https://pure.mpg.de/pubman/faces/ViewItemOverviewPage.jsp?itemId=item_1556167}

\leavevmode\vadjust pre{\hypertarget{ref-cornejo-ramirezStructuralCharacteristics2018}{}}%
Cornejo-Ramírez, Y. I., Martínez-Cruz, O., Del Toro-Sánchez, C. L.,
Wong-Corral, F. J., Borboa-Flores, J., \& Cinco-Moroyoqui, F. J. (2018).
The structural characteristics of starches and their functional
properties. \emph{CyTA - Journal of Food}, \emph{16}(1), 1003--1017.
\url{https://doi.org/10.1080/19476337.2018.1518343}

\leavevmode\vadjust pre{\hypertarget{ref-crowtherDocumentingContamination2014}{}}%
Crowther, A., Haslam, M., Oakden, N., Walde, D., \& Mercader, J. (2014).
Documenting contamination in ancient starch laboratories. \emph{Journal
of Archaeological Science}, \emph{49}, 90--104.
\url{https://doi.org/10.1016/j.jas.2014.04.023}

\leavevmode\vadjust pre{\hypertarget{ref-Rdecontam}{}}%
Davis, N. M., Proctor, D. M., Holmes, S. P., Relman, D. A., \& Callahan,
B. J. (2018). Simple statistical identification and removal of
contaminant sequences in marker-gene and metagenomics data.
\emph{Microbiome}, \emph{6}(1), 226.
\url{https://doi.org/10.1186/s40168-018-0605-2}

\leavevmode\vadjust pre{\hypertarget{ref-dawesCircadianRhythms1972}{}}%
Dawes, C. (1972). Circadian rhythms in human salivary flow rate and
composition. \emph{The Journal of Physiology}, \emph{220}(3), 529--545.
\url{https://www.ncbi.nlm.nih.gov/pmc/articles/PMC1331668/}

\leavevmode\vadjust pre{\hypertarget{ref-doddsHealthBenefits2005}{}}%
Dodds, M. W. J., Johnson, D. A., \& Yeh, C.-K. (2005). Health benefits
of saliva: A review. \emph{Journal of Dentistry}, \emph{33}(3),
223--233. \url{https://doi.org/10.1016/j.jdent.2004.10.009}

\leavevmode\vadjust pre{\hypertarget{ref-edlundBiofilmModel2013}{}}%
Edlund, A., Yang, Y., Hall, A. P., Guo, L., Lux, R., He, X., Nelson, K.
E., Nealson, K. H., Yooseph, S., Shi, W., \& McLean, J. S. (2013). An in
vitrobiofilm model system maintaining a highly reproducible species and
metabolic diversity approaching that of the human oral microbiome.
\emph{Microbiome}, \emph{1}(1), 25.
\url{https://doi.org/10.1186/2049-2618-1-25}

\leavevmode\vadjust pre{\hypertarget{ref-edlundUncoveringComplex2018}{}}%
Edlund, A., Yang, Y., Yooseph, S., He, X., Shi, W., \& McLean, J. S.
(2018). Uncovering complex microbiome activities via metatranscriptomics
during 24 hours of oral biofilm assembly and maturation.
\emph{Microbiome}, \emph{6}(1), 217.
\url{https://doi.org/10.1186/s40168-018-0591-4}

\leavevmode\vadjust pre{\hypertarget{ref-fagernasUnifiedProtocol2020}{}}%
Fagernäs, Z., García-Collado, M. I., Hendy, J., Hofman, C. A., Speller,
C., Velsko, I. M., \& Warinner, C. (2020). A unified protocol for
simultaneous extraction of {DNA} and proteins from archaeological dental
calculus. \emph{Journal of Archaeological Science}, \emph{118}, 105135.
\url{https://doi.org/10.1016/j.jas.2020.105135}

\leavevmode\vadjust pre{\hypertarget{ref-fagernasMicrobialBiogeography2021}{}}%
Fagernäs, Z., Salazar-García, D. C., Avilés, A., Haber, M., Henry, A.,
Maurandi, J. L., Ozga, A., Velsko, I. M., \& Warinner, C. (2021).
Understanding the microbial biogeography of ancient human dentitions to
guide study design and interpretation. \emph{bioRxiv},
2021.08.16.456492. \url{https://doi.org/10.1101/2021.08.16.456492}

\leavevmode\vadjust pre{\hypertarget{ref-fagernasMicrobialBiogeography2022}{}}%
Fagernäs, Z., Salazar-García, D. C., Haber Uriarte, M., Avilés
Fernández, A., Henry, A. G., Lomba Maurandi, J., Ozga, A. T., Velsko, I.
M., \& Warinner, C. (2022). Understanding the microbial biogeography of
ancient human dentitions to guide study design and interpretation.
\emph{FEMS Microbes}, \emph{3}, xtac006.
\url{https://doi.org/10.1093/femsmc/xtac006}

\leavevmode\vadjust pre{\hypertarget{ref-yatesOralMicrobiome2021}{}}%
Fellows Yates, J. A., Velsko, I. M., Aron, F., Posth, C., Hofman, C. A.,
Austin, R. M., Parker, C. E., Mann, A. E., Nägele, K., Arthur, K. W.,
Arthur, J. W., Bauer, C. C., Crevecoeur, I., Cupillard, C., Curtis, M.
C., Dalén, L., Bonilla, M. D.-Z., Fernández-Lomana, J. C. D., Drucker,
D. G., \ldots{} Warinner, C. (2021). The evolution and changing ecology
of the {African} hominid oral microbiome. \emph{Proceedings of the
National Academy of Sciences}, \emph{118}(20).
\url{https://doi.org/10.1073/pnas.2021655118}

\leavevmode\vadjust pre{\hypertarget{ref-flemmingBiofilmMatrix2010}{}}%
Flemming, H.-C., \& Wingender, J. (2010). The biofilm matrix.
\emph{Nature Reviews Microbiology}, \emph{8}(9), 623--633.
\url{https://doi.org/10.1038/nrmicro2415}

\leavevmode\vadjust pre{\hypertarget{ref-friskoppComparativeScanning1980}{}}%
Friskopp, J., \& Hammarström, L. (1980). A {Comparative}, {Scanning
Electron Microscopic Study} of {Supragingival} and {Subgingival
Calculus}. \emph{Journal of Periodontology}, \emph{51}(10), 553--562.
\url{https://doi.org/10.1902/jop.1980.51.10.553}

\leavevmode\vadjust pre{\hypertarget{ref-graneroStarchTaphonomy2020}{}}%
García-Granero, J. J. (2020). Starch taphonomy, equifinality and the
importance of context: {Some} notes on the identification of food
processing through starch grain analysis. \emph{Journal of
Archaeological Science}, \emph{124}, 105267.
\url{https://doi.org/10.1016/j.jas.2020.105267}

\leavevmode\vadjust pre{\hypertarget{ref-gilbertAssessingAncient2005}{}}%
Gilbert, M. T. P., Bandelt, H.-J., Hofreiter, M., \& Barnes, I. (2005).
Assessing ancient {DNA} studies. \emph{Trends in Ecology \& Evolution},
\emph{20}(10), 541--544.
\url{https://doi.org/10.1016/j.tree.2005.07.005}

\leavevmode\vadjust pre{\hypertarget{ref-gilbertBiochemicalPhysical2005}{}}%
Gilbert, M. T. P., Rudbeck, L., Willerslev, E., Hansen, A. J., Smith,
C., Penkman, K. E. H., Prangenberg, K., Nielsen-Marsh, C. M., Jans, M.
E., Arthur, P., Lynnerup, N., Turner-Walker, G., Biddle, M.,
Kjølbye-Biddle, B., \& Collins, M. J. (2005). Biochemical and physical
correlates of {DNA} contamination in archaeological human bones and
teeth excavated at {Matera}, {Italy}. \emph{Journal of Archaeological
Science}, \emph{32}(5), 785--793.
\url{https://doi.org/10.1016/j.jas.2004.12.008}

\leavevmode\vadjust pre{\hypertarget{ref-greeneQuantifyingCalculus2005}{}}%
Greene, T. R., Kuba, C. L., \& Irish, J. D. (2005). Quantifying
calculus: {A} suggested new approach for recording an important
indicator of diet and dental health. \emph{HOMO - Journal of Comparative
Human Biology}, \emph{56}(2), 119--132.
\url{https://doi.org/10.1016/j.jchb.2005.02.002}

\leavevmode\vadjust pre{\hypertarget{ref-haaseComparativeGenomics2017}{}}%
Haase, E. M., Kou, Y., Sabharwal, A., Liao, Y.-C., Lan, T., Lindqvist,
C., \& Scannapieco, F. A. (2017). Comparative genomics and evolution of
the amylase-binding proteins of oral streptococci. \emph{BMC
Microbiology}, \emph{17}(1), 94.
\url{https://doi.org/10.1186/s12866-017-1005-7}

\leavevmode\vadjust pre{\hypertarget{ref-hardyRecoveringInformation2018}{}}%
Hardy, K., Buckley, S., \& Copeland, L. (2018). Pleistocene dental
calculus: {Recovering} information on {Paleolithic} food items,
medicines, paleoenvironment and microbes. \emph{Evolutionary
Anthropology: Issues, News, and Reviews}, \emph{27}(5), 234--246.
\url{https://doi.org/10.1002/evan.21718}

\leavevmode\vadjust pre{\hypertarget{ref-hardyDentalCalculus2016}{}}%
Hardy, K., Radini, A., Buckley, S., Sarig, R., Copeland, L., Gopher, A.,
\& Barkai, R. (2016). Dental calculus reveals potential respiratory
irritants and ingestion of essential plant-based nutrients at {Lower
Palaeolithic Qesem Cave Israel}. \emph{Quaternary International},
\emph{398}, 129--135. \url{https://doi.org/10.1016/j.quaint.2015.04.033}

\leavevmode\vadjust pre{\hypertarget{ref-hayashizakiSiteSpecific2008}{}}%
Hayashizaki, J., Ban, S., Nakagaki, H., Okumura, A., Yoshii, S., \&
Robinson, C. (2008). Site specific mineral composition and
microstructure of human supra-gingival dental calculus. \emph{Archives
of Oral Biology}, \emph{53}(2), 168--174.
\url{https://doi.org/10.1016/j.archoralbio.2007.09.003}

\leavevmode\vadjust pre{\hypertarget{ref-hendyAncientProtein2021}{}}%
Hendy, J. (2021). Ancient protein analysis in archaeology. \emph{Science
Advances}, \emph{7}(3), eabb9314.
\url{https://doi.org/10.1126/sciadv.abb9314}

\leavevmode\vadjust pre{\hypertarget{ref-hendyProteomicCalculus2018}{}}%
Hendy, J., Warinner, C., Bouwman, A., Collins, M. J., Fiddyment, S.,
Fischer, R., Hagan, R., Hofman, C. A., Holst, M., Chaves, E., Klaus, L.,
Larson, G., Mackie, M., McGrath, K., Mundorff, A. Z., Radini, A., Rao,
H., Trachsel, C., Velsko, I. M., \& Speller, C. F. (2018). Proteomic
evidence of dietary sources in ancient dental calculus.
\emph{Proceedings. Biological Sciences}, \emph{285}(1883), 20180977.
\url{https://doi.org/10.1098/rspb.2018.0977}

\leavevmode\vadjust pre{\hypertarget{ref-henryNeanderthalCalculus2014}{}}%
Henry, A. G., Brooks, A. S., \& Piperno, D. R. (2014). Plant foods and
the dietary ecology of {Neanderthals} and early modern humans.
\emph{Journal of Human Evolution}, \emph{69}, 44--54.
\url{https://doi.org/10.1016/j.jhevol.2013.12.014}

\leavevmode\vadjust pre{\hypertarget{ref-henryCookingStarch2009}{}}%
Henry, A. G., Hudson, H. F., \& Piperno, D. R. (2009). Changes in starch
grain morphologies from cooking. \emph{Journal of Archaeological
Science}, \emph{36}(3), 915--922.
\url{https://doi.org/10.1016/j.jas.2008.11.008}

\leavevmode\vadjust pre{\hypertarget{ref-henryCalculusSyria2008}{}}%
Henry, A. G., \& Piperno, D. R. (2008). Using plant microfossils from
dental calculus to recover human diet: A case study from {Tell}
al-{Raqā}'i, {Syria}. \emph{Journal of Archaeological Science},
\emph{35}(7), 1943--1950.
\url{https://doi.org/10.1016/j.jas.2007.12.005}

\leavevmode\vadjust pre{\hypertarget{ref-hillsonDentalAnthropology1996}{}}%
Hillson, S. (1996). \emph{Dental {Anthropology}}. {Cambridge University
Press}.

\leavevmode\vadjust pre{\hypertarget{ref-hublerHOPSAutomated2019}{}}%
Hübler, R., Key, F. M., Warinner, C., Bos, K. I., Krause, J., \& Herbig,
A. (2019). {HOPS}: Automated detection and authentication of pathogen
{DNA} in archaeological remains. \emph{Genome Biology}, \emph{20}(1),
280. \url{https://doi.org/10.1186/s13059-019-1903-0}

\leavevmode\vadjust pre{\hypertarget{ref-immelEffectXray2016}{}}%
Immel, A., Le Cabec, A., Bonazzi, M., Herbig, A., Temming, H.,
Schuenemann, V. J., Bos, K. I., Langbein, F., Harvati, K., Bridault, A.,
Pion, G., Julien, M.-A., Krotova, O., Conard, N. J., Münzel, S. C.,
Drucker, D. G., Viola, B., Hublin, J.-J., Tafforeau, P., \& Krause, J.
(2016). Effect of {X-ray} irradiation on ancient {DNA} in sub-fossil
bones -- {Guidelines} for safe {X-ray} imaging. \emph{Scientific
Reports}, \emph{6}(1, 1), 32969. \url{https://doi.org/10.1038/srep32969}

\leavevmode\vadjust pre{\hypertarget{ref-indriatiCocaPrehistoric2001}{}}%
Indriati, E., \& Buikstra, J. E. (2001). Coca chewing in prehistoric
coastal {Peru}: {Dental} evidence. \emph{American Journal of Physical
Anthropology}, \emph{114}(3), 242--257.
\url{https://doi.org/10.1002/1096-8644(200103)114:3\%3C242::AID-AJPA1023\%3E3.0.CO;2-J}

\leavevmode\vadjust pre{\hypertarget{ref-katzenbergStableIsotope2008}{}}%
Katzenberg, M. A. (2008). Stable {Isotope Analysis}: {A Tool} for
{Studying Past Diet}, {Demography}, and {Life History}. In M. A.
Katzenberg \& S. R. Saunders (Eds.), \emph{Biological {Anthropology} of
the {Human Skeleton}} (pp. 301--340). {John Wiley and Sons}.

\leavevmode\vadjust pre{\hypertarget{ref-kitabaBlackCeramic2017}{}}%
Kitaba, I., \& Nakagawa, T. (2017). Black ceramic spheres as marker
grains for microfossil analyses, with improved chemical, physical, and
optical properties. \emph{Quaternary International}, \emph{455},
166--169. \url{https://doi.org/10.1016/j.quaint.2017.08.052}

\leavevmode\vadjust pre{\hypertarget{ref-knappSettingStage2012}{}}%
Knapp, M., Clarke, A. C., Horsburgh, K. A., \& Matisoo-Smith, E. A.
(2012). Setting the stage -- {Building} and working in an ancient {DNA}
laboratory. \emph{Annals of Anatomy - Anatomischer Anzeiger},
\emph{194}(1), 3--6. \url{https://doi.org/10.1016/j.aanat.2011.03.008}

\leavevmode\vadjust pre{\hypertarget{ref-kolenbranderOralMultispecies2010}{}}%
Kolenbrander, P. E., Palmer, R. J., Periasamy, S., \& Jakubovics, N. S.
(2010). Oral multispecies biofilm development and the key role of
cell--cell distance. \emph{Nature Reviews Microbiology}, \emph{8}(7),
471--480. \url{https://doi.org/10.1038/nrmicro2381}

\leavevmode\vadjust pre{\hypertarget{ref-langejansRemainsDay2010}{}}%
Langejans, G. H. J. (2010). Remains of the day-preservation of organic
micro-residues on stone tools. \emph{Journal of Archaeological Science},
\emph{37}(5), 971--985. \url{https://doi.org/10.1016/j.jas.2009.11.030}

\leavevmode\vadjust pre{\hypertarget{ref-lemoyneCalculusPretreatments2021}{}}%
Le Moyne, C., \& Crowther, A. (2021). Effects of chemical pre-treatments
on modified starch granules: {Recommendations} for dental calculus
decalcification for ancient starch research. \emph{Journal of
Archaeological Science: Reports}, \emph{35}, 102762.
\url{https://doi.org/10.1016/j.jasrep.2020.102762}

\leavevmode\vadjust pre{\hypertarget{ref-leeOralFluid2011}{}}%
Lee, D., Milman, G., Barnes, A. J., Goodwin, R. S., Hirvonen, J., \&
Huestis, M. A. (2011). Oral {Fluid Cannabinoids} in {Chronic}, {Daily
Cannabis Smokers} during {Sustained}, {Monitored Abstinence}.
\emph{Clinical Chemistry}, \emph{57}(8), 1127--1136.
\url{https://doi.org/10.1373/clinchem.2011.164822}

\leavevmode\vadjust pre{\hypertarget{ref-leonardPlantMicroremains2015}{}}%
Leonard, C., Vashro, L., O'Connell, J. F., \& Henry, A. G. (2015). Plant
microremains in dental calculus as a record of plant consumption: {A}
test with {Twe} forager-horticulturalists. \emph{Journal of
Archaeological Science: Reports}, \emph{2}, 449--457.
\url{https://doi.org/10.1016/j.jasrep.2015.03.009}

\leavevmode\vadjust pre{\hypertarget{ref-liInfluenceGrinding2020}{}}%
Li, W., Pagán-Jiménez, J. R., Tsoraki, C., Yao, L., \& Van Gijn, A.
(2020). Influence of grinding on the preservation of starch grains from
rice. \emph{Archaeometry}, \emph{62}(1), 157--171.
\url{https://doi.org/10.1111/arcm.12510}

\leavevmode\vadjust pre{\hypertarget{ref-lindholstLongTerm2010}{}}%
Lindholst, C. (2010). Long term stability of cannabis resin and cannabis
extracts. \emph{Australian Journal of Forensic Sciences}, \emph{42}(3),
181--190. \url{https://doi.org/10.1080/00450610903258144}

\leavevmode\vadjust pre{\hypertarget{ref-llamasFieldLaboratory2017}{}}%
Llamas, B., Valverde, G., Fehren-Schmitz, L., Weyrich, L. S., Cooper,
A., \& Haak, W. (2017). From the field to the laboratory: {Controlling
DNA} contamination in human ancient {DNA} research in the
high-throughput sequencing era. \emph{STAR: Science \& Technology of
Archaeological Research}, \emph{3}(1), 1--14.
\url{https://doi.org/10.1080/20548923.2016.1258824}

\leavevmode\vadjust pre{\hypertarget{ref-maModelingDiffusion2010}{}}%
Ma, R., Liu, J., Jiang, Y., Liu, Z., Tang, Z., Ye, D., Zeng, J., \&
Huang, Z. (2010). Modeling of {Diffusion Transport} through {Oral
Biofilms} with the {Inverse Problem Method}. \emph{International Journal
of Oral Science}, \emph{2}(4, 4), 190--197.
\url{https://doi.org/10.4248/IJOS10075}

\leavevmode\vadjust pre{\hypertarget{ref-maMorphologicalChanges2019}{}}%
Ma, Z., Perry, L., Li, Q., \& Yang, X. (2019). Morphological changes in
starch grains after dehusking and grinding with stone tools.
\emph{Scientific Reports}, \emph{9}(1, 1), 2355.
\url{https://doi.org/10.1038/s41598-019-38758-6}

\leavevmode\vadjust pre{\hypertarget{ref-mannHaveSomething2023}{}}%
Mann, A. E., Fellows Yates, J. A., Fagernäs, Z., Austin, R. M., Nelson,
E. A., \& Hofman, C. A. (2023). Do {I} have something in my teeth? {The}
trouble with genetic analyses of diet from archaeological dental
calculus. \emph{Quaternary International}, \emph{653--654}, 33--46.
\url{https://doi.org/10.1016/j.quaint.2020.11.019}

\leavevmode\vadjust pre{\hypertarget{ref-mannDifferentialPreservation2018}{}}%
Mann, A. E., Sabin, S., Ziesemer, K., Vågene, Å. J., Schroeder, H.,
Ozga, A. T., Sankaranarayanan, K., Hofman, C. A., Fellows Yates, J. A.,
Salazar-García, D. C., Frohlich, B., Aldenderfer, M., Hoogland, M.,
Read, C., Milner, G. R., Stone, A. C., Lewis, C. M., Krause, J., Hofman,
C., \ldots{} Warinner, C. (2018). Differential preservation of
endogenous human and microbial {DNA} in dental calculus and dentin.
\emph{Scientific Reports}, \emph{8}(1, 1), 9822.
\url{https://doi.org/10.1038/s41598-018-28091-9}

\leavevmode\vadjust pre{\hypertarget{ref-marshDentalPlaque2005}{}}%
Marsh, P. D. (2005). Dental plaque: Biological significance of a biofilm
and community life-style. \emph{Journal of Clinical Periodontology},
\emph{32}(s6), 7--15.
\url{https://doi.org/10.1111/j.1600-051X.2005.00790.x}

\leavevmode\vadjust pre{\hypertarget{ref-middletonImprovedMethod1990}{}}%
Middleton, W. D. (1990). An {Improved Method} for {Extraction} of {Opal
Phytoliths} from {Tartar Residues} on {Herbivore Teeth}.
\emph{Phytolitharien Newsletter}, \emph{6}(3), 2--5.

\leavevmode\vadjust pre{\hypertarget{ref-modiCalculusMethodologies2020}{}}%
Modi, A., Pisaneschi, L., Zaro, V., Vai, S., Vergata, C., Casalone, E.,
Caramelli, D., Moggi-Cecchi, J., Mariotti Lippi, M., \& Lari, M. (2020).
Combined methodologies for gaining much information from ancient dental
calculus: Testing experimental strategies for simultaneously analysing
{DNA} and food residues. \emph{Archaeological and Anthropological
Sciences}, \emph{12}(1), 10.
\url{https://doi.org/10.1007/s12520-019-00983-5}

\leavevmode\vadjust pre{\hypertarget{ref-nikitkovaEffectStarch2012}{}}%
Nikitkova, A. E., Haase, E. M., \& Scannapieco, F. A. (2012). Effect of
starch and amylase on the expression of amylase-binding protein {A} in
{Streptococcus} gordonii. \emph{Molecular Oral Microbiology},
\emph{27}(4), 284--294.
\url{https://doi.org/10.1111/j.2041-1014.2012.00644.x}

\leavevmode\vadjust pre{\hypertarget{ref-nikitkovaStarchBiofilms2013}{}}%
Nikitkova, A. E., Haase, E. M., \& Scannapieco, F. A. (2013). Taking the
{Starch} out of {Oral Biofilm Formation}: {Molecular Basis} and
{Functional Significance} of {Salivary} α-{Amylase Binding} to {Oral
Streptococci}. \emph{Applied and Environmental Microbiology},
\emph{79}(2), 416--423. \url{https://doi.org/10.1128/AEM.02581-12}

\leavevmode\vadjust pre{\hypertarget{ref-palmerCoaggregationInteractions2003}{}}%
Palmer, R. J., Jr., Gordon, S. M., Cisar, J. O., \& Kolenbrander, P. E.
(2003). Coaggregation-{Mediated Interactions} of {Streptococci} and
{Actinomyces Detected} in {Initial Human Dental Plaque}. \emph{Journal
of Bacteriology}, \emph{185}(11), 3400--3409.
\url{https://doi.org/10.1128/JB.185.11.3400-3409.2003}

\leavevmode\vadjust pre{\hypertarget{ref-powerSynchrotronRadiationbased2022}{}}%
Power, Robert C., Henry, A. G., Moosmann, J., Beckmann, F., Temming, H.,
Roberts, A., \& Cabec, A. L. (2022). Synchrotron radiation-based
phase-contrast microtomography of human dental calculus allows
nondestructive analysis of inclusions: Implications for archeological
samples. \emph{Journal of Medical Imaging}, \emph{9}(3), 031505.
\url{https://doi.org/10.1117/1.JMI.9.3.031505}

\leavevmode\vadjust pre{\hypertarget{ref-powerChimpCalculus2015}{}}%
Power, R. C., Salazar-Garcia, D. C., Wittig, R. M., Freiberg, M., \&
Henry, A. G. (2015). Dental calculus evidence of {Tai Forest Chimpanzee}
plant consumption and life history transitions. \emph{Scientific
Reports}, \emph{5}, 15161. \url{https://doi.org/10.1038/srep15161}

\leavevmode\vadjust pre{\hypertarget{ref-powerRepresentativenessDental2021}{}}%
Power, Robert C., Wittig, R. M., Stone, J. R., Kupczik, K., \&
Schulz-Kornas, E. (2021). The representativeness of the dental calculus
dietary record: Insights from {Taï} chimpanzee faecal phytoliths.
\emph{Archaeological and Anthropological Sciences}, \emph{13}(6), 104.
\url{https://doi.org/10.1007/s12520-021-01342-z}

\leavevmode\vadjust pre{\hypertarget{ref-proctorSpatialGradient2018}{}}%
Proctor, D. M., Fukuyama, J. A., Loomer, P. M., Armitage, G. C., Lee, S.
A., Davis, N. M., Ryder, M. I., Holmes, S. P., \& Relman, D. A. (2018).
A spatial gradient of bacterial diversity in the human oral cavity
shaped by salivary flow. \emph{Nature Communications}, \emph{9}(1), 681.
\url{https://doi.org/10.1038/s41467-018-02900-1}

\leavevmode\vadjust pre{\hypertarget{ref-radiniFoodPathways2017}{}}%
Radini, A., Nikita, E., Buckley, S., Copeland, L., \& Hardy, K. (2017).
Beyond food: {The} multiple pathways for inclusion of materials into
ancient dental calculus. \emph{American Journal of Physical
Anthropology}, \emph{162}, 71--83.
\url{https://doi.org/10.1002/ajpa.23147}

\leavevmode\vadjust pre{\hypertarget{ref-ramsoeDeamiDATESitespecific2020}{}}%
Ramsøe, A., van Heekeren, V., Ponce, P., Fischer, R., Barnes, I.,
Speller, C., \& Collins, M. J. (2020). {DeamiDATE} 1.0: {Site-specific}
deamidation as a tool to assess authenticity of members of ancient
proteomes. \emph{Journal of Archaeological Science}, \emph{115}, 105080.
\url{https://doi.org/10.1016/j.jas.2020.105080}

\leavevmode\vadjust pre{\hypertarget{ref-rogersRoleStreptococcus2001}{}}%
Rogers, J. D., Palmer, R. J., Kolenbrander, P. E., \& Scannapieco, F. A.
(2001). Role of {Streptococcus} gordonii {Amylase-Binding Protein A} in
{Adhesion} to {Hydroxyapatite}, {Starch Metabolism}, and {Biofilm
Formation}. \emph{Infection and Immunity}, \emph{69}(11), 7046--7056.
\url{https://doi.org/10.1128/IAI.69.11.7046-7056.2001}

\leavevmode\vadjust pre{\hypertarget{ref-scottExoticFoods2021}{}}%
Scott, A., Power, R. C., Altmann-Wendling, V., Artzy, M., Martin, M. A.
S., Eisenmann, S., Hagan, R., Salazar-García, D. C., Salmon, Y.,
Yegorov, D., Milevski, I., Finkelstein, I., Stockhammer, P. W., \&
Warinner, C. (2021). Exotic foods reveal contact between {South Asia}
and the {Near East} during the second millennium {BCE}.
\emph{Proceedings of the National Academy of Sciences}, \emph{118}(2),
e2014956117. \url{https://doi.org/10.1073/pnas.2014956117}

\leavevmode\vadjust pre{\hypertarget{ref-cummingsMayanCalculus1997}{}}%
Scott Cummings, L., \& Magennis, A. (1997). A phytolith and starch
record of food and grit in {Mayan} human tooth tartar. In A. Pinilla, J.
Juan-Tresserras, \& M. J. Machado (Eds.), \emph{The {State-of-the-Art}
of {Phytoliths} in {Soils} and {Plants}}. {CSIC Press}.
\url{https://books.google.com?id=j66CDVfVhwEC}

\leavevmode\vadjust pre{\hypertarget{ref-shawCommonalityElastic2004}{}}%
Shaw, T., Winston, M., Rupp, C. J., Klapper, I., \& Stoodley, P. (2004).
Commonality of {Elastic Relaxation Times} in {Biofilms}. \emph{Physical
Review Letters}, \emph{93}(9), 098102.
\url{https://doi.org/10.1103/PhysRevLett.93.098102}

\leavevmode\vadjust pre{\hypertarget{ref-sissonsArtificialPlaque1997}{}}%
Sissons, C. H. (1997). Artificial {Dental Plaque Biofilm Model Systems}.
\emph{Advances in Dental Research}, \emph{11}(1), 110--126.
\url{https://doi.org/10.1177/08959374970110010201}

\leavevmode\vadjust pre{\hypertarget{ref-sissonsMultistationPlaque1991}{}}%
Sissons, C. H., Cutress, T. W., Hoffman, M. P., \& Wakefield, J. S. J.
(1991). A {Multi-station Dental Plaque Microcosm} ({Artificial Mouth})
for the {Study} of {Plaque Growth}, {Metabolism}, {pH}, and
{Mineralization}: \emph{Journal of Dental Research}.
\url{https://doi.org/10.1177/00220345910700110301}

\leavevmode\vadjust pre{\hypertarget{ref-sissonsPHResponse1994}{}}%
Sissons, C. H., Wong, L., Hancock, E. M., \& Cutress, T. W. (1994). The
{pH} response to urea and the effect of liquid flow in {``artificial
mouth''} microcosm plaques. \emph{Archives of Oral Biology},
\emph{39}(6), 497--505.
\url{https://doi.org/10.1016/0003-9969(94)90146-5}

\leavevmode\vadjust pre{\hypertarget{ref-skoglundSeparatingEndogenous2014}{}}%
Skoglund, P., Northoff, B. H., Shunkov, M. V., Derevianko, A. P., Pääbo,
S., Krause, J., \& Jakobsson, M. (2014). Separating endogenous ancient
{DNA} from modern day contamination in a {Siberian Neandertal}.
\emph{Proceedings of the National Academy of Sciences}, \emph{111}(6),
2229--2234. \url{https://doi.org/10.1073/pnas.1318934111}

\leavevmode\vadjust pre{\hypertarget{ref-sotoCharacterizationDecontamination2019}{}}%
Soto, M., Inwood, J., Clarke, S., Crowther, A., Covelli, D., Favreau,
J., Itambu, M., Larter, S., Lee, P., Lozano, M., Maley, J., Mwambwiga,
A., Patalano, R., Sammynaiken, R., Vergès, J. M., Zhu, J., \& Mercader,
J. (2019). Structural characterization and decontamination of dental
calculus for ancient starch research. \emph{Archaeological and
Anthropological Sciences}, \emph{11}(9), 4847--4872.
\url{https://doi.org/10.1007/s12520-019-00830-7}

\leavevmode\vadjust pre{\hypertarget{ref-springfieldCocaineMetabolites1993}{}}%
Springfield, A. C., Cartmell, L. W., Aufderheide, A. C., Buikstra, J.,
\& Ho, J. (1993). Cocaine and metabolites in the hair of ancient
{Peruvian} coca leaf chewers. \emph{Forensic Science International},
\emph{63}(1-3), 269--275.
\url{https://doi.org/10.1016/0379-0738(93)90280-N}

\leavevmode\vadjust pre{\hypertarget{ref-stephanStudiesChanges1947}{}}%
Stephan, R. M., \& Hemmens, E. S. (1947). Studies of changes in {pH}
produced by pure cultures of oral micro-organisms; effects of varying
the microbic cell concentration; comparison of different micro-organisms
and different substrates; some effects of mixing certain
micro-organisms. \emph{Journal of Dental Research}, \emph{26}(1),
15--41. \url{https://doi.org/10.1177/00220345470260010201}

\leavevmode\vadjust pre{\hypertarget{ref-stewartAntimicrobialTolerance2015}{}}%
Stewart, P. S. (2015). Antimicrobial {Tolerance} in {Biofilms}.
\emph{Microbiology Spectrum}, \emph{3}(3),
10.1128/microbiolspec.mb-0010-2014.
\url{https://doi.org/10.1128/microbiolspec.mb-0010-2014}

\leavevmode\vadjust pre{\hypertarget{ref-takenakaDiffusionMacromolecules2009}{}}%
Takenaka, S., Pitts, B., Trivedi, H. M., \& Stewart, P. S. (2009).
Diffusion of {Macromolecules} in {Model Oral Biofilms}. \emph{Applied
and Environmental Microbiology}, \emph{75}(6), 1750--1753.
\url{https://doi.org/10.1128/AEM.02279-08}

\leavevmode\vadjust pre{\hypertarget{ref-trompEDTACalculus2017}{}}%
Tromp, M., Buckley, H., Geber, J., \& Matisoo-Smith, E. (2017). {EDTA}
decalcification of dental calculus as an alternate means of
microparticle extraction from archaeological samples. \emph{Journal of
Archaeological Science: Reports}, \emph{14}, 461--466.
\url{https://doi.org/10.1016/j.jasrep.2017.06.035}

\leavevmode\vadjust pre{\hypertarget{ref-velskoMicrobialDifferences2019}{}}%
Velsko, I. M., Fellows Yates, J. A., Aron, F., Hagan, R. W., Frantz, L.
A. F., Loe, L., Martinez, J. B. R., Chaves, E., Gosden, C., Larson, G.,
\& Warinner, C. (2019). Microbial differences between dental plaque and
historic dental calculus are related to oral biofilm maturation stage.
\emph{Microbiome}, \emph{7}(1), 102.
\url{https://doi.org/10.1186/s40168-019-0717-3}

\leavevmode\vadjust pre{\hypertarget{ref-velskoHighConservation2023}{}}%
Velsko, I. M., Gallois, S., Stahl, R., Henry, A. G., \& Warinner, C.
(2023). High conservation of the dental plaque microbiome across
populations with differing subsistence strategies and levels of market
integration. \emph{Molecular Ecology}.
\url{https://doi.org/10.1111/mec.16988}

\leavevmode\vadjust pre{\hypertarget{ref-velskoDentalCalculus2017}{}}%
Velsko, I. M., Overmyer, K. A., Speller, C., Klaus, L., Collins, M. J.,
Loe, L., Frantz, L. A. F., Sankaranarayanan, K., Lewis, C. M., Martinez,
J. B. R., Chaves, E., Coon, J. J., Larson, G., \& Warinner, C. (2017).
The dental calculus metabolome in modern and historic samples.
\emph{Metabolomics}, \emph{13}(11), 134.
\url{https://doi.org/10.1007/s11306-017-1270-3}

\leavevmode\vadjust pre{\hypertarget{ref-warinnerEvidenceMilk2014}{}}%
Warinner, C., Hendy, J., Speller, C., Cappellini, E., Fischer, R.,
Trachsel, C., Arneborg, J., Lynnerup, N., Craig, O. E., Swallow, D. M.,
Fotakis, A., Christensen, R. J., Olsen, J. V., Liebert, A., Montalva,
N., Fiddyment, S., Charlton, S., Mackie, M., Canci, A., \ldots{}
Collins, M. J. (2014). Direct evidence of milk consumption from ancient
human dental calculus. \emph{Scientific Reports}, \emph{4}, 7104.
\url{https://doi.org/10.1038/srep07104}

\leavevmode\vadjust pre{\hypertarget{ref-warinnerPathogensHost2014}{}}%
Warinner, C., Rodrigues, J. F., Vyas, R., Trachsel, C., Shved, N.,
Grossmann, J., Radini, A., Hancock, Y., Tito, R. Y., Fiddyment, S.,
Speller, C., Hendy, J., Charlton, S., Luder, H. U., Salazar-Garcia, D.
C., Eppler, E., Seiler, R., Hansen, L. H., Castruita, J. A., \ldots{}
Cappellini, E. (2014). Pathogens and host immunity in the ancient human
oral cavity. \emph{Nature Genetics}, \emph{46}(4), 336--344.
\url{https://doi.org/10.1038/ng.2906}

\leavevmode\vadjust pre{\hypertarget{ref-weinerBiologicalMaterials2010}{}}%
Weiner, S. (2010). Biological {Materials}: {Bones} and {Teeth}. In
\emph{Microarchaeology: {Beyond} the {Visible Archaeological Record}}
(pp. 99--134). {Cambridge University Press}.

\leavevmode\vadjust pre{\hypertarget{ref-yaussyCalculusSurvivorship2019}{}}%
Yaussy, S. L., \& DeWitte, S. N. (2019). Calculus and survivorship in
medieval {London}: {The} association between dental disease and a
demographic measure of general health. \emph{American Journal of
Physical Anthropology}, \emph{168}(3), 552--565.
\url{https://doi.org/10.1002/ajpa.23772}

\leavevmode\vadjust pre{\hypertarget{ref-ziesemerGenomeCalculus2018}{}}%
Ziesemer, K. A., Ramos‐Madrigal, J., Mann, A. E., Brandt, B. W.,
Sankaranarayanan, K., Ozga, A. T., Hoogland, M., Hofman, C. A.,
Salazar‐García, D. C., Frohlich, B., Milner, G. R., Stone, A. C.,
Aldenderfer, M., Lewis, C. M., Hofman, C. L., Warinner, C., \&
Schroeder, H. (2018). The efficacy of whole human genome capture on
ancient dental calculus and dentin. \emph{American Journal of Physical
Anthropology}. \url{https://doi.org/10.1002/ajpa.23763}

\end{CSLReferences}

\hypertarget{refs}{}
\begin{CSLReferences}{1}{0}
\leavevmode\vadjust pre{\hypertarget{ref-abucaCocaTrade2019}{}}%
Abduca, R. (2019). Coca leaf transfers to {Europe}. {Effects} on the
consumption of coca in {North-western Argentina}. In M. Kaller \& F.
Jacob (Eds.), \emph{Transatlantic {Trade} and {Global Cultural Transfers
Since} 1492: {More} than {Commodities}}. {Routledge}.
\url{https://books.google.com?id=13imDwAAQBAJ}

\leavevmode\vadjust pre{\hypertarget{ref-adamsLifeUniverse2002}{}}%
Adams, D. (2002a). \emph{Life, the {Universe} and {Everything}}.
{Picador}.

\leavevmode\vadjust pre{\hypertarget{ref-adamsLongThanks2002}{}}%
Adams, D. (2002b). \emph{So {Long}, and {Thanks} for {All} the {Fish}}.
{Picador}.

\leavevmode\vadjust pre{\hypertarget{ref-adamsHitchhikersGuide2002}{}}%
Adams, D. (2002c). \emph{The {Hitchhiker}'s {Guide} to the {Galaxy}}.
{Picador}.

\leavevmode\vadjust pre{\hypertarget{ref-adamsRestaurantEnd2002}{}}%
Adams, D. (2002d). \emph{The {Restaurant} at the {End} of the
{Universe}}. {Picador}.

\leavevmode\vadjust pre{\hypertarget{ref-adlerSequencingAncient2013}{}}%
Adler, C. J., Dobney, K., Weyrich, L. S., Kaidonis, J., Walker, A. W.,
Haak, W., Bradshaw, C. J., Townsend, G., Sołtysiak, A., Alt, K. W.,
Parkhill, J., \& Cooper, A. (2013). Sequencing ancient calcified dental
plaque shows changes in oral microbiota with dietary shifts of the
{Neolithic} and {Industrial} revolutions. \emph{Nature Genetics},
\emph{45}(4), 450--455, 455e1. \url{https://doi.org/10.1038/ng.2536}

\leavevmode\vadjust pre{\hypertarget{ref-akcaliDentalCalculus2018}{}}%
Akcalı, A., \& Lang, N. P. (2018). Dental calculus: The calcified
biofilm and its role in disease development. \emph{Periodontology 2000},
\emph{76}(1), 109--115. \url{https://doi.org/10.1111/prd.12151}

\leavevmode\vadjust pre{\hypertarget{ref-alanonAssessmentFlavanol2016}{}}%
Alañón, M. E., Castle, S. M., Siswanto, P. J., Cifuentes-Gómez, T., \&
Spencer, J. P. E. (2016). Assessment of flavanol stereoisomers and
caffeine and theobromine content in commercial chocolates. \emph{Food
Chemistry}, \emph{208}, 177--184.
\url{https://doi.org/10.1016/j.foodchem.2016.03.116}

\leavevmode\vadjust pre{\hypertarget{ref-alfaroChronicCoffee2018}{}}%
Alfaro, T. M., Monteiro, R. A., Cunha, R. A., \& Cordeiro, C. R. (2018).
Chronic coffee consumption and respiratory disease: {A} systematic
review. \emph{The Clinical Respiratory Journal}, \emph{12}(3),
1283--1294. \url{https://doi.org/10.1111/crj.12662}

\leavevmode\vadjust pre{\hypertarget{ref-ammannZurichBiofilm2012}{}}%
Ammann, T. W., Gmür, R., \& Thurnheer, T. (2012). Advancement of the
10-species subgingival {Zurich} biofilm model by examining different
nutritional conditions and defining the structure of the in vitro
biofilms. \emph{BMC Microbiology}, \emph{12}, 227.
\url{https://doi.org/10.1186/1471-2180-12-227}

\leavevmode\vadjust pre{\hypertarget{ref-arcaviCigaretteSmoking2004}{}}%
Arcavi, L., \& Benowitz, N. L. (2004). Cigarette {Smoking} and
{Infection}. \emph{Archives of Internal Medicine}, \emph{164}(20),
2206--2216. \url{https://doi.org/10.1001/archinte.164.20.2206}

\leavevmode\vadjust pre{\hypertarget{ref-armitageExtractionIdentification1975}{}}%
Armitage, P. L. (1975). The {Extraction} and {Identification} of {Opal
Phytoliths} from the {Teeth} of {Ungulates}. \emph{Journal of
Archaeological Science}, \emph{2}, 187--197.

\leavevmode\vadjust pre{\hypertarget{ref-aronHalfUDG2020}{}}%
Aron, F., Neumann, G., \& Brandt, G. (2020). Half-{UDG} treated
double-stranded ancient {DNA} library preparation for illumina
sequencing v1 {[}{Data} set{]}. \emph{Protocols. Io}.

\leavevmode\vadjust pre{\hypertarget{ref-asscherAtomicDisorder2011}{}}%
Asscher, Y., Regev, L., Weiner, S., \& Boaretto, E. (2011). Atomic
{Disorder} in {Fossil Tooth} and {Bone Mineral}: {An FTIR Study Using}
the {Grinding Curve Method}. \emph{ArcheoSciences. Revue
d'archéométrie}, \emph{35, 35}, 135--141.
\url{https://doi.org/10.4000/archeosciences.3062}

\leavevmode\vadjust pre{\hypertarget{ref-asscherVariationsAtomic2011}{}}%
Asscher, Y., Weiner, S., \& Boaretto, E. (2011). Variations in {Atomic
Disorder} in {Biogenic Carbonate Hydroxyapatite Using} the {Infrared
Spectrum Grinding Curve Method}. \emph{Advanced Functional Materials},
\emph{21}(17), 3308--3313. \url{https://doi.org/10.1002/adfm.201100266}

\leavevmode\vadjust pre{\hypertarget{ref-aten400Jaar2012}{}}%
Aten, D., Bossaers, K. W. J. M., \& Misset, C. (2012). \emph{400 jaar
Beemster: 1612-2012}. {Stichting Uitgeverij Noord-Holland}.

\leavevmode\vadjust pre{\hypertarget{ref-aufderheidePaleopathology1998}{}}%
Aufderheide, A. C., Rodriguez-Martin, C., \& Langsjoen, O. (1998).
\emph{The {Cambridge} encyclopedia of human paleopathology} (Vol. 478).
{Cambridge University Press Cambridge}.

\leavevmode\vadjust pre{\hypertarget{ref-azarpazhoohSystematicReview2006}{}}%
Azarpazhooh, A., \& Leake, J. L. (2006). Systematic {Review} of the
{Association Between Respiratory Diseases} and {Oral Health}.
\emph{Journal of Periodontology}, \emph{77}(9), 1465--1482.
\url{https://doi.org/10.1902/jop.2006.060010}

\leavevmode\vadjust pre{\hypertarget{ref-badriRegulationFunction2009}{}}%
Badri, D. V., \& Vivanco, J. M. (2009). Regulation and function of root
exudates. \emph{Plant, Cell \& Environment}, \emph{32}(6), 666--681.
\url{https://doi.org/10.1111/j.1365-3040.2009.01926.x}

\leavevmode\vadjust pre{\hypertarget{ref-balajiUnusualPresentation2019}{}}%
Balaji, V. R., Niazi, T. M., \& Dhanasekaran, M. (2019). An unusual
presentation of dental calculus. \emph{Journal of Indian Society of
Periodontology}, \emph{23}(5), 484--486.
\url{https://doi.org/10.4103/jisp.jisp_680_18}

\leavevmode\vadjust pre{\hypertarget{ref-bartholdyMultiproxyAnalysis2023}{}}%
Bartholdy, B. P., Hasselstrøm, J. B., Sørensen, L. K., Casna, M.,
Hoogland, M., Beemster, H. G., \& Henry, A. G. (2023). \emph{Multiproxy
analysis exploring patterns of diet and disease in dental calculus and
skeletal remains from a 19th century {Dutch} population}. {Zenodo}.
\url{https://doi.org/10.5281/zenodo.7649151}

\leavevmode\vadjust pre{\hypertarget{ref-bartholdyInvestigatingBiases2022}{}}%
Bartholdy, B. P., \& Henry, A. G. (2022). Investigating {Biases
Associated With Dietary Starch Incorporation} and {Retention With} an
{Oral Biofilm Model}. \emph{Frontiers in Earth Science}, \emph{10}.
\url{https://www.frontiersin.org/articles/10.3389/feart.2022.886512}

\leavevmode\vadjust pre{\hypertarget{ref-bartholdyAssessingValidity2023}{}}%
Bartholdy, B. P., Velsko, I. M., Gur-Arieh, S., Fagernäs, Z., Warinner,
C., \& Henry, A. G. (2023, May 30). \emph{Assessing the validity of a
calcifying oral biofilm model as a suitable proxy for dental calculus}.
\url{https://doi.org/10.1101/2023.05.23.541904}

\leavevmode\vadjust pre{\hypertarget{ref-bassonEstablishmentCommunity1996}{}}%
Basson, N. J., \& van Wyk, C. W. (1996). The establishment of a
community of oral bacteria that controls the growth of {Candida}
albicans in a chemostat. \emph{Oral Microbiology and Immunology},
\emph{11}(3), 199--202.
\url{https://doi.org/10.1111/j.1399-302X.1996.tb00358.x}

\leavevmode\vadjust pre{\hypertarget{ref-bernfeldAmylase1955}{}}%
Bernfeld, P. (1955). Amylases, α and β. In \emph{Methods in
{Enzymology}} (Vol. 1, pp. 149--158). {Academic Press}.
\url{https://doi.org/10.1016/0076-6879(55)01021-5}

\leavevmode\vadjust pre{\hypertarget{ref-bispoSimultaneousDetermination2002}{}}%
Bispo, M. S., Veloso, M. C. C., Pinheiro, H. L. C., De Oliveira, R. F.
S., Reis, J. O. N., \& De Andrade, J. B. (2002). Simultaneous
{Determination} of {Caffeine}, {Theobromine}, and {Theophylline} by
{High-Performance Liquid Chromatography}. \emph{Journal of
Chromatographic Science}, \emph{40}(1), 45--48.
\url{https://doi.org/10.1093/chromsci/40.1.45}

\leavevmode\vadjust pre{\hypertarget{ref-bjarnsholtVivoBiofilm2013}{}}%
Bjarnsholt, T., Alhede, M., Alhede, M., Eickhardt-Sørensen, S. R.,
Moser, C., Kühl, M., Jensen, P. Ø., \& Høiby, N. (2013). The in vivo
biofilm. \emph{Trends in Microbiology}, \emph{21}(9), 466--474.
\url{https://doi.org/10.1016/j.tim.2013.06.002}

\leavevmode\vadjust pre{\hypertarget{ref-bjorckStarchProcessing1984}{}}%
Björck, I., Asp, N.-G., Birkhed, D., Eliasson, A.-C., Sjöberg, L.-B., \&
Lundquist, I. (1984). Effects of processing on starch availability {In}
vitro and {In} vivo. {II}. {Drum-drying} of wheat flour. \emph{Journal
of Cereal Science}, \emph{2}(3), 165--178.
\url{https://doi.org/10.1016/S0733-5210(84)80030-2}

\leavevmode\vadjust pre{\hypertarget{ref-boocockMaxillarySinusitis1995}{}}%
Boocock, P., Roberts, C. A., \& Manchester, K. (1995). Maxillary
sinusitis in {Medieval Chichester}, {England}. \emph{American Journal of
Physical Anthropology}, \emph{98}(4), 483--495.
\url{https://doi.org/10.1002/ajpa.1330980408}

\leavevmode\vadjust pre{\hypertarget{ref-bosHistoryLicit2006}{}}%
Bos, A. (2006). The {History} of {Licit Cocaine} in the {Netherlands}.
\emph{De Economist}, \emph{154}(4), 581--586.
\url{https://doi.org/10.1007/s10645-006-9031-0}

\leavevmode\vadjust pre{\hypertarget{ref-bosPhysicochemistryInitial1999}{}}%
Bos, R. (1999). Physico-chemistry of initial microbial adhesive
interactions -- its mechanisms and methods for study. \emph{FEMS
Microbiology Reviews}, \emph{23}(2), 179--229.
\url{https://doi.org/10.1016/S0168-6445(99)00004-2}

\leavevmode\vadjust pre{\hypertarget{ref-boumanBegravenis2017}{}}%
Bouman, J. (2017). De Begravenis. \emph{De Nieuwe Schouwschuit},
\emph{15}, 11--15.
\url{https://www.historischgenootschapbeemster.nl/wp-content/uploads/De_Nieuwe_Schouwschuit_15e_jaargang_november_2017.pdf}

\leavevmode\vadjust pre{\hypertarget{ref-bowenOralBiofilms2018}{}}%
Bowen, W. H., Burne, R. A., Wu, H., \& Koo, H. (2018). Oral {Biofilms}:
{Pathogens}, {Matrix} and {Polymicrobial Interactions} in
{Microenvironments}. \emph{Trends in Microbiology}, \emph{26}(3),
229--242. \url{https://doi.org/10.1016/j.tim.2017.09.008}

\leavevmode\vadjust pre{\hypertarget{ref-boyadjianDentalWash2007}{}}%
Boyadjian, C. H. C., Eggers, S., \& Reinhard, K. (2007). Dental wash: A
problematic method for extracting microfossils from teeth. \emph{Journal
of Archaeological Science}, \emph{34}(10), 1622--1628.
\url{https://doi.org/10.1016/j.jas.2006.12.012}

\leavevmode\vadjust pre{\hypertarget{ref-boyan-salyersRelationshipProteolipids1980}{}}%
Boyan-Salyers, B. D., \& Boskey, A. L. (1980). Relationship between
proteolipids and calcium-phospholipid-phosphate complexes
{inBacterionema} matruchotii calcification. \emph{Calcified Tissue
International}, \emph{30}(1), 167--174.
\url{https://doi.org/10.1007/BF02408622}

\leavevmode\vadjust pre{\hypertarget{ref-SucheyBrooks1990}{}}%
Brooks, S., \& Suchey, J. M. (1990). Skeletal age determination based on
the os pubis: {A} comparison of the {Acsádi-Nemeskéri} and
{Suchey-Brooks} methods. \emph{Human Evolution}, \emph{5}(3), 227--238.
\url{https://doi.org/10.1007/BF02437238}

\leavevmode\vadjust pre{\hypertarget{ref-brothwellDiggingBones1981}{}}%
Brothwell, D. (1981). \emph{Digging up {Bones}: {The} excavation,
treatment and study of human skeletal remains} (3rd ed.). {British
Museum (Natural History)}.

\leavevmode\vadjust pre{\hypertarget{ref-bruinsmaBijdragenTot1872}{}}%
Bruinsma, J. J. (1872). \emph{Bijdragen tot de {Geneeskundige
Plaatsbeschrijving} van {Nederland}}. {Van Weelden en Mingelen}.
\url{https://dlcs.io/pdf/wellcome/pdf-item/b24874140/0}

\leavevmode\vadjust pre{\hypertarget{ref-bucchiComparisonsMethods2019}{}}%
Bucchi, A., Burguet-Coca, A., Expósito, I., Aceituno Bocanegra, F. J.,
\& Lozano, M. (2019). Comparisons between methods for analyzing dental
calculus samples from {El Mirador} cave ({Sierra} de {Atapuerca},
{Spain}). \emph{Archaeological and Anthropological Sciences},
\emph{11}(11), 6305--6314.
\url{https://doi.org/10.1007/s12520-019-00919-z}

\leavevmode\vadjust pre{\hypertarget{ref-buckberryAuricular2002}{}}%
Buckberry, J. L., \& Chamberlain, A. T. (2002). Age estimation from the
auricular surface of the ilium: A revised method. \emph{American Journal
of Physical Anthropology}, \emph{119}(3), 231--239.
\url{https://doi.org/10.1002/ajpa.10130}

\leavevmode\vadjust pre{\hypertarget{ref-buckleyDentalCalculus2014}{}}%
Buckley, S., Usai, D., Jakob, T., Radini, A., \& Hardy, K. (2014).
Dental {Calculus Reveals Unique Insights} into {Food Items}, {Cooking}
and {Plant Processing} in {Prehistoric Central Sudan}. \emph{PLOS ONE},
\emph{9}(7), e100808. \url{https://doi.org/10.1371/journal.pone.0100808}

\leavevmode\vadjust pre{\hypertarget{ref-Standards1994}{}}%
Buikstra, J. E., \& Ubelaker, D. H. (1994). Standards for data
collection from human skeletal remains: {Proceedings} of a seminar at
the {Field Museum} of {Natural History} ({Arkansas Archaeology Research
Series} 44). \emph{Fayetteville Arkansas Archaeological Survey}.

\leavevmode\vadjust pre{\hypertarget{ref-burchamPatternsOral2020}{}}%
Burcham, Z. M., Garneau, N. L., Comstock, S. S., Tucker, R. M., Knight,
R., Metcalf, J. L., Genetics of Taste Lab Citizen Scientists, Miranda,
A., Reinhart, B., Meyers, D., Woltkamp, D., Boxer, E., Hutchens, J.,
Kim, K., Archer, M., McAteer, M., Huss, P., Defonseka, R., Stahle, S.,
\ldots{} Reusser, W. (2020). Patterns of {Oral Microbiota Diversity} in
{Adults} and {Children}: {A Crowdsourced Population Study}.
\emph{Scientific Reports}, \emph{10}(1), 2133.
\url{https://doi.org/10.1038/s41598-020-59016-0}

\leavevmode\vadjust pre{\hypertarget{ref-cabanesPhytolithAnalysis2020}{}}%
Cabanes, D. (2020). Phytolith {Analysis} in {Paleoecology} and
{Archaeology}. In A. G. Henry (Ed.), \emph{Handbook for the {Analysis}
of {Micro-Particles} in {Archaeological Samples}} (pp. 255--288).
{Springer International Publishing}.
\url{https://doi.org/10.1007/978-3-030-42622-4_11}

\leavevmode\vadjust pre{\hypertarget{ref-casnaUrbanizationRespiratory2021}{}}%
Casna, M., Burrell, C. L., Schats, R., Hoogland, M. L. P., \& Schrader,
S. A. (2021). Urbanization and respiratory stress in the {Northern Low
Countries}: {A} comparative study of chronic maxillary sinusitis in two
early modern sites from the {Netherlands} ({AD} 1626--1866).
\emph{International Journal of Osteoarchaeology}, \emph{31}(5),
891--901. \url{https://doi.org/10.1002/oa.3006}

\leavevmode\vadjust pre{\hypertarget{ref-ceriCalgaryBiofilm1999}{}}%
Ceri, H., Olson, M. E., Stremick, C., Read, R. R., Morck, D., \& Buret,
A. (1999). The {Calgary Biofilm Device}: {New Technology} for {Rapid
Determination} of {Antibiotic Susceptibilities} of {Bacterial Biofilms}.
\emph{Journal of Clinical Microbiology}, \emph{37}(6), 1771--1776.
\url{https://doi.org/10.1128/JCM.37.6.1771-1776.1999}

\leavevmode\vadjust pre{\hypertarget{ref-charlierSEMCalculus2010}{}}%
Charlier, P., Huynh-Charlier, I., Munoz, O., Billard, M., Brun, L., \&
Grandmaison, G. L. de la. (2010). The microscopic (optical and {SEM})
examination of dental calculus deposits ({DCD}). {Potential} interest in
forensic anthropology of a bio-archaeological method. \emph{Legal
Medicine}, \emph{12}(4), 163--171.
\url{https://doi.org/10.1016/j.legalmed.2010.03.003}

\leavevmode\vadjust pre{\hypertarget{ref-chenCa2Dependent2001}{}}%
Chen, H., Hou, W., Kuć, J., \& Lin, Y. (2001). Ca2+‐dependent and
{Ca2}+‐independent excretion modes of salicylic acid in tobacco cell
suspension culture. \emph{Journal of Experimental Botany},
\emph{52}(359), 1219--1226.
\url{https://doi.org/10.1093/jexbot/52.359.1219}

\leavevmode\vadjust pre{\hypertarget{ref-chenSpecificGenes1999}{}}%
Chen, P., Qi, F., Novak, J., \& Caufield, P. W. (1999). The {Specific
Genes} for {Lantibiotic Mutacin II Biosynthesis} in {Streptococcus}
mutans {T8 Are Clustered} and {Can Be} {Transferred En Bloc}.
\emph{Applied and Environmental Microbiology}, \emph{65}(3), 1356--1360.
\url{https://www.ncbi.nlm.nih.gov/pmc/articles/PMC91190/}

\leavevmode\vadjust pre{\hypertarget{ref-chenStarchGrains2021}{}}%
Chen, T., Hou, L., Jiang, H., Wu, Y., \& Henry, A. G. (2021). Starch
grains from human teeth reveal the plant consumption of proto-{Shang}
people (c. 2000--1600 {BC}) from {Nancheng} site, {Hebei}, {China}.
\emph{Archaeological and Anthropological Sciences}, \emph{13}(9), 153.
\url{https://doi.org/10.1007/s12520-021-01416-y}

\leavevmode\vadjust pre{\hypertarget{ref-chinCaffeineContent2008}{}}%
Chin, J. M., Merves, M. L., Goldberger, B. A., Sampson-Cone, A., \&
Cone, E. J. (2008). Caffeine {Content} of {Brewed Teas}. \emph{Journal
of Analytical Toxicology}, \emph{32}(8), 702--704.
\url{https://doi.org/10.1093/jat/32.8.702}

\leavevmode\vadjust pre{\hypertarget{ref-chovanecOpiumMasses2012}{}}%
Chovanec, Z., Rafferty, S., \& Swiny, S. (2012). Opium for the {Masses}.
\emph{Ethnoarchaeology}, \emph{4}(1), 5--36.
\url{https://doi.org/10.1179/eth.2012.4.1.5}

\leavevmode\vadjust pre{\hypertarget{ref-ciochonOpalPhytoliths1990}{}}%
Ciochon, R. L., Piperno, D. R., \& Thompson, R. G. (1990). Opal
phytoliths found on the teeth of the extinct ape {Gigantopithecus}
blacki: Implications for paleodietary studies. \emph{Proceedings of the
National Academy of Sciences}, \emph{87}(20), 8120--8124.
\url{https://doi.org/10.1073/pnas.87.20.8120}

\leavevmode\vadjust pre{\hypertarget{ref-clarkeCannabisEvolution2013}{}}%
Clarke, R. (2013). \emph{Cannabis : {Evolution} and {Ethnobotany}}.
{University of California Press}.

\leavevmode\vadjust pre{\hypertarget{ref-cogoVitroEvaluation2008}{}}%
Cogo, K., Montan, M. F., Bergamaschi, C. de C., D. Andrade, E., Rosalen,
P. L., \& Groppo, F. C. (2008). In vitro evaluation of the effect of
nicotine, cotinine, and caffeine on oral microorganisms. \emph{Canadian
Journal of Microbiology}, \emph{54}(6), 501--508.
\url{https://doi.org/10.1139/W08-032}

\leavevmode\vadjust pre{\hypertarget{ref-collinsHomelessDental2007}{}}%
Collins, J., \& Freeman, R. (2007). Homeless in {North} and {West
Belfast}: An oral health needs assessment. \emph{British Dental
Journal}, \emph{202}(12, 12), E31--E31.
\url{https://doi.org/10.1038/bdj.2007.473}

\leavevmode\vadjust pre{\hypertarget{ref-marianiCoca1886}{}}%
Company, M. (1886). \emph{Coca {Erythroxylon}: {Its Uses} in the
{Treatment} of {Disease}} (4th ed.). {Mariani \& Co.}
\url{https://dlcs.io/pdf/wellcome/pdf-item/b21069360/0}

\leavevmode\vadjust pre{\hypertarget{ref-coneSalivaTesting1993}{}}%
Cone, E. J. (1993). Saliva {Testing} for {Drugs} of {Abuse}.
\emph{Annals of the New York Academy of Sciences}, \emph{694}(1),
91--127. \url{https://doi.org/10.1111/j.1749-6632.1993.tb18346.x}

\leavevmode\vadjust pre{\hypertarget{ref-coneInterpretationOral2007}{}}%
Cone, E. J., \& Huestis, M. A. (2007). Interpretation of {Oral Fluid
Tests} for {Drugs} of {Abuse}. \emph{Annals of the New York Academy of
Sciences}, \emph{1098}, 51--103.
\url{https://doi.org/10.1196/annals.1384.037}

\leavevmode\vadjust pre{\hypertarget{ref-cooperAncientDNA2000}{}}%
Cooper, A., \& Poinar, H. N. (2000). Ancient {DNA}: {Do} it right or
{Not} at all. \emph{Science}, \emph{289}.
\url{https://pure.mpg.de/pubman/faces/ViewItemOverviewPage.jsp?itemId=item_1556167}

\leavevmode\vadjust pre{\hypertarget{ref-cornejo-ramirezStructuralCharacteristics2018}{}}%
Cornejo-Ramírez, Y. I., Martínez-Cruz, O., Del Toro-Sánchez, C. L.,
Wong-Corral, F. J., Borboa-Flores, J., \& Cinco-Moroyoqui, F. J. (2018).
The structural characteristics of starches and their functional
properties. \emph{CyTA - Journal of Food}, \emph{16}(1), 1003--1017.
\url{https://doi.org/10.1080/19476337.2018.1518343}

\leavevmode\vadjust pre{\hypertarget{ref-costertonBacterialBiofilms1987}{}}%
Costerton, J. W., Cheng, K. J., Geesey, G. G., Ladd, T. I., Nickel, J.
C., Dasgupta, M., \& Marrie, T. J. (1987). Bacterial {Biofilms} in
{Nature} and {Disease}. \emph{Annual Review of Microbiology},
\emph{41}(1), 435--464.
\url{https://doi.org/10.1146/annurev.mi.41.100187.002251}

\leavevmode\vadjust pre{\hypertarget{ref-costertonMicrobialBiofilms1995}{}}%
Costerton, J. William, Lewandowski, Z., Caldwell, D. E., Korber, D. R.,
\& Lappin-Scott, H. M. (1995). Microbial {Biofilms}. \emph{Annual Review
of Microbiology}, \emph{49}(1), 711--745.
\url{https://doi.org/10.1146/annurev.mi.49.100195.003431}

\leavevmode\vadjust pre{\hypertarget{ref-crowtherDocumentingContamination2014}{}}%
Crowther, A., Haslam, M., Oakden, N., Walde, D., \& Mercader, J. (2014).
Documenting contamination in ancient starch laboratories. \emph{Journal
of Archaeological Science}, \emph{49}, 90--104.
\url{https://doi.org/10.1016/j.jas.2014.04.023}

\leavevmode\vadjust pre{\hypertarget{ref-curtisRoleMicrobiota2020}{}}%
Curtis, M. A., Diaz, P. I., \& Dyke, T. E. V. (2020). The role of the
microbiota in periodontal disease. \emph{Periodontology 2000},
\emph{83}(1), 14--25. \url{https://doi.org/10.1111/prd.12296}

\leavevmode\vadjust pre{\hypertarget{ref-dahlenMicrobiologicalStudy2010}{}}%
Dahlén, G., Konradsson, K., Eriksson, S., Teanpaisan, R., Piwat, S., \&
Carlén, A. (2010). A microbiological study in relation to the presence
of caries and calculus. \emph{Acta Odontologica Scandinavica},
\emph{68}(4), 199--206. \url{https://doi.org/10.3109/00016351003745514}

\leavevmode\vadjust pre{\hypertarget{ref-damenSilicicAcid1989}{}}%
Damen, J. J. M., \& Ten Cate, J. M. (1989). The {Effect} of {Silicic
Acid} on {Calcium Phosphate Precipitation}. \emph{Journal of Dental
Research}, \emph{68}(9), 1355--1359.
\url{https://doi.org/10.1177/00220345890680091301}

\leavevmode\vadjust pre{\hypertarget{ref-Rdecontam}{}}%
Davis, N. M., Proctor, D. M., Holmes, S. P., Relman, D. A., \& Callahan,
B. J. (2018). Simple statistical identification and removal of
contaminant sequences in marker-gene and metagenomics data.
\emph{Microbiome}, \emph{6}(1), 226.
\url{https://doi.org/10.1186/s40168-018-0605-2}

\leavevmode\vadjust pre{\hypertarget{ref-dawBacteriocinsNature1996}{}}%
Daw, M. A., \& Falkiner, F. R. (1996). Bacteriocins: Nature, function
and structure. \emph{Micron (Oxford, England: 1993)}, \emph{27}(6),
467--479. \url{https://doi.org/10.1016/s0968-4328(96)00028-5}

\leavevmode\vadjust pre{\hypertarget{ref-dawesEffectsDiet1970}{}}%
Dawes, Colin. (1970). Effects of {Diet} on {Salivary Secretion} and
{Composition}. \emph{Journal of Dental Research}, \emph{49}, 1263--1272.

\leavevmode\vadjust pre{\hypertarget{ref-dawesCircadianRhythms1972}{}}%
Dawes, C. (1972). Circadian rhythms in human salivary flow rate and
composition. \emph{The Journal of Physiology}, \emph{220}(3), 529--545.
\url{https://www.ncbi.nlm.nih.gov/pmc/articles/PMC1331668/}

\leavevmode\vadjust pre{\hypertarget{ref-delafuenteDNAHuman2013}{}}%
De La Fuente, C., Flores, S., \& Moraga, M. (2013). {DNA From Human
Ancient Bacteria}: {A} novel source of genetic evidence from
archaeological dental calculus. \emph{Archaeometry}, \emph{55}(4),
767--778. \url{https://doi.org/10.1111/j.1475-4754.2012.00707.x}

\leavevmode\vadjust pre{\hypertarget{ref-dibdinDiffusionSugars1981}{}}%
Dibdin, G. H. (1981). Diffusion of sugars and carboxylic acids through
human dental plaque in vitro. \emph{Archives of Oral Biology},
\emph{26}(6), 515--523.
\url{https://doi.org/10.1016/0003-9969(81)90010-8}

\leavevmode\vadjust pre{\hypertarget{ref-dibdinOralUrea1998}{}}%
Dibdin, G. H., \& Dawes, C. (1998). A {Mathematical Model} of the
{Influence} of {Salivary Urea} on the {pH} of {Fasted Dental Plaque} and
on the {Changes Occurring} during a {Cariogenic Challenge}. \emph{Caries
Research}, \emph{32}(1), 70--74. \url{https://doi.org/10.1159/000016432}

\leavevmode\vadjust pre{\hypertarget{ref-dobneyMethodEvaluating1987}{}}%
Dobney, K., \& Brothwell, D. (1987). A method for evaluating the amount
of dental calculus on teeth from archaeological sites. \emph{Journal of
Archaeological Science}, \emph{14}(4), 343--351.
\url{https://doi.org/10.1016/0305-4403(87)90024-0}

\leavevmode\vadjust pre{\hypertarget{ref-doddsCarbohydrateRetention1988}{}}%
Dodds, M. W. J., \& Edgar, W. M. (1988). The {Relationship Between
Plaque pH}, {Plaque Acid Anion Profiles}, and {Oral Carbohydrate
Retention After Ingestion} of {Several} '{Reference Foods}' by {Human
Subjects}. \emph{Journal of Dental Research}, \emph{67}(5), 861--865.
\url{https://doi.org/10.1177/00220345880670051301}

\leavevmode\vadjust pre{\hypertarget{ref-doddsHealthBenefits2005}{}}%
Dodds, Michael W. J., Johnson, D. A., \& Yeh, C.-K. (2005). Health
benefits of saliva: A review. \emph{Journal of Dentistry}, \emph{33}(3),
223--233. \url{https://doi.org/10.1016/j.jdent.2004.10.009}

\leavevmode\vadjust pre{\hypertarget{ref-drewettExcavationOval1975}{}}%
Drewett, P. (1975). \emph{The {Excavation} of an {Oval Burial Mound} of
the {Third Millennium} be at {Alfriston}, {East Sussex}, 1974}. 38.

\leavevmode\vadjust pre{\hypertarget{ref-duarteInfluencesStarch2008}{}}%
Duarte, S., Klein, M. I., Aires, C. P., Cury, J. A., Bowen, W. H., \&
Koo, H. (2008). Influences of starch and sucrose on {Streptococcus}
mutans biofilms. \emph{Oral Microbiology and Immunology}, \emph{23}(3),
206--212. \url{https://doi.org/10.1111/j.1399-302X.2007.00412.x}

\leavevmode\vadjust pre{\hypertarget{ref-dudgeonDietGeography2014}{}}%
Dudgeon, J. V., \& Tromp, M. (2014). Diet, {Geography} and {Drinking
Water} in {Polynesia}: {Microfossil Research} from {Archaeological Human
Dental Calculus}, {Rapa Nui} ({Easter Island}). \emph{International
Journal of Osteoarchaeology}, \emph{24}(5), 634--648.
\url{https://doi.org/10.1002/oa.2249}

\leavevmode\vadjust pre{\hypertarget{ref-duthieNaturalSalicylates2011}{}}%
Duthie, G. G., \& Wood, A. D. (2011). Natural salicylates: Foods ,
functions and disease prevention. \emph{Food \& Function}, \emph{2}(9),
515--520. \url{https://doi.org/10.1039/C1FO10128E}

\leavevmode\vadjust pre{\hypertarget{ref-echeverriaNicotineHair2013}{}}%
Echeverría, J., \& Niemeyer, H. M. (2013). Nicotine in the hair of
mummies from {San Pedro} de {Atacama} ({Northern Chile}). \emph{Journal
of Archaeological Science}, \emph{40}(10), 3561--3568.
\url{https://doi.org/10.1016/j.jas.2013.04.030}

\leavevmode\vadjust pre{\hypertarget{ref-edlundBiofilmModel2013}{}}%
Edlund, A., Yang, Y., Hall, A. P., Guo, L., Lux, R., He, X., Nelson, K.
E., Nealson, K. H., Yooseph, S., Shi, W., \& McLean, J. S. (2013). An in
vitrobiofilm model system maintaining a highly reproducible species and
metabolic diversity approaching that of the human oral microbiome.
\emph{Microbiome}, \emph{1}(1), 25.
\url{https://doi.org/10.1186/2049-2618-1-25}

\leavevmode\vadjust pre{\hypertarget{ref-edlundUncoveringComplex2018}{}}%
Edlund, A., Yang, Y., Yooseph, S., He, X., Shi, W., \& McLean, J. S.
(2018). Uncovering complex microbiome activities via metatranscriptomics
during 24 hours of oral biofilm assembly and maturation.
\emph{Microbiome}, \emph{6}(1), 217.
\url{https://doi.org/10.1186/s40168-018-0591-4}

\leavevmode\vadjust pre{\hypertarget{ref-eerkensDentalCalculus2018}{}}%
Eerkens, J. W., Tushingham, S., Brownstein, K. J., Garibay, R., Perez,
K., Murga, E., Kaijankoski, P., Rosenthal, J. S., \& Gang, D. R. (2018).
Dental calculus as a source of ancient alkaloids: {Detection} of
nicotine by {LC-MS} in calculus samples from the {Americas}.
\emph{Journal of Archaeological Science: Reports}, \emph{18}, 509--515.
\url{https://doi.org/10.1016/j.jasrep.2018.02.004}

\leavevmode\vadjust pre{\hypertarget{ref-enneverIntracellularCalcification1960}{}}%
Ennever, J. (1960). Intracellular {Calcification} by {Oral Filamentous
Microorganisms}. \emph{The Journal of Periodontology}, \emph{31}(4),
304--307. \url{https://doi.org/10.1902/jop.1960.31.4.304}

\leavevmode\vadjust pre{\hypertarget{ref-enneverMicrobiologicCalcification1967}{}}%
Ennever, J., \& Creamer, H. (1967). Microbiologic calcification: {Bone}
mineral and bacteria. \emph{Calcified Tissue Research}, \emph{1}(1),
87--93. \url{https://doi.org/10.1007/BF02008078}

\leavevmode\vadjust pre{\hypertarget{ref-enneverCharacterizationBacterionema1978}{}}%
Ennever, J., Riggan, L. J., Vogel, J. J., \& Boyan-Salyers, B. (1978).
Characterization of {Bacterionema} matruchotii {Calcification
Nucleator}. \emph{Journal of Dental Research}, \emph{57}(4), 637--642.
\url{https://doi.org/10.1177/00220345780570041901}

\leavevmode\vadjust pre{\hypertarget{ref-extercateAAA2010}{}}%
Exterkate, R. A. M., Crielaard, W., \& Ten Cate, J. M. (2010). Different
{Response} to {Amine Fluoride} by {Streptococcus} mutans and
{Polymicrobial Biofilms} in a {Novel High-Throughput Active Attachment
Model}. \emph{Caries Research}, \emph{44}(4), 372--379.
\url{https://doi.org/10.1159/000316541}

\leavevmode\vadjust pre{\hypertarget{ref-fagernasUnifiedProtocol2020}{}}%
Fagernäs, Z., García-Collado, M. I., Hendy, J., Hofman, C. A., Speller,
C., Velsko, I. M., \& Warinner, C. (2020). A unified protocol for
simultaneous extraction of {DNA} and proteins from archaeological dental
calculus. \emph{Journal of Archaeological Science}, \emph{118}, 105135.
\url{https://doi.org/10.1016/j.jas.2020.105135}

\leavevmode\vadjust pre{\hypertarget{ref-fagernasMicrobialBiogeography2021}{}}%
Fagernäs, Z., Salazar-García, D. C., Avilés, A., Haber, M., Henry, A.,
Maurandi, J. L., Ozga, A., Velsko, I. M., \& Warinner, C. (2021).
Understanding the microbial biogeography of ancient human dentitions to
guide study design and interpretation. \emph{bioRxiv},
2021.08.16.456492. \url{https://doi.org/10.1101/2021.08.16.456492}

\leavevmode\vadjust pre{\hypertarget{ref-fagernasMicrobialBiogeography2022}{}}%
Fagernäs, Z., Salazar-García, D. C., Haber Uriarte, M., Avilés
Fernández, A., Henry, A. G., Lomba Maurandi, J., Ozga, A. T., Velsko, I.
M., \& Warinner, C. (2022). Understanding the microbial biogeography of
ancient human dentitions to guide study design and interpretation.
\emph{FEMS Microbes}, \emph{3}, xtac006.
\url{https://doi.org/10.1093/femsmc/xtac006}

\leavevmode\vadjust pre{\hypertarget{ref-fagernasDentalCalculus2023}{}}%
Fagernäs, Z., \& Warinner, C. (2023). Dental {Calculus}. In A. M.
Pollard, R. A. Armitage, \& C. Makarevicz (Eds.), \emph{Handbook of
{Archaeological Sciences}} (Second edition).
\url{https://onlinelibrary.wiley.com/doi/epub/10.1002/9781119592112}

\leavevmode\vadjust pre{\hypertarget{ref-fdiOralHealth}{}}%
\emph{{FDI}'s definition of oral health \textbar{} {FDI}}. (n.d.). {FDI
World Dental Federation}. Retrieved March 14, 2022, from
\url{https://www.fdiworlddental.org/fdis-definition-oral-health}

\leavevmode\vadjust pre{\hypertarget{ref-yatesEAGER2020}{}}%
Fellows Yates, J. A., Lamnidis, T. C., Borry, M., Valtueña, A. A.,
Fagernäs, Z., Clayton, S., Garcia, M. U., Neukamm, J., \& Peltzer, A.
(2020). Reproducible, portable, and efficient ancient genome
reconstruction with nf-core/eager. \emph{bioRxiv}, 2020.06.11.145615.
\url{https://doi.org/10.1101/2020.06.11.145615}

\leavevmode\vadjust pre{\hypertarget{ref-yatesOralMicrobiome2021}{}}%
Fellows Yates, J. A., Velsko, I. M., Aron, F., Posth, C., Hofman, C. A.,
Austin, R. M., Parker, C. E., Mann, A. E., Nägele, K., Arthur, K. W.,
Arthur, J. W., Bauer, C. C., Crevecoeur, I., Cupillard, C., Curtis, M.
C., Dalén, L., Bonilla, M. D.-Z., Fernández-Lomana, J. C. D., Drucker,
D. G., \ldots{} Warinner, C. (2021). The evolution and changing ecology
of the {African} hominid oral microbiome. \emph{Proceedings of the
National Academy of Sciences}, \emph{118}(20).
\url{https://doi.org/10.1073/pnas.2021655118}

\leavevmode\vadjust pre{\hypertarget{ref-filocheFluorescenceAssay2007}{}}%
Filoche, Sara K., Coleman, M. J., Angker, L., \& Sissons, C. H. (2007).
A fluorescence assay to determine the viable biomass of microcosm dental
plaque biofilms. \emph{Journal of Microbiological Methods},
\emph{69}(3), 489--496.
\url{https://doi.org/10.1016/j.mimet.2007.02.015}

\leavevmode\vadjust pre{\hypertarget{ref-filochePlaqueMicrocosm2007}{}}%
Filoche, S. K., Soma, K. J., \& Sissons, C. H. (2007). Caries-related
plaque microcosm biofilms developed in microplates. \emph{Oral
Microbiology and Immunology}, \emph{22}(2), 73--79.
\url{https://doi.org/10.1111/j.1399-302X.2007.00323.x}

\leavevmode\vadjust pre{\hypertarget{ref-fiorinCombiningDental2021}{}}%
Fiorin, E., Moore, J., Montgomery, J., Lippi, M. M., Nowell, G., \&
Forlin, P. (2021). Combining dental calculus with isotope analysis in
the {Alps}: {New} evidence from the {Roman} and medieval cemeteries of
{Lamon}, northern {Italy}. \emph{Quaternary International}.
\url{https://doi.org/10.1016/j.quaint.2021.11.022}

\leavevmode\vadjust pre{\hypertarget{ref-flemmingBiofilmMatrix2010}{}}%
Flemming, H.-C., \& Wingender, J. (2010). The biofilm matrix.
\emph{Nature Reviews Microbiology}, \emph{8}(9), 623--633.
\url{https://doi.org/10.1038/nrmicro2415}

\leavevmode\vadjust pre{\hypertarget{ref-flemmingBiofilmsEmergent2016}{}}%
Flemming, H.-C., Wingender, J., Szewzyk, U., Steinberg, P., Rice, S. A.,
\& Kjelleberg, S. (2016). Biofilms: An emergent form of bacterial life.
\emph{Nature Reviews Microbiology}, \emph{14}(9), 563--575.
\url{https://doi.org/10.1038/nrmicro.2016.94}

\leavevmode\vadjust pre{\hypertarget{ref-foxPhytolithCalculus1996}{}}%
Fox, C. L., Juan, J., \& Albert, R. M. (1996). Phytolith analysis on
dental calculus, enamel surface, and burial soil: {Information} about
diet and paleoenvironment. \emph{American Journal of Physical
Anthropology}, \emph{101}(1), 101--113.
\url{https://doi.org/10.1002/(SICI)1096-8644(199609)101:1\%3C101::AID-AJPA7\%3E3.0.CO;2-Y}

\leavevmode\vadjust pre{\hypertarget{ref-francoStarchDegradation1992}{}}%
Franco, C. M. L., Preto, S. J. do R., \& Ciacco, C. F. (1992). Factors
that {Affect} the {Enzymatic Degradation} of {Natural Starch Granules}
-{Effect} of the {Size} of the {Granules}. \emph{Starch - Stärke},
\emph{44}(11), 422--426. \url{https://doi.org/10.1002/star.19920441106}

\leavevmode\vadjust pre{\hypertarget{ref-friskoppUltrastructureNondecalcified1983}{}}%
Friskopp, J. (1983). Ultrastructure of {Nondecalcified Supragingival}
and {Subgingival Calculus}. \emph{Journal of Periodontology},
\emph{54}(9), 542--550. \url{https://doi.org/10.1902/jop.1983.54.9.542}

\leavevmode\vadjust pre{\hypertarget{ref-friskoppComparativeScanning1980}{}}%
Friskopp, J., \& Hammarström, L. (1980). A {Comparative}, {Scanning
Electron Microscopic Study} of {Supragingival} and {Subgingival
Calculus}. \emph{Journal of Periodontology}, \emph{51}(10), 553--562.
\url{https://doi.org/10.1902/jop.1980.51.10.553}

\leavevmode\vadjust pre{\hypertarget{ref-froehlichEffectOral1987}{}}%
Froehlich, D. A., Pangborn, R. M., \& Whitaker, J. R. (1987). The effect
of oral stimulation on human parotid salivary flow rate and
alpha-amylase secretion. \emph{Physiology \& Behavior}, \emph{41}(3),
209--217. \url{https://doi.org/10.1016/0031-9384(87)90355-6}

\leavevmode\vadjust pre{\hypertarget{ref-graneroStarchTaphonomy2020}{}}%
García-Granero, J. J. (2020). Starch taphonomy, equifinality and the
importance of context: {Some} notes on the identification of food
processing through starch grain analysis. \emph{Journal of
Archaeological Science}, \emph{124}, 105267.
\url{https://doi.org/10.1016/j.jas.2020.105267}

\leavevmode\vadjust pre{\hypertarget{ref-gilbertAssessingAncient2005}{}}%
Gilbert, M. T. P., Bandelt, H.-J., Hofreiter, M., \& Barnes, I. (2005).
Assessing ancient {DNA} studies. \emph{Trends in Ecology \& Evolution},
\emph{20}(10), 541--544.
\url{https://doi.org/10.1016/j.tree.2005.07.005}

\leavevmode\vadjust pre{\hypertarget{ref-gilbertBiochemicalPhysical2005}{}}%
Gilbert, M. T. P., Rudbeck, L., Willerslev, E., Hansen, A. J., Smith,
C., Penkman, K. E. H., Prangenberg, K., Nielsen-Marsh, C. M., Jans, M.
E., Arthur, P., Lynnerup, N., Turner-Walker, G., Biddle, M.,
Kjølbye-Biddle, B., \& Collins, M. J. (2005). Biochemical and physical
correlates of {DNA} contamination in archaeological human bones and
teeth excavated at {Matera}, {Italy}. \emph{Journal of Archaeological
Science}, \emph{32}(5), 785--793.
\url{https://doi.org/10.1016/j.jas.2004.12.008}

\leavevmode\vadjust pre{\hypertarget{ref-gismondiMultidisciplinaryApproach2020}{}}%
Gismondi, A., Baldoni, M., Gnes, M., Scorrano, G., D'Agostino, A.,
Marco, G. D., Calabria, G., Petrucci, M., Müldner, G., Tersch, M. V.,
Nardi, A., Enei, F., Canini, A., Rickards, O., Alexander, M., \&
Martínez-Labarga, C. (2020). A multidisciplinary approach for
investigating dietary and medicinal habits of the {Medieval} population
of {Santa Severa} (7th-15th centuries, {Rome}, {Italy}). \emph{PLOS
ONE}, \emph{15}(1), e0227433.
\url{https://doi.org/10.1371/journal.pone.0227433}

\leavevmode\vadjust pre{\hypertarget{ref-gismondiStarchGranules2019}{}}%
Gismondi, A., D'Agostino, A., Canuti, L., Di Marco, G., Basoli, F., \&
Canini, A. (2019). Starch granules: A data collection of 40 food
species. \emph{Plant Biosystems - An International Journal Dealing with
All Aspects of Plant Biology}, \emph{153}(2), 273--279.
\url{https://doi.org/10.1080/11263504.2018.1473523}

\leavevmode\vadjust pre{\hypertarget{ref-glasBiophysicalStudies1962}{}}%
Glas, J.-E., \& Krasse, B. (1962). Biophysical {Studies} on {Dental
Calculus} from {Germfree} and {Conventional Rats}. \emph{Acta
Odontologica Scandinavica}, \emph{20}(2), 127--134.
\url{https://doi.org/10.3109/00016356209026100}

\leavevmode\vadjust pre{\hypertarget{ref-gloorMicrobiomeDatasets2017}{}}%
Gloor, G. B., Macklaim, J. M., Pawlowsky-Glahn, V., \& Egozcue, J. J.
(2017). Microbiome {Datasets Are Compositional}: {And This Is Not
Optional}. \emph{Frontiers in Microbiology}, \emph{8}, 2224.
\url{https://doi.org/10.3389/fmicb.2017.02224}

\leavevmode\vadjust pre{\hypertarget{ref-goodmanTobaccoHistory1994}{}}%
Goodman, J. (1994). \emph{Tobacco in history: The cultures of
dependence}. {Routledge}.

\leavevmode\vadjust pre{\hypertarget{ref-grahamEnterococcusFaecalis2017}{}}%
Graham, C. E., Cruz, M. R., Garsin, D. A., \& Lorenz, M. C. (2017).
Enterococcus faecalis bacteriocin {EntV} inhibits hyphal morphogenesis,
biofilm formation, and virulence of {Candida} albicans.
\emph{Proceedings of the National Academy of Sciences}, \emph{114}(17),
4507--4512. \url{https://doi.org/10.1073/pnas.1620432114}

\leavevmode\vadjust pre{\hypertarget{ref-greeneSimplifiedOral1964}{}}%
Greene, J. G., \& Vermillion, J. R. (1964). The {Simplified Oral Hygiene
Index}. \emph{The Journal of the American Dental Association},
\emph{68}(1), 7--13.
\url{https://doi.org/10.14219/jada.archive.1964.0034}

\leavevmode\vadjust pre{\hypertarget{ref-greeneQuantifyingCalculus2005}{}}%
Greene, T. R., Kuba, C. L., \& Irish, J. D. (2005). Quantifying
calculus: {A} suggested new approach for recording an important
indicator of diet and dental health. \emph{HOMO - Journal of Comparative
Human Biology}, \emph{56}(2), 119--132.
\url{https://doi.org/10.1016/j.jchb.2005.02.002}

\leavevmode\vadjust pre{\hypertarget{ref-haaseComparativeGenomics2017}{}}%
Haase, E. M., Kou, Y., Sabharwal, A., Liao, Y.-C., Lan, T., Lindqvist,
C., \& Scannapieco, F. A. (2017). Comparative genomics and evolution of
the amylase-binding proteins of oral streptococci. \emph{BMC
Microbiology}, \emph{17}(1), 94.
\url{https://doi.org/10.1186/s12866-017-1005-7}

\leavevmode\vadjust pre{\hypertarget{ref-haffajeeBiofilmPosition2009}{}}%
Haffajee, A. D., Teles, R. P., Patel, M. R., Song, X., Yaskell, T., \&
Socransky, S. S. (2009). Factors affecting human supragingival biofilm
composition. {II}. {Tooth} position. \emph{Journal of Periodontal
Research}, \emph{44}(4), 520--528.
\url{https://doi.org/10.1111/j.1600-0765.2008.01155.x}

\leavevmode\vadjust pre{\hypertarget{ref-hardyStarchGranules2009}{}}%
Hardy, K., Blakeney, T., Copeland, L., Kirkham, J., Wrangham, R., \&
Collins, M. (2009). Starch granules, dental calculus and new
perspectives on ancient diet. \emph{Journal of Archaeological Science},
\emph{36}(2), 248--255. \url{https://doi.org/10.1016/j.jas.2008.09.015}

\leavevmode\vadjust pre{\hypertarget{ref-hardyNeanderthalMedics2012}{}}%
Hardy, K., Buckley, S., Collins, M. J., Estalrrich, A., Brothwell, D.,
Copeland, L., García-Tabernero, A., García-Vargas, S., de la Rasilla,
M., Lalueza-Fox, C., Huguet, R., Bastir, M., Santamaría, D., Madella,
M., Wilson, J., Cortés, Á. F., \& Rosas, A. (2012). Neanderthal medics?
{Evidence} for food, cooking, and medicinal plants entrapped in dental
calculus. \emph{Naturwissenschaften}, \emph{99}(8), 617--626.
\url{https://doi.org/10.1007/s00114-012-0942-0}

\leavevmode\vadjust pre{\hypertarget{ref-hardyRecoveringInformation2018}{}}%
Hardy, K., Buckley, S., \& Copeland, L. (2018). Pleistocene dental
calculus: {Recovering} information on {Paleolithic} food items,
medicines, paleoenvironment and microbes. \emph{Evolutionary
Anthropology: Issues, News, and Reviews}, \emph{27}(5), 234--246.
\url{https://doi.org/10.1002/evan.21718}

\leavevmode\vadjust pre{\hypertarget{ref-hardyDentalCalculus2016}{}}%
Hardy, K., Radini, A., Buckley, S., Sarig, R., Copeland, L., Gopher, A.,
\& Barkai, R. (2016). Dental calculus reveals potential respiratory
irritants and ingestion of essential plant-based nutrients at {Lower
Palaeolithic Qesem Cave Israel}. \emph{Quaternary International},
\emph{398}, 129--135. \url{https://doi.org/10.1016/j.quaint.2015.04.033}

\leavevmode\vadjust pre{\hypertarget{ref-haslamDecompositionStarch2004}{}}%
Haslam, M. (2004). The decomposition of starch grains in soils:
Implications for archaeological residue analyses. \emph{Journal of
Archaeological Science}, \emph{31}(12), 1715--1734.
\url{https://doi.org/10.1016/j.jas.2004.05.006}

\leavevmode\vadjust pre{\hypertarget{ref-hayashizakiSiteSpecific2008}{}}%
Hayashizaki, J., Ban, S., Nakagaki, H., Okumura, A., Yoshii, S., \&
Robinson, C. (2008). Site specific mineral composition and
microstructure of human supra-gingival dental calculus. \emph{Archives
of Oral Biology}, \emph{53}(2), 168--174.
\url{https://doi.org/10.1016/j.archoralbio.2007.09.003}

\leavevmode\vadjust pre{\hypertarget{ref-hendyAncientProtein2021}{}}%
Hendy, J. (2021). Ancient protein analysis in archaeology. \emph{Science
Advances}, \emph{7}(3), eabb9314.
\url{https://doi.org/10.1126/sciadv.abb9314}

\leavevmode\vadjust pre{\hypertarget{ref-hendyProteomicCalculus2018}{}}%
Hendy, J., Warinner, C., Bouwman, A., Collins, M. J., Fiddyment, S.,
Fischer, R., Hagan, R., Hofman, C. A., Holst, M., Chaves, E., Klaus, L.,
Larson, G., Mackie, M., McGrath, K., Mundorff, A. Z., Radini, A., Rao,
H., Trachsel, C., Velsko, I. M., \& Speller, C. F. (2018). Proteomic
evidence of dietary sources in ancient dental calculus.
\emph{Proceedings. Biological Sciences}, \emph{285}(1883), 20180977.
\url{https://doi.org/10.1098/rspb.2018.0977}

\leavevmode\vadjust pre{\hypertarget{ref-henryNeanderthalCalculus2014}{}}%
Henry, A. G., Brooks, A. S., \& Piperno, D. R. (2014). Plant foods and
the dietary ecology of {Neanderthals} and early modern humans.
\emph{Journal of Human Evolution}, \emph{69}, 44--54.
\url{https://doi.org/10.1016/j.jhevol.2013.12.014}

\leavevmode\vadjust pre{\hypertarget{ref-henryCookingStarch2009}{}}%
Henry, A. G., Hudson, H. F., \& Piperno, D. R. (2009). Changes in starch
grain morphologies from cooking. \emph{Journal of Archaeological
Science}, \emph{36}(3), 915--922.
\url{https://doi.org/10.1016/j.jas.2008.11.008}

\leavevmode\vadjust pre{\hypertarget{ref-henryCalculusSyria2008}{}}%
Henry, A. G., \& Piperno, D. R. (2008). Using plant microfossils from
dental calculus to recover human diet: A case study from {Tell}
al-{Raqā}'i, {Syria}. \emph{Journal of Archaeological Science},
\emph{35}(7), 1943--1950.
\url{https://doi.org/10.1016/j.jas.2007.12.005}

\leavevmode\vadjust pre{\hypertarget{ref-henryDietAustralopithecus2012}{}}%
Henry, A. G., Ungar, P. S., Passey, B. H., Sponheimer, M., Rossouw, L.,
Bamford, M., Sandberg, P., de Ruiter, D. J., \& Berger, L. (2012). The
diet of {Australopithecus} sediba. \emph{Nature}, \emph{487}(7405,
7405), 90--93. \url{https://doi.org/10.1038/nature11185}

\leavevmode\vadjust pre{\hypertarget{ref-hidakaDietCalculus2007}{}}%
Hidaka, S., \& Oishi, A. (2007). An in vitro study of the effect of some
dietary components on calculus formation: Regulation of calcium
phosphate precipitation. \emph{Oral Diseases}, \emph{13}(3), 296--302.
\url{https://doi.org/10.1111/j.1601-0825.2006.01283.x}

\leavevmode\vadjust pre{\hypertarget{ref-hidakaStarchRole2008}{}}%
Hidaka, Saburo, Okamoto, Y., Tsukamoto, S., \& Oishi, A. (2008). The
{Possible Role} of {Starch} in {Oral Calcification}: {The In Vitro
Formation} of {Hydroxyapatite} is {Regulated} by a {Combination} of
{Protein} and {Mineral Content} in {Dietary Starch Flour}. \emph{The
Open Food Science Journal}, \emph{2}(1), 10--22.
\url{https://doi.org/10.2174/1874256400802010010}

\leavevmode\vadjust pre{\hypertarget{ref-hillsonDentalAnthropology1996}{}}%
Hillson, S. (1996). \emph{Dental {Anthropology}}. {Cambridge University
Press}.

\leavevmode\vadjust pre{\hypertarget{ref-honraetModifiedRobbins2006}{}}%
Honraet, K., \& Nelis, H. J. (2006). Use of the modified robbins device
and fluorescent staining to screen plant extracts for the inhibition of
{S}. Mutans biofilm formation. \emph{Journal of Microbiological
Methods}, \emph{64}(2), 217--224.
\url{https://doi.org/10.1016/j.mimet.2005.05.005}

\leavevmode\vadjust pre{\hypertarget{ref-huangFactorsAssociated2012}{}}%
Huang, X., Exterkate, R. A. M., \& ten Cate, J. M. (2012). Factors
{Associated} with {Alkali Production} from {Arginine} in {Dental
Biofilms}. \emph{Journal of Dental Research}, \emph{91}(12), 1130--1134.
\url{https://doi.org/10.1177/0022034512461652}

\leavevmode\vadjust pre{\hypertarget{ref-huangEffectArginine2017}{}}%
Huang, Xuelian, Zhang, K., Deng, M., Exterkate, R. A. M., Liu, C., Zhou,
X., Cheng, L., \& ten Cate, J. M. (2017). Effect of arginine on the
growth and biofilm formation of oral bacteria. \emph{Archives of Oral
Biology}, \emph{82}, 256--262.
\url{https://doi.org/10.1016/j.archoralbio.2017.06.026}

\leavevmode\vadjust pre{\hypertarget{ref-hublerHOPSAutomated2019}{}}%
Hübler, R., Key, F. M., Warinner, C., Bos, K. I., Krause, J., \& Herbig,
A. (2019). {HOPS}: Automated detection and authentication of pathogen
{DNA} in archaeological remains. \emph{Genome Biology}, \emph{20}(1),
280. \url{https://doi.org/10.1186/s13059-019-1903-0}

\leavevmode\vadjust pre{\hypertarget{ref-immelEffectXray2016}{}}%
Immel, A., Le Cabec, A., Bonazzi, M., Herbig, A., Temming, H.,
Schuenemann, V. J., Bos, K. I., Langbein, F., Harvati, K., Bridault, A.,
Pion, G., Julien, M.-A., Krotova, O., Conard, N. J., Münzel, S. C.,
Drucker, D. G., Viola, B., Hublin, J.-J., Tafforeau, P., \& Krause, J.
(2016). Effect of {X-ray} irradiation on ancient {DNA} in sub-fossil
bones -- {Guidelines} for safe {X-ray} imaging. \emph{Scientific
Reports}, \emph{6}(1, 1), 32969. \url{https://doi.org/10.1038/srep32969}

\leavevmode\vadjust pre{\hypertarget{ref-indriatiCocaPrehistoric2001}{}}%
Indriati, E., \& Buikstra, J. E. (2001). Coca chewing in prehistoric
coastal {Peru}: {Dental} evidence. \emph{American Journal of Physical
Anthropology}, \emph{114}(3), 242--257.
\url{https://doi.org/10.1002/1096-8644(200103)114:3\%3C242::AID-AJPA1023\%3E3.0.CO;2-J}

\leavevmode\vadjust pre{\hypertarget{ref-jainIsolationCharacterization2013}{}}%
Jain, K., Parida, S., Mangwani, N., Dash, H. R., \& Das, S. (2013).
Isolation and characterization of biofilm-forming bacteria and
associated extracellular polymeric substances from oral cavity.
\emph{Annals of Microbiology}, \emph{63}(4), 1553--1562.
\url{https://doi.org/10.1007/s13213-013-0618-9}

\leavevmode\vadjust pre{\hypertarget{ref-janeAnthologyStarch1994}{}}%
Jane, J.-L., Kasemsuwan, T., Leas, S., Zobel, H., \& Robyt, J. F.
(1994). Anthology of {Starch Granule Morphology} by {Scanning Electron
Microscopy}. \emph{Starch - Stärke}, \emph{46}(4), 121--129.
\url{https://doi.org/10.1002/star.19940460402}

\leavevmode\vadjust pre{\hypertarget{ref-jepsenCalculusRemoval2011}{}}%
Jepsen, S., Deschner, J., Braun, A., Schwarz, F., \& Eberhard, J.
(2011). Calculus removal and the prevention of its formation.
\emph{Periodontology 2000}, \emph{55}(1), 167--188.
\url{https://doi.org/10.1111/j.1600-0757.2010.00382.x}

\leavevmode\vadjust pre{\hypertarget{ref-jiFluorideMagnesium2000}{}}%
Ji, H., Nakagaki, H., Hayashizaki, J., Tsuboi, S., Kato, K., Toyama, A.,
Arai, K., Thuy, T. T., Ha, N. T. T., Kameyama, Y., Kirkham, J., \&
Robinson, C. (2000). Fluoride and magnesium concentrations in human
dental calculus obtained from {Japanese} and {Chinese} patients.
\emph{Archives of Oral Biology}, \emph{45}(7), 611--615.
\url{https://doi.org/10.1016/S0003-9969(00)00021-2}

\leavevmode\vadjust pre{\hypertarget{ref-jinSupragingivalCalculus2002}{}}%
Jin, Y., \& Yip, H.-K. (2002). Supragingival {Calculus}: {Formation} and
{Control}. \emph{Critical Reviews in Oral Biology \& Medicine}.
\url{https://doi.org/10.1177/154411130201300506}

\leavevmode\vadjust pre{\hypertarget{ref-jovanovicNeolithicCalculus2021}{}}%
Jovanović, J., Power, R. C., de Becdelièvre, C., Goude, G., \&
Stefanović, S. (2021). Microbotanical evidence for the spread of cereal
use during the {Mesolithic-Neolithic} transition in the {Southeastern
Europe} ({Danube Gorges}): {Data} from dental calculus analysis.
\emph{Journal of Archaeological Science}, \emph{125}, 105288.
\url{https://doi.org/10.1016/j.jas.2020.105288}

\leavevmode\vadjust pre{\hypertarget{ref-kashketFoodRetention1991}{}}%
Kashket, S., Van Houte, J., Lopez, L. R., \& Stocks, S. (1991). Lack of
{Correlation Between Food Retention} on the {Human Dentition} and
{Consumer Perception} of {Food Stickiness}. \emph{Journal of Dental
Research}, \emph{70}(10), 1314--1319.
\url{https://doi.org/10.1177/00220345910700100101}

\leavevmode\vadjust pre{\hypertarget{ref-kashketFoodParticles1996}{}}%
Kashket, S., Zhang, J., \& Houte, J. V. (1996). Accumulation of
{Fermentable Sugars} and {Metabolic Acids} in {Food Particles} that
{Become Entrapped} on the {Dentition}. \emph{Journal of Dental
Research}, 8.

\leavevmode\vadjust pre{\hypertarget{ref-katzenbergStableIsotope2008}{}}%
Katzenberg, M. A. (2008). Stable {Isotope Analysis}: {A Tool} for
{Studying Past Diet}, {Demography}, and {Life History}. In M. A.
Katzenberg \& S. R. Saunders (Eds.), \emph{Biological {Anthropology} of
the {Human Skeleton}} (pp. 301--340). {John Wiley and Sons}.

\leavevmode\vadjust pre{\hypertarget{ref-kazarinaPostmedievalMicrobial2021}{}}%
Kazarina, A., Petersone-Gordina, E., Kimsis, J., Kuzmicka, J., Zayakin,
P., Griškjans, Ž., Gerhards, G., \& Ranka, R. (2021). The {Postmedieval
Latvian Oral Microbiome} in the {Context} of {Modern Dental Calculus}
and {Modern Dental Plaque Microbial Profiles}. \emph{Genes},
\emph{12}(2), 309. \url{https://doi.org/10.3390/genes12020309}

\leavevmode\vadjust pre{\hypertarget{ref-kearnsMasterRegulator2005}{}}%
Kearns, D. B., Chu, F., Branda, S. S., Kolter, R., \& Losick, R. (2005).
A master regulator for biofilm formation by {Bacillus} subtilis.
\emph{Molecular Microbiology}, \emph{55}(3), 739--749.
\url{https://doi.org/10.1111/j.1365-2958.2004.04440.x}

\leavevmode\vadjust pre{\hypertarget{ref-kinastonOrtnerDentition2019}{}}%
Kinaston, R., Willis, A., Miszkiewicz, J. J., Tromp, M., \& Oxenham, M.
F. (2019). The {Dentition}: {Development}, {Disturbances}, {Disease},
{Diet}, and {Chemistry}. In J. E. Buikstra (Ed.), \emph{Ortner's
{Identification} of {Pathological Conditions} in {Human Skeletal
Remains} ({Third Edition})} (pp. 749--797). {Academic Press}.
\url{https://doi.org/10.1016/B978-0-12-809738-0.00021-1}

\leavevmode\vadjust pre{\hypertarget{ref-kitabaBlackCeramic2017}{}}%
Kitaba, I., \& Nakagawa, T. (2017). Black ceramic spheres as marker
grains for microfossil analyses, with improved chemical, physical, and
optical properties. \emph{Quaternary International}, \emph{455},
166--169. \url{https://doi.org/10.1016/j.quaint.2017.08.052}

\leavevmode\vadjust pre{\hypertarget{ref-knappSettingStage2012}{}}%
Knapp, M., Clarke, A. C., Horsburgh, K. A., \& Matisoo-Smith, E. A.
(2012). Setting the stage -- {Building} and working in an ancient {DNA}
laboratory. \emph{Annals of Anatomy - Anatomischer Anzeiger},
\emph{194}(1), 3--6. \url{https://doi.org/10.1016/j.aanat.2011.03.008}

\leavevmode\vadjust pre{\hypertarget{ref-knightsSourceTracker2011}{}}%
Knights, D., Kuczynski, J., Charlson, E. S., Zaneveld, J., Mozer, M. C.,
Collman, R. G., Bushman, F. D., Knight, R., \& Kelley, S. T. (2011).
Bayesian community-wide culture-independent microbial source tracking.
\emph{Nature Methods}, \emph{8}(9), 761--763.
\url{https://doi.org/10.1038/nmeth.1650}

\leavevmode\vadjust pre{\hypertarget{ref-kolenbranderAdhereToday1993}{}}%
Kolenbrander, P. E., \& London, J. (1993). Adhere today, here tomorrow:
Oral bacterial adherence. \emph{Journal of Bacteriology},
\emph{175}(11), 3247--3252.
\url{https://doi.org/10.1128/jb.175.11.3247-3252.1993}

\leavevmode\vadjust pre{\hypertarget{ref-kolenbranderOralMultispecies2010}{}}%
Kolenbrander, P. E., Palmer, R. J., Periasamy, S., \& Jakubovics, N. S.
(2010). Oral multispecies biofilm development and the key role of
cell--cell distance. \emph{Nature Reviews Microbiology}, \emph{8}(7),
471--480. \url{https://doi.org/10.1038/nrmicro2381}

\leavevmode\vadjust pre{\hypertarget{ref-kondoAssociationCoffee2021}{}}%
Kondo, K., Suzuki, K., Washio, M., Ohfuji, S., Adachi, S., Kan, S.,
Imai, S., Yoshimura, K., Miyashita, N., Fujisawa, N., Maeda, A.,
Fukushima, W., \& Hirota, Y. (2021). Association between coffee and
green tea intake and pneumonia among the {Japanese} elderly: A
case-control study. \emph{Scientific Reports}, \emph{11}(1, 1), 5570.
\url{https://doi.org/10.1038/s41598-021-84348-w}

\leavevmode\vadjust pre{\hypertarget{ref-langejansRemainsDay2010}{}}%
Langejans, G. H. J. (2010). Remains of the day-preservation of organic
micro-residues on stone tools. \emph{Journal of Archaeological Science},
\emph{37}(5), 971--985. \url{https://doi.org/10.1016/j.jas.2009.11.030}

\leavevmode\vadjust pre{\hypertarget{ref-lemoyneCalculusPretreatments2021}{}}%
Le Moyne, C., \& Crowther, A. (2021). Effects of chemical pre-treatments
on modified starch granules: {Recommendations} for dental calculus
decalcification for ancient starch research. \emph{Journal of
Archaeological Science: Reports}, \emph{35}, 102762.
\url{https://doi.org/10.1016/j.jasrep.2020.102762}

\leavevmode\vadjust pre{\hypertarget{ref-leeOralFluid2011}{}}%
Lee, D., Milman, G., Barnes, A. J., Goodwin, R. S., Hirvonen, J., \&
Huestis, M. A. (2011). Oral {Fluid Cannabinoids} in {Chronic}, {Daily
Cannabis Smokers} during {Sustained}, {Monitored Abstinence}.
\emph{Clinical Chemistry}, \emph{57}(8), 1127--1136.
\url{https://doi.org/10.1373/clinchem.2011.164822}

\leavevmode\vadjust pre{\hypertarget{ref-lemmersMiddenbeemster2013}{}}%
Lemmers, S. A. M., Schats, R., Hoogland, M. L. P., \& Waters-Rist, A.
(2013). Fysisch antropologische analyse Middenbeemster. In \emph{De
begravingen bij de Keyserkerk te Middenbeemster} (pp. 35--60).

\leavevmode\vadjust pre{\hypertarget{ref-leonardPlantMicroremains2015}{}}%
Leonard, C., Vashro, L., O'Connell, J. F., \& Henry, A. G. (2015). Plant
microremains in dental calculus as a record of plant consumption: {A}
test with {Twe} forager-horticulturalists. \emph{Journal of
Archaeological Science: Reports}, \emph{2}, 449--457.
\url{https://doi.org/10.1016/j.jasrep.2015.03.009}

\leavevmode\vadjust pre{\hypertarget{ref-leuwProhibitionLegalization1994}{}}%
Leuw, E., \& Marshall, I. H. (1994). \emph{Between {Prohibition} and
{Legalization}: {The Dutch Experiment} in {Drug Policy}}. {Kugler
Publications}. \url{https://books.google.com?id=2mAVkStNG5EC}

\leavevmode\vadjust pre{\hypertarget{ref-BWA}{}}%
Li, H., \& Durbin, R. (2009). Fast and accurate short read alignment
with {Burrows}--{Wheeler} transform. \emph{Bioinformatics},
\emph{25}(14), 1754--1760.
\url{https://doi.org/10.1093/bioinformatics/btp324}

\leavevmode\vadjust pre{\hypertarget{ref-liInfluenceGrinding2020}{}}%
Li, W., Pagán-Jiménez, J. R., Tsoraki, C., Yao, L., \& Van Gijn, A.
(2020). Influence of grinding on the preservation of starch grains from
rice. \emph{Archaeometry}, \emph{62}(1), 157--171.
\url{https://doi.org/10.1111/arcm.12510}

\leavevmode\vadjust pre{\hypertarget{ref-lieverseDietAetiology1999}{}}%
Lieverse, A. R. (1999). Diet and the aetiology of dental calculus.
\emph{International Journal of Osteoarchaeology}, \emph{9}(4), 219--232.
\url{https://doi.org/10.1002/(SICI)1099-1212(199907/08)9:4\%3C219::AID-OA475\%3E3.0.CO;2-V}

\leavevmode\vadjust pre{\hypertarget{ref-lieverseDentalHealth2007}{}}%
Lieverse, A. R., Link, D. W., Bazaliiskiy, V. I., Goriunova, O. I., \&
Weber, A. W. (2007). Dental health indicators of hunter--gatherer
adaptation and cultural change in {Siberia}'s {Cis-Baikal}.
\emph{American Journal of Physical Anthropology}, \emph{134}(3),
323--339. \url{https://doi.org/10.1002/ajpa.20672}

\leavevmode\vadjust pre{\hypertarget{ref-limSalivaMicrobiome2017}{}}%
Lim, Y., Totsika, M., Morrison, M., \& Punyadeera, C. (2017). The saliva
microbiome profiles are minimally affected by collection method or {DNA}
extraction protocols. \emph{Scientific Reports}, \emph{7}(1, 1), 8523.
\url{https://doi.org/10.1038/s41598-017-07885-3}

\leavevmode\vadjust pre{\hypertarget{ref-linANCOMBC2020}{}}%
Lin, H., \& Peddada, S. D. (2020). Analysis of compositions of
microbiomes with bias correction. \emph{Nature Communications},
\emph{11}(1, 1), 3514. \url{https://doi.org/10.1038/s41467-020-17041-7}

\leavevmode\vadjust pre{\hypertarget{ref-lindholstLongTerm2010}{}}%
Lindholst, C. (2010). Long term stability of cannabis resin and cannabis
extracts. \emph{Australian Journal of Forensic Sciences}, \emph{42}(3),
181--190. \url{https://doi.org/10.1080/00450610903258144}

\leavevmode\vadjust pre{\hypertarget{ref-lingstromStarchyFood1994}{}}%
Lingstrom, P., Birkhed, D., Ruben, J., \& Arends, J. (1994). Effect of
{Frequent Consumption} of {Starchy Food Items} on {Enamel} and {Dentin
Demineralization} and on {Plaque pH} in situ. \emph{Journal of Dental
Research}, \emph{73}(3), 652--660.
\url{https://doi.org/10.1177/00220345940730031101}

\leavevmode\vadjust pre{\hypertarget{ref-liuNicotinedegradingMicroorganisms2015}{}}%
Liu, J., Ma, G., Chen, T., Hou, Y., Yang, S., Zhang, K.-Q., \& Yang, J.
(2015). Nicotine-degrading microorganisms and their potential
applications. \emph{Applied Microbiology and Biotechnology},
\emph{99}(9), 3775--3785.
\url{https://doi.org/10.1007/s00253-015-6525-1}

\leavevmode\vadjust pre{\hypertarget{ref-llamasFieldLaboratory2017}{}}%
Llamas, B., Valverde, G., Fehren-Schmitz, L., Weyrich, L. S., Cooper,
A., \& Haak, W. (2017). From the field to the laboratory: {Controlling
DNA} contamination in human ancient {DNA} research in the
high-throughput sequencing era. \emph{STAR: Science \& Technology of
Archaeological Research}, \emph{3}(1), 1--14.
\url{https://doi.org/10.1080/20548923.2016.1258824}

\leavevmode\vadjust pre{\hypertarget{ref-lovejoyAuricular1985}{}}%
Lovejoy, C. O., Meindl, R. S., Pryzbeck, T. R., \& Mensforth, R. P.
(1985). Chronological metamorphosis of the auricular surface of the
ilium: {A} new method for the determination of adult skeletal age at
death. \emph{American Journal of Physical Anthropology}, \emph{68}(1),
15--28. \url{https://doi.org/10.1002/ajpa.1330680103}

\leavevmode\vadjust pre{\hypertarget{ref-lustmannScanningElectron1976}{}}%
Lustmann, J., Lewin-Epstein, J., \& Shteyer, A. (1976). Scanning
electron microscopy of dental calculus. \emph{Calcified Tissue
Research}, \emph{21}(1), 47--55.
\url{https://doi.org/10.1007/BF02547382}

\leavevmode\vadjust pre{\hypertarget{ref-maModelingDiffusion2010}{}}%
Ma, R., Liu, J., Jiang, Y., Liu, Z., Tang, Z., Ye, D., Zeng, J., \&
Huang, Z. (2010). Modeling of {Diffusion Transport} through {Oral
Biofilms} with the {Inverse Problem Method}. \emph{International Journal
of Oral Science}, \emph{2}(4, 4), 190--197.
\url{https://doi.org/10.4248/IJOS10075}

\leavevmode\vadjust pre{\hypertarget{ref-maHumanDiet2022}{}}%
Ma, Z., Liu, S., Li, Z., Ye, M., \& Huan, X. (2022). Human {Diet
Patterns During} the {Qijia Cultural Period}: {Integrated Evidence} of
{Stable Isotopes} and {Plant Micro-remains From} the {Lajia Site},
{Northwest China}. \emph{Frontiers in Earth Science}, \emph{10}.
\url{https://www.frontiersin.org/articles/10.3389/feart.2022.884856}

\leavevmode\vadjust pre{\hypertarget{ref-maMorphologicalChanges2019}{}}%
Ma, Z., Perry, L., Li, Q., \& Yang, X. (2019). Morphological changes in
starch grains after dehusking and grinding with stone tools.
\emph{Scientific Reports}, \emph{9}(1, 1), 2355.
\url{https://doi.org/10.1038/s41598-019-38758-6}

\leavevmode\vadjust pre{\hypertarget{ref-maatManualPhysical2005}{}}%
Maat, G., \& Mastwijk, R. (2005). Manual for the physical
anthropological report. \emph{Barge's Anthropologica}, \emph{6}.

\leavevmode\vadjust pre{\hypertarget{ref-machtHistoryOpium1915}{}}%
Macht, D. I. (1915). The history of opium and some of its preparations
and alkaloids. \emph{The Journal of the American Medical Association},
\emph{LXIV}(6), 5.

\leavevmode\vadjust pre{\hypertarget{ref-mackiePreservationMetaproteome2017}{}}%
Mackie, M., Hendy, J., Lowe, A. D., Sperduti, A., Holst, M., Collins, M.
J., \& Speller, C. F. (2017). Preservation of the metaproteome:
Variability of protein preservation in ancient dental calculus.
\emph{STAR: Science \& Technology of Archaeological Research},
\emph{3}(1), 58--70. \url{https://doi.org/10.1080/20548923.2017.1361629}

\leavevmode\vadjust pre{\hypertarget{ref-malakarNaturallyOccurring2017}{}}%
Malakar, S., Gibson, P. R., Barrett, J. S., \& Muir, J. G. (2017).
Naturally occurring dietary salicylates: {A} closer look at common
{Australian} foods. \emph{Journal of Food Composition and Analysis},
\emph{57}, 31--39. \url{https://doi.org/10.1016/j.jfca.2016.12.008}

\leavevmode\vadjust pre{\hypertarget{ref-mannHaveSomething2023}{}}%
Mann, A. E., Fellows Yates, J. A., Fagernäs, Z., Austin, R. M., Nelson,
E. A., \& Hofman, C. A. (2023). Do {I} have something in my teeth? {The}
trouble with genetic analyses of diet from archaeological dental
calculus. \emph{Quaternary International}, \emph{653--654}, 33--46.
\url{https://doi.org/10.1016/j.quaint.2020.11.019}

\leavevmode\vadjust pre{\hypertarget{ref-mannDifferentialPreservation2018}{}}%
Mann, A. E., Sabin, S., Ziesemer, K., Vågene, Å. J., Schroeder, H.,
Ozga, A. T., Sankaranarayanan, K., Hofman, C. A., Fellows Yates, J. A.,
Salazar-García, D. C., Frohlich, B., Aldenderfer, M., Hoogland, M.,
Read, C., Milner, G. R., Stone, A. C., Lewis, C. M., Krause, J., Hofman,
C., \ldots{} Warinner, C. (2018). Differential preservation of
endogenous human and microbial {DNA} in dental calculus and dentin.
\emph{Scientific Reports}, \emph{8}(1, 1), 9822.
\url{https://doi.org/10.1038/s41598-018-28091-9}

\leavevmode\vadjust pre{\hypertarget{ref-marshRoleMicrobiology1995}{}}%
Marsh, P. D. (1995). The {Role} of {Microbiology} in {Models} of {Dental
Caries}. \emph{Advances in Dental Research}, \emph{9}(3), 244--254.
\url{https://doi.org/10.1177/08959374950090030901}

\leavevmode\vadjust pre{\hypertarget{ref-marshDentalPlaque2005}{}}%
Marsh, Philip D. (2005). Dental plaque: Biological significance of a
biofilm and community life-style. \emph{Journal of Clinical
Periodontology}, \emph{32}(s6), 7--15.
\url{https://doi.org/10.1111/j.1600-051X.2005.00790.x}

\leavevmode\vadjust pre{\hypertarget{ref-marshDentalPlaque2006}{}}%
Marsh, Philip D. (2006). Dental plaque as a biofilm and a microbial
community -- implications for health and disease. \emph{BMC Oral
Health}, \emph{6}(S1), S14.
\url{https://doi.org/10.1186/1472-6831-6-S1-S14}

\leavevmode\vadjust pre{\hypertarget{ref-marshMicrobiologyDental2010}{}}%
Marsh, Philip D. (2010). Microbiology of {Dental Plaque Biofilms} and
{Their Role} in {Oral Health} and {Caries}. \emph{Dental Clinics of
North America}, \emph{54}(3), 441--454.
\url{https://doi.org/10.1016/j.cden.2010.03.002}

\leavevmode\vadjust pre{\hypertarget{ref-marshPhysiologicalApproaches1997}{}}%
Marsh, Philip D., \& Bradshaw, D. J. (1997). Physiological {Approaches}
to the {Control} of {Oral Biofilms}. \emph{Advances in Dental Research},
\emph{11}(1), 176--185.
\url{https://doi.org/10.1177/08959374970110010901}

\leavevmode\vadjust pre{\hypertarget{ref-marshDentalPlaque2016}{}}%
Marsh, Philip D., Lewis, M. A. O., Rogers, H., Williams, D. W., \&
Wilson, M. (2016). Dental {Plaque}. In \emph{Marsh and {Martin}'s {Oral
Microbiology}} (6th Edition, pp. 81--111). {Elsevier Health Sciences}.

\leavevmode\vadjust pre{\hypertarget{ref-maughanCaffeineIngestion2003}{}}%
Maughan, R. J., \& Griffin, J. (2003). Caffeine ingestion and fluid
balance: A review. \emph{Journal of Human Nutrition and Dietetics},
\emph{16}(6), 411--420.
\url{https://doi.org/10.1046/j.1365-277X.2003.00477.x}

\leavevmode\vadjust pre{\hypertarget{ref-mcbainBiofilmModels2009}{}}%
McBain, A. J. (2009). In {Vitro Biofilm Models}: {An Overview}. In
\emph{Advances in {Applied Microbiology}} (Vol. 69, pp. 99--132).
{Academic Press}. \url{https://doi.org/10.1016/S0065-2164(09)69004-3}

\leavevmode\vadjust pre{\hypertarget{ref-meindlSutureClosure1985}{}}%
Meindl, R. S., \& Lovejoy, C. O. (1985). Ectocranial suture closure: {A}
revised method for the determination of skeletal age at death based on
the lateral-anterior sutures. \emph{American Journal of Physical
Anthropology}, \emph{68}(1), 57--66.
\url{https://doi.org/10.1002/ajpa.1330680106}

\leavevmode\vadjust pre{\hypertarget{ref-mentzerDistributionAuthigenic2014}{}}%
Mentzer, S. M., Miller, C. E., Kloos, P., Wadley, L., \& Conard, N. J.
(2014). The distribution of authigenic minerals in the {Middle Stone
Age} deposits of {Sibudu} ({South Africa}), and implications for the
preservation of archaeological features. \emph{European Society for the
Study of Human Evolution, {4thAnnual} Meeting, Florence, Italy}.

\leavevmode\vadjust pre{\hypertarget{ref-mercaderExaggeratedExpectations2018}{}}%
Mercader, J., Akeju, T., Brown, M., Bundala, M., Collins, M. J.,
Copeland, L., Crowther, A., Dunfield, P., Henry, A., Inwood, J., Itambu,
M., Kim, J.-J., Larter, S., Longo, L., Oldenburg, T., Patalano, R.,
Sammynaiken, R., Soto, M., Tyler, R., \& Xhauflair, H. (2018).
Exaggerated expectations in ancient starch research and the need for new
taphonomic and authenticity criteria. \emph{FACETS}, \emph{3}(1),
777--798. \url{https://doi.org/10.1139/facets-2017-0126}

\leavevmode\vadjust pre{\hypertarget{ref-mickleburghNewInsights2012}{}}%
Mickleburgh, H. L., \& Pagán-Jiménez, J. R. (2012). New insights into
the consumption of maize and other food plants in the pre-{Columbian
Caribbean} from starch grains trapped in human dental calculus.
\emph{Journal of Archaeological Science}, \emph{39}(7), 2468--2478.
\url{https://doi.org/10.1016/j.jas.2012.02.020}

\leavevmode\vadjust pre{\hypertarget{ref-middletonVitroCalculus1965}{}}%
Middleton, J. D. (1965). Human salivary proteins and artificial calculus
formation in vitro. \emph{Archives of Oral Biology}, \emph{10}(2),
227--235. \url{https://doi.org/10.1016/0003-9969(65)90024-5}

\leavevmode\vadjust pre{\hypertarget{ref-middletonImprovedMethod1990}{}}%
Middleton, W. D. (1990). An {Improved Method} for {Extraction} of {Opal
Phytoliths} from {Tartar Residues} on {Herbivore Teeth}.
\emph{Phytolitharien Newsletter}, \emph{6}(3), 2--5.

\leavevmode\vadjust pre{\hypertarget{ref-middletonOpalPhytoliths1994}{}}%
Middleton, W. D., \& Rovner, I. (1994). Extraction of {Opal Phytoliths}
from {Herbivore Dental Calculus}. \emph{Journal of Archaeological
Science}, \emph{21}(4), 469--473.
\url{https://doi.org/10.1006/jasc.1994.1046}

\leavevmode\vadjust pre{\hypertarget{ref-milmanOralFluid2011}{}}%
Milman, G., Schwope, D. M., Schwilke, E. W., Darwin, W. D., Kelly, D.
L., Goodwin, R. S., Gorelick, D. A., \& Huestis, M. A. (2011). Oral
{Fluid} and {Plasma Cannabinoid Ratios} after {Around-the-Clock
Controlled Oral Δ9-Tetrahydrocannabinol Administration}. \emph{Clinical
Chemistry}, \emph{57}(11), 1597--1606.
\url{https://doi.org/10.1373/clinchem.2011.169490}

\leavevmode\vadjust pre{\hypertarget{ref-modiCalculusMethodologies2020}{}}%
Modi, A., Pisaneschi, L., Zaro, V., Vai, S., Vergata, C., Casalone, E.,
Caramelli, D., Moggi-Cecchi, J., Mariotti Lippi, M., \& Lari, M. (2020).
Combined methodologies for gaining much information from ancient dental
calculus: Testing experimental strategies for simultaneously analysing
{DNA} and food residues. \emph{Archaeological and Anthropological
Sciences}, \emph{12}(1), 10.
\url{https://doi.org/10.1007/s12520-019-00983-5}

\leavevmode\vadjust pre{\hypertarget{ref-moorerCalcificationCariogenic1993}{}}%
Moorer, W. R., Ten Cate, J. M., \& Buijs, J. F. (1993). Calcification of
a {Cariogenic Streptococcus} and of {Corynebacterium} ({Bacterionema})
matruchotii. \emph{Journal of Dental Research}, \emph{72}(6),
1021--1026. \url{https://doi.org/10.1177/00220345930720060501}

\leavevmode\vadjust pre{\hypertarget{ref-mortimerHistoryCoca1901}{}}%
Mortimer, W. G. (1901). \emph{Peru. {History} of coca, "the divine
plant" of the {Incas}; with an introductory account of the {Incas}, and
of the {Andean Indians} of to-day}. {New York, J. H. Vail \& Company}.
\url{http://archive.org/details/peruhistoryofcoc00mortrich}

\leavevmode\vadjust pre{\hypertarget{ref-Rhere}{}}%
Müller, K. (2020). \emph{Here: {A} simpler way to find your files}
{[}Manual{]}. \url{https://CRAN.R-project.org/package=here}

\leavevmode\vadjust pre{\hypertarget{ref-naterHumanAmylase2005}{}}%
Nater, U. M., Rohleder, N., Gaab, J., Berger, S., Jud, A., Kirschbaum,
C., \& Ehlert, U. (2005). Human salivary alpha-amylase reactivity in a
psychosocial stress paradigm. \emph{International Journal of
Psychophysiology}, \emph{55}(3), 333--342.
\url{https://doi.org/10.1016/j.ijpsycho.2004.09.009}

\leavevmode\vadjust pre{\hypertarget{ref-nearingAssessingVariation2020}{}}%
Nearing, J. T., DeClercq, V., Van Limbergen, J., \& Langille, M. G. I.
(2020). Assessing the {Variation} within the {Oral Microbiome} of
{Healthy Adults}. \emph{mSphere}, \emph{5}(5), e00451--20.
\url{https://doi.org/10.1128/mSphere.00451-20}

\leavevmode\vadjust pre{\hypertarget{ref-nierstraszTeaTrade2015}{}}%
Nierstrasz, C. (2015). \emph{Rivalry for {Trade} in {Tea} and
{Textiles}: {The English} and {Dutch East India} companies
(1700--1800)}. {Springer}.
\url{https://books.google.com?id=uwtaCwAAQBAJ}

\leavevmode\vadjust pre{\hypertarget{ref-nikitkovaEffectStarch2012}{}}%
Nikitkova, A. E., Haase, E. M., \& Scannapieco, F. A. (2012). Effect of
starch and amylase on the expression of amylase-binding protein {A} in
{Streptococcus} gordonii. \emph{Molecular Oral Microbiology},
\emph{27}(4), 284--294.
\url{https://doi.org/10.1111/j.2041-1014.2012.00644.x}

\leavevmode\vadjust pre{\hypertarget{ref-nikitkovaStarchBiofilms2013}{}}%
Nikitkova, A. E., Haase, E. M., \& Scannapieco, F. A. (2013). Taking the
{Starch} out of {Oral Biofilm Formation}: {Molecular Basis} and
{Functional Significance} of {Salivary} α-{Amylase Binding} to {Oral
Streptococci}. \emph{Applied and Environmental Microbiology},
\emph{79}(2), 416--423. \url{https://doi.org/10.1128/AEM.02581-12}

\leavevmode\vadjust pre{\hypertarget{ref-nobbsStreptococcusAdherence2009}{}}%
Nobbs, A. H., Lamont, R. J., \& Jenkinson, H. F. (2009). Streptococcus
{Adherence} and {Colonization}. \emph{Microbiology and Molecular Biology
Reviews}, \emph{73}(3), 407--450.
\url{https://doi.org/10.1128/MMBR.00014-09}

\leavevmode\vadjust pre{\hypertarget{ref-ogaldeIdentificationPsychoactive2009}{}}%
Ogalde, J. P., Arriaza, B. T., \& Soto, E. C. (2009). Identification of
psychoactive alkaloids in ancient {Andean} human hair by gas
chromatography/mass spectrometry. \emph{Journal of Archaeological
Science}, \emph{36}(2), 467--472.
\url{https://doi.org/10.1016/j.jas.2008.09.036}

\leavevmode\vadjust pre{\hypertarget{ref-Rvegan}{}}%
Oksanen, J., Simpson, G. L., Blanchet, F. G., Kindt, R., Legendre, P.,
Minchin, P. R., O'Hara, R. B., Solymos, P., Stevens, M. H. H., Szoecs,
E., Wagner, H., Barbour, M., Bedward, M., Bolker, B., Borcard, D.,
Carvalho, G., Chirico, M., De Caceres, M., Durand, S., \ldots{} Weedon,
J. (2022). \emph{Vegan: {Community} ecology package} {[}Manual{]}.
\url{https://CRAN.R-project.org/package=vegan}

\leavevmode\vadjust pre{\hypertarget{ref-omelonReviewPhosphate2013}{}}%
Omelon, S., Ariganello, M., Bonucci, E., Grynpas, M., \& Nanci, A.
(2013). A {Review} of {Phosphate Mineral Nucleation} in {Biology} and
{Geobiology}. \emph{Calcified Tissue International}, \emph{93}(4),
382--396. \url{https://doi.org/10.1007/s00223-013-9784-9}

\leavevmode\vadjust pre{\hypertarget{ref-omoriComparativeEvaluation2021}{}}%
Omori, M., Kato-Kogoe, N., Sakaguchi, S., Fukui, N., Yamamoto, K.,
Nakajima, Y., Inoue, K., Nakano, H., Motooka, D., Nakano, T., Nakamura,
S., \& Ueno, T. (2021). Comparative evaluation of microbial profiles of
oral samples obtained at different collection time points and using
different methods. \emph{Clinical Oral Investigations}, \emph{25}(5),
2779--2789. \url{https://doi.org/10.1007/s00784-020-03592-y}

\leavevmode\vadjust pre{\hypertarget{ref-whoOralHealth}{}}%
\emph{Oral health}. (n.d.). {World Health Organization}. Retrieved March
14, 2022, from
\url{https://www.who.int/news-room/fact-sheets/detail/oral-health}

\leavevmode\vadjust pre{\hypertarget{ref-ortnerIdentificationPathological2003}{}}%
Ortner, D. J. (2003). \emph{Identification of {Pathological Conditions}
in {Human Skeletal Remains}}. {Academic Press}.

\leavevmode\vadjust pre{\hypertarget{ref-palmerActivityReconstruction2016}{}}%
Palmer, J. L. A., Hoogland, M. H. L., \& Waters‐Rist, A. L. (2016).
Activity {Reconstruction} of {Post}‐{Medieval Dutch Rural Villagers}
from {Upper Limb Osteoarthritis} and {Entheseal Changes}.
\emph{International Journal of Osteoarchaeology}, \emph{26}(1), 78--92.
\url{https://doi.org/10.1002/oa.2397}

\leavevmode\vadjust pre{\hypertarget{ref-palmerCoaggregationInteractions2003}{}}%
Palmer, Robert J., Jr., Gordon, S. M., Cisar, J. O., \& Kolenbrander, P.
E. (2003). Coaggregation-{Mediated Interactions} of {Streptococci} and
{Actinomyces Detected} in {Initial Human Dental Plaque}. \emph{Journal
of Bacteriology}, \emph{185}(11), 3400--3409.
\url{https://doi.org/10.1128/JB.185.11.3400-3409.2003}

\leavevmode\vadjust pre{\hypertarget{ref-palmerInterbacterialAdhesion2017}{}}%
Palmer, Robert J., Shah, N., Valm, A., Paster, B., Dewhirst, F., Inui,
T., \& Cisar, J. O. (2017). Interbacterial {Adhesion Networks} within
{Early Oral Biofilms} of {Single Human Hosts}. \emph{Applied and
Environmental Microbiology}, \emph{83}(11), e00407--17.
\url{https://doi.org/10.1128/AEM.00407-17}

\leavevmode\vadjust pre{\hypertarget{ref-pearceConcomitantDeposition1987}{}}%
Pearce, E. I. F., \& Sissons, C. H. (1987). The {Concomitant Deposition}
of {Strontium} and {Fluoride} in {Dental Plaque}. \emph{Journal of
Dental Research}, \emph{66}(10), 1518--1522.
\url{https://doi.org/10.1177/00220345870660100101}

\leavevmode\vadjust pre{\hypertarget{ref-Rpatchwork}{}}%
Pedersen, T. L. (2020). \emph{Patchwork: {The} composer of plots}
{[}Manual{]}. \url{https://CRAN.R-project.org/package=patchwork}

\leavevmode\vadjust pre{\hypertarget{ref-petersConstantDepth1988}{}}%
Peters, A., \& Wimpenny, J. W. T. (1988). A {Constant-Depth Laboratory
Model Film Fermenter}. In \emph{{CRC Handbook} of {Laboratory Model
Systems} for {Microbial Ecosystems}}. {CRC Press}.

\leavevmode\vadjust pre{\hypertarget{ref-petersonViscoelasticityBiofilms2015}{}}%
Peterson, B. W., He, Y., Ren, Y., Zerdoum, A., Libera, M. R., Sharma, P.
K., van Winkelhoff, A.-J., Neut, D., Stoodley, P., van der Mei, H. C.,
\& Busscher, H. J. (2015). Viscoelasticity of biofilms and their
recalcitrance to mechanical and chemical challenges. \emph{FEMS
Microbiology Reviews}, \emph{39}(2), 234--245.
\url{https://doi.org/10.1093/femsre/fuu008}

\leavevmode\vadjust pre{\hypertarget{ref-petrovaEscapingBiofilm2016}{}}%
Petrova, O. E., \& Sauer, K. (2016). Escaping the biofilm in more than
one way: Desorption, detachment or dispersion. \emph{Current Opinion in
Microbiology}, \emph{30}, 67--78.
\url{https://doi.org/10.1016/j.mib.2016.01.004}

\leavevmode\vadjust pre{\hypertarget{ref-pilloudOutliningDefinition2019}{}}%
Pilloud, M. A., \& Fancher, J. P. (2019). Outlining a {Definition} of
{Oral Health} within the {Study} of {Human Skeletal Remains}: {Defining
Oral Health}. \emph{Dental Anthropology Journal}, \emph{32}(2, 2),
3--11. \url{https://doi.org/10.26575/daj.v32i2.297}

\leavevmode\vadjust pre{\hypertarget{ref-pipernoStarchGrains2008}{}}%
Piperno, D. R., \& Dillehay, T. D. (2008). Starch grains on human teeth
reveal early broad crop diet in northern {Peru}. \emph{Proceedings of
the National Academy of Sciences}, \emph{105}(50), 19622--19627.
\url{https://doi.org/10.1073/pnas.0808752105}

\leavevmode\vadjust pre{\hypertarget{ref-powerSynchrotronRadiationbased2022}{}}%
Power, Robert C., Henry, A. G., Moosmann, J., Beckmann, F., Temming, H.,
Roberts, A., \& Cabec, A. L. (2022). Synchrotron radiation-based
phase-contrast microtomography of human dental calculus allows
nondestructive analysis of inclusions: Implications for archeological
samples. \emph{Journal of Medical Imaging}, \emph{9}(3), 031505.
\url{https://doi.org/10.1117/1.JMI.9.3.031505}

\leavevmode\vadjust pre{\hypertarget{ref-powerChimpCalculus2015}{}}%
Power, R. C., Salazar-Garcia, D. C., Wittig, R. M., Freiberg, M., \&
Henry, A. G. (2015). Dental calculus evidence of {Tai Forest Chimpanzee}
plant consumption and life history transitions. \emph{Scientific
Reports}, \emph{5}, 15161. \url{https://doi.org/10.1038/srep15161}

\leavevmode\vadjust pre{\hypertarget{ref-powerSEMCalculus2014}{}}%
Power, R. C., Salazar-García, D. C., Wittig, R. M., \& Henry, A. G.
(2014). Assessing use and suitability of scanning electron microscopy in
the analysis of micro remains in dental calculus. \emph{Journal of
Archaeological Science}, \emph{49}, 160--169.
\url{https://doi.org/10.1016/j.jas.2014.04.016}

\leavevmode\vadjust pre{\hypertarget{ref-powerRepresentativenessDental2021}{}}%
Power, Robert C., Wittig, R. M., Stone, J. R., Kupczik, K., \&
Schulz-Kornas, E. (2021). The representativeness of the dental calculus
dietary record: Insights from {Taï} chimpanzee faecal phytoliths.
\emph{Archaeological and Anthropological Sciences}, \emph{13}(6), 104.
\url{https://doi.org/10.1007/s12520-021-01342-z}

\leavevmode\vadjust pre{\hypertarget{ref-prattenVitroStudies1998}{}}%
Pratten, Wills, Barnett, \& Wilson. (1998). In vitro studies of the
effect of antiseptic-containing mouthwashes on the formation and
viability of {Streptococcus} sanguis biofilms. \emph{Journal of Applied
Microbiology}, \emph{84}(6), 1149--1155.
\url{https://doi.org/10.1046/j.1365-2672.1998.00462.x}

\leavevmode\vadjust pre{\hypertarget{ref-proctorSpatialGradient2018}{}}%
Proctor, D. M., Fukuyama, J. A., Loomer, P. M., Armitage, G. C., Lee, S.
A., Davis, N. M., Ryder, M. I., Holmes, S. P., \& Relman, D. A. (2018).
A spatial gradient of bacterial diversity in the human oral cavity
shaped by salivary flow. \emph{Nature Communications}, \emph{9}(1), 681.
\url{https://doi.org/10.1038/s41467-018-02900-1}

\leavevmode\vadjust pre{\hypertarget{ref-Rbase}{}}%
R Core Team. (2020). \emph{R: {A} language and environment for
statistical computing} {[}Manual{]}. {R Foundation for Statistical
Computing}; {R Foundation for Statistical Computing}.
\url{https://www.R-project.org/}

\leavevmode\vadjust pre{\hypertarget{ref-radiniDirtyTeeth2022}{}}%
Radini, A., \& Nikita, E. (2022). Beyond dirty teeth: {Integrating}
dental calculus studies with osteoarchaeological parameters.
\emph{Quaternary International}.
\url{https://doi.org/10.1016/j.quaint.2022.03.003}

\leavevmode\vadjust pre{\hypertarget{ref-radiniFoodPathways2017}{}}%
Radini, A., Nikita, E., Buckley, S., Copeland, L., \& Hardy, K. (2017).
Beyond food: {The} multiple pathways for inclusion of materials into
ancient dental calculus. \emph{American Journal of Physical
Anthropology}, \emph{162}, 71--83.
\url{https://doi.org/10.1002/ajpa.23147}

\leavevmode\vadjust pre{\hypertarget{ref-radiniMedievalWomen2019}{}}%
Radini, A., Tromp, M., Beach, A., Tong, E., Speller, C., McCormick, M.,
Dudgeon, J. V., Collins, M. J., Rühli, F., Kröger, R., \& Warinner, C.
(2019). Medieval women's early involvement in manuscript production
suggested by lapis lazuli identification in dental calculus.
\emph{Science Advances}, \emph{5}(1), eaau7126.
\url{https://doi.org/10.1126/sciadv.aau7126}

\leavevmode\vadjust pre{\hypertarget{ref-raffertyCurrentResearch2012}{}}%
Rafferty, S. M., Lednev, I., Virkler, K., \& Chovanec, Z. (2012).
Current research on smoking pipe residues. \emph{Journal of
Archaeological Science}, \emph{39}(7), 1951--1959.
\url{https://doi.org/10.1016/j.jas.2012.02.001}

\leavevmode\vadjust pre{\hypertarget{ref-ramsoeDeamiDATESitespecific2020}{}}%
Ramsøe, A., van Heekeren, V., Ponce, P., Fischer, R., Barnes, I.,
Speller, C., \& Collins, M. J. (2020). {DeamiDATE} 1.0: {Site-specific}
deamidation as a tool to assess authenticity of members of ancient
proteomes. \emph{Journal of Archaeological Science}, \emph{115}, 105080.
\url{https://doi.org/10.1016/j.jas.2020.105080}

\leavevmode\vadjust pre{\hypertarget{ref-rehImpactTobacco2012}{}}%
Reh, D. D., Higgins, T. S., \& Smith, T. L. (2012). Impact of {Tobacco
Smoke} on {Chronic Rhinosinusitis} -- {A Review} of the {Literature}.
\emph{International Forum of Allergy \& Rhinology}, \emph{2}(5), 362.
\url{https://doi.org/10.1002/alr.21054}

\leavevmode\vadjust pre{\hypertarget{ref-reichertStarchBible1913b}{}}%
Reichert, E. T. (1913). \emph{The differentiation and specificity of
starches in relation to genera, species, etc: Stereochemistry applied to
protoplasmic processes and products, and as a strictly scientific basis
for the classification of plants and animals} (Vol. 2). {Carnegie
institution of Washington}.

\leavevmode\vadjust pre{\hypertarget{ref-reimerBacDive2022}{}}%
Reimer, L. C., Sardà Carbasse, J., Koblitz, J., Ebeling, C., Podstawka,
A., \& Overmann, J. (2022). {BacDive} in 2022: The knowledge base for
standardized bacterial and archaeal data. \emph{Nucleic Acids Research},
\emph{50}(D1), D741--D746. \url{https://doi.org/10.1093/nar/gkab961}

\leavevmode\vadjust pre{\hypertarget{ref-renduelesMechanismsCompetition2015}{}}%
Rendueles, O., \& Ghigo, J.-M. (2015). Mechanisms of {Competition} in
{Biofilm Communities}. \emph{Microbiology Spectrum}, \emph{3}(3),
3.3.28. \url{https://doi.org/10.1128/microbiolspec.MB-0009-2014}

\leavevmode\vadjust pre{\hypertarget{ref-rennerPhysicochemicalRegulation2011}{}}%
Renner, L. D., \& Weibel, D. B. (2011). Physicochemical regulation of
biofilm formation. \emph{MRS Bulletin}, \emph{36}(5), 347--355.
\url{https://doi.org/10.1557/mrs.2011.65}

\leavevmode\vadjust pre{\hypertarget{ref-Rpsych}{}}%
Revelle, W. (2022). \emph{Psych: {Procedures} for psychological,
psychometric, and personality research} {[}Manual{]}. {Northwestern
University}. \url{https://CRAN.R-project.org/package=psych}

\leavevmode\vadjust pre{\hypertarget{ref-robertsDentalDisease2007}{}}%
Roberts, C. A., \& Manchester, K. (2007). Dental {Disease}. In \emph{The
{Archaeology} of {Disease}} (3rd Edition, pp. 63--83). {Cornell
University Press}.

\leavevmode\vadjust pre{\hypertarget{ref-Rbroom}{}}%
Robinson, D., Hayes, A., \& Couch, S. (2021). \emph{Broom: {Convert}
statistical objects into tidy tibbles} {[}Manual{]}.
\url{https://CRAN.R-project.org/package=broom}

\leavevmode\vadjust pre{\hypertarget{ref-roderStudyingBacterial2016}{}}%
Røder, H. L., Sørensen, S. J., \& Burmølle, M. (2016). Studying
{Bacterial Multispecies Biofilms}: {Where} to {Start}? \emph{Trends in
Microbiology}, \emph{24}(6), 503--513.
\url{https://doi.org/10.1016/j.tim.2016.02.019}

\leavevmode\vadjust pre{\hypertarget{ref-rogersPalaeopathologyJoint2000}{}}%
Rogers, J. (2000). The palaeopathology of joint disease. In M. Cox \& S.
Mays (Eds.), \emph{Human osteology : {In} archaeology and forensic
science.} (1st ed, pp. 163--182). {Cambridge University Press}.
\url{https://login.ezproxy.leidenuniv.nl:2443/login?URL=https://search.ebscohost.com/login.aspx?direct=true\&db=e000xww\&AN=40641\&site=ehost-live}

\leavevmode\vadjust pre{\hypertarget{ref-rogersRoleStreptococcus2001}{}}%
Rogers, J. D., Palmer, R. J., Kolenbrander, P. E., \& Scannapieco, F. A.
(2001). Role of {Streptococcus} gordonii {Amylase-Binding Protein A} in
{Adhesion} to {Hydroxyapatite}, {Starch Metabolism}, and {Biofilm
Formation}. \emph{Infection and Immunity}, \emph{69}(11), 7046--7056.
\url{https://doi.org/10.1128/IAI.69.11.7046-7056.2001}

\leavevmode\vadjust pre{\hypertarget{ref-rohanizadehUltrastructuralStudy2005}{}}%
Rohanizadeh, R., \& LeGeros, R. Z. (2005). Ultrastructural study of
calculus--enamel and calculus--root interfaces. \emph{Archives of Oral
Biology}, \emph{50}(1), 89--96.
\url{https://doi.org/10.1016/j.archoralbio.2004.07.001}

\leavevmode\vadjust pre{\hypertarget{ref-RmixOmics}{}}%
Rohart, F., Gautier, B., Singh, A., \& Le Cao, K.-A. (2017). {mixOmics}:
{An R} package for 'omics feature selection and multiple data
integration. \emph{PLoS Computational Biology}, \emph{13}(11), e1005752.
\url{http://www.mixOmics.org}

\leavevmode\vadjust pre{\hypertarget{ref-sagneStudiesPeriodontal1977}{}}%
Sagne, S., \& Olsson, G. (1977). Studies of the {Periodontal Status} of
a {Medieval Population}. \emph{Dentomaxillofacial Radiology},
\emph{6}(1), 46--52. \url{https://doi.org/10.1259/dmfr.1977.0006}

\leavevmode\vadjust pre{\hypertarget{ref-scannapiecoRoleOral1999}{}}%
Scannapieco, F. A. (1999). Role of {Oral Bacteria} in {Respiratory
Infection}. \emph{Journal of Periodontology}, \emph{70}(7), 793--802.
\url{https://doi.org/10.1902/jop.1999.70.7.793}

\leavevmode\vadjust pre{\hypertarget{ref-scannapiecoPotentialAssociations2001}{}}%
Scannapieco, F. A., \& Ho, A. W. (2001). Potential {Associations Between
Chronic Respiratory Disease} and {Periodontal Disease}: {Analysis} of
{National Health} and {Nutrition Examination Survey III}. \emph{Journal
of Periodontology}, \emph{72}(1), 50--56.
\url{https://doi.org/10.1902/jop.2001.72.1.50}

\leavevmode\vadjust pre{\hypertarget{ref-scannapiecoSalivaryAmylase1993}{}}%
Scannapieco, F. A., Torres, G., \& Levine, M. J. (1993). Salivary
α-amylase: Role in dental plaque and caries formation. \emph{Critical
Reviews in Oral Biology \& Medicine}, \emph{4}(3), 301--307.

\leavevmode\vadjust pre{\hypertarget{ref-scheltemaOpiumTrade1907}{}}%
Scheltema, J. F. (1907). The {Opium Trade} in the {Dutch East Indies}.
{I}. \emph{American Journal of Sociology}, \emph{13}(1), 79--112.

\leavevmode\vadjust pre{\hypertarget{ref-AdapterRemovalv2}{}}%
Schubert, M., Lindgreen, S., \& Orlando, L. (2016). {AdapterRemoval} v2:
Rapid adapter trimming, identification, and read merging. \emph{BMC
Research Notes}, \emph{9}, 88.
\url{https://doi.org/10.1186/s13104-016-1900-2}

\leavevmode\vadjust pre{\hypertarget{ref-schuijtemakerTeTheegasten2011}{}}%
Schuijtemaker, D. (2011). Te Theegasten. \emph{De Nieuwe Schouwschuit},
\emph{9}, 16--17.

\leavevmode\vadjust pre{\hypertarget{ref-scottExoticFoods2021}{}}%
Scott, A., Power, R. C., Altmann-Wendling, V., Artzy, M., Martin, M. A.
S., Eisenmann, S., Hagan, R., Salazar-García, D. C., Salmon, Y.,
Yegorov, D., Milevski, I., Finkelstein, I., Stockhammer, P. W., \&
Warinner, C. (2021). Exotic foods reveal contact between {South Asia}
and the {Near East} during the second millennium {BCE}.
\emph{Proceedings of the National Academy of Sciences}, \emph{118}(2),
e2014956117. \url{https://doi.org/10.1073/pnas.2014956117}

\leavevmode\vadjust pre{\hypertarget{ref-cummingsMayanCalculus1997}{}}%
Scott Cummings, L., \& Magennis, A. (1997). A phytolith and starch
record of food and grit in {Mayan} human tooth tartar. In A. Pinilla, J.
Juan-Tresserras, \& M. J. Machado (Eds.), \emph{The {State-of-the-Art}
of {Phytoliths} in {Soils} and {Plants}}. {CSIC Press}.
\url{https://books.google.com?id=j66CDVfVhwEC}

\leavevmode\vadjust pre{\hypertarget{ref-scottBriefHistory2015}{}}%
Scott, G. R. (2015). A {Brief History} of {Dental Anthropology}. In J.
D. Irish \& G. R. Scott (Eds.), \emph{A {Companion} to {Dental
Anthropology}} (pp. 7--17). {John Wiley \& Sons, Ltd}.
\url{https://doi.org/10.1002/9781118845486.ch18}

\leavevmode\vadjust pre{\hypertarget{ref-seidemannStarchAtlas1966}{}}%
Seidemann, J. (1966). \emph{St\{\textbackslash"a\}rke-{Atlas}:
{Grundlagen} der {St}\{\textbackslash"a\}rke-{Mikroskopie} und
{Beschreibung} der wichtigsten {St}\{\textbackslash"a\}rkearten}.
{Parey}.

\leavevmode\vadjust pre{\hypertarget{ref-shawCommonalityElastic2004}{}}%
Shaw, T., Winston, M., Rupp, C. J., Klapper, I., \& Stoodley, P. (2004).
Commonality of {Elastic Relaxation Times} in {Biofilms}. \emph{Physical
Review Letters}, \emph{93}(9), 098102.
\url{https://doi.org/10.1103/PhysRevLett.93.098102}

\leavevmode\vadjust pre{\hypertarget{ref-shellisSyntheticSaliva1978}{}}%
Shellis, R. P. (1978). A synthetic saliva for cultural studies of dental
plaque. \emph{Archives of Oral Biology}, \emph{23}(6), 485--489.
\url{https://doi.org/10.1016/0003-9969(78)90081-X}

\leavevmode\vadjust pre{\hypertarget{ref-sidawayMicrobiologicalStudy1978a}{}}%
Sidaway, D. A. (1978). A microbiological study of dental calculus.
\emph{Journal of Periodontal Research}, \emph{13}(4), 360--366.
\url{https://doi.org/10.1111/j.1600-0765.1978.tb00190.x}

\leavevmode\vadjust pre{\hypertarget{ref-simonsoroOralGeography2013}{}}%
Simón-Soro, A., Tomás, I., Cabrera-Rubio, R., Catalan, M. D., Nyvad, B.,
\& Mira, A. (2013). Microbial geography of the oral cavity.
\emph{Journal of Dental Research}, \emph{92}(7), 616--621.
\url{https://doi.org/10.1177/0022034513488119}

\leavevmode\vadjust pre{\hypertarget{ref-sissonsArtificialPlaque1997}{}}%
Sissons, C. H. (1997). Artificial {Dental Plaque Biofilm Model Systems}.
\emph{Advances in Dental Research}, \emph{11}(1), 110--126.
\url{https://doi.org/10.1177/08959374970110010201}

\leavevmode\vadjust pre{\hypertarget{ref-sissonsMultistationPlaque1991}{}}%
Sissons, C. H., Cutress, T. W., Hoffman, M. P., \& Wakefield, J. S. J.
(1991). A {Multi-station Dental Plaque Microcosm} ({Artificial Mouth})
for the {Study} of {Plaque Growth}, {Metabolism}, {pH}, and
{Mineralization}: \emph{Journal of Dental Research}.
\url{https://doi.org/10.1177/00220345910700110301}

\leavevmode\vadjust pre{\hypertarget{ref-sissonsPHResponse1994}{}}%
Sissons, C. H., Wong, L., Hancock, E. M., \& Cutress, T. W. (1994). The
{pH} response to urea and the effect of liquid flow in {``artificial
mouth''} microcosm plaques. \emph{Archives of Oral Biology},
\emph{39}(6), 497--505.
\url{https://doi.org/10.1016/0003-9969(94)90146-5}

\leavevmode\vadjust pre{\hypertarget{ref-skoglundSeparatingEndogenous2014}{}}%
Skoglund, P., Northoff, B. H., Shunkov, M. V., Derevianko, A. P., Pääbo,
S., Krause, J., \& Jakobsson, M. (2014). Separating endogenous ancient
{DNA} from modern day contamination in a {Siberian Neandertal}.
\emph{Proceedings of the National Academy of Sciences}, \emph{111}(6),
2229--2234. \url{https://doi.org/10.1073/pnas.1318934111}

\leavevmode\vadjust pre{\hypertarget{ref-slavinDiagnosisManagement2005}{}}%
Slavin, R. G., Spector, S. L., Bernstein, I. L., Slavin, R. G., Kaliner,
M. A., Kennedy, D. W., Virant, F. S., Wald, E. R., Khan, D. A.,
Blessing-Moore, J., Lang, D. M., Nicklas, R. A., Oppenheimer, J. J.,
Portnoy, J. M., Schuller, D. E., Tilles, S. A., Borish, L., Nathan, R.
A., Smart, B. A., \& Vandewalker, M. L. (2005). The diagnosis and
management of sinusitis: {A} practice parameter update. \emph{Journal of
Allergy and Clinical Immunology}, \emph{116}, S13--S47.
\url{https://doi.org/10.1016/j.jaci.2005.09.048}

\leavevmode\vadjust pre{\hypertarget{ref-smithDetectionOpium2018}{}}%
Smith, R. K., Stacey, R. J., Bergström, E., \& Thomas-Oates, J. (2018).
Detection of opium alkaloids in a {Cypriot} base-ring juglet.
\emph{Analyst}, \emph{143}(21), 5127--5136.
\url{https://doi.org/10.1039/C8AN01040D}

\leavevmode\vadjust pre{\hypertarget{ref-songEffectsMaterial2015}{}}%
Song, F., Koo, H., \& Ren, D. (2015). Effects of {Material Properties}
on {Bacterial Adhesion} and {Biofilm Formation}. \emph{Journal of Dental
Research}, \emph{94}(8), 1027--1034.
\url{https://doi.org/10.1177/0022034515587690}

\leavevmode\vadjust pre{\hypertarget{ref-sorensenEffectAntioxidants2018}{}}%
Sørensen, L. K., \& Hasselstrøm, J. B. (2018). The effect of
antioxidants on the long-term stability of {THC} and related
cannabinoids in sampled whole blood. \emph{Drug Testing and Analysis},
\emph{10}(2), 301--309. \url{https://doi.org/10.1002/dta.2221}

\leavevmode\vadjust pre{\hypertarget{ref-sorensenDrugsCalculus2021}{}}%
Sørensen, L. K., Hasselstrøm, J. B., Larsen, L. S., \& Bindslev, D. A.
(2021). Entrapment of drugs in dental calculus -- {Detection} validation
based on test results from post-mortem investigations. \emph{Forensic
Science International}, \emph{319}, 110647.
\url{https://doi.org/10.1016/j.forsciint.2020.110647}

\leavevmode\vadjust pre{\hypertarget{ref-sotoCharacterizationDecontamination2019}{}}%
Soto, M., Inwood, J., Clarke, S., Crowther, A., Covelli, D., Favreau,
J., Itambu, M., Larter, S., Lee, P., Lozano, M., Maley, J., Mwambwiga,
A., Patalano, R., Sammynaiken, R., Vergès, J. M., Zhu, J., \& Mercader,
J. (2019). Structural characterization and decontamination of dental
calculus for ancient starch research. \emph{Archaeological and
Anthropological Sciences}, \emph{11}(9), 4847--4872.
\url{https://doi.org/10.1007/s12520-019-00830-7}

\leavevmode\vadjust pre{\hypertarget{ref-springfieldCocaineMetabolites1993}{}}%
Springfield, A. C., Cartmell, L. W., Aufderheide, A. C., Buikstra, J.,
\& Ho, J. (1993). Cocaine and metabolites in the hair of ancient
{Peruvian} coca leaf chewers. \emph{Forensic Science International},
\emph{63}(1-3), 269--275.
\url{https://doi.org/10.1016/0379-0738(93)90280-N}

\leavevmode\vadjust pre{\hypertarget{ref-squierOralMucosa1998}{}}%
Squier, C. A., \& Finkelstein, M. W. (1998). Oral {Mucosa}. In A. R. Ten
Cate (Ed.), \emph{Oral {Histology}: {Development}, {Structure}, and
{Function}} (5th ed., pp. 345--385). {Mosby}.

\leavevmode\vadjust pre{\hypertarget{ref-srdjenovicSimultaneousHPLC2008}{}}%
Srdjenovic, B., Djordjevic-Milic, V., Grujic, N., Injac, R., \&
Lepojevic, Z. (2008). Simultaneous {HPLC Determination} of {Caffeine},
{Theobromine}, and {Theophylline} in {Food}, {Drinks}, and {Herbal
Products}. \emph{Journal of Chromatographic Science}, \emph{46}(2),
144--149. \url{https://doi.org/10.1093/chromsci/46.2.144}

\leavevmode\vadjust pre{\hypertarget{ref-stahlDoublestrandedIndexing2019}{}}%
Stahl, R., Warinner, C., Velsko, I., Orfanou, E., Aron, F., \& Brandt,
G. (2019). Illumina double-stranded {DNA} dual indexing for ancient
{DNA} v1 {[}{Data} set{]}. \emph{Protocols. Io}.

\leavevmode\vadjust pre{\hypertarget{ref-stavricVariabilityCaffeine1988}{}}%
Stavric, B., Klassen, R., Watkinson, B., Karpinski, K., Stapley, R., \&
Fried, P. (1988). Variability in caffeine consumption from coffee and
tea: {Possible} significance for epidemiological studies. \emph{Food and
Chemical Toxicology}, \emph{26}(2), 111--118.
\url{https://doi.org/10.1016/0278-6915(88)90107-X}

\leavevmode\vadjust pre{\hypertarget{ref-stephanStudiesChanges1947}{}}%
Stephan, R. M., \& Hemmens, E. S. (1947). Studies of changes in {pH}
produced by pure cultures of oral micro-organisms; effects of varying
the microbic cell concentration; comparison of different micro-organisms
and different substrates; some effects of mixing certain
micro-organisms. \emph{Journal of Dental Research}, \emph{26}(1),
15--41. \url{https://doi.org/10.1177/00220345470260010201}

\leavevmode\vadjust pre{\hypertarget{ref-stewartAntimicrobialTolerance2015}{}}%
Stewart, P. S. (2015). Antimicrobial {Tolerance} in {Biofilms}.
\emph{Microbiology Spectrum}, \emph{3}(3),
10.1128/microbiolspec.mb-0010-2014.
\url{https://doi.org/10.1128/microbiolspec.mb-0010-2014}

\leavevmode\vadjust pre{\hypertarget{ref-sunMetabolomicsEvaluation2016}{}}%
Sun, J., Jin, J., Beger, R. D., Cerniglia, C. E., Yang, M., \& Chen, H.
(2016). Metabolomics evaluation of the impact of smokeless tobacco
exposure on the oral bacterium {Capnocytophaga} sputigena.
\emph{Toxicology in Vitro}, \emph{36}, 133--141.
\url{https://doi.org/10.1016/j.tiv.2016.07.020}

\leavevmode\vadjust pre{\hypertarget{ref-sutherlandBiofilmMatrix2001}{}}%
Sutherland, I. W. (2001). The biofilm matrix -- an immobilized but
dynamic microbial environment. \emph{Trends in Microbiology},
\emph{9}(5), 222--227.
\url{https://doi.org/10.1016/S0966-842X(01)02012-1}

\leavevmode\vadjust pre{\hypertarget{ref-takahashiOralMicrobiome2015}{}}%
Takahashi, N. (2015). Oral {Microbiome Metabolism}: {From} {``{Who Are
They}?''} To {``{What Are They Doing}?''} \emph{Journal of Dental
Research}, \emph{94}(12), 1628--1637.
\url{https://doi.org/10.1177/0022034515606045}

\leavevmode\vadjust pre{\hypertarget{ref-takazoeCalciumHydroxyapatite1970}{}}%
Takazoe, I., Vogel, J., \& Ennever, J. (1970). Calcium {Hydroxyapatite
Nucleation} by {Lipid Extract} of {Bacterionema} matruchotii.
\emph{Journal of Dental Research}, \emph{49}(2), 395--398.
\url{https://doi.org/10.1177/00220345700490023301}

\leavevmode\vadjust pre{\hypertarget{ref-takenakaDiffusionMacromolecules2009}{}}%
Takenaka, S., Pitts, B., Trivedi, H. M., \& Stewart, P. S. (2009).
Diffusion of {Macromolecules} in {Model Oral Biofilms}. \emph{Applied
and Environmental Microbiology}, \emph{75}(6), 1750--1753.
\url{https://doi.org/10.1128/AEM.02279-08}

\leavevmode\vadjust pre{\hypertarget{ref-tanCalculusUltrastructure2004}{}}%
Tan, B. T. K., Gillam, D. G., Mordan, N. J., \& Galgut, P. N. (2004). A
preliminary investigation into the ultrastructure of dental calculus and
associated bacteria. \emph{Journal of Clinical Periodontology},
\emph{31}(5), 364--369.
\url{https://doi.org/10.1111/j.1600-051X.2004.00484.x}

\leavevmode\vadjust pre{\hypertarget{ref-tanBacterialViability2004}{}}%
Tan, B. T. K., Mordan, N. J., Embleton, J., Pratten, J., \& Galgut, P.
N. (2004). Study of {Bacterial Viability} within {Human Supragingival
Dental Calculus}. \emph{Journal of Periodontology}, \emph{75}(1),
23--29. \url{https://doi.org/10.1902/jop.2004.75.1.23}

\leavevmode\vadjust pre{\hypertarget{ref-tanAllTogether2017}{}}%
Tan, C. H., Lee, K. W. K., Burmølle, M., Kjelleberg, S., \& Rice, S. A.
(2017). All together now: Experimental multispecies biofilm model
systems. \emph{Environmental Microbiology}, \emph{19}(1), 42--53.
\url{https://doi.org/10.1111/1462-2920.13594}

\leavevmode\vadjust pre{\hypertarget{ref-taoWheatCalculus2020}{}}%
Tao, D., Zhang, G., Zhou, Y., \& Zhao, H. (2020). Investigating wheat
consumption based on multiple evidences: {Stable} isotope analysis on
human bone and starch grain analysis on dental calculus of humans from
the {Laodaojing} cemetery, {Central Plains}, {China}.
\emph{International Journal of Osteoarchaeology}, \emph{30}(5),
594--606. \url{https://doi.org/10.1002/oa.2884}

\leavevmode\vadjust pre{\hypertarget{ref-theiladeGermfreeCalculus1964}{}}%
Theilade, J., Fitzgerald, R. J., Scott, D. B., \& Nylen, M. U. (1964).
Electron microscopic observations of dental calculus in germfree and
conventional rats. \emph{Archives of Oral Biology}, \emph{9}(1),
97--IN17. \url{https://doi.org/10.1016/0003-9969(64)90051-2}

\leavevmode\vadjust pre{\hypertarget{ref-tianUsingDGGE2010}{}}%
Tian, Y., He, X., Torralba, M., Yooseph, S., Nelson, K. e., Lux, R.,
McLean, J. s., Yu, G., \& Shi, W. (2010). Using {DGGE} profiling to
develop a novel culture medium suitable for oral microbial communities.
\emph{Molecular Oral Microbiology}, \emph{25}(5), 357--367.
\url{https://doi.org/10.1111/j.2041-1014.2010.00585.x}

\leavevmode\vadjust pre{\hypertarget{ref-tonjumNeisseria2017}{}}%
Tønjum, T., \& van Putten, J. (2017). 179 - {Neisseria}. In J. Cohen, W.
G. Powderly, \& S. M. Opal (Eds.), \emph{Infectious {Diseases} ({Fourth
Edition})} (pp. 1553--1564.e1). {Elsevier}.
\url{https://doi.org/10.1016/B978-0-7020-6285-8.00179-9}

\leavevmode\vadjust pre{\hypertarget{ref-toppingResistantStarch2003}{}}%
Topping, D. L., Fukushima, M., \& Bird, A. R. (2003). Resistant starch
as a prebiotic and synbiotic: State of the art. \emph{Proceedings of the
Nutrition Society}, \emph{62}(1), 171--176.
\url{https://doi.org/10.1079/PNS2002224}

\leavevmode\vadjust pre{\hypertarget{ref-townsendDentalAnthropology2012}{}}%
Townsend, G., Kanazawa, E., \& Takayama, H. (Eds.). (2012). \emph{New
{Directions} in {Dental Anthropology}: {Paradigms}, {Methodologies} and
{Outcomes}}. {The University of Adelaide Press}.
\url{https://doi.org/10.1017/9780987171870}

\leavevmode\vadjust pre{\hypertarget{ref-trompEDTACalculus2017}{}}%
Tromp, M., Buckley, H., Geber, J., \& Matisoo-Smith, E. (2017). {EDTA}
decalcification of dental calculus as an alternate means of
microparticle extraction from archaeological samples. \emph{Journal of
Archaeological Science: Reports}, \emph{14}, 461--466.
\url{https://doi.org/10.1016/j.jasrep.2017.06.035}

\leavevmode\vadjust pre{\hypertarget{ref-trompDietaryNondietary2015}{}}%
Tromp, M., \& Dudgeon, J. V. (2015). Differentiating dietary and
non-dietary microfossils extracted from human dental calculus: The
importance of sweet potato to ancient diet on {Rapa Nui}. \emph{Journal
of Archaeological Science}, \emph{54}, 54--63.
\url{https://doi.org/10.1016/j.jas.2014.11.024}

\leavevmode\vadjust pre{\hypertarget{ref-tushinghamHuntergathererTobacco2013}{}}%
Tushingham, S., Ardura, D., Eerkens, J. W., Palazoglu, M., Shahbaz, S.,
\& Fiehn, O. (2013). Hunter-gatherer tobacco smoking: Earliest evidence
from the {Pacific Northwest Coast} of {North America}. \emph{Journal of
Archaeological Science}, \emph{40}(2), 1397--1407.
\url{https://doi.org/10.1016/j.jas.2012.09.019}

\leavevmode\vadjust pre{\hypertarget{ref-uzelMicrobialShifts2011}{}}%
Uzel, N. G., Teles, F. R., Teles, R. P., Song, X. Q., Torresyap, G.,
Socransky, S. S., \& Haffajee, A. D. (2011). Microbial shifts during
dental biofilm re-development in the absence of oral hygiene in
periodontal health and disease. \emph{Journal of Clinical
Periodontology}, \emph{38}(7), 612--620.
\url{https://doi.org/10.1111/j.1600-051X.2011.01730.x}

\leavevmode\vadjust pre{\hypertarget{ref-valenDetermination212017}{}}%
Valen, A., Leere Øiestad, Å. M., Strand, D. H., Skari, R., \& Berg, T.
(2017). Determination of 21 drugs in oral fluid using fully automated
supported liquid extraction and {UHPLC-MS}/{MS}. \emph{Drug Testing and
Analysis}, \emph{9}(5), 808--823. \url{https://doi.org/10.1002/dta.2045}

\leavevmode\vadjust pre{\hypertarget{ref-vandeveldeStarchMorphology2002}{}}%
van de Velde, F., van Riel, J., \& Tromp, R. H. (2002). Visualisation of
starch granule morphologies using confocal scanning laser microscopy
({CSLM}). \emph{Journal of the Science of Food and Agriculture},
\emph{82}(13), 1528--1536. \url{https://doi.org/10.1002/jsfa.1165}

\leavevmode\vadjust pre{\hypertarget{ref-vandermeerschMiddlePaleolithic1994}{}}%
Vandermeersch, B., Arensburg, B., Tillier, A. M., Rak, Y., Weiner, S.,
Spiers, M., \& Aspillaga, E. (1994). Middle {Paleolithic Dental Bacteria
From Kebara}, {Israel}. \emph{Comptes Rendus De L Academie Des Sciences
Serie Ii}, \emph{319}(6), 727--731.
\url{https://weizmann.esploro.exlibrisgroup.com/esploro/outputs/journalArticle/MIDDLE-PALEOLITHIC-DENTAL-BACTERIA-FROM-KEBARA/993266802803596}

\leavevmode\vadjust pre{\hypertarget{ref-velskoCytokineResponse2017}{}}%
Velsko, I. M., Cruz-Almeida, Y., Huang, H., Wallet, S. M., \& Shaddox,
L. M. (2017). Cytokine response patterns to complex biofilms by
mononuclear cells discriminate patient disease status and biofilm
dysbiosis. \emph{Journal of Oral Microbiology}, \emph{9}(1), 1330645.
\url{https://doi.org/10.1080/20002297.2017.1330645}

\leavevmode\vadjust pre{\hypertarget{ref-velskoMicrobialDifferences2019}{}}%
Velsko, I. M., Fellows Yates, J. A., Aron, F., Hagan, R. W., Frantz, L.
A. F., Loe, L., Martinez, J. B. R., Chaves, E., Gosden, C., Larson, G.,
\& Warinner, C. (2019). Microbial differences between dental plaque and
historic dental calculus are related to oral biofilm maturation stage.
\emph{Microbiome}, \emph{7}(1), 102.
\url{https://doi.org/10.1186/s40168-019-0717-3}

\leavevmode\vadjust pre{\hypertarget{ref-velskoHighConservation2023}{}}%
Velsko, I. M., Gallois, S., Stahl, R., Henry, A. G., \& Warinner, C.
(2023). High conservation of the dental plaque microbiome across
populations with differing subsistence strategies and levels of market
integration. \emph{Molecular Ecology}.
\url{https://doi.org/10.1111/mec.16988}

\leavevmode\vadjust pre{\hypertarget{ref-velskoDentalCalculus2017}{}}%
Velsko, I. M., Overmyer, K. A., Speller, C., Klaus, L., Collins, M. J.,
Loe, L., Frantz, L. A. F., Sankaranarayanan, K., Lewis, C. M., Martinez,
J. B. R., Chaves, E., Coon, J. J., Larson, G., \& Warinner, C. (2017).
The dental calculus metabolome in modern and historic samples.
\emph{Metabolomics}, \emph{13}(11), 134.
\url{https://doi.org/10.1007/s11306-017-1270-3}

\leavevmode\vadjust pre{\hypertarget{ref-velskoConsistentReproducible2018}{}}%
Velsko, I. M., \& Shaddox, L. M. (2018). Consistent and reproducible
long-term in vitro growth of health and disease-associated oral
subgingival biofilms. \emph{BMC Microbiology}, \emph{18}(1), 70.
\url{https://doi.org/10.1186/s12866-018-1212-x}

\leavevmode\vadjust pre{\hypertarget{ref-vigeantReversibleIrreversible2002}{}}%
Vigeant, M. A.-S., Ford, R. M., Wagner, M., \& Tamm, L. K. (2002).
Reversible and {Irreversible Adhesion} of {Motile Escherichia} coli
{Cells Analyzed} by {Total Internal Reflection Aqueous Fluorescence
Microscopy}. \emph{Applied and Environmental Microbiology},
\emph{68}(6), 2794--2801.
\url{https://doi.org/10.1128/AEM.68.6.2794-2801.2002}

\leavevmode\vadjust pre{\hypertarget{ref-waldronPalaeopathology2020}{}}%
Waldron, T. (2020). \emph{Palaeopathology}. {Cambridge University
Press}.

\leavevmode\vadjust pre{\hypertarget{ref-warinnerEvidenceMilk2014}{}}%
Warinner, C., Hendy, J., Speller, C., Cappellini, E., Fischer, R.,
Trachsel, C., Arneborg, J., Lynnerup, N., Craig, O. E., Swallow, D. M.,
Fotakis, A., Christensen, R. J., Olsen, J. V., Liebert, A., Montalva,
N., Fiddyment, S., Charlton, S., Mackie, M., Canci, A., \ldots{}
Collins, M. J. (2014). Direct evidence of milk consumption from ancient
human dental calculus. \emph{Scientific Reports}, \emph{4}, 7104.
\url{https://doi.org/10.1038/srep07104}

\leavevmode\vadjust pre{\hypertarget{ref-warinnerPathogensHost2014}{}}%
Warinner, C., Rodrigues, J. F., Vyas, R., Trachsel, C., Shved, N.,
Grossmann, J., Radini, A., Hancock, Y., Tito, R. Y., Fiddyment, S.,
Speller, C., Hendy, J., Charlton, S., Luder, H. U., Salazar-Garcia, D.
C., Eppler, E., Seiler, R., Hansen, L. H., Castruita, J. A., \ldots{}
Cappellini, E. (2014). Pathogens and host immunity in the ancient human
oral cavity. \emph{Nature Genetics}, \emph{46}(4), 336--344.
\url{https://doi.org/10.1038/ng.2906}

\leavevmode\vadjust pre{\hypertarget{ref-warinnerNewEra2015}{}}%
Warinner, C., Speller, C., \& Collins, M. J. (2015). A new era in
palaeomicrobiology: Prospects for ancient dental calculus as a long-term
record of the human oral microbiome. \emph{Philosophical Transactions of
the Royal Society B: Biological Sciences}, \emph{370}(1660), 20130376.
\url{https://doi.org/10.1098/rstb.2013.0376}

\leavevmode\vadjust pre{\hypertarget{ref-weinerBiologicalMaterials2010}{}}%
Weiner, S. (2010a). Biological {Materials}: {Bones} and {Teeth}. In
\emph{Microarchaeology: {Beyond} the {Visible Archaeological Record}}
(pp. 99--134). {Cambridge University Press}.

\leavevmode\vadjust pre{\hypertarget{ref-weinerInfraredSpectroscopy2010}{}}%
Weiner, S. (2010b). Infrared {Spectroscopy} in {Archaeology}. In
\emph{Microarchaeology: {Beyond} the {Visible Archaeological Record}}
(1st ed., pp. 275--316). {Cambridge University Press}.
\url{https://doi.org/10.1017/CBO9780511811210}

\leavevmode\vadjust pre{\hypertarget{ref-weinerStatesPreservation1990}{}}%
Weiner, S., \& Bar-Yosef, O. (1990). States of preservation of bones
from prehistoric sites in the {Near East}: {A} survey. \emph{Journal of
Archaeological Science}, \emph{17}(2), 187--196.
\url{https://doi.org/10.1016/0305-4403(90)90058-D}

\leavevmode\vadjust pre{\hypertarget{ref-wesolowskiEvaluatingMicrofossil2010}{}}%
Wesolowski, V., Ferraz Mendonça de Souza, S. M., Reinhard, K. J., \&
Ceccantini, G. (2010). Evaluating microfossil content of dental calculus
from {Brazilian} sambaquis. \emph{Journal of Archaeological Science},
\emph{37}(6), 1326--1338.
\url{https://doi.org/10.1016/j.jas.2009.12.037}

\leavevmode\vadjust pre{\hypertarget{ref-whiteDentalCalculus1997}{}}%
White, D. J. (1997). Dental calculus: Recent insights into occurrence,
formation, prevention, removal and oral health effects of supragingival
and subgingival deposits. \emph{European Journal of Oral Sciences},
\emph{105}(5), 508--522.
\url{https://doi.org/10.1111/j.1600-0722.1997.tb00238.x}

\leavevmode\vadjust pre{\hypertarget{ref-whiteHumanOsteology2011}{}}%
White, T. D., Black, M. T., \& Folkens, P. A. (2011). \emph{Human
{Osteology}} (3rd edition). {Academic Press}.

\leavevmode\vadjust pre{\hypertarget{ref-whiteBoneManual2005}{}}%
White, T. D., \& Folkens, P. A. (2005). \emph{The {Human Bone Manual}}
(1st edition). {Academic Press}.

\leavevmode\vadjust pre{\hypertarget{ref-ggplot2}{}}%
Wickham, H. (2016). \emph{Ggplot2: {Elegant Graphics} for {Data
Analysis}}. {Springer-Verlag}. \url{https://ggplot2.tidyverse.org}

\leavevmode\vadjust pre{\hypertarget{ref-tidyverse2019}{}}%
Wickham, Hadley, Averick, M., Bryan, J., Chang, W., McGowan, L. D.,
François, R., Grolemund, G., Hayes, A., Henry, L., Hester, J., Kuhn, M.,
Pedersen, T. L., Miller, E., Bache, S. M., Müller, K., Ooms, J.,
Robinson, D., Seidel, D. P., Spinu, V., \ldots{} Yutani, H. (2019).
Welcome to the {tidyverse}. \emph{Journal of Open Source Software},
\emph{4}(43), 1686. \url{https://doi.org/10.21105/joss.01686}

\leavevmode\vadjust pre{\hypertarget{ref-willeRelationshipOral2009}{}}%
Wille, S. M. R., Raes, E., Lillsunde, P., Gunnar, T., Laloup, M., Samyn,
N., Christophersen, A. S., Moeller, M. R., Hammer, K. P., \& Verstraete,
A. G. (2009). Relationship {Between Oral Fluid} and {Blood
Concentrations} of {Drugs} of {Abuse} in {Drivers Suspected} of {Driving
Under} the {Influence} of {Drugs}. \emph{Therapeutic Drug Monitoring},
\emph{31}(4), 511. \url{https://doi.org/10.1097/FTD.0b013e3181ae46ea}

\leavevmode\vadjust pre{\hypertarget{ref-wongCalciumPhosphate2002}{}}%
Wong, L., Sissons, C. H., Pearce, E. I. F., \& Cutress, T. W. (2002).
Calcium phosphate deposition in human dental plaque microcosm biofilms
induced by a ureolytic {pH-rise} procedure. \emph{Archives of Oral
Biology}, \emph{47}(11), 779--790.
\url{https://doi.org/10.1016/S0003-9969(02)00114-0}

\leavevmode\vadjust pre{\hypertarget{ref-kraken2}{}}%
Wood, D. E., Lu, J., \& Langmead, B. (2019). Improved metagenomic
analysis with {Kraken} 2. \emph{Genome Biology}, \emph{20}(1), 257.
\url{https://doi.org/10.1186/s13059-019-1891-0}

\leavevmode\vadjust pre{\hypertarget{ref-wrightAdvancingRefining2021}{}}%
Wright, S. L., Dobney, K., \& Weyrich, L. S. (2021). Advancing and
refining archaeological dental calculus research using multiomic
frameworks. \emph{STAR: Science \& Technology of Archaeological
Research}, \emph{7}(1), 13--30.
\url{https://doi.org/10.1080/20548923.2021.1882122}

\leavevmode\vadjust pre{\hypertarget{ref-wuDietEarliest2021}{}}%
Wu, Y., Tao, D., Wu, X., \& Liu, W. (2021). \emph{Diet of the earliest
modern humans in {East Asia}} {[}Preprint{]}. {In Review}.
\url{https://doi.org/10.21203/rs.3.rs-442096/v1}

\leavevmode\vadjust pre{\hypertarget{ref-yaoIdentificationProtein2003}{}}%
Yao, Y., Berg, E. A., Costello, C. E., Troxler, R. F., \& Oppenheim, F.
G. (2003). Identification of protein components in human acquired enamel
pellicle and whole saliva using novel proteomics approaches. \emph{J
Biol Chem}, \emph{278}(7), 5300--5308.
\url{https://doi.org/10.1074/jbc.M206333200}

\leavevmode\vadjust pre{\hypertarget{ref-yaussyCalculusSurvivorship2019}{}}%
Yaussy, S. L., \& DeWitte, S. N. (2019). Calculus and survivorship in
medieval {London}: {The} association between dental disease and a
demographic measure of general health. \emph{American Journal of
Physical Anthropology}, \emph{168}(3), 552--565.
\url{https://doi.org/10.1002/ajpa.23772}

\leavevmode\vadjust pre{\hypertarget{ref-zeroSituCaries1995}{}}%
Zero, D. T. (1995). In {Situ Caries Models}. \emph{Advances in Dental
Research}, \emph{9}(3), 214--230.
\url{https://doi.org/10.1177/08959374950090030501}

\leavevmode\vadjust pre{\hypertarget{ref-zhangMeasurementPolysaccharides1998}{}}%
Zhang, X., Bishop, P. L., \& Kupferle, M. J. (1998). Measurement of
polysaccharides and proteins in biofilm extracellular polymers.
\emph{Water Science and Technology}, \emph{37}(4), 345--348.
\url{https://doi.org/10.1016/S0273-1223(98)00127-9}

\leavevmode\vadjust pre{\hypertarget{ref-zhangDentalDisease1982}{}}%
Zhang, Y. (1982). Dental disease of neolithic age skulls excavated in
shaanxi province. \emph{Chinese Medical Journal}, \emph{95}(06),
391--396. \url{https://doi.org/10.5555/cmj.0366-6999.95.06.p391.01}

\leavevmode\vadjust pre{\hypertarget{ref-ziesemer16SChallenges2015}{}}%
Ziesemer, K. A., Mann, A. E., Sankaranarayanan, K., Schroeder, H., Ozga,
A. T., Brandt, B. W., Zaura, E., Waters-Rist, A., Hoogland, M.,
Salazar-Garcia, D. C., Aldenderfer, M., Speller, C., Hendy, J., Weston,
D. A., MacDonald, S. J., Thomas, G. H., Collins, M. J., Lewis, C. M.,
Hofman, C., \& Warinner, C. (2015). Intrinsic challenges in ancient
microbiome reconstruction using {16S rRNA} gene amplification. \emph{Sci
Rep}, \emph{5}, 16498. \url{https://doi.org/10.1038/srep16498}

\leavevmode\vadjust pre{\hypertarget{ref-ziesemerGenomeCalculus2018}{}}%
Ziesemer, K. A., Ramos‐Madrigal, J., Mann, A. E., Brandt, B. W.,
Sankaranarayanan, K., Ozga, A. T., Hoogland, M., Hofman, C. A.,
Salazar‐García, D. C., Frohlich, B., Milner, G. R., Stone, A. C.,
Aldenderfer, M., Lewis, C. M., Hofman, C. L., Warinner, C., \&
Schroeder, H. (2018). The efficacy of whole human genome capture on
ancient dental calculus and dentin. \emph{American Journal of Physical
Anthropology}. \url{https://doi.org/10.1002/ajpa.23763}

\leavevmode\vadjust pre{\hypertarget{ref-zijngeBiofilmArchitecture2010}{}}%
Zijnge, V., van Leeuwen, M. B. M., Degener, J. E., Abbas, F., Thurnheer,
T., Gmür, R., \& M. Harmsen, H. J. (2010). Oral {Biofilm Architecture}
on {Natural Teeth}. \emph{PLoS ONE}, \emph{5}(2), e9321.
\url{https://doi.org/10.1371/journal.pone.0009321}

\end{CSLReferences}

\backmatter
\end{document}
